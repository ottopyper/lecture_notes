\documentclass{article}

\usepackage{amsmath}
\usepackage{amssymb}
\usepackage{parskip}

\renewcommand{\rm}[1]{\mathrm{#1}}
\newcommand{\mc}[1]{\mathcal{#1}}
\newcommand{\bb}[1]{\mathbb{#1}}
\newcommand{\C}{\mc{C}}
\newcommand{\D}{\mc{D}}
\newcommand{\T}{\mc{T}}
\newcommand{\eps}{\varepsilon}
\newcommand{\ket}[1]{|#1\rangle}
\newcommand{\bra}[1]{\langle #1 |}
\newcommand{\ra}{\rightarrow}

\title{Quantum Information Theory Revision Questions}

\author{Otto Pyper}

\date{}

\begin{document}
    
\maketitle



\begin{enumerate}
    \item Define the surprisial of a random variable.
    \item Define the Shanon entropy of a discrete random variable $X$.
    \item Define binary entropy.
    \item Define a source alphabet.
    \item Define a memoryless source, and characterise it with random variables.
    \item Define the Shannon entropy of a memoryless source.
    \item Generally speaking, why is data compression possible?
    \item Give two methods of data encoding and describe their differences.
    \item Defina a typical/atypical signal.
    \item Define a compression map $C^n$, and its rate.
    \item Define a corresponding decompression map $D^n$.
    \item The triple $\mathcal{C}_n$...?
    \item Write down the average probability of error of $\mathcal{C}_n$.
    \item Define what it means for $\mathcal{C}_n$ to be reliable.
    \item Define the data compression limit for a source.
    \item Define the $\eps$-typical set $T_\eps^{(n)}$, and the typical sequences.
    \item What is the approximate probability of a typical sequence?
    \item Demonstrate that this definition agrees with our intuitive notion of what a typical sequence should be.
    \item State the Typical Sequence Theorem.
    \item Prove (some bits of) the Typical Sequence Theorem.
    \item State a corollary of the TST.
    \item State Shannon's Source Coding Theorem.
    \item State a lemma bounding the probability of a sufficiently small set of sequences.
    \item Sketch a proof.
    \item Explain how this implies the converse of Shannon.
    \item Define joint entropy.
    \item Define conditional entropy.
    \item State and prove the chain rule identity relating joint and conditional entropy.
    \item Define the relative entropy/Kullback Leibler divergence.
    \item State Jensen's inequality.
    \item Show that $D(p||q)\ge 0$. When does equality hold?
    \item Define the mutual information $I(X:Y)$.
    \item \textbf{LECTURE 3 DIAGRAM}
    \item Define $D(P||Q)$ for a pairt of functions (instead of random variables).
    \item Express $H(X), I(X:Y), H(X|Y)$ in terms of the relative entropy.
    \item State the chain rule for entropies.
    \item Define the conditional mutual information $I(X:Y|Z)$.
    \item State the data-processing inequality.
    \item Prove the data-processing inequality.
    \item Show that $D(p||q) \ge 0 $, with equality iff $p = q$.
    \item Show that $H(X) \ge 0$, with equality iff $X$ is deterministic.
    \item Show that $H(X|Y) \ge 0$, or equivalently that $H(X,Y)\ge H(Y)$.
    \item Show that if $X$ takes values in $J$, then $H(X) \le \log |J|$
    \item Show that $H(X,Y) \le H(X) + H(Y)$ (subadditivity).
    \item Show that the Shannon entropy is concave.
    \item Show that $I(X:Y) \ge 0$, with equality iff $X$ and $Y$ are independent.
    \item Define a discrete channel.
    \item Define a memoryless (discrete) channel.
    \item Define a symmetric memoryless channel.
    \item Define the memoryless, binary, symmetric channel.
    \item What is majority voting?
    \item What is the probability of error in this scheme?
    \item How much has this scheme improved the reliability of transmission?
    \item What is this type of error-correcting code called?
    \item \textbf{LECTURE 4 DIAGRAM.}
    \item Define $\mc{C}_n$, an error-correcting code.
    \item Define the probability of error $p(\mc{C}_n)$.
    \item Define an achievable rate for a channel.
    \item Define the capacity $C(\mc{N})$ of a memoryless channel.
    \item State Shannon's Noisy Channel Coding Theorem.
    \item State three properties of $C(\mc{N})$.
    \item Give some intuition behind the proof of SNCCT.
    \item Use SNCCT to calculate the capacity of the binary memoryless symmetric channel.
    \item Give two examples of physical realisations of qubit.s
    \item What are reliably distinguishable states?
    \item What is the Hamming space?
    \item Define $\mc{B}(\mc{H})$.
    \item Define the Hilbert-Schmidt inner product.
    \item Define the Pauli matrices.
    \item What is an open system?
    \item What is decoherence?
    \item State the four postulates of quantum mechanics.
    \item What postulates are no longer valid in open systems?
    \item How can we view open systems to get around this issue?
    \item Describe the density matrix formalism.
    \item State and prove two properties of density matrices.
    \item What can we conclude from these properties?
    \item Give an equivalent, but more abstract, definition of a density matrix.
    \item Give an important property of $\mc{D}(\mc{H})$.
    \item Define a pure state.
    \item Define a mixed state.
    \item Give an equivalent characterisation of pure/mixed states.
    \item What is the purity of a state? What range of values can this take?
    \item \textbf{LECTURE 6 FIGURE AND DISCUSSION.}
    \item Show that $\mc{D}(\mc{H})$ is convex. Show that pure states are extremal points.
    \item Define the expectation of an observable $A$.
    \item Show that this is linear, positive and normal.
    \item Define the reduced density operator/matrix of a bipartite state $\rho_{AB}$.
    \item Define the partial trace.
    \item \textbf{LECTURE 6 DETAILED EXPRESSIONS AND DERIVATIONS.}
    \item Showt that a reduced density matrix is a valid density matrix.
    \item Let $M_{AB} = M_A\otimes I_B$. Show that $\langle M_{AB}\rangle = \textrm{Tr}(M_A\rho_A)$.
    \item Define a separable bipartite pure state, and write down its reduced density operators.
    \item Define an entangled bipartite pure state. What form do its reduced density operators have?
    \item Define maximally entangeled states.
    \item Write down the Bell/EPR states. Who are they named after?
    \item What does it mean for a bipartite \textbf{mixed} state $\rho_{AB}$ to be separable?
    \item What does it mean for a bipartite mixed state to be entangled?
    \item Show that separable bipartite states can always be expressed as a convex combination of pure product states.
    \item State the Schmidt Decomposition Theorem.
    \item Prove the Schmidt Decomposition Theorem.
    \item Given an immediate consequence of this theorem for the reduced density matrices of a pure state $\Psi_{AB}$.
    \item Under what condition is the Schmdit decomposition of $\ket{\Psi_{AB}}$ uniquely determined by $\rho_A$ and $\rho_B$?
    \item Define the Schmidt rank of a bipartite pure state $\ket{\Psi_{AB}}$. How is it denoted?
    \item A bipartite pure state is entangled iff...?
    \item Prove that a BPS is a product state iff its Schimdt number is equal to one.
    \item Prove that a BPS is a product state iff its reduced density matrices are pure states.
    \item When can we apply Schmidt, more generally?
    \item Define the purification of a state $\rho_A$.
    \item Define a reference system for a state $\rho_A$.
    \item Prove that any state $\rho_A$ can be purified.
    \item Give a more general purification of $\rho_A$.
    \item How can we express $\ket{\Psi_{AR}}$ with reduced state $\rho_A$ having eigenvectors $\{\ket{i}\}$.
    \item State the No-Cloning Theorem.
    \item Prove the No-Cloning Theorem.
    \item Why does this extend to full generality?
    \item Why is the No-Cloning Theorem a Big Problem?
    \item Who (independently) devised the first quantum error-correcting codes, and when?
    \item Demonstrate how No-Cloning prevents superluminal communication.
    \item Define a quantum operation, and give a simple example.
    \item Define a linear CPTP map.
    \item Why are quantum operations given by linear CPTPs?
    \item Define $\mc{M}_n^+$.
    \item Why is complete positivity a physically reasonable condition? \textbf{LECTURE 8/9 DISCUSSION.}
    \item Give a map that is positive but not completely positive.
    \item State a theorem that describes when a linear operator map is completely positive.
    \item Prove this theorem.
    \item Define the Choi matrix/state $J(\Lambda)$ for a quantum operation $\Lambda$.
    \item State (a simplified version of) Stinespring's Dilation Theorem.
    \item Therefore, any quantum operation can be composed of which three building blocks?
    \item What does ``going to the church of the larger Hilbert space'' mean? \textbf{SCHEMATIC IN LECTURES.}
    \item State the Kraus Representation Theorem.
    \item Prove one direction of the Kraus Representation Theorem.
    \item Show that it is essentially a restatement of Stinespring.
    \item Is Kraus decomposition unique?
    \item State the Choi-Jamilkowski Isomorphism (Theorem).
    \item Define the adjoint $\Lambda^\ast$ to $\Lambda$.
    \item \textbf{LECTURE 9/10 VERIFY ONE C-J PART.}
    \item What suffices to prove that the C-J maps are mutual inverses?
    \item Prove this.
    \item Prove the other direction of Kraus.
    \item State Stinespring's Dilation Theorem (in full generality).
    \item Prove Stinespring's Dilation Theorem.
    \item Why are standard projective measurements insufficient for us?
    \item Describe the Generalised Measurement Postulate.
    \item What does POVM stand for, and why?
    \item How can projective measurements be viewed as a special case of a generalised measurement?
    \item What is the POVM formalism? When is it used?
    \item What does a POVM not do?
    \item Define a POVM.
    \item What is a pure POVM?
    \item State Neumark's Theorem.
    \item Show that a projective measurement is a special case of a POVM.
    \item Give an example (/case study?) of a POVM being useful.
    \item Describe how to implement a generalised measurement using an ancilla, unitary dynamics and projective measurements.
    \item Define the trace distance $D(\rho,\sigma)$ of two states $\rho,\sigma\in \mc{D}(\mc{H})$.
    \item Define $||A||_1$.
    \item Give a helpful decomposition of the difference operator.
    \item State and prove an identity relating $D$ and this decomposition.
    \item Prove that $D(\rho,\sigma) = \max_{0\le P\le I}\textrm{Tr}(P(\rho-\sigma))$.
    \item Prove that $D$ is a metric on the space of density operators.
    \item Prove the monotonicity of $D$ under quantum operations.
    \item Relate the trace distance to quantum hypothesis testing with binary POVMs.
    \item Which measurement maximises $p^\ast_{\textrm{success}}$?
    \item The trace distance is the...?
    \item Define the fidelity $F(\rho,\sigma)$ for $\rho,\sigma \in \mc{B}(\mc{H})$.
    \item What form does $F$ take when $[\rho,\sigma] = 0$?
    \item What form does $F$ take when one of $\rho,\sigma$ is pure? What about both states pure?
    \item Prove that fidelity is invariant under unitary transformation.
    \item State Uhlmann's Theorem.
    \item Prove that $||A||_1 = \sup_{U}|\textrm{Tr}(UA)|$, $U$ unitary.
    \item Prove Uhlmann's Theorem.
    \item Show that $0\le F(\rho,\sigma) \le 1$, with $=1$ iff $\rho = \sigma$.
    \item Show that $F(\rho,\sigma) = F(\sigma,\rho)$.
    \item Prove that $F(\rho_{AB},\sigma_{AB})\le F(\rho_A,\sigma_A)$.
    \item Define the entanglement fidelity $F_\rm{e}(\rho,\Lambda)$.
    \item Express $F_\rm{e}(\rho,\Lambda)$ in terms of fidelity $F$.
    \item Prove that $F_\rm{e}(\rho,\Lambda) = \sum_{k}|\rm{Tr}(A_k\rho)|^2$, for $A_k$ Kraus operators of $\Lambda$.
    \item Show that $F_\rm{e}(\rho,\Lambda)\le F(\rho,\Lambda(\rho))^2$.
    \item Define the von Neumann entropy of a density matrix $\rho$.
    \item What is its classical analogue?
    \item State four properties of the von Neumann entropy.
    \item Prove concavity.
    \item For $\rho \in \mc{D}(\mc{H})$ and $\sigma \in \mc{B}(\mc{H})$, define the quantum relative entropy $D(\rho||\sigma)$.
    \item State Klein's inequality.
    \item Prove Klein's inequality.
    \item Now prove $S(\rho)\le \log d$.
    \item State the data-processing inequality.
    \item State joint convexity of quantum relative entropy.
    \item \textbf{LIEB'S CONCAVITY THEOREM DISCUSSION LECTURE 14.}
    \item State two properties of $D(\rho||\sigma)$ involving products and unitaries.
    \item Define the Heisenberg-Weyl operators.
    \item Prove that $$\frac{1}{d^2}\sum_{k,m=0}^{d-1}W_{k,m}AW^\dagger_{k,m} = (\rm{Tr}A)I/d$$
    \item Define the quantum joint entropy.
    \item Define the quantum conditional entropy.
    \item Define the quantum mutual information.
    \item Express all of the above quantities in terms of the relative entropy.
    \item What property of classical entropy is not upheld in the quantum case?
    \item Demonstrate the above through an example.
    \item Prove additivity of $S$.
    \item State subadditivity of $S$.
    \item Prove subadditivity of $S$. [or outline a proof] To what classical property is this analogous?
    \item Prove equality of entropies of subsystems of a bipartite pure state.
    \item State and prove the triangle inequality for $S$.
    \item Prove that $S(\sum_i p_i \rho_i) = H(p) + \sum_i p_i S(\rho_i)$.
    \item State strong subadditivity. How is it proved?
    \item \textbf{LECTURE 15 CHECK FOR PROOF OF SSA.}
    \item State three consequences of SSA. [prove them?]
    \item Prove the last of these.
    \item Define a quantum information source.
    \item Given an equivalent formulation of the above.
    \item What is a (quantum) compression scheme?
    \item What is a corresponding (quantum) decompression scheme?
    \item What property must both $\mc{C}^{(n)}$ and $\mc{D}^{(n)}$ have?
    \item Define the \textit{rate} of a compression-decompression scheme.
    \item What does it mean for a compression scheme $(\C^{(n)},\D^{(n)})$ to have rate $R$?
    \item What quantity is being compressed in a quantum compression scheme?
    \item What key difference between classical and quantum compression makes state reconstruction difficult?
    \item Define the ensemble average fidelity.
    \item Define what it means for a compression-decompression scheme to be \textit{reliable}.
    \item What is a typical subspace?
    \item Give an expression for the density matrix of a memoryless/iid quantum channel.
    \item Write down the spectral decomposition for the density matrix $\rho^{(n)}$ of a memoryless source.
    \item What is the von Neumann entropy of such a source?
    \item Define $T_\eps^(n)$.
    \item Give a bound on the probability of a (classical) typical sequence $(i_1\dots i_n)$.
    \item Define $\T^{(n)}$. What is it called?
    \item State the Typical Subspace Theorem.
    \item What is the probability of the typical subspace $\T_\eps^{(n)}$?
    \item State Schumacher's Theorem.
    \item \textbf{Prove Schumacher's Theorem.} [This might be asking a bit much lmao.]
    \item What is a quantum channel?
    \item Describe the Bloch sphere representation of a qubit.
    \item Give four examples of single qubit channels.
    \item Write down the Bit-flip channel and its Kraus operators.
    \item How does it act on the Bloch sphere?
    \item Describe the class or random unitary channels/mixing-enhancing channels.
    \item Describe the Phase flip channel and its Kraus operators.
    \item How does it act on the Bloch sphere?
    \item Describe the Depolarising channel and its Kraus operators.
    \item How does it act on the Bloch sphere?
    \item Give an alternative characterisation of the Depoloarising channel.
    \item Prove this equivalence.
    \item Generalise the depolarising channel to $d > 2$ dimensions.
    \item Describe the context of an amplitude damping channel.
    \item Write down the unitary transformation describing the evolution of a $2$-level atom.
    \item Write down the Kraus operators for the channel.
    \item Now write down the CPTP map. Is it unital?
    \item Describe the evolution of the atom over time (repeated applications of the channel). What is its limiting state?
    \item How then does this map change the input state? Why is this surprising?
    \item But why is the above not as surprising as it initially seems?
    \item \textbf{LECTURE 18 DIAGRAM}
    \item Define the accessible information of an ensemble $\{p_x,\rho_x\}$.
    \item \textbf{State the Holevo Bound.}
    \item Define the Holveo $\chi$ quantity.
    \item For an ensemble $\mc{E}$ of pure states, what does $\chi(\mc{E})$ reduce to? Why?
    \item \textbf{Prove the Holveo Bound.}
    \begin{enumerate}
        \item Define the enlarged Hilbert space representation.
        \item What is the initial state of the quantum system $AQB$?
        \item Define the action $\Lambda$ of the quantum operation of measuring and recording.
        \item Show that $\Lambda$ is indeed a quantum operation.
        \item Express the state $\rho_{A'Q'B'}$ after application of $\Lambda$.
        \item Prove the Holveo bound in terms of mutual information.
        \item Demonstrate that this is in fact the Holevo bound.
    \end{enumerate}
    \item Show that the Holevo $\chi$ quantity is non-negative.
    \item Express $\chi(\mc{E})$ using the relative entropy.
    \item Can a quantum operation increase $\chi$?
    \item Show that the von Neumann entropy is not monotonic under quantum operations.
    \item Define a memoryless quantum channel.
    \item Describe four conditions affecting the type of relevant capactiy of a quantum channel.
    \item \textbf{LECTURE 19 DIAGRAM}
    \item Define the rate of information transmission (for classical info through a quantum channel).
    \item Define what it means for the transmission of classical information through $\Lambda$ to be reliable.
    \item Define an achievable rate.
    \item Define the capacity of a quantum channel (in this context).
    \item Define the product-state classical capacity of a quantum channel.
    \item State the Holevo-Schumacher-Westmoreland (HSW) Theorem.
    \item Define a classical-quantum (c-q) state.
    \item Prove that the maximisation in the Holveo capacity can be restricted to pure state ensembles.
    \item Prove that the Holveo capacity is superadditive.
    \item Use the HSW theorem to find the product state capacity of the qubit depolarising channel.
    \item Prove that an arbitrary quantum channel $\Lambda$ can be used to transmit classical information, provided the channel is not simply a constant \textit{i.e.} $\Lambda(\rho)$ is not identical for all $\rho$.
    \item Show in a separate way that it suffices to consider pure state ensembles.
    \item Use the Holevo bound to justify that by transmitting $n$ qubits to Bob, Alice can send at most $n$ bits of classical information to him.
    \item Prove the converse of the HSW theorem.
    \item What question does the HSW theorem naturally lead to?
    \item State the additivity conjecture of the Holveo capacity.
    \item Define $C_{\textrm{classical}}(\Lambda)$, and give an expression for it.
    \item Explain how the additivity conjecture relates to this capacity.
    \item Who gave a counterexample to the additivity conjecture, and when? What is the consequence of this?
    \item Define the coherent information of a bipartite quantum state.
    \item Express this in terms of conditional von Neumann entropy.
    \item In some sense, what is this analgous to in the classical case?
    \item Let $\rho_{RQ} = \frac{1}{2}\sum_{i=0}^{1}\ket{i}\bra{i}_R\otimes\ket{i}\bra{i}_Q$. Evaluate $I(R\rangle Q))_\rho$ and $I(R:Q)_\rho$.
    \item Now let $\ket{\psi}_{RQE}$ be a purification of $\rho_{RQ}$.
    \begin{enumerate}
        \item Show that $I(R\rangle Q) = S(Q)_\psi - S(E)_\psi$.
        \item Show that $-S(R|Q)_\rho = I(R\rangle Q)_\rho = S(R|E)_\psi$.
    \end{enumerate}
    \item Show that $|S(R\rangle Q)_\rho|\le \log\dim(\mc{H}_R)$. When do we have equality?
    \item Define the coherent information of a quantum channel $\Lambda$ with respect to the input state $\rho = \rho_Q$.
    \item Show that $I_c(\Lambda,\rho)\le S(\rho_Q)$. When do we have equality, and why?
    \item When do we have $I_c(\lambda,\rho) = S(\rho)$, in general?
    \item State the quantum data processing inequality.
    \item Prove the quantum data processing inequality.
    \begin{enumerate}
        \item What are the two key tools in this proof?
    \end{enumerate}
    \item Define $Q^{(1)}(\Lambda)$.
    \item Define $Q^{(n)}(\Lambda)$.
    \item State a theorem of Lloyd, Shor and Devetak.
    \item What does LOCC stand for?
    \item Define what LOCC is.
    \item Describe a LOCC transformation.
    \item Define $\rm{LOCC}^{\ra}$, $\rm{LOCC}^{\leftarrow}$, and $\rm{LOCC}^\leftrightarrow$.
    \item Define what it means for a state $\rho_{AB}$ to be distillable.
    \item Define what it means to be LOCC-distillable.
    \item $\rho_{AB}$ is distillable iff...?
    \item Or equivalently...?
    \item Describe the superactivation phenomenon.
    \item  q[Unsure if this business is really examinable, on the whole.]
\end{enumerate}


\end{document}