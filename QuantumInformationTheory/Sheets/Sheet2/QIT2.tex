\documentclass[]{article}

\usepackage{amsmath}
\usepackage{amssymb}
\usepackage{amsthm}
\usepackage{graphicx}
\usepackage{parskip}
\usepackage{xcolor}
\usepackage{pagecolor}
\usepackage[margin=1.2in]{geometry}
\usepackage{enumerate}
\usepackage{enumitem}
\usepackage{tikz}
\newcommand*\circled[1]{%
   \tikz[baseline=(C.base)]\node[draw,circle,inner sep=1.2pt,line width=0.2mm,](C) {#1};}
\newcommand*\Myitem{%
   \stepcounter{enumi}\item[\circled{\theenumi}]}

\usepackage[utf8]{inputenc}
\usepackage[english]{babel}

\usepackage{mathtools}
\DeclarePairedDelimiter\bra{\langle}{\rvert}
\DeclarePairedDelimiter\ket{\lvert}{\rangle}
\DeclarePairedDelimiterX\braket[2]{\langle}{\rangle}{#1 \delimsize\vert #2}

\definecolor{thmcolour}{rgb}{0,0,0}
\definecolor{defcolour}{rgb}{0,0,0}
\definecolor{textcolour}{rgb}{0,0,0}
\definecolor{backgroundcolour}{rgb}{1,1,1}

\pagecolor{backgroundcolour}
\color{textcolour}

\newtheoremstyle{custhm}
{%space above
}{%space below
}{%body font
\color{thmcolour}\em
}{%indent amount
-0em
}{%head font
\bfseries\color{thmcolour}
}{%head punct
}{%after head space
1em
}{%head spec
\thmname{#1}\if\relax\detokenize{#2}\relax:\else\thmnumber{ #2}:\fi\if\relax\detokenize{#3}\relax\else\thmnote{ (#3)}\fi
}

\newtheoremstyle{ex}
{%space above
}{%space below
}{%body font
\color{thmcolour}
}{%indent amount
-0em
}{%head font
\bfseries\color{thmcolour}
}{%head punct
}{%after head space
1em
}{%head spec
\thmname{#1}\if\relax\detokenize{#2}\relax:\else\thmnumber{ #2}:\fi\if\relax\detokenize{#3}\relax\else\thmnote{(#3)}\fi
}

\newtheoremstyle{remark}
{%space above
}{%space below
}{% body font
}{%indent amount
-0em
}{%head font
\bfseries
}{%head punct
}{%after head space
1em
}{%head spec
\if\relax\detokenize{#3}\relax\thmname{#1}:\else\thmname{#3}:\fi
}

\newtheoremstyle{numremark}
{%space above
}{%space below
}{% body font
}{%indent amount
-0em
}{%head font
\bfseries
}{%head punct
}{%after head space
1em
}{%head spec
\thmname{#1}\thmnumber{ #2}:
}

\newtheoremstyle{cusdef}
{%space above
}{%space below
}{%body font
\color{defcolour}
}{%indent amount
-0em
}{%head font
\bfseries\color{defcolour}
}{%head punct
}{%after head space
1em
}{%head spec
%if numbered, include number
%if named, include name
\thmname{#1}\if\relax\detokenize{#2}\relax:\else\thmnumber{ #2}:\fi\if\relax\detokenize{#3}\relax\else\thmnote{ (#3)}\fi
}

\theoremstyle{custhm}
\newtheorem{theorem}{Theorem}[section]
\theoremstyle{cusdef}
\newtheorem{defin}[theorem]{Definition}
\theoremstyle{custhm}
\newtheorem{lemma}[theorem]{Lemma}
\theoremstyle{custhm}
\newtheorem{cor}[theorem]{Corollary}

\theoremstyle{custhm}
\newtheorem{prop}[theorem]{Proposition}

\theoremstyle{ex}
\newtheorem{ex}[theorem]{Example}

\theoremstyle{custhm}
\newtheorem*{theorem*}{Theorem}

\theoremstyle{cusdef}
\newtheorem*{defin*}{Definition}

\theoremstyle{remark}
\newtheorem*{remark*}{Remark}

\theoremstyle{remark}
\newtheorem{remark}[theorem]{Remark}

\theoremstyle{numremark}
\newtheorem{numremark}[theorem]{Remark}

\setcounter{section}{-1}

%\marginpar{to describe which lecture it is}

\newcommand{\N}{\mathbb{N}}
\newcommand{\Z}{\mathbb{Z}}
\newcommand{\Q}{\mathbb{Q}}
\newcommand{\R}{\mathbb{R}}
\newcommand{\C}{\mathbb{C}}
\newcommand{\e}{\mathrm{e}}
\newcommand{\ra}{\rightarrow}
\newcommand{\lef}{\left(}
\newcommand{\res}{\right)}
\newcommand{\ie}{\textit{i.e.}}
\newcommand{\eps}{\varepsilon}
\newcommand{\E}{\mathbb{E}}
\newcommand{\suminf}{\sum_{n=0}^{\infty}}
\newcommand{\suminfa}[1]{\sum_{#1=0}^{\infty}}
\renewcommand{\P}{\mathbb{P}}
\newcommand{\undf}[1]{\textit{\textbf{#1}}}
\renewcommand{\L}{\mathcal{L}}
\renewcommand{\it}[1]{\textit{#1}}
\newcommand{\M}{\mathcal{M}}
\renewcommand{\phi}{\varphi}
\newcommand{\proves}{\vdash}
\newcommand{\lra}{\leftrightarrow}

\renewcommand{\bar}{\overline}
\renewcommand{\O}{\mathcal{O}}


\newcommand{\ac}[1]{\mathcal{#1}}
\newcommand{\A}{\mathcal{A}}


\renewcommand{\subset}{\subseteq}

\renewcommand{\th}{\textrm{th}}

\newcommand{\av}{\textrm{av}}
\newcommand{\un}{\underline{u}^{(n)}}
\newcommand{\ten}{T_\eps^{(n)}}
\newcommand{\aen}{A_\eps^{(n)}}
\renewcommand{\H}{\mathcal{H}}

\title{Quantum Information Theory: Sheet 2}
\author{Otto Pyper}
\date{}

\begin{document}
\maketitle

\textbf{Exercise 1}.

\begin{enumerate}
    \item For any $v$ we have $\langle v|Av\rangle  \ge 0$, and in particular real. So $\langle v|Av \rangle = (\langle v | Av \rangle)^\ast = \langle Av|v\rangle = \langle v | A^\dagger v\rangle$. Hence for all $v$ we have $\langle v | (A - A^\dagger)v\rangle = 0$. $(A-A^\dagger)$ is skew-Hermitian and hence normal, hence diagonalisable. So in some basis it is diagonal, and using the above relation we see that all the elements on the diagonal are zero. So $(A - A^\dagger)$ is zero in this basis and hence every basis. Thus $A = A^\dagger$.
    
    \item $\mathbb{F}\ket{ij} = \ket{ji}$, which defines its action on an orthonormal basis for $\H_A\otimes \H_B$. Hence we can represent $\mathbb{F}$ as: $$\mathbb{F} = \sum_{i,j}\ket{ji}\bra{ij}$$
    
    Let $v = \sum_{i,j}a_{ij}\ket{ij}$ be an eigenvector with eigenvalue $\lambda$. Then $\mathbb{F}v = \lambda v = \sum_{i,j}a_{ij}\ket{ji} = \sum_{i,j}a_{ji}\ket{ij} = \sum_{i,j}\lambda a_{ij}\ket{ij}$. So we must have $a_{ji} = \lambda a_{ij}$ for each $i,j$, and so $a_{ji} = \lambda^2 a_{ji}$. Since $v$ is non-zero, there must be some non-zero $a_{ij}$. Hence $\lambda^2 = 1$, and $\lambda = \pm 1$.

    To calculate their multiplicities we can consier the degrees of freedom of the vector elements. Let $\lambda = 1$, and consider the matrix given by $A_{ij} = a_{ij}$. Since $a_{ij} = a_{ji}$, this must be symmetric, and the diagonal is unconstratined. So there are $d$ + $(d-1)d/2 = d(d+1)/2$ degrees of freedom, hence this is the multiplicity.

    So the multiplicity of $\lambda = -1$ is $d(d-1)/2$, which we can also see by remarking that in this case the diagonal elements $a_{ii}$ must all be zero, and then the upper right of the matrix is determined entirely by the lower left, which has $d(d-1)/2$ free elements.

    Slightly more rigorously, the `degrees of freedom' correspond to basis vectors in the eigenspace.

    We form the operator $\Omega = \ket{\Omega}\bra{\Omega} = \sum_{i,j}\ket{j}\bra{i}\otimes\ket{j}\bra{i}$, and what we want is $\mathbb{F} = \sum_{i,j}$
\end{enumerate}


\end{document}