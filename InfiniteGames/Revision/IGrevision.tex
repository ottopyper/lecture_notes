\documentclass[]{article}

\usepackage{amsmath}
\usepackage{amssymb}
\usepackage{amsthm}
\usepackage{graphicx}
\usepackage{parskip}
\usepackage{xcolor}
\usepackage{pagecolor}
\usepackage[margin=1.2in]{geometry}
\usepackage{enumerate}
\usepackage{enumitem}
\usepackage{tikz}
\newcommand*\circled[1]{%
   \tikz[baseline=(C.base)]\node[draw,circle,inner sep=1.2pt,line width=0.2mm,](C) {#1};}
\newcommand*\Myitem{%
   \stepcounter{enumi}\item[\circled{\theenumi}]}

\usepackage[utf8]{inputenc}
\usepackage[english]{babel}

\usepackage{mathtools}
\DeclarePairedDelimiter\bra{\langle}{\rvert}
\DeclarePairedDelimiter\ket{\lvert}{\rangle}
\DeclarePairedDelimiterX\braket[2]{\langle}{\rangle}{#1 \delimsize\vert #2}

\definecolor{thmcolour}{rgb}{0,0,0}
\definecolor{defcolour}{rgb}{0,0,0}
\definecolor{textcolour}{rgb}{0,0,0}
\definecolor{backgroundcolour}{rgb}{1,1,1}

\pagecolor{backgroundcolour}
\color{textcolour}

\newtheoremstyle{custhm}
{%space above
}{%space below
}{%body font
\color{thmcolour}\em
}{%indent amount
-0em
}{%head font
\bfseries\color{thmcolour}
}{%head punct
}{%after head space
1em
}{%head spec
\thmname{#1}\if\relax\detokenize{#2}\relax:\else\thmnumber{ #2}:\fi\if\relax\detokenize{#3}\relax\else\thmnote{ (#3)}\fi
}

\newtheoremstyle{ex}
{%space above
}{%space below
}{%body font
\color{thmcolour}
}{%indent amount
-0em
}{%head font
\bfseries\color{thmcolour}
}{%head punct
}{%after head space
1em
}{%head spec
\thmname{#1}\if\relax\detokenize{#2}\relax:\else\thmnumber{ #2}:\fi\if\relax\detokenize{#3}\relax\else\thmnote{(#3)}\fi
}

\newtheoremstyle{remark}
{%space above
}{%space below
}{% body font
}{%indent amount
-0em
}{%head font
\bfseries
}{%head punct
}{%after head space
1em
}{%head spec
\if\relax\detokenize{#3}\relax\thmname{#1}:\else\thmname{#3}:\fi
}

\newtheoremstyle{numremark}
{%space above
}{%space below
}{% body font
}{%indent amount
-0em
}{%head font
\bfseries
}{%head punct
}{%after head space
1em
}{%head spec
\thmname{#1}\thmnumber{ #2}:
}

\newtheoremstyle{cusdef}
{%space above
}{%space below
}{%body font
\color{defcolour}
}{%indent amount
-0em
}{%head font
\bfseries\color{defcolour}
}{%head punct
}{%after head space
1em
}{%head spec
%if numbered, include number
%if named, include name
\thmname{#1}\if\relax\detokenize{#2}\relax:\else\thmnumber{ #2}:\fi\if\relax\detokenize{#3}\relax\else\thmnote{ (#3)}\fi
}

\theoremstyle{custhm}
\newtheorem{theorem}{Theorem}[section]
\theoremstyle{cusdef}
\newtheorem{defin}[theorem]{Definition}
\theoremstyle{custhm}
\newtheorem{lemma}[theorem]{Lemma}
\theoremstyle{custhm}
\newtheorem{cor}[theorem]{Corollary}

\theoremstyle{custhm}
\newtheorem{prop}[theorem]{Proposition}

\theoremstyle{ex}
\newtheorem{ex}[theorem]{Example}

\theoremstyle{custhm}
\newtheorem*{theorem*}{Theorem}

\theoremstyle{cusdef}
\newtheorem*{defin*}{Definition}

\theoremstyle{remark}
\newtheorem*{remark*}{Remark}

\theoremstyle{remark}
\newtheorem{remark}[theorem]{Remark}

\theoremstyle{numremark}
\newtheorem{numremark}[theorem]{Remark}

\setcounter{section}{-1}

%\marginpar{to describe which lecture it is}

\newcommand{\N}{\mathbb{N}}
\newcommand{\Z}{\mathbb{Z}}
\newcommand{\Q}{\mathbb{Q}}
\newcommand{\R}{\mathbb{R}}
\newcommand{\C}{\mathbb{C}}
\newcommand{\e}{\mathrm{e}}
\newcommand{\ra}{\rightarrow}
\newcommand{\lef}{\left(}
\newcommand{\res}{\right)}
\newcommand{\ie}{\textit{i.e.}}
\newcommand{\eps}{\varepsilon}
\newcommand{\E}{\mathbb{E}}
\newcommand{\suminf}{\sum_{n=0}^{\infty}}
\newcommand{\suminfa}[1]{\sum_{#1=0}^{\infty}}
\renewcommand{\P}{\mathbb{P}}
\newcommand{\undf}[1]{\textit{\textbf{#1}}}
\renewcommand{\L}{\mathcal{L}}
\renewcommand{\it}[1]{\textit{#1}}
\newcommand{\M}{\mathcal{M}}
\renewcommand{\phi}{\varphi}
\newcommand{\proves}{\vdash}
\newcommand{\lra}{\leftrightarrow}

\renewcommand{\bar}{\overline}
\renewcommand{\O}{\mathcal{O}}


\newcommand{\ac}[1]{\mathcal{#1}}
\newcommand{\A}{\mathcal{A}}


\renewcommand{\subset}{\subseteq}

\renewcommand{\th}{\textrm{th}}

\usepackage[cmtip,all]{xy}
\newcommand{\longsquiggly}{\xymatrix{{}\ar@{<~>}[r]&{}}}

\newcommand{\I}{\textrm{I}}
\newcommand{\II}{\textrm{II}}
\newcommand{\om}{\omega}
\newcommand{\lom}{{<\omega}}
\newcommand{\lh}{\ell h}
\renewcommand{\ac}{\textrm{AC}}
\newcommand{\bosig}{\bm{\Sigma}}
\newcommand{\bopi}{\bm{\Pi}}
\newcommand{\bodel}{\bm{\Delta}}
\newcommand{\bg}{{\breve \Gamma}}
\newcommand{\br}[1]{{\breve #1}}
\newcommand{\Det}{\textrm{Det}}
\newcommand{\eomg}{\exists^{\om^\om}\Gamma}
\newcommand{\eom}{\exists^{\om^\om}}
\newcommand{\psp}{\textrm{PSP}}
\newcommand{\rk}{\textrm{rk}}
\newcommand{\fld}{\textrm{fld}}
\newcommand{\wf}{\textrm{WF}}
\newcommand{\hit}{\textrm{ht}}
\newcommand{\bij}[1]{\lceil #1 \rceil}
\newcommand{\bog}{\bm{\Gamma}}
\newcommand{\op}{\textrm{OP}}
\newcommand{\zfc}{\textrm{ZFC}}
\newcommand{\cons}{\textrm{Cons}}
\newcommand{\ic}{\textrm{IC}}
\newcommand{\ad}{\textrm{AD}}
\newcommand{\zf}{\textrm{ZF}}
\newcommand{\tri}{\mathop{\triangle}}
\newcommand{\aux}{\textrm{aux}}
\newcommand{\lkb}{<_{\textrm{KB}}}
\newcommand{\dom}{\textrm{dom}}
\newcommand{\led}{\le_{\mathrm{D}}}
\newcommand{\D}{\mathcal{D}}
\newcommand{\cone}{\textrm{Cone}}
\newcommand{\cd}{\textrm{code}}
%\renewcommand{\ht}{\textrm{ht}}
%\newcommand{\wf}{\textrm{WF}}
\usepackage{bm}

\newcommand{\game}{
    \begin{center}
        \begin{tabular}{c|ccccccc}
            I & $m_0$ & & $m_2$ & & $m_4 $& & $\dots $\\ \hline
            II & & $m_1$ & & $m_3 $& &$ m_5$ & $\dots$ 
        \end{tabular}
    \end{center}
}

\newcommand{\gamec}[2]{
    \begin{center}
        \begin{tabular}{c|ccccccc}
            I & $#1_0$ & & $#1_1$ & & $#1_2 $& & $\dots $\\ \hline
            II & & $#2_0$ & & $#2_1 $& &$ #2_2$ & $\dots$ 
        \end{tabular}
    \end{center}
}

\newcommand{\gamed}[1]{
    \begin{center}
        \begin{tabular}{c|ccccccc}
            I & $#1_0$ & & $#1_2$ & & $#1_4 $& & $\dots $\\ \hline
            II & & $#1_1$ & & $#1_3 $& &$ #1_5$ & $\dots$ 
        \end{tabular}
    \end{center}
}

\title{Infinite Games Revision Questions}
\author{Otto Pyper}
\date{}

\begin{document}
\maketitle
\begin{enumerate}
    \item Which five properties do the games we consider have?
    \item Define $M^\lom$, $M^\om$.
    \item Given $x \in M^\om$, define $x_\I$ and $x_\II$.
    \item Define the interleaving of $x,y\in M^\om$.
    \item What is $A \subset M^\om$ called?
    \item What is a strategy?
    \item Given two strategies $\sigma, \tau$ define $\sigma \ast \tau$.
    \item Define a winning strategy.
    \item Define a determined set.
    \item Define a tree on $M$.
    \item Given a tree $T$, define a branch through $T$.
    \item Define $[T]$. What is it called?
    \item Define $G(A;T)$ and represent it as $G(B)$ for some $B$.
    \item Define a I/II-strategic tree.
    \item Define a strategic tree in general.
    \item What is $[T^\I_\sigma]$?
    \item $\sigma$ is a winning strategy for player I in $G(A)$ $\iff$...?
    \item $A$ is determined iff...?
    \item For $s \in M^\lom$, what is $\lh(s)$?
    \item Define a splitting node.
    \item Define a perfect tree.
    \item Define a perfect set.
    \item State Cantor's Theorem about perfect, non-empty subsets of $2^\om$.
    \item Prove Cantor's Theorem.
    \item State and prove two corollaries of Cantor's Theorem for $|M| \ge 2$.
    \item Prove that if $A$ is countable then player II has a winning strategy in $G(A)$.
    \item Prove that if $|A| < 2^{\aleph_0}$, then player II has a winning strategy in $G(A)$ (and a similar result for player I).
    \item Define a blindfolded strategy.
    \item Prove (in ZFC) that there is a non-determined subset $A\subset \om^\om$.
    \item Define a quasistrategy.
    \item Define a (winning) quasistrategic tree.
    \item Define a quasidetermined set.
    \item What condition on $M$ allows the construction of strategies from quasistrategies?
    \item Define a closed set.
    \item Represent Zermelo's finite games with closed payoff sets.
    \item State the Gale-Stewart Theorem.
    \item Prove the Gale-Stewart Theorem.
    \item Define Baire space.
    \item What are the open balls in this metric?
    \item Define Cantor space.
    \item What is alternative characterisation of these topologies?
    \item Show that Cantor space is compact, but that Baire space is (very) disconnected.
    \item Given $A$ in Baire space, define $T_A$.
    \item Prove that $[T_A]$ is the closure of $A$.
    \item State and prove the tree representation theorem for closed sets.
    \item Basic open sets are...? Spaces with this property are called...?
    \item Singletons are...?
    \item Prove that these spaces are Hausdorff.
    \item A function $f$ on $\om^\om$ is continuous iff what?
    \item Prove this.
    \item What is the general rule of thumb for determining whether or not $f$ is continuous?
    \item Show that $(\om^\om)^2$ and $\om^\om$ are homeomorphic.
    \item Baire space is homeomorphic to...? How do we thus sometimes refer to elements of Baire space?
    \item What is $\ac_X(Y)$?
    \item Define the Borel Hierarchy.
    \item Define a $G_\delta$ space.
    \item Give some spaces that are $G_\delta$.
    \item Prove that if $X$ has a countable, clopen topology base then $X$ is $G_\delta$.
    \item When does the Borel Hierarchy terminate if:
    \begin{enumerate}
        \item $X$ is discrete?
        \item singletons are closed and $X$ is countable?
    \end{enumerate}
    \item Prove that for arbitrary $X$, $\bodel_{\aleph_1} = \bosig_{\aleph_1} = \bopi_{\aleph_1}$
    \item (In ZFC) what is the height of the Borel Hierarchy for Cantor space/Baire space/$\R$?
    \item What technique does the proof of the above use?
    \item Define a pointclass.
    \item Define the dual pointclass, and the ambiguous pointclass.
    \item What does it mean for a pointclass to be boldface? [Why is this silly?]
    \item What does it mean for $\Gamma$ to be closed under continuous images?
    \item Define what it means for a set $U$ to be $X$-universal for $\Gamma(Y)$.
    \item Prove that if $U$ is $X$-universal for $\Gamma(X)$ and $\Gamma$ is boldface, then $\Gamma(X) \ne \breve{\Gamma}(X)$.
    \item Prove that for every $\alpha < \aleph_1$, $\bosig^0_\alpha$ has an $\om^\om$-universal set.
    \item Prove that if $U\subset X\times X$ is $X$-universal for $\Gamma(X)$, then $X\times X\backslash U$ is $X$-universal for $\bg(X)$.
    \item Let $\lambda < \om_1$. Suppose that for each $\alpha < \lambda$ there is an $\om^\om$-universal set $U_\alpha$ for $\bopi^0_\alpha(\om^\om)$. Then there is an $\om^\om$-universal set for $\bosig^0_\lambda$.
    \item Deduce the Borel Hierarchy Theorem.
    \item Where did we use/need AC in the above proof?
    \item What does $\Det(\Gamma)$ mean?
    \item Show that in general the class of determined sets is not closed under complementation.
    \item Who proved the following, and when?
    \begin{enumerate}
        \item $\Det(\bosig^0_2)$
        \item $\Det(\bosig^0_3)$
        \item $\Det(\bosig^0_4)$
    \end{enumerate}
    \item What did who prove about $\Det(\bosig^0_5)$?
    \item This paved the way for who to prove what, and when?
    \item Prove (in ZFC) that $|\mathcal{B}| = 2^{\aleph_0} < 2^{2^{\aleph_0}}$.
    \item What is the Feferman-Levy Model $\mathcal{M}$?
    \item Use it show that we need choice in the above proof.
    \item What is the famous mistake of Henri Lebesgue?
    \item Define a projection.
    \item Define $\eom\Gamma$.
    \item Define what it means for $\Gamma$ to be closed under projections.
    \item Define the projective hierarchy.
    \item Prove that the projective hierarchy does not collapse.
    \item Prove that every Borel set is $\bosig^1_1$.
    \item Deduce Suslin's Theorem.
    \item State the Continuum Hypothesis, and an equivalent formulation of it under ZFC.
    \item Define what it means for $A\subset \om^\om$ to have the perfect set property.
    \item State the Cantor-Bendixson Theorem.
    \item Sketch a proof.
    \item Why was this proof important?
    \item Define $\psp(\Gamma)$, and re-state Cantor-Bendixson with this notation.
    \item Define $\psp$ and state an observation involving it.
    \item State a theorem of Bernstein.
    \item How can it be proven?
    \item State a theorem of Haudorff. How will we prove this?
    \item Prove that if $\Gamma$ is boldface, then $\Det(\Gamma)\implies \psp(\Gamma)$.
    \begin{enumerate}
        \item Define the asymmetric game $G^\ast(A)$.
        \item If $A \in \Gamma$ and $\Det(\Gamma)$ then $G^\ast(A)$ is determined.
        \item If player I was a winning strategy in $G^\ast(A)$ then $A$ contains a perfect subset.
        \item If player II has a winning strategy, then $A$ is countable.
        \begin{enumerate}
            \item Define a $\tau$-decisive for $x$ position.
            \item If $\tau$ is winning for II, then for each $x \in A$ ther is a $\tau$-decisive position $p$ for $x$
            \item Every position $p$ is $\tau$-decisive for at most one $x \in 2^\om$.
        \end{enumerate}
    \end{enumerate}
    \item Deduce $\psp(\textrm{Borel})$, and some further corollaries.
    \item State a theorem of Godel and Addison.
    \item What is Godel's Constructible Universe? How is it denoted? Why?
    \item State explicitly the definition of $\bosig^1_1$, and reformulate it in terms of $T_x$ (giving the definition).
    \item Define an illfounded/wellfounded tree. With a bit of AC, what is this equivalent to?
    \item Hence state the tree representation of analytic and co-analytic sets.
    \item Describe how to code a tree on $\om$ or $\om\times\om$ as elements of Baire space.
    \item Define WF.
    \item Define the rank function, and the height function.
    \item Prove that $||\cdot||:\wf \ra \om_1$ is a surjection.
    \item Define $\wf_\alpha$, $\wf_{<\alpha}$, $\wf_{\le \alpha}$.
    \item WF can be thought of as [...] in [...] many levels.
    \item Prove that $\wf$ is $\bopi^1_1$.
    \item What is the general proof technique here?
    \item Show that $\wf_{<\alpha},\wf_{\le \alpha},\wf_{\alpha}$ are $\bopi^1_1$.
    \item Show further that $\wf_{\le\alpha}$ is also $\bosig^1_1$.
    \item Deduce that $\wf_\alpha,\wf_{\le\alpha},\wf_{<\alpha}$ are all $\bodel^1_1$.
    \item Prove that $\bodel^1_1 = $Borel.
    \item Write $\wf$ as a union of [...].
    \item Let $\Gamma$ be boldface. Define what it means for $A$ to be $\Gamma$-hard, and $\Gamma$-complete.
    \item Prove that $\wf$ is $\bopi^1_1$-complete.
    \item Show that $\wf$ is not $\bosig^1_1$.
    \item Deduce that every $\bopi^1_1$ set is an $\om_1$-union of Borel sets.
    \item State the Weak CH for $\bopi^1_1$ sets.
    \item Prove the Weak CH for $\bopi^1_1$ sets.
    \item State the Boundedness Lemma.
    \item Prove the Boundedness Lemma.
    \item Define a set of unique codes.
    \item Prove that if $C$ is an SUC it cannot have PSP.
    \item Prove that if there is a $\bodel^1_n$ wellorder of $\om^\om$, then there is a $\bopi^1_n$ set without PSP.
    \item Give an outline of the corresponding notions between large cardinals, determinacy, and definable wellorders.
    \item Roughly describe what it means if $\Phi$ is an LCP.
    \item Use two of Godel's theorems to contextuatlise this notion.
    \item Define what it means if $\Phi C < \Psi C$ and if $\Phi C$, $\Psi C$ are equiconsistent.
    \item Define a (strong) limit.
    \item Define regular.
    \item Define an inaccessible cardinal.
    \item State the GCH.
    \item Show that IC is a large cardinal axiom.
    \item Prove that if $\kappa$ is inaccessible, then $V_\kappa \models \zfc$.
    \item Define an inner model.
    \begin{enumerate}
        \item Define an inner model.
        \item What does it mean for $\phi$ to define an inner model?
        \item Define a canonical model family.
        \item Define what it means for a canonical model family to be $\bodel^1_n$-wellordered.
    \end{enumerate}
    \item State and prove some (basic) correspondences between models and inner models of ZFC.
    \item State (and prove?) another two transfers between $M$ and $V$.
    \item So what is $\aleph_1^M$?
    \item for $\mu$ a canonical model family, we have that [...].
    \item Suppose there is a $\bodel^1_n$-wellordered canonical model family and $\Det(\bopi^1_n)$. Then there is an inner model of ZFC + IC.
    \begin{enumerate}
        \item Define a weak set of unique codes.
        \item Show $M\models \aleph_1^V$ inaccessible.
    \end{enumerate}
    \item Define a filter on $\kappa$.
    \item Define an ultrafilter.
    \item Define $\lambda$-complete.
    \item If $\lambda = \aleph_1$, what is this completness often called?
    \item Define principal/non-principal ultrafilters.
    \item How many ultrafilters are there on $\kappa$ (in ZFC)? How many of these are principal?
    \item Define a measurable cardinal.
    \item Prove (in ZFC) that every measurable cardinal is inaccessible. [Why do we need AC here?]
    \item Define Erdos-Rado Arrow Notation, and all the terms associated with it.
    \item Define a weakly compact cardinal.
    \item State two facts about weak compactness of cardinals.
    \item Prove that every weakly compact cardinal is inaccessible.
    \item Define $n-c-$homogeneity and some more arrow notation.
    \item State Rowbottom's Theorem.
    \item Prove Rowbottom's Theorem.
    \item Prove (in ZFC) that $\aleph_1$ is not weakly compact.
    \item Define the diagonal intersection of a family of subsets of $\kappa$.
    \item Define a normal ultrafilter.
    \item Prove that if $U$ is an ultrafilter on $\kappa$ such that all elements of $U$ have size $\kappa$ and $U$ is normal, then $U$ is $\kappa$-complete.
    \item Fact (ZFC): If $\kappa$ is measurable, then there is a [...] on $\kappa$.
    \item Prove (hence) that measurable cardinals are weakly compact.
    \item Define $p[T]$ for $T$ a tree on $\kappa\times\om$.
    \item Define a $\kappa$-Suslin set $A\subset\om^\om$.
    \item So being $\aleph_0$-Suslin is equivalent to...?
    \item Define the Auxiliary Game.
    \item State Schoenfield's Theorem.
    \item Prove Schoenfield's Theorem.
    \begin{enumerate}
        \item Define the Kleene-Brouwer order.
        \item Define the Schoenfield tree.
    \end{enumerate}
    \item State a 1969/70 theorem of Martin.
    \item Define what it means for $\kappa$ to satisfy Rowbottom's Theorem.
    \item State Martin's Theorem.
    \item Prove Martin's Theorem.
    \item What aspect of $H$ did we fail to use in this proof? We don't need the full strength of what?
    \item State a ZFC Theorem: $\Det(\bosig^1_1) \iff $...?
    \item Which fragment of AC have we used liberally so far? In particular...?
    \item Prove that $\ad\implies\ac_{\om}(\om^\om)$.
    \item How does this proof generalise?
    \item State the uniformisation principle.
    \item $U$ is a non-principal ultrafilter iff it extends which filter?
    \item Define the image filter.
    \item Shwo that if there is an ultrafilter $U$ on $X$ that is not $\aleph_1$-complete, then there is a non-principal ultrafilter on $\N$.
    \item Corollary: ZF + there is no non-principal ultrafilter on $\N$ $\implies$...?
    \item Prove that $\ad\implies$ there is no non-principal ultrafilter on $\N$.
    \begin{enumerate}
        \item What is the technique here called?
    \end{enumerate}
    \item What does it mean if $x$ is definable from $y$?
    \item State six properties of $\le_D$.
    \item Define a preorder.
    \item Prove that if $x,y\in\om^\om$ and $z$ is such that $x\le_Dz$ and $y\le_D z$ then $x\ast y \le_Dz$.
    \item Define the Turing Join of $x$ and $y$.
    \item Define the cone of $x$.
    \item Define the cone filter $F_D$.
    \item Define $\equiv_D$-invariant.
    \item Describe the relationship between strategies and $\mathcal{D}_D$.
    \item State Martin's Lemma.
    \item State a corollary of Martin's Lemma (involving the Martin Measure).
    \item State Solovay's Theorem.
    \item Prove Solovay's Theorem.
\end{enumerate}


\end{document}