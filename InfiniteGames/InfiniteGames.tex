\documentclass[]{article}

\usepackage{amsmath}
\usepackage{amssymb}
\usepackage{amsthm}
\usepackage{graphicx}
\usepackage{parskip}
\usepackage{xcolor}
\usepackage{pagecolor}
\usepackage[margin=1.2in]{geometry}
\usepackage{enumerate}
\usepackage{enumitem}
\usepackage{tikz}
\newcommand*\circled[1]{%
   \tikz[baseline=(C.base)]\node[draw,circle,inner sep=1.2pt,line width=0.2mm,](C) {#1};}
\newcommand*\Myitem{%
   \stepcounter{enumi}\item[\circled{\theenumi}]}

\usepackage[utf8]{inputenc}
\usepackage[english]{babel}

\usepackage{mathtools}
\DeclarePairedDelimiter\bra{\langle}{\rvert}
\DeclarePairedDelimiter\ket{\lvert}{\rangle}
\DeclarePairedDelimiterX\braket[2]{\langle}{\rangle}{#1 \delimsize\vert #2}

\definecolor{thmcolour}{rgb}{0,0,0}
\definecolor{defcolour}{rgb}{0,0,0}
\definecolor{textcolour}{rgb}{0,0,0}
\definecolor{backgroundcolour}{rgb}{1,1,1}

\pagecolor{backgroundcolour}
\color{textcolour}

\newtheoremstyle{custhm}
{%space above
}{%space below
}{%body font
\color{thmcolour}\em
}{%indent amount
-0em
}{%head font
\bfseries\color{thmcolour}
}{%head punct
}{%after head space
1em
}{%head spec
\thmname{#1}\if\relax\detokenize{#2}\relax:\else\thmnumber{ #2}:\fi\if\relax\detokenize{#3}\relax\else\thmnote{ (#3)}\fi
}

\newtheoremstyle{ex}
{%space above
}{%space below
}{%body font
\color{thmcolour}
}{%indent amount
-0em
}{%head font
\bfseries\color{thmcolour}
}{%head punct
}{%after head space
1em
}{%head spec
\thmname{#1}\if\relax\detokenize{#2}\relax:\else\thmnumber{ #2}:\fi\if\relax\detokenize{#3}\relax\else\thmnote{(#3)}\fi
}

\newtheoremstyle{remark}
{%space above
}{%space below
}{% body font
}{%indent amount
-0em
}{%head font
\bfseries
}{%head punct
}{%after head space
1em
}{%head spec
\if\relax\detokenize{#3}\relax\thmname{#1}:\else\thmname{#3}:\fi
}

\newtheoremstyle{numremark}
{%space above
}{%space below
}{% body font
}{%indent amount
-0em
}{%head font
\bfseries
}{%head punct
}{%after head space
1em
}{%head spec
\thmname{#1}\thmnumber{ #2}:
}

\newtheoremstyle{cusdef}
{%space above
}{%space below
}{%body font
\color{defcolour}
}{%indent amount
-0em
}{%head font
\bfseries\color{defcolour}
}{%head punct
}{%after head space
1em
}{%head spec
%if numbered, include number
%if named, include name
\thmname{#1}\if\relax\detokenize{#2}\relax:\else\thmnumber{ #2}:\fi\if\relax\detokenize{#3}\relax\else\thmnote{ (#3)}\fi
}

\theoremstyle{custhm}
\newtheorem{theorem}{Theorem}[section]
\theoremstyle{cusdef}
\newtheorem{defin}[theorem]{Definition}
\theoremstyle{custhm}
\newtheorem{lemma}[theorem]{Lemma}
\theoremstyle{custhm}
\newtheorem{cor}[theorem]{Corollary}

\theoremstyle{custhm}
\newtheorem{prop}[theorem]{Proposition}

\theoremstyle{ex}
\newtheorem{ex}[theorem]{Example}

\theoremstyle{custhm}
\newtheorem*{theorem*}{Theorem}

\theoremstyle{cusdef}
\newtheorem*{defin*}{Definition}

\theoremstyle{remark}
\newtheorem*{remark*}{Remark}

\theoremstyle{remark}
\newtheorem{remark}[theorem]{Remark}

\theoremstyle{numremark}
\newtheorem{numremark}[theorem]{Remark}

\setcounter{section}{-1}

%\marginpar{to describe which lecture it is}

\newcommand{\N}{\mathbb{N}}
\newcommand{\Z}{\mathbb{Z}}
\newcommand{\Q}{\mathbb{Q}}
\newcommand{\R}{\mathbb{R}}
\newcommand{\C}{\mathbb{C}}
\newcommand{\e}{\mathrm{e}}
\newcommand{\ra}{\rightarrow}
\newcommand{\lef}{\left(}
\newcommand{\res}{\right)}
\newcommand{\ie}{\textit{i.e.}}
\newcommand{\eps}{\varepsilon}
\newcommand{\E}{\mathbb{E}}
\newcommand{\suminf}{\sum_{n=0}^{\infty}}
\newcommand{\suminfa}[1]{\sum_{#1=0}^{\infty}}
\renewcommand{\P}{\mathbb{P}}
\newcommand{\undf}[1]{\textit{\textbf{#1}}}
\renewcommand{\L}{\mathcal{L}}
\renewcommand{\it}[1]{\textit{#1}}
\newcommand{\M}{\mathcal{M}}
\renewcommand{\phi}{\varphi}
\newcommand{\proves}{\vdash}
\newcommand{\lra}{\leftrightarrow}

\renewcommand{\bar}{\overline}
\renewcommand{\O}{\mathcal{O}}


\newcommand{\ac}[1]{\mathcal{#1}}
\newcommand{\A}{\mathcal{A}}


\renewcommand{\subset}{\subseteq}

\renewcommand{\th}{\textrm{th}}

\usepackage[cmtip,all]{xy}
\newcommand{\longsquiggly}{\xymatrix{{}\ar@{<~>}[r]&{}}}

\newcommand{\I}{\textrm{I}}
\newcommand{\II}{\textrm{II}}
\newcommand{\om}{\omega}
\newcommand{\lom}{{<\omega}}
\newcommand{\lh}{\ell h}
\renewcommand{\ac}{\textrm{AC}}
\newcommand{\bosig}{\bm{\Sigma}}
\newcommand{\bopi}{\bm{\Pi}}
\newcommand{\bodel}{\bm{\Delta}}
\newcommand{\bg}{{\breve \Gamma}}
\newcommand{\br}[1]{{\breve #1}}
\newcommand{\Det}{\textrm{Det}}
\newcommand{\eomg}{\exists^{\om^\om}\Gamma}
\newcommand{\eom}{\exists^{\om^\om}}
\newcommand{\psp}{\textrm{PSP}}
\newcommand{\rk}{\textrm{rk}}
\newcommand{\fld}{\textrm{fld}}
\newcommand{\wf}{\textrm{WF}}
\newcommand{\hit}{\textrm{ht}}
\newcommand{\bij}[1]{\lceil #1 \rceil}
\newcommand{\bog}{\bm{\Gamma}}
\newcommand{\op}{\textrm{OP}}
\newcommand{\zfc}{\textrm{ZFC}}
\newcommand{\cons}{\textrm{Cons}}
\newcommand{\ic}{\textrm{IC}}
\newcommand{\ad}{\textrm{AD}}
\newcommand{\zf}{\textrm{ZF}}
\newcommand{\tri}{\mathop{\triangle}}
\newcommand{\aux}{\textrm{aux}}
\newcommand{\lkb}{<_{\textrm{KB}}}
\newcommand{\dom}{\textrm{dom}}
\newcommand{\led}{\le_{\mathrm{D}}}
\newcommand{\D}{\mathcal{D}}
\newcommand{\cone}{\textrm{Cone}}
\newcommand{\cd}{\textrm{code}}
%\renewcommand{\ht}{\textrm{ht}}
%\newcommand{\wf}{\textrm{WF}}
\usepackage{bm}

\newcommand{\game}{
    \begin{center}
        \begin{tabular}{c|ccccccc}
            I & $m_0$ & & $m_2$ & & $m_4 $& & $\dots $\\ \hline
            II & & $m_1$ & & $m_3 $& &$ m_5$ & $\dots$ 
        \end{tabular}
    \end{center}
}

\newcommand{\gamec}[2]{
    \begin{center}
        \begin{tabular}{c|ccccccc}
            I & $#1_0$ & & $#1_1$ & & $#1_2 $& & $\dots $\\ \hline
            II & & $#2_0$ & & $#2_1 $& &$ #2_2$ & $\dots$ 
        \end{tabular}
    \end{center}
}

\newcommand{\gamed}[1]{
    \begin{center}
        \begin{tabular}{c|ccccccc}
            I & $#1_0$ & & $#1_2$ & & $#1_4 $& & $\dots $\\ \hline
            II & & $#1_1$ & & $#1_3 $& &$ #1_5$ & $\dots$ 
        \end{tabular}
    \end{center}
}

\title{Infinite Games}
\author{Lectures by Benedikt L{\"o}we}
\date{}

\begin{document}
\maketitle


\reversemarginpar{Lecture 1}

\section{Introduction}
Before this course kicks off, we will first discuss a few things that this course is \it{not} about. Do bear in mind though that this is still an advanced set theory course, building off the content found in IID Logic and Set Theory.

\underline{Literature}: Some useful literature can be found in `The Higher Infinite' - in particular, Chapter 6 ``Determinacy'', sections 27 (Infinite Games) and 28 (AD and Combinatorics).

The term ``Infinite Games'' can evoke different reactions in different mathematicians. We will get on to formal definitions later, but the games we will consider have the following properties. They are:
\begin{itemize}
    \item Two-player
    \item Length $\omega$
    \item Win-lose
    \item Perfect information
    \item Perfect recall
\end{itemize}
Essentially, they are infinite versions of board games.

A notable theorem from the finite analogue of this theory is that of Zermelo, which is that very finite such game is determined - this is considered rather trivial, and we will see the difference between the application of this to the finite case versus the infinite case.

\subsection{Two Players}

We have two players: I and II.

Three-player games do not, in general, admit winning strategies. Consider the following example:

I, II, III are playing. Player I is given a gold coin.

\underline{Round 1}: I can give it to II or to III.

\underline{Round 2}: Whoever has the coin can give it to I or to II.

The person with the coin wins.

So if II gets the coin, they will keep it and win. Player I can either hand the win to II, or leave it to chance with III. III has no win condition, and II cannot force anyone to give them the coin. So no-one has a winning strategy. However, \undf{coalitions} $\{I,II\}$, $\{I,III\}$, $\{II,III\}$ can each ensure a win.

What we have encountered here is \undf{cooperation}, which fundamentally changes the strategies and payoffs; this phenomenon cannot arise in two-player games, as they players are always competing directly with each other.

\subsection{Length $\omega$}

In game theory in economics, there is research on potentially infinite games.

Consider the prisoner's dilemma, concisecly represented by:

\begin{center}
    \begin{tabular}{cc|cc}
        2 & 2 & 1 & 3 \\ \hline
        3 & 1 & 0 & 0
    \end{tabular}
\end{center}

There is a lot of research on this game as a single-move game, but it can also be thought of as a repeated game, where many rounds are played one after the other. In fact, if you fix a length of the game in advance this will affect the strategy of the players. Economists have deal with games which are, while certainly not infinte, of unknown length, and as such it happens that infinite games can serve as a useful model for this situation From there we can study asymptotic behaviour, or evolutionary phenomena.

The point here is that if we only think of \it{truly} infinite games, \it{i.e.} it will take an infinite amount of time to win, then the situation is fundamentally different.

\begin{remark}[Example] The Prime Factor Game\ \\

\begin{center}
    \begin{tabular}{cccccccccc}
        I & $k_0$ & & $k_1$ & & $k_2$ & & $k_3$ & &\dots \\
        II & & $p_0$ & & $p_1$ & & $p_2$ & & $p_3$ &\dots
    \end{tabular}
\end{center}

The $k_i\ge 2$ are natural numbers, and the $p_i$ are prime numbers. At the end of the game, we look at $K = \{k_i:i\in \N\}$ and $P = \{p_i:i \in \N\}$. We say that Player II wins if $P$ is the set of all prime factors in $K$ (and no more).

\underline{Observe}: Player II has a winning strategy.

Let $k_0 = q_0^{\ell_0}\dots q_m^{\ell_m}$. Then play as though $p_0 = q_0$, $p_1 = q_1$, $p_2 = q_2$, exhausting the finitely many prime factors of $k_0$ before moving on to $k_1$. Repeating ad infinitum, it is clear that the set $P$ is precisely what is desired.

However, if we look at how this is going for II after any finite number of moves $N \in \N$, then in most runs of the game, the finite sets $\{k_0,\dots,k_N\}$ and $\{p_0,\dots,p_N\}$ do not look like a win for player II; it will seems as though things are going worse and worse for II.

This critically highlights how the asymptotic behaviour can be drastically different from the outcome of the game after infinite time.
\end{remark}

A common objection to this type of material is that you can't play these games, so how could you know who wins?

The above example clearly highlights that even though, of course, the games cannot actually be played, we may still be able to prove that a winning strategy does or does not exist for either player. So we are replacing actually playing the game with thinking about the different strategies for it.

Let's modify the PFG slightly:

\begin{center}
    \begin{tabular}{cccccccc}
        I & & $k_0$ & & $k_1$ & & $k_2$ & \dots \\ 
        II & $p_0$ & & $p_1$ & & $p_2$ & & \dots
    \end{tabular}
\end{center}

Now Player I has a winning strategy instead. Take $p_0$, find another prime $q \ne p_0$ and play $k_i = q^{i+1}$. Then $K = \{q^{i+1}:i \in \N\}$. So II wins iff $P = \{q\}$, but $p_0 \in P$. So II loses.

\subsection{Win-Lose}

There is a related notion here called \undf{zero-sum}; in these games, there is a fixed payoff that is split between the two players. So, for instance, the Prisoner's Dilemma is \undf{not} zero-sum because the total payoff differs between some outcomes. However, the game
\begin{center}
    \begin{tabular}{cc|cc}
        1&1&0&2\\ \hline
        2&0&1&1
    \end{tabular}
\end{center}
\it{is} zero-sum.

\undf{Win-lose} simply means that the payoff is an indivisible $1$. So in our case, payoff functions are characteristic functions of a \undf{payoff set}.

\subsection{Perfect Information}

Paradigmatic: board games, after which this idea was modelled.

A non-example is \it{card games}, in which your own hand is only known to you. Unsurprisingly, this scenario is called \it{imperfect information}.

Consider yet another variant of PFG:

\begin{center}
    \begin{tabular}{cccccccccc}
        I & $k_0$ & & $k_1$ & & $k_2$ & & $k_3$ & &\dots \\
        II & & $p_0$ & & $p_1$ & & $p_2$ & & $p_3$ &\dots
    \end{tabular}
\end{center}

Here, player I picks $k_i$, but does not have to reveal $k_i$ before II has played $p_i$. Here, although it may happen with probability zero, it is possible for II to beat any set of moves that I makes simply by being lucky, and guessing only prime factors for numbers chosen by I. However, it is clear that it is impossible to ensure that this is the case.

So neither of the two players has a winning strategy in this variant. The study of these imperfect information games is closely related to probability.

\subsection{Perfect Recall}

This means that both of the players remember everything that has happened before; the opposite of course would be that the players have a finite, bounded memory.

For instance, take PFG with the additional constraint that Player II can only remember the last 1000 moves. Now Player I has a strategy that might win; on the first move, pick a natural number with at least 1001 distinct prime factors. Then II will have no idea what move to make at $N = 1001$; they might guess, and so they can still win, but they have no way to ensure this (and again it will not be very likely).

Imperfect recall is very relevant in applications of infinite games in computer science.

\reversemarginpar{Lecture 2}

We fix a set $M$ of moves. In most cases, $M$ will simply be $\N$ - but we will aim to keep this slightly more general for now.

Note that from the perspective of a set theorist, we think of $\N$ as equal to $\omega$, and in particular $n = \{0,1,\dots,n-1\}$. Moreover, functions are \it{set-theoretic} functions, \it{i.e.} sets of ordered pairs with the function property. For instance:
\begin{align*}
    M^n = \{s; s:n\ra M\}
\end{align*}
is the set of functions from the \it{set} $n$ to the set $M$. If $s\in M^n$ and $t \in M^k$, with $k>n$, then $s\subset t$ is the same as saying ``$s$ is an initial segment of $t$'', or that ``$t$ is an extension of $s$''.

Since we are thinking of sequences as functions, we can also write: if $m<n$ and $s \in M^n$, then $s\restriction m \in M^m$.

These are well-known formal definitions of these objects, but they will be used particularly ruthelessly here.

We make another important definition:
\begin{align*}
    M^{<\omega} \coloneqq \bigcup_{n\in \N}M^n
\end{align*}
This is the set of all finite sequences of elements of $M$; these will be called the \undf{positions} of the game. We also have:
\begin{align*}
    M^\omega \coloneqq \{x;x:\N\ra M\}
\end{align*}
is the set of all \undf{runs} or \undf{plays} of the game \it{i.e.} the set of all sequences of $M$ of length $\omega$. Note that if $x\in M^\omega$ is a run and $n\in \N$, then $$x\restriction n:n\ra M$$ is the position that the play producing $x$ was in after $n$ rounds.

\underline{The games on $M$}:

\begin{center}
    \begin{tabular}{c|ccccccc}
        I & $m_0$ & & $m_2$ & & $m_4 $& & $\dots $\\ \hline
        II & & $m_1$ & & $m_3 $& &$ m_5$ & $\dots$ 
    \end{tabular}
\end{center}
We are restricting our attention to games where I,II play in alternation and player I starts. [Remark: more general games can be described by these; see later.]

Then $x(i)\coloneqq m_i$ is the run produced by the game, and $s\coloneqq x\restriction n$ is the $n^\th$ position.

If $x \in M^\omega$, we write
\begin{align*}
    x_\I(i) &\coloneqq x(2i)\\
    x_\II(i) &\coloneqq x(2i+1)
\end{align*}
$x_I,x_{II}\in M^\omega$ correspond to the moves made by players I,II respectively. If $x,y\in M^\omega$, we write $x\ast y$ (\undf{interleaving}) for the sequence $z$ defined by:
\begin{align*}
    z(n) \coloneqq \left\lbrace  \begin{array}{ccc}x(k)&n = 2k\\ y(k) &n = 2k+1 \end{array}\right.
\end{align*}

Clearly, $x_\I \ast x_\II = x$.

If $A\subset M^\omega$, we call $A$ a \undf{payoff set}. In the game $G(A)$, we say that player I wins a run $x \in M^\om$ if $x \in A$; otherwise player II wins.

We call any function $$\sigma:M^{<\om} \ra M$$ a \undf{strategy}. Note that a strategy looks at the entire game up until that point to decide the next move; this is the perfect recall aspect. You may wonder why we bother defining a strategy for player I at odd length positions, and this is largely for notational convenience; it is a little easier to have this notational overkill.

Note that each strategy in this sense can be thought of a strategy for I, plus a strategy for II. Let
\begin{align*}
    O&\coloneqq \bigcup_{n\textrm{ odd}}M^n\\
    E &\coloneqq \bigcup_{n\textrm{ even}}M^n
\end{align*}
Then $\sigma\restriction E$ is a strategy for I, and $\sigma \restriction O$ is a strategy for II. So there is redundance in the notation.

If $\sigma, \tau$ are strategies, we can play them against each other by interleaving them as $\sigma\ast\tau \in M^\om$, which is defined by:
\begin{align*}
    (\sigma\ast\tau)(2n)&\coloneqq \sigma((\sigma\ast\tau)\restriction 2n)\\
    (\sigma\ast\tau)(2n+1)&\coloneqq \tau((\sigma\ast\tau)\restriction 2n+1)
\end{align*}
We say that $\sigma$ is \undf{winning for I in $G(A)$} if $\forall \tau(\sigma\ast\tau\in A)$, \it{i.e.} I always wins regardless of II's strategy. Similarly, we say that $\tau$ is \undf{winning for II in $G(A)$} if $\forall \sigma(\sigma\ast\tau\not\in A)$.

We say that a set $A$ is \undf{determined} if one of the two players has a winning strategy in $G(A)$.

\begin{remark}\ 
    \begin{itemize}
        \item Clearly, at most one player can have a winning strategy (otherwise, play them against each other).
        \item However, it is not obvious (and not true, up to the axiom of choice) that every set is determined.
        \item In fact, we will see that AC implies that there are non-determined sets, but ``every set is determined'' (Axiom of Determinacy, AD) is consistent ZF - though this requires more nuance, but we will discuss all of this later.
    \end{itemize}
\end{remark}

You may ask: is that really the most general form of games that we want to look at? What if we wanted to include things like `forbidden' moves, or allowing one or two players to make several moves at a time, or having two different move sets? It turns out that we don't need to worry about this:

A set $T\subset M^{\lom}$ is called a \undf{tree} if it is closed under initial segments, \it{i.e.} if $s\in T$ and $t\subset s$ then $t \in T$.

These trees look very much like the trees one might encounter in graph theory/combinatorics; though there are some small differences. In this set-theoretic notion of a tree, each node in the tree contains within it all of the information abou the path from the root (\it{i.e.} the empty set, $\emptyset$) to it.

If $T$ is a tree on $M$ and $x\in M^\om$, we say that $x$ is a \undf{branch through $T$} if for all $n \in \N$, $x\restriction n \in T$. We write $[T]$ for the set of branches through $T$; in some literature this is referred to as the \undf{body of $T$}.

\begin{remark}[Example]
    $M^\lom$ is a tree; $[M^\lom] = M^\om$.
\end{remark}

We can think of a tree $T$ as ``finitary'' rules for a game: if $x\not\in [T]$, then there is a least $n$ for which $x\restriction n\not\in T$. If $n$ is odd, then player I left the tree, and if $n$ is even then player II left the tree.

Define a game $G(A;T)$ where $A\subset [T]$ and $T$ is a tree on $M$.
The game is as usual:
\game
If $x \in A$, then player I wins. If $x\not\in [T]$ and the least $n$ for which $x\restriction n\not\in T$ is even, then player I wins. In all other cases, player II wins.

Now, even though this looks more general due to the introduction of the tree $T$, it can be seen that this is in fact a special case of a $G(A)$ game, since we can define: $$A_T \coloneqq \{x \in M^\om; x\in A\textrm{ or }x\not\in [T]\textrm{ and the least }n\textrm{ s.t. } x\restriction n\not\in T\textrm{ is even}\}$$. Then $G(A;T)$ and $G(A_T)$ are \undf{the same game}.

Note that we haven't quite defined what it means to be the same game, but in this particular case it should be rather clear that these two are indeed the same game.

This idea of using trees gives us a lot of flexibility with the move set.

\begin{remark}[Example 1]
    Suppose the moves for I are in $X$, and the moves for II are in $Y$. We can take $M\coloneqq X\cup Y$, $A\subset M^\om$, and
    \begin{align*}
        T\coloneqq \{s;\ \forall n\in \N,\ s(2n)\in X\textrm{ and }s(2n+1)\in Y\}
    \end{align*}
    Then $G(A;T)$ is the game we desire.
\end{remark}

\begin{remark}[Example 2]
    Suppose I can always make two moves in $X$, but II can only make one move. We then take $M\coloneqq X^2\cup X^1$, and apply the idea of Ex. 1 with $X = X^2$ and $Y = X^1$.
\end{remark}

\begin{remark}[Example 3]
    If $X\subset Y$, then every game $G(A)$ on $X$ can be thought of as a game on $Y$ by $G(A;T)$, where $$T\coloneqq X^{\lom} \subset Y^{\lom}$$
\end{remark}

\begin{defin*}[Strategic Tree]
    Let $\sigma$ be a strategy. We define the \undf{I-strategic tree} and the \undf{II-strategic tree} on $M$ as follows:
    \begin{align*}
        T_\sigma^\I &\coloneqq \{s\in M^{\lom}; \forall n(s(2n)=\sigma(s\restriction 2n))\}\\
        T_\sigma^\II &\coloneqq \{s\in M^\lom; \forall n(s(2n+1)=\sigma(s\restriction 2n+1))\}
    \end{align*}

    When drawing out these trees, for instance $T_\sigma^\I$, the layers alternate between making any choice from $M$ (representing II's moves) and making the only choice dictated by $\sigma$ for I.

    II-strategic trees look the same except that we have branching in odd length nodes and no branching in even length nodes.

    $T$ is called \undf{strategic} if there is $\sigma$ such that $T = T^\I_\sigma$ or $T = T^\II_\sigma$.
\end{defin*}

\underline{Observe}: $$T_\sigma^\I = \{(\sigma\ast \tau)\restriction n; \tau \textrm{ any strategy and }n\in \N\}$$ $$T_\sigma^\I = \{(\tau\ast \sigma)\restriction n; \tau \textrm{ any strategy and }n\in \N\}$$

\underline{Therefore}: $$ [T_\sigma^\I] = \{\sigma\ast \tau; \tau \textrm{ any strategy}\}$$
$$ [T_\sigma^\II] = \{\tau\ast\sigma; \tau \textrm{ any strategy}\}$$

\begin{remark}[Proposition]\ 
    \begin{enumerate}
        \item $\sigma$ is a winning strategy for I in $G(A)\iff [T_\sigma^\I]\subset A$
        \item $\sigma$ is a winning strategy for II in $G(AS)\iff [T_\sigma^\II]\cap A = \emptyset \iff [T_\sigma^\I]\subset M^\omega \backslash A$
    \end{enumerate}

    \underline{Also}: $A$ is determined iff either $A$ contains $[T_\sigma^\I]$ for some $\sigma$ or $M^\om \backslash A$ contains $[T_\sigma^\II]$ for some $\sigma$.
\end{remark}

\underline{Notation}: If $s,t\in M^\lom$, we write $st$ for the concatenation of $s$ and $t$. This also works if $t$ is infinite; if $x \in M^\om$ and $s\in M^\lom$, then similarly $sx\in M^\om$ is the concatenation.

If $t$ is a length 1 sequence, say $t = \langle m\rangle$, we also write $sm$ for $st = s\langle m\rangle$; this is usually unambiguous.

For the length of a sequence we write $\lh(s) = \textrm{dom}(s)$.

\begin{defin*}[Splitting Node, Perfect Tree, Perfect Set]\ 
    \begin{enumerate}[label=\arabic*)]
        \item If $T$ is a tree and $s \in T$ we say $s$ is a \undf{splitting node} if there are $m\ne m'$ such that both $sm,sm' \in T$.
        \item $T$ is \undf{perfect} if for each $s \in T$ there is a $t\supseteq s$ such that $t\in T$ and $t$ is splitting in $T$.
        
        [Remark: every strategic tree is perfect]

        \item $A\subset M^\om$ is \undf{perfect} if there is a perfect tree $T$ such that $A = [T]$.
        
        \underline{Remark}: Compare to the topological notion of a \undf{perfect set}: clsoed without isolated points. We will find out later that, with the right toplology on $M^\om$, these notions will coincide.

    \end{enumerate}
\end{defin*}

\begin{theorem*}[Cantor]
    Suppose $A\subset 2^\om \equiv \{0,1\}^\om$ is perfect and non-empty. Then $A$ has cardinality $2^{\aleph_0}$.
\end{theorem*}
\begin{proof}
    $A\subset 2^\om$ and $|2^\om| = 2^{\aleph_0}$, so $|A| \le 2^{\aleph_0}$. So by Cantor-Schr{\"o}der-Bernstein, it is enough to show that there is an injection from $2^\om$ into $A$.

    We define this injection via a function $\phi : 2^{\lom} \ra T$, where $T$ is perfect such that $A = [T]$. We will define this by recursion (this is known as a \it{Cantor scheme}):
    \begin{align*}
        \phi(\emptyset) &\coloneqq \emptyset\\
    \end{align*}
    Suppose $\phi(s) = t \in T$. Since $T$ was perfect, find $u\supseteq t, u \in T$ that is splitting: $u0,u1\in T$. To ensure this is uniquely defined (to potentially avoid issues with Choice), take the minimal one. Then:
    \begin{align*}
        \phi(s0) &\coloneqq u0\\
        \phi(s1) &\coloneqq u1
    \end{align*}
    This finishes the definition of $\phi$. We then define:
    \begin{align*}
        \hat{\phi}&:2^\om \ra [T] = A\\
        \hat{\phi}(x)&\coloneqq \bigcup_{n\in \N}\phi(x\restriction n)
    \end{align*}
    We need to check some things:
    \begin{enumerate}
        \item $\lh(\phi(x\restriction n)) \ge n$
        \item $\phi(x\restriction n)\subset \phi(x\restriction m)$ if $n\le m$
        
        $\implies \hat{\phi}:2^\om\ra2^\om$

        \item $\hat{\phi}(x)\restriction u\subset \phi(x\restriction k)$ for some $k$ so $\hat{\phi}(x)\restriction u\in T$, so $\hat{\phi}(x)\in [T]$.
        
        So we indeed have that $\hat{\phi}:2^\om \ra [T]$.
    \end{enumerate}

    It remains to show that $\hat{\phi}$ is an injection:

    Suppose $x\ne y$. Find $n$ such that $x\restriction n = y\restriction n$, but $x(n)\ne y(n)$. WLOG, say $x(n) = 0$ and $y(n) = 1$. But then $\phi(x\restriction n+1)\ne \phi(y\restriction n+1)$, since the former ends in $0$ and the latter in $1$. This implies that $\bigcup_{k\in \N}\phi(x\restriction k)\ne \bigcup_{k\in \N}\phi(y\restriction k)$, hence $\hat{\phi}(x)\ne \hat{\phi}(y)$.
\end{proof}

\begin{remark}
    If $|M|\ge 2$ and $T$ is a perfect tree on $M$, then the same proof shows that $2^{\aleph_0}\le |[T]|$.
\end{remark}

\begin{remark}[Corollary]
    If $|M|\ge 2$, then:
    \begin{enumerate}[label = (\roman*)]
        \item if player I has a winning strategy in $G(A)$, then $|A|\ge 2^{\aleph_0}$
        \item if player II has a winning strategy in $G(A)$ then $|M^\om \backslash A| \ge 2^{\aleph_0}$.
    \end{enumerate}
    This follows from:
    \begin{enumerate}
        \item strategic trees are perfect
        \item perfect sets are large
        \item winning startegy means ``includes strategic tree''
    \end{enumerate}
\end{remark}

Note that if $A\subset M^\om$ with $|M|\ge 2$, then either $|A|\ge 2^{\aleph_0}$ or $|M\backslash A| \ge 2^{\aleph_0}$.

The corollary gives a necessary condition on when a fixed player has a winning strategy, but no non-trivial necessary condition for determinacy.

\subsection*{Sufficient Conditions}

Let's do the following as a warmup.

Prove that if $A$ is countable, then player II has a winning strategy in $G(A)$.

\begin{remark}[Proposition]
    If $A = \{a_i;i \in \N\}$ is countable, then player II has a winning strategy in $G(A)$.
\end{remark}
\begin{proof}
    In II's round $k$ [that means digit $2k+1$], II takes care of $a_k$, simply by playing $1 - a_k(2k+1)$ (assume again we are playing on $M = \{0,1\}$; on anything else just pick something different to $a_k(2k+1)$).

    So the strategy $\tau$ is:
    \begin{itemize}
        \item ignore everything player I does
        \item blindly play $1 - a_k(2k+1)$ in your $k^\th$ move.
    \end{itemize}

    Clearly then for any $\sigma$,
    \begin{align*}
        (\sigma\ast \tau)_\II(k) &= (\sigma \ast \tau)(2k+1)\\
        &= 1 - a_k(2k+1)\\
        &\ne a_k(2k+1)
    \end{align*}
    So $\sigma \ast \tau\ne a_k$ for arbitrary $k$, so $\sigma\ast \tau \not \in A$. Thus $\tau$ is winning.
\end{proof}

\reversemarginpar{Lecture 3}

\underline{Necessary}: we have determined some necessary conditions for wins:
\begin{itemize}
    \item I wins $G(A)\implies |A| = 2^{\aleph_0}$
    \item II wins $G(A)\implies |\om^\om\backslash A| = 2^{\aleph_0}$
\end{itemize}
\underline{Sufficient}: we also have the sufficient condition:
\begin{itemize}
    \item if $A$ is countable, then player II wins.
\end{itemize}

Note that we write `I/II' wins as shorthand for `I/II has a w.s.'
\ \\

\begin{theorem*}\ \\
    \begin{enumerate}
        \item If $A\subset \om^\om$ such that $|A| < 2^{\aleph_0}$, then player II has a winning strategy in $G(A)$.
        \item If $A\subset \om^\om$ such that $|\om^\om\backslash A| < 2^{\aleph_0}$ then player I has a winning strategy in $G(A)$.
    \end{enumerate}
\end{theorem*}
\begin{proof}
    The proofs of 1 and 2 are essentially just switching the roles of I,II. So we are just going to prove 1.

    \underline{Caution}: our games are \it{not} fully symmetric; I is not in the same situation as II. Moving first can sometimes be an advantage, and sometimes a disadvantage. The above claim must thus be checkd carefully.

    Let $|A| < 2^{\aleph_0}$. Define an equivalence relation $\sim$ on $\om^\om$ by $$x\sim y \iff x_{II} = y_{II}$$. So equivalence classes look like this:
    \begin{align*}
        C_z \coloneqq \{x; x_{II} = z\}
    \end{align*}
    So there is a bijection between the $\sim$-equivalence classes and $\om^\om$. In particular, there are $2^{\aleph_0}$ such equivalence classes. By the pigeonhole principle, we find $z$ such that $$C_z \cap A = \emptyset$$. Then define $\tau$ as:
    \begin{itemize}
        \item ignore everything player I does
        \item just play the next digit of $z$
    \end{itemize}
    Formally, we can say $\tau(s)\coloneqq z(n)$ if $\lh(s) = 2n+1$, and whatevery you like on the even entries. Then if $\sigma$ is any strategy, we have
    \begin{align*}
        (\sigma\ast\tau)_{\II} &= z\\
        \implies \sigma\ast\tau &\in C_z\\
        \implies \sigma\ast\tau&\not\in A
    \end{align*}
    So $\tau$ is a w.s. for II.
\end{proof}

This is called a \undf{blindfolded strategy}, since II doesn't care about what I is doing (and doesn't need to know). Formally, this is when there is $z\in \om^\om$ such that for player II, $\tau(s)\coloneqq z(n)$ if $\lh(s) = 2n+1$, or similarly for player I $\sigma(s)\coloneqq z(n)$ if $\lh(s) = 2n$. We normally denote this as $\tau_z$ if for player II, and $\sigma_z$ if for player I.

\underline{Consequence}: If we have any set $A$ that is not determined, then it must be the case that $|A| = |\om^\om \backslash A| = 2^{\aleph_0}$. So we have a bit of an idea that a non-determined set must look a bit symmetric.

\underline{Next goal}: Find such a non-determined set.

\begin{theorem*}[uses AC]
    There is a non-determined subset $A\subset \om^\om$.
\end{theorem*}
\begin{proof}
    The idea here is to ensure that $A$ contains no strategic trees, which we do by enumerating them and then distributing the branches between $A$ and its complement.

    We did prove in Lecture 3 that if $T$ is a strategic tree, then $|[T]| = 2^{\aleph_0}$.
    
    \underline{Question}: How many strategic trees are there?

    \underline{Notation}: We write Trees $\coloneqq \{T;T\textrm{ is a tree on }\om\}$, and STrees$\coloneqq \{T;T\textrm{ is a strategic tree on }\om\}$.

    We remark that a tree $T\subset \om^\om$, which is countable, so this gives an upper bound on the size of Trees. So we have $|$STrees$|\le|$Trees$|\le2^{\aleph_0}$. Can we also find a lower bound? This is where the blindfolded strats come in...

    If $z\ne z'\in \om^\om$, then $[T^\I_{\sigma_z}]\cap[T^\I_{\sigma_{z'}}] = \emptyset$. So $T_{\sigma_z}^\I \ne T^\I_{\sigma_{z'}}$.

    Together (with Cantor-Schr{\"o}der-Bernstein), we get a bijection between $2^{\aleph_0}$ and STrees.

    Note that so far we haven't used AC, with the mild exception that the notation $2^{\aleph_0}$ implies the existence of an ordinal in bijection with the powerset of $\om$, which requires AC. However, the above can be reformulated as saying that we have injections $\om^\om \xhookrightarrow{} \dots \xhookrightarrow{} \om^\om$, and then use CSB; choiceless.

    However, we need a little bit of choice now. We aren't going to use the full AC, but only that the set $\om^\om$ is wellorderable. This implies:
    \begin{itemize}
        \item $2^\aleph_0$ \it{is} an ordinal (so we can do transfinite recursion on it)
        \item there is a choice funcntion $c:\mathcal{P}\om^\om\backslash \emptyset \ra \om^\om$ (\it{i.e.} $c(A)\in A$)
    \end{itemize}

    Equipped with this, we do the construction. We had that $|\textrm{STrees}| = 2^{\aleph_0}$, so write STrees $=\{T_\alpha; \alpha < 2^{\aleph_0}\}$ and do the following transfinite recursion:

    We are going to define sets $A_\alpha,B_\alpha$ in stages for $\alpha < 2^{\aleph_0}$ in such a way that $|A_\alpha| = |B_\alpha| = |\alpha|$ ($\ast$). In the end we will see $A_\alpha \cap B_\alpha = \emptyset$.

    \underline{$\alpha = 0$}: $A_0 = B_0 = \emptyset$.

    \underline{$\alpha = \beta+1$}: Suppose we have $A_\beta,B_\beta$. Consider $T_\beta\in$ STrees. Then $|[T_\beta]| = 2^{\aleph_0}$. By $(\ast)$, $|A_\beta| = |B_\beta| = |\beta|<2^{\aleph_0}$. This implies that $|A_\beta\cup B_\beta|<2^{\aleph_0}$. Thus $[T_\beta]\backslash (A_\beta \cup B_\beta)\ne\emptyset$ (even better, it has $2^{\aleph_0}$ many elements).

    Define:
    \begin{align*}
        a_\beta & \coloneqq c([T_\beta]\backslash(A_\beta\cup B_\beta))\\
        b_\beta & \coloneqq c([T_\beta]\backslash (A_\beta\cup B_\beta\cup \{a_\beta\}))
    \end{align*}
    Note that the latter input for $c$ is still non-empty since $|[T_\beta]\backslash (A_\beta\cup B_\beta)| = 2^{\aleph_0}$.

    Then let $A_\alpha \coloneqq A_\beta \cup \{a_\beta\}$, and similarly $B_\alpha \coloneqq B_\beta\cup\{b_\beta\}$. Moreover, $|A_\alpha| = |\beta + 1| = |\alpha| = |B_\alpha|$, satisfying IH.

    \underline{$\alpha$ is a limit}: For all $\beta < \alpha$, $A_\beta$, $B_\beta$ are defined and satisfy $(\ast)$. So we let:
    \begin{align*}
        A_\alpha &\coloneqq \bigcup_{\beta<\alpha}A_\beta\\
        B_\alpha &\coloneqq \bigcup_{\beta<\alpha}B_\beta
    \end{align*}
    Obviously, $|A_\alpha| = |\alpha| = |B_\alpha|$ since they are each unions of things of the right size (there is an increasing sequence of bijections that converges to what we want, formally), so $(\ast)$ is still satisfied.

    Now we let
    \begin{align*}
        A &\coloneqq \bigcup_{\alpha <2^{\aleph_0}}A_\alpha\\
        B &\coloneqq \bigcup_{\alpha < 2^{\aleph_0}}B_\alpha
    \end{align*}
    \underline{Note 1}: $A\cap B = \emptyset$; if not, then $a_\alpha = b_\beta$ for some $\alpha,\beta$. WLOG $\alpha \le \beta$. This then contradicts the choice of $b_\beta$.

    \underline{Note 2}: $|A| = 2^{\aleph_0} = |B|$. This is good, since it was a necessary condition for $A$ and $B$ being non-determined, as we found earlier.

    \underline{Claim}: $A$ is not determined (and similarly $B$).

    \underline{Proof of Claim}: Suppose $A$ is determiend. Then there is a strategic tree $T_\alpha$, $\alpha <2^{\aleph_0}$, such that either $[T_\alpha] \subset A$ or $[T_\alpha]\cap A = \emptyset$. But consider $a_\alpha,b_\alpha \in [T_\alpha]$. We have $a_\alpha \in A$, $b_\alpha \in B$, \it{i.e.} $b_\alpha\not\in B$. This rules out both cases, so contradiction. This proves the theorem.
\end{proof}

You may notice that this is the very same diagonalisation argument that crops up all over the place.

\underline{Discussion}: we used AC to produce a non-determined set. [Usually, AC implies the existence of pathologies, \it{e.g.} a non-Lebesgue measurable set, or the Banach-Tarski decomposition of the unit ball. AC does not produce a constructive method for teh pathological objects, since the construction depends on the choice function.

\underline{Motto (hope)}: If a set $A$ is `nice' or `simple', then it is not pathological.

Similarly here; though we have just proved that non-determined sets may exist, the ones we have found are all pathological. We might ask if non-determined sets only arise this way. Hence:

\underline{Goal}: Make `nice' and `simple' precise, and prove that nice sets are determined.

\reversemarginpar{Lecture 5}

\begin{defin*}[Quasistrategy]
    A function $$\sigma:M^{\lom}\ra \mathcal{P}(M)\backslash \{\emptyset\}$$ is called a \undf{quasi-strategy}. Strategies can be identified as a special case of quasistrategies: $$(\forall s)|\sigma(s)| = 1$$ which clearly induced a strategy as we have defined it.

    We also have a notion of \undf{quasistrategic trees}:
    \begin{align*}
        Q_\sigma^\I &\coloneqq \{s\in M^\lom; (\forall n) s(2n)\in \sigma(s\restriction 2n)\}\\
        Q_\sigma^\II &\coloneqq \{s\in M^{\lom}; (\forall n) s(2n+1)\in \sigma(s\restriction 2n+1)\}
    \end{align*}
    A quasistrategy is \undf{winning for I in $G(A)$} if $[Q_\sigma^\I]\subset A$, and \undf{winning for II in $G(A)$} if $[Q_\sigma^\II]\cap A = \emptyset$.

    As before, at worst one of the two players can have a winning quasistrategy.
\end{defin*}

\begin{defin*}[Quasidetermined Set]
    A set $A\subset M^\om$ is called \undf{quasidetermined} if either I or II has a winning qs in $G(A)$.
\end{defin*}

\begin{remark*}[Lemma]
    If $M$ is wellorderable, then every quasistrategic tree on $M$ contains a strategic tree on $M$.
\end{remark*}

[\underline{Consequence}: if $M$ is wellorderable and $A\subset M^\om$ is quasidetermined, then it is determined.]

\begin{proof}
    Let $\sigma$ be a quasistrategy $\sigma :M^\lom \ra \mathcal{P}(M)\backslash \{\emptyset\}$. If $M$ is wellorderable, then there is a choice function $c:\mathcal{P}(M)\backslash \{\emptyset\}\ra M$ (\it{i.e.} such that $c(A)\in A$ for each $A$).

    Then define $\sigma^\ast \coloneqq c\circ \sigma:M^\lom \ra M$. By construction, $T_{\sigma^\ast}^\I \subset Q_\sigma^\I$ and $T_{\sigma^\ast}^\II \subset Q_\sigma^\II$.
\end{proof}

Note that we needed a bit of choice here - that $M$ is wellorderable.

Compare with out proof of the existence of a non-determined set from Lecture 4 (we used a wellordering of $M^\om$ to get $A\subset M^\om$ non-determined) to this, where we just use a wellordering of $M$. This requires significantly less choice, and it is often the case that $M$ comes equipped with a wellordering but $M^\om$ does not, \it{e.g.} $M = \N$.

\begin{defin*}[Closed Set]
    A set $A\subset M^\om$ is called \undf{closed} if there is a tree $T$ on $M$ such that $A = [T]$.
\end{defin*}

\begin{remark*}[Remarks]\ 
    \begin{enumerate}
        \item This is actually the notion of being closed in a topological space; we will see this space in the next lecture.
        \item Zermelo's finite games can be represented by closed payoff sets.
        
        Finite game of length $n$: let's say $f:M^n\ra \{\I,\II\}$ is the functino labelling the leaves according to which player wins.

        Let $A\coloneq \{x\in M^\om ; f(x\restriction n) = \I\}$. Then the finite game is just $G(A)$. This is just playing an infinite game, but having decided the outcome already after $n$ moves.

        We can also define:
        \begin{align*}
            T\coloneqq \{s\in M^\lom; f(s\restriction n) = \I\textrm{ and }\lh(s)\ge n \textrm{ or }\lh(s) < n\textrm{ and there is a }t\supseteq s \textrm{ s.t. } f(t)=\I\}
        \end{align*}
        Clearly, $[T] = A$. Thus all finite games are closed games.
    \end{enumerate}
\end{remark*}

\begin{theorem*}[Gale-Stewart]
    All closed sets $A\subset M^\om$ are quasidetermined.
\end{theorem*}
\begin{proof}
    If $A = [T]$, this means that if $x\not \in A$, then $x\not\in [T] = \{x\in M^\lom; (\forall n)x\restriction n\in T\}$. This implies there is some $n$ such that $x\restriction n\not \in T$. These are positions thare are won for sure by player II.

    Define a partial function $$\ell:M^\lom \ra \{\I,\II\}$$ by $\ell(s) = \II\iff s\not\in T$. Apply the following recursion rules to the partial labellings:

    We extend $\ell$ to $\ell^+\supseteq \ell$ according to the following rules. If $\ell(s)$ is undefined:
    \begin{itemize}
        \item and $\lh(s)$ is even [I moves] and $\forall m\in M$, $\ell(sm)=\II$, then $\ell^+(s)\coloneqq = \II$
        \item and $\lh(s)$ is odd [II moves] and $\exists m\in M$, $\ell(sm) = \II$, then $\ell^+(s) = \II$
    \end{itemize}
    Our transfinite recursion is then in the normal way:
    \begin{itemize}
        \item $\ell_0\coloneqq \ell$
        \item $\ell_{\alpha+1} \coloneqq (\ell_\alpha)^+$
        \item $\ell_{\lambda} \coloneqq \bigcup_{\alpha < \lambda} \ell_\alpha$
    \end{itemize}
    It's easy to construct examples of truly transfinite processes like this [needs $|M|\ge \aleph_0$].

    \underline{Claim}: this process terminates at some ordinal $\alpha$, \it{i.e.} $\ell_{\alpha} = \ell_{\alpha+1} = (\ell_{\alpha})^+$.

    \underline{Proof of Claim}: for $s\in M^\om$, we can define an \it{age function}:
    \begin{align*}
        \textrm{age}(s)\coloneqq \left\lbrace \begin{array}{c}\textrm{least }\beta\textrm{ s.t. }\ell_{\beta}(s)\textrm{ is defined}\\ 0\quad\textrm{ if }\ell_{\beta}(s)\textrm{ is never defined}\end{array}\right.
    \end{align*}
    So if the process never terminates, then age is a surjection from the set $M^\lom$ onto the (proper) class Ord. This contradicts the Axiom of Replacement.

    Let $\alpha$ be this termination point. We can then define:
    \begin{align*}
        \hat{\ell}(s)\coloneqq \left\lbrace \begin{array}{c}\II\quad \textrm{ if }s\in \textrm{dom}(\ell_\alpha)\\ \I\qquad \ \ \ \textrm{otherwise}\end{array}\right.
    \end{align*}
    Then $\hat{\ell}\supseteq \ell_\alpha$, effectively just taking $\ell_\alpha$ and filling everything else in with $\I$s. So $\hat{\ell}$ is a total function.

    \underline{Claim}: If $\hat{\ell}(\emptyset) = \I$, then player I has a winning quasistrategy; otherwise, $\hat{\ell}(\emptyset) = \II$ and player II has a winning q.s.

    Given this, the proof is complete.

    \marginpar{Lecture 6}

    \underline{Subclaim 1}: If $\hat{\ell}(\emptyset) = \I$, define
    \begin{align*}
        Q_\I \coloneqq \{s \in \om^\lom; \hat{\ell}(s) = \I\}
    \end{align*}
    and if $\hat{\ell}(\emptyset) = \II$, define $Q_\II$ similarly.l

    Then $Q_\I/Q_\II$ is a I/II-quasistrategic tree.

    [Need to show:
    \begin{enumerate}
        \item if $\hat{\ell}(s) = \I$ and $\lh(s)$ is even then there is $m$ such that $\hat{\ell}(sm) = \I$]
        \item if $\hat{\ell}(s) = \I$ and $\lh(s)$ is odd then for all $m$, $\hat{\ell}(sm) = \I$
        \item if $\hat{\ell}(s) = \II$ and $\lh(s)$ is even then for all $m$, $\hat{\ell}(sm) = \II$
        \item if $\hat{\ell}(s) = \II $ and $\lh(s)$ is odd then there is $m$ such that $\hat{\ell}(sm) = \II$
    \end{enumerate}

    We had $\ell_\alpha = \ell_{\alpha+1} = (\ell_\alpha)^+$, so 3 and 4 follow immediately from the recursion definition of $\ell^+$.

    Similarly, if $\hat{\ell}(s) = \I$ and $\lh(s)$ is even then $s\not\in \textrm{dom}(\ell_\alpha)$. This implies there is an $m$ such that $sm \not\in \textrm{dom}(\ell_\alpha)$, so $\hat{\ell}(sm) = \I$. 2 is similar.]

    \underline{Subclaim 2}: If $\hat{\ell}(\emptyset) = \I$, then $Q_\I$ is a w.q.s for $\I$.

    [Need to show that $[Q_\I] \subset A$. So fix some $x \in [Q_\I]$. So for all $x$,
    \begin{align*}
        x\restriction u \in Q_\I &\implies \hat{\ell}(x\restriction u) = \I\\
        &\implies \hat{\ell}(x\restriction u)\ne \II\\
        &\implies \ell(x\restriction u)\not\in \II\\
        &\implies x\restriction u \in T
    \end{align*}
    So $x \in A$.]


\begin{remark*}: What we have done so far hasn't really needed trees/closed sets. Instead, we could let $$\ell :M^\lom \ra \{\II\}$$ be any partial labelling. Do the transfinite recursion $\ell_{\alpha+1} \coloneqq (\ell_\alpha)^+$, $\ell_0 \coloneqq \ell,\dots$, find fixed point $\ell_\alpha$; define $\hat{\ell}$, define $Q_\I$. Then $\Q_\I$ is a q.s. that avoids all $s \in \textrm{dom}(\ell)$.
\end{remark*}

For player II, it is not always the case that $Q_\II$ is winning [\it{c.f.} Example 12 on ES\#1] since you could stay on labels II without ever leaving the tree. So, we need to work slightly harder and find a sub-quasistrategy that is in fact winning.

Recall the age function:
\begin{align*}
    &\textrm{age}:Q_\II \ra \alpha + 1\\
    &\textrm{age}(s) \coloneqq \textrm{the least }\beta\textrm{ such that } s \in \textrm{dom}(\ell_{\beta})
\end{align*}
This means:
\begin{align*}
    \textrm{age}(s) =  0 &\iff s \in \textrm{dom}(\ell_0) = \textrm{dom}(\ell)\\
    &\iff s\not\in T
\end{align*}

\underline{Subclaim 3}: If $\hat{\ell}(s) = \II$ and $\lh(s)$ is even, then if $\textrm{age}(s) = 0$ or for all $m$, $\textrm{age}(sm) < \textrm{age}(s)$. If $\hat{\ell}(s) = \II$ and $\lh(s)$ is odd then either $\textrm{age}(s) = 0$ or there is $m$ such that $\textrm{age}(sm) < \textrm{age}(s)$.

[This follows directly from the recursive construction step.]

Now define $\hat{Q}_\II \subset Q_\II$ by:
\begin{itemize}
    \item $\emptyset \in \hat{Q}_{\II}$
    \item $sm \in \hat{Q}_\II :\iff \textrm{age}(sm) = 0\textrm{ or }sm \in Q_\II \textrm{ and }\textrm{age}(sm) < \textrm{age}(s)$
\end{itemize}

Informatlly, playing by $\hat{Q}_\II$ means: ``play into positions labelled II reducing the age if you can''.

If $\hat{\ell}(\emptyset) = \II$, then by Subclaim 3 $\hat{Q}_\II$ is a q.s.

\underline{Subclaim 4}: If $\hat{\ell}(\emptyset) = \II$, then $\hat{Q}_\II$ is winning for II: $[\hat{Q}_\II]\cap A = \emptyset$.

[Suppose $x \in [\hat{Q}_\II]$. Consider $a_n \coloneqq \textrm{age}(x\restriction n)$. This is a decreasing sequence of ordinals until it hits 0 by construction of $\hat{Q}_\II$. Since no infinite, strictly decreasing sequence of ordinals exists, we find $k$ such that $a_k = \textrm{age}(x\restriction k) = 0$. This implies $x\restriction k \not \in T$, hence $x\not\in A$.]

This finishes the claim, and hence concludes the entire proof.
\end{proof}

\begin{remark*}
    If $M$ is wellorderable (\it{e.g.} $M = \N$), then GST says that every closed subset of $M^\om$ is determined, and also that every complement of a closed set is determined. [Follows directly from the proof.]
\end{remark*}

Recall our motto/hope from the end of lecture 2: if a set is \it{nice} or \it{simple}, then it is determined. If by `nice' we mean closed, then this is just GST.

\underline{Next goal}: Define a topology on $M^\lom$.

Let's focus on the case $M = \N$, or even $M = 2 = \{0,1\}$.

\begin{defin*}[Baire Space]
    If $x,y \in \om^\om$, we can define
    \begin{align*}
        d(x,y) \coloneqq \left\lbrace \begin{array}{cc} 0\quad & \textrm{ if }x = y\\ 2^{-m} & \quad \textrm{ if }x\restriction m = y\restriction m\textrm{ and }x(m)\ne y(m) \end{array}\right.
    \end{align*}
    which is a metric on $\om^\om$.

    What are the open balls for this metric? Let $\eps = 2^{-n}$:
    \begin{align*}
        B_\eps(x) &= \{y\in \om^\om : d(x,y) < \eps\}\\
        &= \{y; y\restriction (n+1) = x \restriction (n+1)\}
    \end{align*}
    In particular, the open balls are determined by finite sequences.

    If $s \in \om^\om$, write:
    \begin{align*}
        [s]\coloneqq \{x \in \om^\om ;s \subset x\}
    \end{align*}
    So $B_{2^{-n}}(x) = [x\restriction (n+1)]$. Thus the topology of the metric space is generated by the basic open sets $\{[s];s \in \om^\om\}$.

    This topological space on $\om^\om$ is called \undf{Baire space}. If we restrict to $2^\om$, then it is called \undf{Cantor space}.
\end{defin*}

If you think of $\om^\om$ as $$\prod_{i\in \om}Y_i$$ with $Y_i = \om$, and $2^\om$ as $$\prod_{i\in \om}Y_i$$ with $Y_i = 2 = \{0,1\}$, then Baire space is just the product topology on $\prod_{i\in \om}X_i$ with the discrete topology on $\om$, and Cantor space is the product topology on $\prod_{i\in \om}Y_i$ with the discrete topology on 2.

Tychonoff implies that Cantor space is compact, but Baire space is not. Indeed, the latter can even by seen with Tychonoff:
\begin{align*}
    \om^\om = \bigcup_{m\in \om}[<m>]
\end{align*}
Since $[<m>] = \{x;x(0)=m\}$, this union is disjoint so this open cover clearly has no finite subcover, and tells us that Baire space is (very) disconnected.

Next time, we show that $A = [T] \subset \om^\om \iff A$ is closed in Baire space.

\marginpar{Lecture 7}

We now study the Baire space and Cantor space in detail. Firstly, consider convergence:

\begin{align*}
    x_n \ra x &\iff \forall \eps\ \exists N\ \forall x > N\ d(x_n,x) < \eps\\
    &\iff \forall m\ \exists N \ \forall n>N\ d(x_n,x) < 2^{-m} \\
    &\iff \forall m\ \exists N\ \forall n > N\ x_n\restriction m  = x\restriction m
\end{align*}
If $A\subset \om^\om$, we can define
\begin{align*}
    T_A \coloneqq \{x\restriction n; x \in A, n \in \N\}
\end{align*}
\underline{Observe}: $A \subset [T_A]$, since $x \in A \implies x\restriction n \in T_A\textrm{ for all }n\implies x \in [T_A]$.

\begin{remark*}[Proposition]\emph{
$[T_A]$ is the closure of $A$, \it{i.e.} $\{x;\exists(x_n)\textrm{ with }x_n\in A\textrm{ and }x_n\ra x\}$
}
\end{remark*}
\begin{proof}
    Suppose $x_n \in A$ and $x_n\ra x$. By our characterisation of convergence, this means that $x\restriction k = x_n\restriction k$ for some $n$, so $x\restriction k \in T_A$. Since $k$ was arbitrary, we have that $x \in [T_A]$.

    Conversely, suppose that $x \in [T_A]$. For every $k \in \N$, $x\restriction k \in T_A$, so there is some $x_k \in A$ such that $x\restriction k = x_k\restriction k$. Then again by the characterisation of convergence, we have that $x_k \ra x$. So $x$ is in the closure of $A$.
\end{proof}
\begin{remark*}[Corollary]
    $A\subset \om^\om$ is closed $\iff (\exists T)(A = [T])$ (we know that $T\coloneqq T_A$ does it).

    This is sometimes known as the \undf{tree representation theorem for closed sets}.
\end{remark*}

Some more topological properties:

Basic open sets are $[s] = [T_s]$, where $T_s \coloneqq \{t;s\subset t\textrm{ of }t\subset s\}$. So basic open sets are closed; we call these \undf{clopen}. Spaces like this are called \undf{zero-dimensional}

If $x \in \om^\om$, then $\{x\} = [T_x]$, with $T_x \coloneqq \{x\restriction n;n\in \N\}$. So singletons are closed, and not open.

We can easily see that this set is \undf{Hausdorff}: if $x \ne y$, find $n$ such that $x\restriction n \ne y \restriction n$. Then $x \in [x\restriction n]$, $y \in [y\restriction n]$, but $[x\restriction n]\cap[y\restriction n] = \emptyset$.

\underline{Continuous Functions}: [proof on ES\#2] If $g:\om^\lom \ra \om^\lom$ such that:
\begin{enumerate}
    \item $g$ is order-preserving, \it{i.e.} $s\subset t \ra g(s)\subset g(t)$
    \item $g$ is ``unbounded'', \it{i.e.} if $x \in \om^\om$, then $\lh(g(x\restriction n))\ra\infty$
\end{enumerate}
then define $$\hat{g}(x) \coloneqq \bigcup_{n\in \N}g(x\restriction n)$$ which is a function $\om^\om$.

\begin{remark*}[Proposition]
    $f:\om^\om$ is continuous iff there is $g:\om^\lom \ra \om^\lom$ with 1 \& 2 satisfied such that $f = \hat{g}$
\end{remark*}

The \undf{rule of thumb} here is that $f$ is continuous iff in order to determine $f(x)(k)$ you only need $x\restriction n$ for some finite $n$.

Now consider functions from $(\om^\om)^2$ to $\om^\om$ and vice versa and $\om^\om$ to $\om^\om$:
\begin{itemize}
    \item $x \mapsto x_\I$
    \item $x \mapsto x_{\II}$
    \item $(x,y)\mapsto x\ast y$
    \item $x \mapsto (x_\I,x_\II)$
\end{itemize}
These are all continuous. Moreover, we see that $(\om^\om)^2$ and $(\om^\om)$ are homeomorphic by $(x,y)\mapsto x\ast y$ with inverse $x\mapsto (x_\I,x_\II)$. This is unusual, and this phenomenon may explain the name \it{zero-dimensional}. [Similarly, $(\om^\om)^k \cong (\om^\om)^\ell$ for any $k,\ell > 0$]. 

\begin{theorem*}
    Baire space is homeomorphic to $\R\backslash \Q$.
\end{theorem*}
[The proof uses continued fractions: if $x \in \R\backslash \Q$, then there is a sequence $a_i \in \Z^\om$ such that $x = [0;a_0,a_1,a_2,\dots]$.]

So while topologically quite different from $\R$, Baire space is set-theoretically very close to $\R$: many set theoretic properties/proofs depend only on cardinality, and we have only removed a countable (dense) subset.

\begin{remark*}[Example]
    ES\#1 (4) has choice principles $\textrm{AC}_X(Y)$. These are invariant under replacing $X$ or $Y$ with $X',Y'$ such that $X$ is in bijection with $X'$ and $Y$ is in bijection with $Y'$.

    In particular,
    \begin{align*}
        \textrm{AC}_\om(\R) &\iff \ac_\om(\om^\om)\\
        &\iff \ac_\om(2^\om)\\
        &\iff \ac_\om(X)
    \end{align*}
    where $X$ is any set in bijection with $\R$.
\end{remark*}

In set theory, we often refer to elements of $2^\om$ or $\om^\om$ as ``reals'' and abuse the notation by sometimes writing $\R \coloneqq \om^\om$.

We now repeate our motto/hope: if $A$ is ``simple'', then $A$ is determined.

To make precise what `simple' means, we need a complexity hierarchy on Baire space:

\begin{defin*}[Borel Hierarchy]
    Let $X$ be any topological space. We then define \undf{[boldface] sigma zero one} as
    \begin{align*}
        \bm{\Sigma}_{0}^{1}\coloneqq \{A\subset X; A\textrm{ is open in }X\}
    \end{align*}
    Then if $\bm{\Sigma}^0_\alpha$ is defined, we define
    \begin{align*}
        \bm{\Pi}^0_\alpha \coloneqq \{X\backslash A; A \in \bm{\Sigma}^0_\alpha \}
    \end{align*}
    If $\alpha$ is an ordinal and for all $\gamma < \alpha$, $\bopi^0_\gamma$ is defined, then
    \begin{align*}
        \bosig^0_\alpha \coloneqq \{A; \exists (A_n)\textrm{ such that }\forall n\ A_n\in \bigcup_{\gamma<\alpha}\bopi^0_\gamma\textrm{ and }A = \bigcup_{n\in\N}A_n\}
    \end{align*}
    We also have
    \begin{align*}
        \bodel^0_\alpha \coloneqq \bosig^0_\alpha \cap \bopi^0_\alpha
    \end{align*}

    So we get the $\bosig$s from countable unions, and the $\bopi$s from complements.
\end{defin*}

\underline{Properties}: By definition, $\bodel ^0_\alpha \subset \bosig^0_\alpha,\bopi^0_\alpha$.

Moreover, by definition $\alpha \le \beta \implies \bosig^0_\alpha \subset \bosig^0_\beta$. This also gives us that $\bopi^0_\alpha \subset \bopi^0_\beta$.

To see the full structure of the hierarchy, we need to show that for $\alpha < \beta$, $\bopi^0_\alpha \subset \bosig^0_\beta$ (equivalently, $\bosig^0_\alpha \subset \bopi^0_\beta$). This follows from the definition of $\sigma^0_\beta$ by letting $A_n\coloneqq A$, so $\bigcup_{n\in \N}A_n = A$.

\underline{Question}: When does the Borel hierarchy terminate? This will, of course, depend on the topological space.

\marginpar{Lecture 8}

\begin{defin*}[$G_\delta$ space]
    A topological space is called a \undf{$G_\delta$ space} if $\bopi^0_1 \subseteq \bopi^0_2$. That is to say, `every closed set is a $G_\delta$ set'.
\end{defin*}

\underline{Note}: every metric space is a $G_\delta$ space.

For Cantor and Baire space, we proved that there is a countable topology base of clopen sets, and this implies being in $G_\delta$.

\begin{remark*}[Proposition]
    If $X$ has a countable, clopen topology base then $X$ is $G_\delta$.
\end{remark*}
\begin{proof}
    Let $F\subset X$ be closed, and let $G = X\backslash F$; $G$ is open. For every $x \in G$, find $B_x$ in the topology base such that $x \in B_x\subset G$. Since $B_x$ is clopen, $X\backslash B_x$ is also open.

    Now consider $\{B_x ; x \in G\}$. This is countable, since the basis is countable, so write it as $\{B_n; n\in \N\}$. Then $F = \bigcap_{n\in\N}X\backslash B_n \in \bopi_2^0$.

    Note that since countability implies wellorderability, no choice is needed.
\end{proof}

This is slightly irrelevant to what we wanted to do, but it felt like it was worth covering.

In principle, the Borel Hierarchy is defined on arbitary spaces, but it of course cannot go on forever; it must terminate on some ordinal, and this may differ between spaces.

\underline{Question}: By the Axiom of Replacement, the Borel Hierarchy termiantes at some ordinal $\alpha$ [\it{i.e.}, $\bosig_\alpha^0 = \bopi_\alpha^0$]; what can we say about $\alpha$?

The upper bound that we get by just using Replacement is pretty bad.

\begin{remark*}[Observations]\ 
    \begin{enumerate}
    \item If $X$ is discrete, then every subset of $X$ is clopen, so $$\bodel_1^0 = \bosig_1^0 = \bopi_1^0$$
    \item If singletons are closed and $X$ is countable, then $$\bodel_2^0 = \bosig_2^0 = \bopi_2^0$$
    
    [If $A = \{x; x \in A\}$ is countable, then $A = \bigcup_{x\in A}\{x\}$] 
    \end{enumerate}
\end{remark*}

We can obtain a better upper bound than by the cardinality of $X$.

\begin{remark*}[Proposition (ZFC)]
    \emph{
    For arbitrary $X$, $\bodel_{\aleph_1} = \bosig_{\aleph_1} = \bopi_{\aleph_1}$.
    }
\end{remark*}
\begin{proof}
    It is enough to show that $$ \bosig_{\aleph_1}^0 = \bigcup_{\alpha < \aleph_1} \bopi_{\alpha}^0$$
    The $\supseteq$ inclusion is clear. For the other direction, consider the following.

    If $A \in \bosig_{\aleph_1}^0$, there are $A_n$ such that $A = \bigcup_{n\in\N}A_n$ and $\alpha_n < \aleph_1$ such that $A_n \in \bopi_{\alpha_n}^0$. Since $\aleph_1$ is a regular cardinal, every countable subset $A \subset \aleph_1$ is bounded, \it{i.e.} there is $\beta < \aleph_1$ such that $A\subset \beta$.

    So we then look at the countable subset $\{\alpha_n ; n \in \N\}\subset \aleph_1$, and find for it a countable bound $\beta < \aleph_1$. Then $\{\alpha_n ; n\in\N\} \subset \beta$. Then for all $n$, $A_n \in \bigcup_{\alpha < \beta}\bopi_\alpha^0$. Hence $A \in \bosig_{\beta+1}^0\subset \bopi_{\beta+2}^0$. But $\beta+2$ is still countable, so $A$ is in the union as in the claim.

    The AC required was in showing that $\aleph_1$ is regular.
\end{proof}

Hence the height of the Borel hierarchy (in ZFC) is $1 \le \beta \le \aleph_1$.

\begin{theorem*}[ZFC]
    If $X$ is Cantor space, Baire space or $\R$, then the height of the Borel hierarchy is $\aleph_1$.

    [This is not just the case for these spaces; in general, this holds if $X$ is an uncountable Polish space.]
\end{theorem*}

This means that if $\alpha < \aleph_1$, then $\bosig_{\alpha}^0 \ne \bopi_{\alpha}^0$. The proof of the theorem uses the \undf{method of universal sets}.

\begin{defin*}[Pointclass]
    A \undf{pointclass} is an operation that assigns to each topological space $X$ a set of subsets of $X$.
\end{defin*}
\begin{remark*}[Examples]\ 
    \begin{itemize}
        \item ``open''/$\bosig_1^0$
        \item ``closed''/$\bopi_1^0$
        \item $\bosig_\alpha^0/\bopi_\alpha^0/\bodel_\alpha^0$
    \end{itemize}
\end{remark*}
If $\Gamma$ is a pointclass, we define ${\breve \Gamma}$ by $$\bg(X) \coloneqq \{X\backslash A; A \in \bg(X)\}$$ called the \undf{dual pointclass} of $\Gamma$, pronounced ``Gamma dual'', and $\Delta_\Gamma$ by $$\Delta_\Gamma(X)\coloneqq \Gamma(X) \cap \bg (X)$$ called the \undf{ambiguous pointclass} of $\Gamma$.

For example, $\bopi_\alpha^0$ is $\br{\bosig_\alpha^0}$, and $\bodel_\alpha^0$ is  $\Delta_{\bosig_{\alpha}^0}$ (and $\Delta_{\bopi_\alpha^0}$).

\subsection*{Closure Properties of Pointclasses}

\begin{center}
    \begin{tabular}{cc|c}
        & YES & NO\\
        closed under finite unions & $\bosig_\alpha^0,\bopi_\alpha^0,\bodel_\alpha^0$& \\
        closed under finite intersections &$\bosig_\alpha^0,\bopi_\alpha^0,\bodel_\alpha^0$ & \\
        closed under countable unions & $\bosig_{\alpha}^0$ & $\bopi_\alpha^0,\bodel_\alpha^0$ \\
        closed under countable intersections &$\bopi_\alpha^0$ &$\bosig_\alpha^0,\bodel_\alpha^0$ \\
        closed under complements &$\bodel_\alpha^0$ &$\bosig_{\alpha}^0,\bopi_\alpha^0$ \\
    \end{tabular}
\end{center}
Here, `no' means `if the space is Baire space and $\alpha$ is countable then no' (\it{i.e.} correctly chosen $\alpha$ and space $X$).

These properties are entirely local; we have another closure property that links the meaning of pointclasses in different spaces.

We say that $\Gamma$ is \undf{closed under continuous pre-images} if whenever $f : X \ra Y$ is continuous and $A \in \Gamma(Y)$, then $f^{-1}[A] \in \Gamma(X)$. This property is also called \undf{boldface}; this is silly because the notion has been named after the notation used. A simple inductive argument shows that this is consistent with the `boldface' notation we have been using for the Borel classes $\bosig_\alpha^0$, $\bopi_\alpha^0$, $\bodel_\alpha^0$.

Similarly, $\Gamma$ is \undf{closed under continuous images} if whenever $f : X \ra Y$ is continuous and $A \in \Gamma(X)$ then $f[A] \in \Gamma(Y)$.

We're going to talk more about continuous images in Lectures 9 \& 10.

By restricting our focus to boldface pointclasses, we eliminate any pathological issues whereby pointclasses among different spaces do not correspond in any way.

\begin{defin*}[]
    Let $X,Y$ be topological spaces, and $\Gamma$ a pointclass. A set $U \subset X \times Y$ is called \undf{$X$-universal for $\Gamma(Y)$} if:
    \begin{enumerate}
        \item $U \in \Gamma(X\times Y)$ (under the product topology)
        \item For every $A \in \Gamma(Y)$, there is some $x \in X$ such that $U_x = A$, where $U_x = \{y \in Y ; (x,y)\in U\}$ is the `section of $U$ at $x$'.
    \end{enumerate}
\end{defin*}

\begin{remark*}[Lemma]
    \emph
    {
        Suppose $U$ is $X$-universal for $\Gamma(X)$ and $\Gamma$ is boldface. Then $\Gamma(X) \ne \bg(X)$ (\undf{non-selfdual}).
    }
\end{remark*}
\begin{proof}
    Consider $X\times X\backslash U \in \bg(X\times X)$, and consider $x\mapsto (x,x)$ the diagonal map $X \ra X \times X$, which is continuous. Let $D\coloneqq \{x;(x,x)\not\in U\} \in \bg(X)$.

    Assume that $\Gamma(X) = \bg(X)$. Then find $d \in X$ such that $D = U_d$. Then
    \begin{align*}
        d \in D &\iff (d,d) \in U\\
        &\iff d \in U_d\\
        &\iff d\not\in D
    \end{align*}
    which is a contradiction.
\end{proof}
This is once again a standard diagonalisation proof that one often encounters in Logic \& Set Theory, \it{e.g.} uncountability of the reals, the halting problem, \it{etc}...

In Lecture 9, will prove that for $\alpha < \aleph_1$, $\bosig_\alpha^0$ has an $\om^\om$-universal set. Then by the Lemma, this implies our theorem.

\marginpar{Lecture 9}

We will use AC fairly liberally, but keeping track of our uses of it when we are done.

\begin{proof} (Proof of Theorem).
    
    Proof by induction:
    \begin{itemize}
        \item Induction base $\bosig_1^0$
        \item Complementation step $\bosig \ra \bopi$
        \item Countable union step $\bopi \ra \bosig$
    \end{itemize}
    This will be done in three lemmas.

    \begin{remark*}[Lemma 1]\emph{
        There is an $\om^\om$-universal set for $\bosig_1^0(\om^\om$)
    }   
    \end{remark*}

    \underline{Proof of Lemma 1}: an open set is an arbitrary union of basic open sets. Since there are only countably many basic open sets, we can say this is in fact a countable union. So we can write $$P = \bigcup_{i\in I}[s_i]$$ where $I$ is a countable index set.

    Let $\{s_i ; i \in \N\}$ be your favourite enumeration of $\om^\lom$. Thus $\{[s_i];i\in\N\}$ is an enumeration of the basic open sets. Then define $$U\coloneqq \{(x,y); \exists i \in \N, (x(i)\ne 0)\land (s_i\subset y)\}$$

    This is univeral for $\bosig_1^0(\om^\om)$:
    \begin{enumerate}
        \item $U$ is open: if $(x,y)\in U$ then there is $i\in \N$ with $x(i)\ne 0$ and $s_i\subset y$. Define $t = x\restriction (i+1)$. Then $[t,s_i]\subset U$. Note that we haven't exactly defined basic open sets on Baire space squared, but the notation is intuitive/obvious.
        \item If $P$ is open, then let
        \begin{align*}
            x(i)\coloneqq \left\lbrace \begin{array}{cc} 1 & \textrm{ if }[s_i]\subset P\\ 0 & \textrm{ if }[s_i]\not\subset P\end{array}\right.
        \end{align*}
        Then $P = \bigcup_{x(i)\ne 0}[s_i]$. Thus $P = U_x$.
    \end{enumerate}
    Wo we are done. \qedhere

    \begin{remark*}[Lemma 2]\emph{
        If $U\subset X\times X$ is $X$-universal for $\Gamma(X)$, then $X\times X \backslash U$ is $X$-universal for $\bg(X)$.
    }
    \end{remark*}
    \underline{Proof of Lemma 2}: Obvious.

    \begin{remark*}[Lemma 3]\emph{
        Let $\lambda < \om_1$. Suppose that for each $\alpha < \lambda$, there is an $\om^\om$-universal set $U_\alpha$ for $\bopi_\alpha^0(\om^\om)$. Then there is an $\om^\om$-universal set for $\bosig_\lambda^0$.
    }
    \end{remark*}
    \underline{Proof of Lemma 3}:

    If $lambda$ is a successor ordinal, \it{i.e.} $\lambda = \mu + 1$, then let $\alpha_n \coloneqq \mu$ for all $n$. If $\lambda$ is a limit, pick a sequence $\alpha_n$ such that $\bigcup \alpha_n = \lambda$.

    Observe that if $A = \bigcup_{n\in\N}A_n$, where $A_n \in \bigcup_{\alpha < \lambda}\bopi_\alpha^0$, then I find (by ``postponing if necessary'') a sequence $A_n'\in \bopi_{\alpha_n}^0$ such that $$\bigcup_{n\in\N}A_n = \bigcup_{n\in\N}A_n'\qquad (\ast)$$

    To simplify notation, write $U_n \coloneqq U_{\alpha_n}$.

    \underline{Encoding countable sequences of $\om^\om$ by an element of $\om^\om$}:

    Take your favourite bijection $\lceil \cdot , \dot \rceil:\om\times \om \ra \om$. If $x \in \om^\om$ and $n \in \N$, define $(x)_n \in \om^\om$ by $(x)_n(m) \coloneqq x(\lceil n,m\rceil)$. Then $x\mapsto ((x)_n;n\in\N)$ is a bijection between $\om^\om$ and $(\om^\om)^\om$.
    
    Note also that the map $x\mapsto (x)_n$ is continuous, because we only require a finite amount of information to determine what $x$ at $n$ is, and by our earlier characterisation this means we have continuity.

    Now define $$U\coloneqq \{(x,y);\exists n((x)_n,y)\in U_n\}$$

    \underline{Claim}: $U$ is $\om^\om$-universal for $\bosig_\lambda^0$.
    \begin{itemize}
        \item $\overline{U}_n\coloneqq \{(x,y);((x)_n,y)\in U_n\}$ is the pre-image of $U_n$ under the continuous map $x,y\mapsto ((x)_n,y)$, so it's in $\bopi_{\alpha_n}^0$ [closure of the Borel classes under continuous pre-images]. But $U = \bigcup_{n\in\N}\overline{U}_n$, so $U$ is $\sigma_{\lambda}^0$.
        \item Let $A$ be $\sigma_\lambda^0$. By ($\ast$), we find $A_n \in \bopi_{\alpha_n}^0$ such that $A = \bigcup_{n\in\N}A_n$. By universality of $U_n$, we find $x_n$ such that $A_n = (U_n)_{x_n}$. Now fold up the sequence $(x_n;n\in \om)$ into a single element of $\om^\om$ such that $(x)_n = x_n$ for all $n$. We do this by defining $x(\lceil n,m\rceil)\coloneqq x_n(m)$.
        
        Here's the situation:
        \begin{enumerate}
            \item $A_n = (U_n)_{x_n}$ 
            \item $(x)_n = x_n$ 
            \item $A = \bigcup_{n\in\N}A_n$
            \item $U = \{(x,y);\exists n ((x)_n,y)\in U_n\}$
        \end{enumerate}

        \underline{Claim}: $U_x = A$:
        \begin{align*}
            y \in A &\xLeftrightarrow{3} \exists n\ y\in A_n\\
            &\xLeftrightarrow{1}\exists n\ y \in (U_n)_{x_n}\\
            &\xLeftrightarrow{ } \exists n\ (x_n,y)\in U_n\\
            &\xLeftrightarrow{2} \exists n\ ((x)_n,y)\in U_n\\
            &\xLeftrightarrow{4} (x,y)\in U\\
            &\xLeftrightarrow y \in U_x
        \end{align*}
    \end{itemize}
\end{proof}

\begin{remark*}[Corollary] \textbf{Borel Hierarchy Theorem}
\end{remark*}
\begin{proof}
    This is an inductive proof using L1 - L3. [Recursive definition of a $\om^\om$-universal set for $\bosig_{\lambda}^0$ for each $\lambda < \om_1$.]
\end{proof}

\begin{remark*}
    Let's check how much choice we used.

    \underline{Lemma 1} is a ZF theorem.

    \underline{Lemma 2} is a ZF theorem.

    \underline{Lemma 3}: $\lambda \mapsto $ pick $\alpha_n$ such that $\bigcup \alpha_n = \lambda$; if $\lambda$ fixed, no choice is needed. However, after declaring that there \it{exists} a universal set $U_n$ for each $n$, we then picked a concrete example of one for each $n$ to play around with. So this needed choice. Perhaps it would have been more prudent to simply formulate Lemma 3 by just assuming that there exists a choice function \it{only where we need it} for the universal sets. Then we don't need to use the more general Axiom of Choice.

    We also made another choice in the presentation of $A_n$; if $A \in \bosig_\lambda^0$, then $$S_A\coloneqq \{(A_n)_n; A = \bigcup_{n\in\N}A_n\}\ne\emptyset$$ for every such $A$, so we use a choice function for this family too.

    But we are still not done; while we didn't need choice for $\lambda$ fixed in Lemma 3, we have to use Lemma 3 infinitely many times in the proof of the corollary; so we need choice to pick for each limit $\lambda < \om_1$ a sequence $\alpha_n$ in order to apply L3.
\end{remark*}

Back to games.

\underline{Notation}: if $\Gamma$ is a pointclass, write $\Det(\Gamma)$ for ``$\forall A \in \Gamma(\om^\om)$, $A$ is determined''.

G-S proved $\Det(\bopi_1^0)$. The proof also implies $\Det(\bosig_1^0)$.

ES\#1 (11): in general, the class of determined sets is not closed under complementation.

ES\#1 (10): if a pointclass $\Gamma$ is closed under the operation $A\mapsto \{mx;x\in A\}$, then $\Det(\Gamma) \implies \Det(\bg)$. Note that Borel pointclasses are closed under this operation.

Wolfe (1955) proved $\Det(\bosig_2^0)$; Davis (1964) proved $\Det(\bosig_3^0)$; Paris (1972) proved $\Det(\bosig_4^0)$.

Friedman proved that you cannot prove $\Det(\bosig_5^0)$ without using iterations of the power set axiom. This paved the way for Martin (1975) to prove that \it{all} Borel sets are determined.

\marginpar{Lecture 10}

The Borel sets, from the point of view of the ordinary analyst, look extremely complicated already. So while they are not all the sets, you might expect that all `reasonable' (for some definition of reasonable) sets \it{are} Borel, so a proof that every Borel set has some property is in practice equivalent to proving that all sets have that property. But this might not be the case, which begs that question: what \it{is} the size of the set of Borel sets?

Denote the set of all Borel sets by $\mathcal{B}$. Clearly then $|\mathcal{B}| \le 2^{2^{\aleph_0}}$, but we can show that it is in fact much smaller:

\begin{theorem*}\textbf{(ZFC)}
    $$|\mathcal{B}| = 2^{\aleph_0} < 2^{2^{\aleph_0}}$$
\end{theorem*}
\begin{proof}
    Since $\{x\}\in \mathcal{B}$ for every $x \in \om^\om$, $x\mapsto \{x\}$ is an injectino from $\om^\om$ into $\mathcal{B}$. So $2^{\aleph_0} \le |\mathcal{B}|$.

    \underline{Upper Bound}: Proof by induction.

    We prove that $|\bosig_\alpha^0| = 2^{\aleph_0}$ for all $\alpha$. This then implies the Theorem:

    $$ |\mathcal{B}| = \left|\bigcup_{\alpha < \om_1} \bosig_{\alpha}^0 \right| \le \aleph_1 \cdot 2^{\aleph_0} = 2^{\aleph_0}$$

    \underline{Base case}: for $\bosig_{1}^{0}$, every open set is of the form $$\bigcup_{i\in I}[s_i]$$ where $I\subset \N$ and $s_i$ is our enumeration of $\om^\lom$. So $I\mapsto \bigcup_{i\in I}[s_i]$ is a surjection from $\mathcal{P}\N$ onto $\bosig_{1}^{0}$. Hence $|\bosig_1^0| \le 2^{\aleph_0}$.

    Going from $\bosig \ra \bopi$ is easy , since $A\mapsto \om^\om \backslash A$ is a bijection between $\bosig_\alpha^0$ and $\bopi_\alpha^0$, hence $|\bosig_\alpha^0| = |\bopi_\alpha^0|$.

    From $\bopi \ra \bosig$: Suppose for each $\alpha < \lambda$, we have $|\bopi_\alpha^0| \le 2^{\aleph_0}$. Using AC, pick surjections $s_\alpha : \om^\om \twoheadrightarrow \bopi_\alpha^0$. Define surjection $S$ by
    \begin{align*}
        S : \lambda^\om \times \om^\om &\ra \bosig_\lambda ^0 \\
        ((\alpha_i;i\in \om),x)&\mapsto \bigcup_{i\in \N}S_{\alpha_i}((x)_i)
    \end{align*}
    This clearly is a surjection. Since $\lambda$ is countable, $|\lambda^\om| = 2^{\aleph_0}$. So $|\lambda^\om \times \om^\om| = 2^{\aleph_0}$.
\end{proof}

\begin{remark*}
    We used AC in this proof, and that is not avoidable; consider the \undf{Feferman-Levy Model $\M$}, which models the theory `ZF + ``$\R$ is a countable union of countable sets'''.

    In the FL model, $\R = \bigcup_{n\in\N}A_n$. So if $X \subset \R$, then $X = \bigcup_{n\in\N} A_n\cap X$. Since $A_n\cap X \subset A_n$, this is still countable. Hence, in this model, every subset of the reals is a countable union of countable sets. Now, all countable sets are Borel since they are a countable union of closed sets. But the Borel sets are closed under countable unions, so this means that \it{every} set is Borel. In particular, $|\mathcal{B}| = \mathcal{P}\R$. So we needed AC after all.
\end{remark*}

So the Borel sets are in fact only a very small part of the collection of all sets in Baire space. What else is out there?

\undf{The famous mistake of Henri Lebesgue}.

Measures are defined on $\sigma$-algebras $\mathcal{A}$. In general, $\mathcal{A} \ne \mathcal{P}\R$. [Involves AC; Vitali set is a non-Lebesgue-measurable set that can be constructed AC.]

The smallest $\sigma$-algebra, namely the Borel $\sigma$-algebra (the smallest one containing the open sets), is the minimal setting for measures, and Lebesgue made an argument for this:

\begin{remark*}[Lebesgue]
    \emph{
        ``All sets we care about are Borel.''
    }
\end{remark*}

He believed that if $f:\R \ra \R$ is continuous and if $A\subset \R$ is Borel, then $f[A]$ is Borel. However, this is in fact FALSE.

The mistake was spotted in 1917, when when Suslin proved that there are non-Borel sets which are continuous images of Borel sets - these are called \undf{analytic sets}. We can in fact prove this. We don't even need the continuous functions to be complicated at all (projection).

\begin{defin*}[Projection]
    Let $A\subset X^{n+1}$. Then we call $$B = \{(x_1,\dots,x_n); \exists x \in X (x,x_1,\dots,x_n)\in A\}$$ the \undf{projection} of $A$. Write $pA = B$.

    The map $$\pi : (x,x_1,\dots,x_n) \mapsto (x_1,\dots,x_n)$$ is clearly continuous, so $pA = \pi[A]$, and thus projections are indeed special cases of continuous images.
\end{defin*}

\begin{defin*}
    Let $\Gamma$ be a pointclass. We define $\exists^{\om^\om} \Gamma$ a pointclass by:
    \begin{align*}
        \exists^{\om^\om}\Gamma(X) \coloneqq \{pA; A \in \Gamma(\om^\om\times X)\}
    \end{align*}

    We say $\Gamma$ is \undf{closed under projections} if $\eomg \subset \Gamma$.
\end{defin*}

Suslin's Theorem now says: ``the pointclass BOREL is not closed under projections''.

\begin{defin*}[The Projective Hierarchy]
    We start by defining $\bopi_0^1(\om^\om) = \bopi_1^0(\om^\om)$. [Note that if $X \ne \om^\om$, this won't necessarily be the right starting point; we might want to use $\bopi^0_2$ instead.]

    We then define 
    \begin{align*}
        \bosig_{n+1}^1 &\coloneqq \eom \bopi_n^1\\
        \bopi_{n+1}^1 &\coloneqq \breve{\bosig}_{n+1}^1
    \end{align*}

    $\bosig_1^1$ is called the \undf{analytic sets}, and $\bopi_1^1$ is called the $\undf{co-analytic sets}$.

    We also define
    \begin{align*}
        \bodel_n^1(X) \coloneqq \bosig_n^1 (X) \cap \bopi_n^1(X)
    \end{align*}
\end{defin*}

Lebesgue believed that all of these were contained in $\mathcal{B}$; he was wrong. We already have the technique to prove this, which is the technique of universal sets.

\begin{theorem*}
    The projective hierarchy does not collapse.
\end{theorem*}
\begin{proof}
    Using the technique of universal sets. we know that $\bopi_0^1 = \bopi_1^0$ has a universal set. We know that if $\bosig_n^1$ has a universal set, then $\bopi_n^1$ has a universal set. So all that is left to show is that if $V$ is universal for $\bopi_n^1$, then we can find $U$ universal for $\bosig_{n+1}^1$.

    Let $V\subset \om^\om \times \om^\om \times (\om^\om)^k$ be universal fotr $\bopi_n^1$. Define $$U \coloneqq \{(0,\vec{x})\in \om^\om \times (\om^\om)^k; \exists v \in \om^\om (u,v,\vec{x}) \in V\}$$
    It is clear that $U \in \bosig_{n+1}^1(\om^\om \times (\om^\om)^k)$, and the same $x$ that is the code for the set $A$ [in $V$] is the code for $pA$ in $U$.
\end{proof}

\begin{remark*}[Proposition]
\emph{
    Every Borel set is $\bosig_1^1$
}    
\end{remark*}
\begin{proof}
    (ES\#2)
\end{proof}

\begin{remark*}[Corollary]
    \emph{
        Suslin's Theorem: there is a $\bosig_1^1$ set that is not Borel.
    }
\end{remark*}

\underline{Question}: Does $\Det(\bosig_1^1)$ hold? Perhaps $\bosig_2^1$, or $\bosig_3^1$ \it{etc...}? These questions are much more interesting set-theoretically. These are not ZFC theorems, but are closely connected to:
\begin{enumerate}
    \item Large Carindal Axioms
    \item Definability of wellorders of $\om^\om$
\end{enumerate}
These are the two topics that will cover the remainder of this course.

\marginpar{Lecture 11}

\section*{Applications of Infinite Games}

\subsection*{The Continuum Problem}

The first, and arguably the most significant application, is to the famous Continuum Hypothesis: $2^{\aleph_0} = \aleph_1$.

In ZFC, there is an equivalent formulation of the problem, which is that every uncountable set of reals $(A\subset \om^\om)$ is in bijection with the set of all reals. $(\ast\ast)$

Note that the first formulation implies that $\om^\om$ is wellordered, whereas the second one does not. So we may think of the second formulation as the choice-free version of CH.

\begin{defin*}[Perfect Set Property]
    We say that a set $A\subset \om^\om$ has the \undf{perfect set property} if it is either countable or there is a perfect tree $T$ such that $[T]\subset A$
\end{defin*}
\begin{remark*}
    Since every perfect set $[T]$ has size $2^{\aleph_0}$, having the perfect set property implies not being a counterexample to $(\ast\ast)$.
\end{remark*}
\begin{theorem*}[Cantor-Bendixson]
    Every uncountable closed set of reals contains a non-empty perfect subset.

    Equivalently, every closed set has the perfect set propety.
\end{theorem*}
\begin{proof}[Sketch.]
    Take $A\subset \om^\om$, remove isolated points to obtain $A' = \{x\in A;x\textrm{ is not isolated}\}$, also known as the \undf{Cantor-Bendixson derivative}. But removing isolated points might create new isolated points, so we need to repeat this:

    $A_0 = A$, $A_{\alpha+1} = (A_\alpha)'$, $A_\lambda \coloneqq \bigcap_{\alpha < \lambda}A_{\alpha}$. Since $\om^\om$ is second countable, each $A_{\alpha+1} \backslash A_\alpha$ is countable. There is then a fixed point $A_\beta = A_{\beta+1}$, with $\beta < \aleph_1$.
    
    \underline{Case 1}: $A_\beta = \emptyset$. Then $A$ was countable.

    \underline{Case 2}: $A_\beta \ne \emptyset $ and is perfect.
\end{proof}
This was one of the first proofs requiring a transfinite recursion, and as such was very important for the development of ordinals and set theory.

\begin{defin*}
    If $\Gamma$ is a pointclass, write $\psp(\Gamma)$ to mean ``for every $A \in \Gamma$, $A$ has the p.s.p.''
\end{defin*}

Cantor-Bendixson now says $\psp(\bopi_1^0)$.

\begin{defin*}
    Write $\psp$ for ``every set has the p.s.p.''
\end{defin*}

\underline{Observation}: $\psp \implies \textrm{CH}$. [In the case of $(\ast\ast)$.]

\begin{theorem*}[Bernstein]
    $$\ac \implies \neg\psp$$
\end{theorem*}
[So, this approach is not going to solve the Continuum Problem, unless you are willing to give up AC.]

We already saw the proof of this theorem when we constructed a non-determined set using AC; just replace the notion of `strategic tree' with `perfect tree'.

\begin{theorem*}[Hausdorff]
    PSP(Borel)
\end{theorem*}

We are going to prove Hausdorff's Theorem from Borel determinacy, using games.

\begin{theorem*}
    If $\Gamma$ is a boldface pointclass, then $\Det(\Gamma)\implies \psp(\Gamma)$.
\end{theorem*}
\begin{proof}
    For technical simplicity, we do this on Cantor space.

    So fix $A\subset 2^\om$. We define the \undf{asymmetric game $G^\ast(A)$} played with moves in $2^{\lom}$ by player I and moves in $2$ by player II:

    \gamec{s}{b}

    If $z = (s_0,b_0,s_1,b_2,\dots) \in (2^\lom \cup 2)^\om$, we form $z^\ast = s_0b_0s_1b_1\dots \in 2^\om$ by concatenation. We then say player I wins if $z^\ast \in A$.

    \underline{Notation}: a \it{position} in the game has the form $p = (s_0,b_0,\dots,s_n/b_n)$ (depending on the parity of $n$). If $\tau$ is a strategy for II and $(s_0,\dots,s_n)$ is a sequence of elements of $2^\lom$, then $t\ast \tau$ is the position obtained by playing $\tau$ against $t$.

    If $p = (s_0,b_0,\dots,s_n,b_n),s\in 2^\lom$, $\tau$ is a strategy for II, write $ps\tau$ for the position obtained by playing $s$ \& $\tau$.

    \underline{Claim}: If $A \in \Gamma$ and $\Det(\Gamma)$ holds then $G^\ast(A)$ is determined.

    \underline{Proof}: Find $A^\ast \subset \om^\om$ such that $G(A^\ast)$ and $G^\ast(A)$ are the same game and $A^\ast$ is a continuous pre-image of $A$.

    Fix your favourite bijection $\pi : 2^\lom \ra \om$, and define for $x \in \om^\om$:
    \begin{align*}
        x^\pi(n) \coloneqq \left\lbrace \begin{array}{cc}\pi(x(n))&\textrm{ if }n\textrm{ is even}\\ x(n)\mod 2&\textrm{ if }n\textrm{ is odd}\end{array}\right.
    \end{align*}
    Then $x^\pi$ is a run of $G^\ast(A)$. Define $A^\ast \coloneqq \{x\in\om^\om;(x^\pi)^\ast \in A\}$. Then $A^\ast$ is the preimage of the map $x\mapsto (x^\pi)^\ast$ of $A$. But this map is continuous, since we need only finite information to determine $x^\pi(n)$ from $x$. \qedsymbol

    We now continue the proof with some claims.

    \underline{Claim 1}: If player I has a winning strategy in $G^\ast(A)$, then $A$ contains a perfect subset.

    \underline{Proof of Claim 1}: By construction, a strategic tree for $G^\ast(A)$ is a perfect tree on $2$. \qedsymbol

    \underline{Claim 2}: If II has a winning strategy, then $A$ is countable.

    Let $p$ be a position, $x \in 2^\om$ and $\tau$ any strategy for II in $G^\ast(A)$. We say that $p$ is \undf{$\tau$-decisive for $x$} if, for $p = (s_0,b_0,\dots,s_n,b_n)$, $p^\ast = s_0b_0\dots s_nb_n \in 2^\lom$, we have $p^\ast \subset x$ but for all $ s \in 2^\lom$, $(ps\tau)^\ast \not\subset x$.
    [So $p$ is the `maximal' position consistent with $\tau$ \& $x$.] \qedsymbol

    \underline{Subclaim 2a}: If $\tau$ is winning for II, then for each $x\in A$ there is a $\tau$-decisive position $p$ for $x$.

    \underline{Proof of Subclaim 2a}: Suppose not. Then for every $p$ we find $s$ such that $ps\tau^\ast \subset x$. Recursively define a sequence $s = (s_i;i\in \N)$ such that $s_{i+1}$ is the witness that $(s_0,\dots,s_i)\ast \tau$ is not $\tau$-decisive. Then $s\ast \tau$ is a sequence that is entirely consistent with $\tau$, and $(s\ast \tau)^\ast = x \in A$. So $\tau$ is not winning. \qedsymbol

    \underline{Subclaim 2b}: Every position $p$ is $\tau$-decisive for at most one $x \in 2^\om$.

    \marginpar{Lecture 12}

    \underline{Proof of Subclaim 2b}: Let $p$ be $\tau$-decisive for $x$ and show that every $x(k)$ is determined uniquely by $p$ and $\tau$.

    By definition $p^\ast \subset x$. If $\ell \coloneqq \lh(p^\ast)$ and $ k < \ell$, then $x(k) = p^\ast(k)$, so determined by $p$.

    Consider now $x(\ell + n)$, where $n\in\N$. We determine this recursively:
    \begin{itemize}
        \item $x(\ell + 0) = x(\ell)$. If $s_0 \coloneqq \emptyset$, then $ps_0\tau^\ast \not\subset x$. $\lh(ps_0\tau^\ast) = \ell + 1$. So $ps_0\tau^\ast(\ell)\ne x(\ell)$. Thus $x(\ell) = 1 - ps_0\tau^\ast(\ell)$ [since we are in Cantor space].
        
        This determines $x(\ell)$ by just $p,\tau$.

        \item Assume we know $x(\ell + 0),\dots,x(\ell + n - 1)$ and determine $x(\ell + n)$.
        
        Let $s_n\coloneqq (x(\ell + 0),\dots,x(\ell + n - 1))$. So $\lh(s_n) = n$. Consider $ps_n\tau$. By decisiveness, we have $ps_n\tau^\ast\not\subset x$, of length $\ell + n + 1$. So by choice of $s_n$, we must have that $ps_n\tau^\ast(\ell + n)\ne x(\ell + n)$. Hence $x(\ell + n) = 1 - ps_n\tau^\ast(\ell + n)$.

        So, once more, $x(\ell + n)$ is determined just by $p$ and $\tau$. \qedsymbol
    \end{itemize}

    This concludes Subclaim 2b, hence Claim 2, and hence the theorem.
\end{proof}

\begin{remark*}[Corollary]
    \emph{
        ZFC $\proves$PSP(Borel)
    [Our proof is modulo Borel Determinacy, which we did not prove.]}
    \emph{Moreover}:
    \begin{itemize}
        \item $\Det(\bopi_1^1) \implies \psp(\bopi_1^1)$
        \item $\Det(\bopi_n^1) \implies \psp(\bopi_n^1)$
    \end{itemize}
\end{remark*}

This yields necessary conditions for axioms of determinacy in the projective hierarchy: if $\Det(\bopi_2^1)$, then we can't have $\bopi_2^1$ sets violating CH.

These conditions are non-trivial:

\begin{theorem*}[G{\"o}del-Addison]
    There is a model of ZFC + $\neg \psp(\bopi_1^1)$
\end{theorem*}
\begin{remark*}
    This is \undf{G{\"o}del's Construcible Universe}, usually denoted by $\mathbf{L}$. The reason for this is that $\mathbf{L}$ has a $\bodel_2^1$ wellorder of $\om^\om$.

    What does that even mean?
    
    If $\le$ is a wellorder of $\om^\om$, then it is a binary relation on $\om^\om$, so $\le \subset \om^\om\times\om^\om$.

    Therefore, it is perfectly reasonable to ask whether $\le \in \bodel^1_2((\om^\om)^2)$
\end{remark*}

Our next goal:

\begin{theorem*}
    If there is a $\bodel_n^1$ wellorder of $\om^\om$, then there is a set in $\bopi^1_n$ without the perfect set property.
\end{theorem*}

\begin{remark*}
    This is not optimal, as the G{\"o}del-Addison theorem shows.
\end{remark*}

Proving this theorem will require:
\begin{enumerate}
    \item a structural analysis of $\bopi^1_1$
    \item a relation between $\bopi^1_1$ and the ordinal $\om_1$
\end{enumerate}

\subsection*{Structure Theory of Co-Analytic Sets}

Tree representation theorem for closed sets:

$ A \in \bopi^0_1 \iff $ there is a tree $T$ such that $A = [T]$. ($\ast$)

The pointclass $\bosig^1_1$ was defined in terms of projections and closed sets. In particular,
\begin{align*}
A \in \bosig^1_1&\iff \exists C\in \bopi^0_1 \textrm{ s.t. } A = pC\\
&\xLeftrightarrow{(\ast)} \exists T\textrm{ tree s.t. }A = p[T]
\end{align*}

Let's slightly reformulate this. If $T$ is a tree on $\om \times \om$ and $x \in \om^\om$, we can define
\begin{align*}
    T_x\coloneqq \left\lbrace s; (s,x\restriction \lh(s))\in T\right\rbrace
\end{align*}
Then $A$ is $\bosig^1_1$ if and only if: $\exists T \textrm{ tree s.t. } x \in A \iff [T_x]\ne \emptyset$.

\begin{defin*}
    A tree $T$ is called \undf{illfounded} if $[T] \ne \emptyset$ and \undf{wellfounded} if $[T] = \emptyset$.

    With some axiom of choice, this is equivalent to $(T,\supseteq)$ being ill/wellfounded. So we can reformulate this as
    \begin{align*}
        A \in \bosig^1_1 \iff \exists T\textrm{ tree s.t. }\forall x(x\in A \iff T_x\textrm{ is illfounded})\\
        A \in \bopi^1_1 \iff \exists T\textrm{ tree s.t.}\forall x(x\in A \iff T_x\textrm{ is wellfounded})
    \end{align*}

    This is \undf{tree representation of analytic and co-analytic sets}.
\end{defin*}

\underline{Coding trees on $\om$ or $\om\times\om$ as elements of Baire space}:

Pick your favourite bijection between $\om\ra \om^\lom$ and write $\{s_i;i\in \om\} = \om^\lom$. Define $R$ by $iRj$ iff $s_i\supseteq s_j$. Then $F_T\coloneqq \{i;s_i \in T\}$. Then $(T,\supseteq)\cong (F_T,R)$. In particular, $T$ is wellfounded $\iff (F_T,R)$ is wellfounded.

One of the benefits of wellfounded relations is that we can define rank functions on them. If I have a wellfounded relation $R$ on $F$, I can recursively define a rank function:
\begin{itemize}
    \item $i\in F$: let rk$(i)\coloneqq \sup \{\rk(j)+1;jRi\}$
\end{itemize}
By the recursion theorem, if $R$ is wellfounded, then $\rk$ is a function assigning an ordinal to each element of $F$: $\{\rk(i);i\in F\}$ is an ordinal. But the $i$s in question are natural numbers, so this must be a countable ordinal.

We now identify our relation on subsets of $\om$ with elements of Baire space:
$$ \lceil n,m\rceil :\om\times\om\ra\om$$
your favourite bijcetion; $x \in \om^\om$. Define:
\begin{align*}
    \fld(x)&\coloneqq \{i;x(\lceil i,i\rceil) \ne 0\}\\
    R_x &\coloneqq \{(i,j); x(\lceil i,j\rceil)\ne 0\}
\end{align*}
Then $(\fld(x),R_x)$ is a reflexive relation.

If $(A,R)$ is any such structure, \it{i.e.} $A\subset \om$, $R\subset A\times A$ reflexive, then define
\begin{align*}
    x_A(\lceil i,j\rceil)\coloneqq \left\lbrace \begin{array}{cc} 1 & \textrm{ if }i,j\in A,iRj\\ 0 & \textrm{ o/w}\end{array}\right.
\end{align*}
Then $\fld(x_A) = A$, and $iRj\iff iR_{x_A}j$.

\marginpar{Lecture 13}

 \begin{defin*}
     $\wf\subset\om^\om$ is defined to be $$\wf \coloneqq \{x \in \om^\om ; (\fld(x),R_x)\textrm{ is wellfounded}\}$$

     Then the rank function $$\rk : \fld(x) \ra \alpha$$ gives an order-preserving map from $\fld(x)$ into $\alpha =: \hit(\fld(x),R_x)$, the \undf{height} of the relation.

     If $x \in \wf$, $||x|| \coloneqq \hit(\fld(x),R_x)$. This operation $||\cdot || : \wf \ra \omega_1$ is a surjection.

     [Indeed, let $\alpha < \omega_1$. There is some injection $f : \alpha \ra \N$. Define $F\coloneqq f[\alpha]$. Then define $f(\beta) R f(\gamma)$ by $\beta\le \gamma$. Then by construction $(F,R)\simeq (\alpha,\le)$. So if $A\coloneqq (F,R)$ and $x \coloneqq x_A$, then $||x_A|| = \alpha$.]

     Define:
     \begin{align*}
         \wf_\alpha &\coloneqq \{x \in \wf; ||x|| = \alpha \}\\
         \wf_{<\alpha} &\coloneqq \bigcup_{\beta<\alpha}\wf_\beta\\
         \wf_{\le \alpha} &\coloneqq \bigcup_{\beta\le\alpha}\wf_\beta
     \end{align*}
     Thus $\wf$ can be thought of as \undf{stratified} in $\om_1$ may levels.
 \end{defin*}

 \begin{theorem*}
     $\wf$ is $\bopi^1_1$.
 \end{theorem*}
 \begin{proof}
     If $A = (F,R)$ is a relation on $\N$, we can define ``$y\in\om^\om$ is an \undf{$A$-descending sequence}'' if $\forall i[(y(i+1)Ry(i))\land(y(i+1)\ne y(i))]$. Then $x\in \wf \iff \forall y\ y$ is not an $(\fld(x),R_x)$-descending sequence, and $x \not\in \wf\iff \exists y\ y$ is a $(\fld(x),R_x)$-descending sequence.

     Hence $ x \not\in \wf \iff \exists y [\forall i\ x(\lceil y(i+1),y(i)\rceil)\ne0\ \land\ y(i+1)\ne y(i)]$.

     Now consider $C\coloneqq \{(y,x); \forall i\ x(\lceil y(i+1),y(i)\rceil)\ne 0\ land\ y(i+1)\ne y(i)\}$. $C$ is closed in $\om^\om\times\om^\om$ (can easily check that these conditions form a tree). So by definition, $\om^\om \backslash \wf$ is $\bosig^1_1$, and hence $\wf$ is $\bopi^1_1$.
 \end{proof}

\begin{remark*}
    The general proof technique extracted from this is that if $C$ is $\bopi^1_x(\om^\om\times\om^\om)$ and $x \in A \iff \exists y (y,x)\in C$ then $A$ is $\bosig^1_{n+1}$, and similarly if $C$ is $\bopi^1_n$ and $x \in A \iff \forall y (y,x)\in C$ then $A$ is $\bopi^1_{n+1}$.
\end{remark*}

Now check the complexity of the sets $\wf_\alpha,\wf_{<\alpha},\wf_{\le\alpha}$. We have $x\in \wf_{<\alpha} \iff x \in \wf$ and there is no order-preserving map from $\alpha$ into $(\fld(x),R_x)$; define $$N_\alpha \coloneqq \{x; \textrm{ there is no o.p. map from }\alpha \textrm{ into } (\fld(x),R_x)\}$$

Fix some $a \in \om^\om$ such that $(\fld(a),R_a)\cong (\alpha,\le)$. [We saw that this exists.] Hence we have that
\begin{align*}
    N_\alpha =&\{x;\textrm{ there is no o.p. map from }(\fld(a),R_a)\textrm{ into }(\fld(x),R_x)\}\\
    =& \big\{x; \forall y\textrm{ it is not the case that: }\\
    &[\forall i\ a(\lceil i,i\rceil)\ne 0 \implies x(\lceil y(i),y(i)\rceil)\ne 0\textrm{ and }\forall i,j\ a(\lceil i,j\rceil)\ne0 \implies x(\lceil y(i),y(j)\rceil)\ne 0]\big\}
\end{align*}
So $N_\alpha$ is $\bopi^1_1$, and since $\wf_{<\alpha} = \wf \cap N_\alpha$, we thus have that $\wf_{<\alpha}$ is $\bopi^1_1$. Similarly, $\wf_{\le \alpha},\wf_\alpha$ are $\bopi^1_1$.

$\wf_{\le\alpha}$ is also $\bosig^1_1$: $\wf_{\le\alpha} = \{x; (\fld(x),R_x)\textrm{ o.p. maps into }(\fld(a),R_a)\}$. We can write this as:
\begin{align*}
    x \in \wf_{\le\alpha} &\iff \exists y[\forall i,j x(\bij{i,j})\ne 0\implies a(\bij{y(i),y(j)})\ne 0]
\end{align*}
and again we see the bit in square brackets defines a closed set of $x$s, and hence $\wf_{\le\alpha}$ is $\bosig^1_1$.

\underline{Summary}: For every $\alpha < \om_1$, $\wf_\alpha,\wf_{<\alpha},\wf_{\le\alpha}$ are $\bodel^1_1$.

On ES\#2, we show that $\bodel^1_1 = \textrm{Borel}$.

\begin{remark*}[Corollary]
    \emph{
        $\wf$ can be written as a union of $\om_1$ many Borel sets: $$\wf = \bigcup_{\alpha < \om_1}\wf_\alpha$$
    }
\end{remark*}

\begin{defin*}[$\bog$-complete]
    Let $\bog$ be a boldface pointclass. A set $A\subset\om^\om$ is called \undf{$\bog$-hard} if for all $B \in \bog(\om^\om)$ there is a continuous function $f : \om^\om \ra \om^\om$ such that $B = f^{-1}[A]$. $A$ is \undf{$\bog$-complete} if it is $\bog$-hard and $A \in \bog(\om^\om)$.
\end{defin*}

\begin{theorem*}
    $\wf$ is $\bopi^1_1$-complete.
\end{theorem*}
\begin{proof}
    Tree representation theorem says: if $B$ is $\bopi^1_1$, then there is a tree $T$ such that $\forall x\ x\in B\iff T_x$ is wellfounded. This almost has the form that we want, but we're talking about trees instead of elements of $\om^\om$. So we map $x\mapsto c_{T_x} \in \om^\om$ such that:
    \begin{align*}
        c_{T_x}(\bij{i,j}) \coloneqq \left\lbrace \begin{array}{cl} 1 & \textrm{ if }s_i,s_j\in T_x\textrm{ and } s_i\supseteq s_j\\ 0 & \textrm{ o/w}\end{array}\right.
    \end{align*}
    Consider $x\mapsto c_{T_x}$ and check that it is continuous. Given $i,j$, how much information about $x$ do I need to determine whether $c_{T_x}(\bij{i,j}) = 0$ or $1$? Note that whether $s_i\supseteq s_j$ or not does not depend on $x$ at all. What does $s_i \in T_x$ mean? It means $(s_i,x\restriction \lh(s_i))\in T$.

    So if we know $x\restriction \max(\lh(s_i),\lh(s_j))$, then we can calculate $c_{T_x}(\bij{i,j})$. So $x\mapsto c_{T_x}$ is a continuous function. So we have:
    \begin{align*}
        x \in B &\iff [T_x]\ne\emptyset\\
        &\iff T_x\textrm{ is wellfounded}\\
        &\iff c_{T_x}\in \wf
    \end{align*}
    So $B$ is the continuous preimage of $\wf$.
\end{proof}

\begin{remark*}[Corollary]
    \emph{
        $\wf$ is not $\bosig^1_1$
    }
\end{remark*}
\begin{proof}
    We know that $\bosig^1_1 \ne \bosig^1_1$. However, if $B \in \bopi^1_1$ arbitrary, by completeness of $\wf$ if $\wf$ is $\bosig^1_1$ then $B$ is $\bosig^1_1$. Contradiction.
\end{proof}

\begin{remark*}[Corollary]
    \emph{
        Every $\bopi^1_1$ set is an $\om_1$-union of Borel sets.
    }
\end{remark*}
\begin{proof}
    If $B \in \bopi^1_1$, find $f$ such that $B = f^{-1}[\wf]$. So:
    \begin{align*}
        B &= f^{-1}\left[ \bigcup_{\alpha < \om_1}\wf_\alpha\right]\\
        &=\bigcup_{\alpha<\om_1}f^{-1}[\wf_\alpha]        
    \end{align*}
    and each $f^{-1}[\wf_\alpha]$ is Borel.
\end{proof}

\marginpar{Lecture 14}

\begin{remark*}[Proposition (Weak CH for $\bopi^1_1$ sets)]
    \emph{
        If $A \in \bopi^1_1$, then $|A|$ is either $\le \aleph_0$, or $\aleph_1$, or $2^{\aleph_0}$.
    }
\end{remark*}
\begin{proof}
    By our analysis, we know $$A = \bigcup_{\alpha < \om_1}A_\alpha$$ where $A_\alpha$ is Borel. By Borel determinacy \& the characterisation of PSP in terms of games, we know that $A_\alpha$ has the p.s.p for each $\alpha$. So for each $\alpha$, $A_\alpha$ is countable or there is a $T_\alpha$ perfect such that $[T_\alpha] \subset A_\alpha$.

    \underline{Case 1}: There is some $\alpha$ such that $[T_\alpha]\subset A_\alpha \subset A$, so $A$ contains a perfect subset and hence $|A| = 2^{\aleph_0}$.

    \underline{Case 2}: $A_\alpha$ countable for all $\alpha$. So then $|A| \le \aleph_1 \cdot \aleph_0 = \aleph_1$.
\end{proof}

\begin{theorem*}[Boundedness Lemma]
    If $A\subset \wf$ such that $A$ is $\bosig^1_1$, then there is a bound $\alpha < \om_1$ such that: $$A\subset \wf_{<\alpha}$$
\end{theorem*}
\begin{proof}
    Let us write $\op(y,x,z)$ for ``$y$ is an order-preserving map from $(\fld(x),R_x)$ to $(\fld(z),R_z)$''. In the proof last lecture we saw that this is a closed set.

    [This allows us to express $||x|| = ||z||$, then this is just $\exists y \exists y' (\op(y,x,z)\land\op(y',z,x))$. $\op(y,x,z)\land \op(y',z,x)$ is closed, so under the $\exists$ quantifiers the set is $\bosig^1_1$.]

    We prove boundedness by contradiction. Assume $A \in \bosig^1_1, A\subset \wf$ unbounded $(\ast)$ and show that $\wf \in \bosig^1_1$.

    Under the assumption $(\ast)$, we have: 
    \begin{align*}{}
        x \in \wf \iff \underbrace{\exists a \exists y\underbrace{(\underbrace{a\in A}_{\bosig^1_1} \land \exists y\ \underbrace{\op(y,x,a)}_{\bopi^0_1})}_{\bosig^1_1}}_{\bosig^1_1}
    \end{align*}
    since $\bosig^1_1$ is closed under existential quantifiers. This contradicts the $\bopi^1_1$-completeness of $\wf$.
\end{proof}

\begin{defin*}
    $C\subset \wf$ is called a \undf{set of unique codes} if for all $\alpha \in \om_1$, $|C\cap \wf_\alpha| = 1$.
\end{defin*}

Clearly, if $C$ is a s.u.c. then $|C| = \aleph_1$. Also clearly, $\ac$ implies the existence of s.u.c.: $$\{\wf_{\alpha};\alpha \in \om_1\}$$ is a family of non-empty sets, and a choice function for this family a s.u.c. as the range.

So the fragment of choice needed here is $\ac_{\om_1}(\om^\om)$.

\begin{theorem*}
    If $C$ is a s.u.c., it cannot have the p.s.p.
\end{theorem*}
\begin{proof}
    Since $C$ is uncountable, if it has the p.s.p. it must contain some $[T]\subset C$, with $T$ perfect. So we have $[T]\subset C\subset \wf$, and thus $[T]$ is a $\bosig^1_1$ subset of $\wf$. By boundedness, we find $\alpha < \om_1$ such that $[T]\subset \wf_{<\alpha}\cap C$; but $|\wf_{<\alpha}\cap C| = |\alpha| \le \aleph_0$, but $|[T]| = 2^{\aleph_0}$ is uncountable.
\end{proof}

\begin{theorem*}
    If there is a $\bodel^1_1$ wellorder of $\om^\om$, then there is a $\bopi^1_n$ set without the perfect set property.
\end{theorem*}
\begin{proof}
    Produce a set of unique codes. In each $\wf_\alpha$, there is a $<$-least element if $<$ is the $\bodel^1_n$-wellorder. Then the set
    \begin{align*}
        C\coloneqq \{x;\exists \alpha < \om_1 \textrm{ s.t. }x\textrm{ is the }<\textrm{-least element of }\wf_\alpha\}
    \end{align*}
    is an s.u.c. and thus doesn't have the p.s.p.

    We need to analyse the definition of $C$ a bit better. What does it mean to be in $C$?

    $x \in \wf$ and if $z \in \wf$ and $||x|| = ||z||$, then $x\le z$ in the $\bodel^1_n$ wellorder. But recall our characterisation of $||x|| = ||z||$. So we combine this into one formula and analyse the complexity:
    \begin{align*}
        x \in C \iff & \underbrace{\underbrace{x \in \wf}_{\bopi^1_1} \land \underbrace{\forall z (\underbrace{\underbrace{\exists y\exists y' [\underbrace{\op(y,x,z)\land\op(y',z,x)}_{\bopi^0_1}}_{\bosig^1_1}\implies \underbrace{x\le z}_{\bodel^1_n}]}_{\cup \textrm{ of } \bopi^1_1,\bodel^1_n\textrm{ which is }\bopi^1_1\textrm{ or }\bodel^1_n\textrm{ if }n\ge2})}_{\bopi^1_n}}_{\bopi^1_n}
    \end{align*}
    So $C$ is a s.u.c. which is $\bopi^1_n$.
\end{proof}

\subsection*{Close connection between Games, Wellorders, and Large Cardinals}
We have some (somewhat, \it{n.b.} not equivalent) correspondent notions:

\begin{center}
    \begin{tabular}{ccccc}\\
        \textbf{Large Cardinals} & &\textbf{Determinacy} & &\textbf{Definable Wellorders}\\ 
        &   &  & & \\
        $\omega$ many Woodin cardinals $+$ measurable above &$\longsquiggly$ & $\Det(\textrm{Definable})$  & $\longsquiggly$ & no definable wellorders\\
        $\omega$ many Woodin cardinals & $\longsquiggly$ & $\Det(\textrm{Proj})$ & $\longsquiggly$ & no projective wellorder\\
        &   & $\vdots$ & & \\
        $n$ Woodin cardinals $+$ a measurable above them & $\longsquiggly$ & $\Det(\bopi^1_{n+1})$ & $\longsquiggly$ & no $\bodel^1_{n+2}$ wellorders\\
            &   & $\vdots$ & & \\
         $\kappa < \lambda$, $\kappa$ Woodin, $\lambda$ measurable &$\longsquiggly$ & $\Det(\bopi^1_2)$ & $\longsquiggly$ & no $\bodel^1_3$ wellorders \\
        measurable cardinal & $\longsquiggly$ & $\Det(\bopi^1_1)$ & $\longsquiggly$ & no $\bodel^1_2$ wellorders \\
        ZFC && $\Det(\bodel^1_1)$ && -
    \end{tabular}
\end{center}

The results of the first two columns and all but the first two rows are the famous \it{Martin-Steel Theorem of projective determinacy} (1985).

\marginpar{Lecture 15}

The remainder of this course will be devoted to exploring these interactions.

\subsection*{Large Cardinals}
\begin{remark*}[Imprecise Definition]
    Let $\Phi$ be a property of cardinals. We write $\Phi C$ for $\exists x \Phi(x)$.

    $\Phi$ is going to be the \it{large cardinal property}, and $\Phi C$ is going to be the \it{large cardinal axiom}.

    We say that $\Phi$ is an LCP if:
    \begin{enumerate}[label = (\roman*)]
        \item $\Phi(\kappa)$ implies that $\kappa$ is ``large'' in some sense
        \item $\Phi C \proves \textrm{Cons}(\textrm{ZFC})$ (so $\Phi C$ cannot be proven in ZFC)
    \end{enumerate}
\end{remark*}

\underline{Comments}:\ 
    \begin{enumerate}[label=(\roman*)]
        \item G{\"o}del's Incompleteness Theorem says if $\zfc \proves \Phi C$ and $\zfc + \Phi C \proves \textrm{Cons(ZFC)}$ then ZFC is inconsistent. So, under reasonable assumptions, it means $\zfc \not\proves \Phi C$.
        \item G{\"o}del's Completeness Theorem tells us that Cons(ZFC) is equivalent to $\exists M (M\models \zfc)$. The model property is what we will be using.
    \end{enumerate}

If $\Phi,\Psi$ are two LCPs, we can say \underline{$\Phi C < \Psi C$} if $\zfc + \Psi C \proves \textrm{Cons}(\zfc+\Phi C)$, and $\Phi C,\Psi C$ are \undf{equiconsistent} if $\zfc + \cons(\zfc + \Phi C) \iff \zfc + \cons(\zfc + \Psi C)$.

\begin{defin*}[Inaccessible Cardinal]
    A cardinal  is called \undf{inaccessible} if it is \it{regular} and a \it{strong limit}.

    [A cardinal $\kappa$ is a strong limit if for all $\lambda < \kappa$, $2^\lambda < \kappa$.]

    We write $I$ for the property of being inaccessible. Then as before, $IC$ is the statement $\exists x I(x)$ asserting that there exists an inaccessible cardinal.
\end{defin*}
\begin{remark*}
    $\kappa$ is a limit $\iff$ $\forall \lambda < \kappa (\lambda^+ < \kappa)$.
\end{remark*}
\begin{remark*}
    Under GCH, $[\forall \lambda (\lambda^+ = 2^\lambda)]$ we have $\kappa$ is a limit $\iff \kappa$ is a strong limit.
\end{remark*}

We now show that $IC$ is indeed a large cardinal axiom, by checking properties (1) and (2).
\begin{enumerate}[label = (\roman*)]
    \item Clearly, if $I(\kappa)$ then $\aleph_1,\aleph_2,\aleph_3 < \kappa $ trivially since these are successor cardinals. Also $\aleph_{\omega},\aleph_{\om_1},\aleph_{\om_2} < \kappa$ since these limit cardinals are not regular. Furthermore, we know there are ordinals $\alpha$ such that $\alpha = \aleph_\alpha$. Let $lambda$ be the least of these. This is pretty massive, but they not regular either since in showing its existence we show it has countable cofinaltiy. So we even have $\lambda < \kappa$.
    
    So these are pretty big (though this is currently an imprecise notion).

    \item Need to show $\zfc + IC \proves \cons(\zfc)$, which we do by the following theorem:
    
    \begin{theorem*}
        If $\kappa$ is inaccessible, then $V_\kappa \models \zfc$
    \end{theorem*}
    \begin{proof}
        In L\&ST 2019/20, ES\#4 (9): if $\lambda$ is any limit larger than $\omega$, then $V_\lambda$ models all $\zfc$ axioms, minus Replacement ($\zfc - $Replacement is sometimes known as \it{Zermelo Set Theory}).

        So all we need to show is that $V_\kappa$ models Replacement.

        \begin{remark*}[Lemma 1]
            If $\kappa$ is inaccessible, then $\forall \alpha < \kappa$, $|V_\alpha| < \kappa$.
        \end{remark*}
        \begin{proof}
            This is an induction on $\alpha$. $\alpha = 0$ is obvious, $\alpha$ successor done by strong limit, $\alpha$ limit done by regularity.
        \end{proof}

        \begin{remark*}[Lemma 2]
            Let $\kappa$ be inaccessible. Then TFAE:
            \begin{enumerate}[label=(\roman*)]
                \item $x \in V_\kappa$
                \item $x\subset V_\kappa$ and $|X| < \kappa$
            \end{enumerate}
        \end{remark*}
        \begin{proof} \underline{(i)$\implies$(ii)}:
            $x \in V_\kappa \implies x\subset V_\kappa$ [transitivity of $V_\kappa$]
            
            $x \in V_\kappa = \bigcup_{\alpha < \kappa}V_\alpha \implies $ there is $\alpha < \kappa$ such that $x \in V_\alpha$, so $x \subset V_\alpha$. Then $|x| \le |V_\alpha| < \kappa$ by L1.

            \underline{(ii)$\implies$(i)}: Suppose $x\subset V_\kappa$. For every $y \in x$, define $\alpha_y = \rho(y)$, the rank of $y$ ($y \in V_{\alpha_y+1}\backslash V_{\alpha_y}$). Let $A\coloneqq \{\alpha_y+1; y \in x\}$. Clearly $|A| \le |x| < \kappa$. Hence, by regularity of $\kappa$, $\bigcup A =:\alpha < \kappa$. But then for all $y \in x$, $y \in V_\alpha$, so $x \subset V_\alpha$. Hence $x \in V_{\alpha+1}\subset V_\kappa$.
        \end{proof}

        Now we can show something even stronger than Replacement:

        Take any function $F: V_\kappa \ra V_\kappa$ and show that if $x \in V_\kappa$, then $F[x] \in V_\kappa$.

        If $x \in V_\kappa$, then by L2 we know that $|x| <\kappa$, but then $|F[x]|\le |x| < \kappa$, and clearly $F[x]\subset V_\kappa$. Hence by L2 $F[x] \in V_\kappa$.

        So this holds for all functions, including those defined by formulae.
    \end{proof}
\end{enumerate}

\underline{Next goal}: If there is a family of models with nice wellorders of $\om^\om$ and $\psp(\bopi^1_n)$ holds, then there is a model of $\zfc + IC$. This will require a bit of work; we need to make precise the notion of `nice'.

\begin{defin*}[Inner Model,]\ 
    \begin{enumerate}
        \item $M$ is an \undf{inner model} if it is a transitive class containing all ordinals and a model of $\zfc$.
        
        Alternatively, suppose $(V,\in)\models \zfc$ is a set model. Then $M\subset V$ is called an \undf{inner model} if it contains all of the ordinals of $V$ and is transitive in $V$ and $(M,\in)\models \zfc$. [\it{c.f.} ES\#3]

        \item A formula $\phi$ \undf{defines} an inner model if $x \in M \iff \phi(x)$.
        
        \item A formula $\mu$ is called a \undf{canonical model family} if the following classes are inner models: $M = \{w;\phi(\emptyset,w)\}$, $M_x \coloneqq \{w;\phi(x,w)\}$ for each $x \in \om^\om$, with the properties:
        \begin{enumerate}[label = (\alph*)]
            \item $\forall x (M\subset M_x)$
            \item $\forall x (x \in M_x)$
            \item $M\models \textrm{GCH}$
        \end{enumerate}

        We call $M$ the \undf{root}.

        \item If $\mu$ is a canonical model family, we say that $\mu$ is \undf{$\bodel^1_n$-wellordered} if for each $x$, there is a wellordering $<_x$ of $\om^\om \cap M_x$ such that $\{(u,v); u,v\in M_x \land u <_x v\}$ is $\bodel^1_n$.
    \end{enumerate}
\end{defin*}

\marginpar{Lecture 16}

This lecture is devoted to proving a theorem which, while of little practical use to us, will help to demonstrate the techniques of relating the various concepts that we are interested in.

\subsection*{Preserving of basic features in Inner Models}

The main thing about inner models that we assumed was that, for $M\subset V$, $M$ is transitive. Suppose $M,V\models \zfc$.

We have (ES\#3), for $x,y,f \in M$:
\begin{itemize}    
    \item $M\models f : x\ra y \iff V\models f : x \ra y$
    \item $M\models f:x\ra y $ is surjective $\iff$ $V\models f:x \ra y$ is surjective
    \item $M\models x,y$ are ordinals, $f : x \ra y$, $f$ cofinal $\iff$ $V\models x,y$ are ordinals, $f : x \ra y$, $f$ cofinal
\end{itemize}

\underline{Consequence}:
\begin{itemize}
    \item $V\models \kappa$ is a cardinal $\implies$ $M\models \kappa$ is a cardinal
    \item $V\models \kappa$ regular $\implies$ $M\models \kappa$ is regular
\end{itemize}
Note that this consequence is not in general reversible; regularity of a cardinal in $M$ doesn't necessarily imply regularity in $V$. For example, there are possibly other cardinals in $M$, as well as those inherited from $V$. So perhaps the models disagree on what the label `$\aleph_0$' refers to; we can distinguish this by $\aleph_\alpha^M$ (and $\aleph_\alpha^V$).

If we write ``$\aleph_1$ is a cardinal'', this is technically not a first order formula of set theory; what we mean by this is ``the first uncountable ordinal is a cardinal''. It is important to understand that when we say that $\kappa$ is regular in $V$ and then look at its properties in $M$, we refer in $M$ to that exact object, no the cardinal with those properties. For instance, if we have $V\models \aleph_2$ regular, then the conseqeunce ``$M\models \aleph_2$ regular'' means that the exact same set theoretic object is still regular in $M$, not that the second uncountable cardinal is also regular in $M$.

If we mean $\aleph_n$ to be interpreted by a formula in $M$, we write $\aleph_n^M$. If we mean the object interpreted in $V$, we write $\aleph_n = \aleph_n^V$. In general, $\aleph_n^M < \aleph_n^V$ is possible.

We have another transfer between $M$ and $V$: $$M\models x \in \wf \implies V \models x \in \wf$$
which is proved on ES\#3 also.
\begin{align*}
    M\models \alpha\textrm{ is countable} \iff & \exists x (||x|| = \alpha \ \& \ x \in \wf\cap M)\\
    \iff & \wf_\alpha \cap M \ne \emptyset 
\end{align*}
In this situation, this means that $\aleph_1^M < \aleph_1$, so in $V$, $\wf_{\aleph_1^M}\ne \emptyset$. But $M \models \aleph_1^M$ is not countable (clearly), hence $\wf_{\aleph_1^M}\cap M = \emptyset$.

\underline{Summary}: $\aleph_1^M = \min \{\alpha ; \wf_\alpha \cap M = \emptyset\}$. 

Let $\mu$ be a canonical model family, with models $M,M_x$. Since $M\subset M_x$, we have that $\aleph_1^M\le \aleph_1^{M_x}\le \aleph_1$. 

Here comes the `weird' theorem:

\begin{theorem*}
    Suppose there is a $\bodel^1_n$-wellordered canonical model family and $\Det(\bopi^1_n)$. Then there is an inner model of $\zfc + \ic$.
\end{theorem*}
\begin{proof}
    We fix models $M,M_x$ from the canonical model family $\mu$.

    \begin{defin*}[Weak set of unique codes]
        $C\subset \wf$ is called a \undf{weak set of unique codes} if for all $\alpha$, $|\wf_\alpha \cap C| \le 1$.
    \end{defin*}

    We get from our general theorem about s.u.c. that  if $C$ is a wsuc, then $C$ is uncountable $\iff$ $C$ does not have the psp.

    By assumption, we know that $\{(u,v);u,v,\in M_x\textrm{ and }u<_x v\}$ is $\bodel^1_n$. The proof of ``$\bodel^1_n$-wellordering $\implies$ suc that is $\bopi^1_n$'' shows here that there is a wsuc $C_x$ that is $\bopi^1_n$. Then $C_x \cap \wf_\alpha = \emptyset \iff \alpha < \aleph_1^{M_x}$.

    Clearly $|C_x| = |\aleph_1^{M_x}|$, so $C_x$ has the psp $\iff$ $\aleph_1^{M_x} < \aleph_1$. BY $\Det(\bopi^1_n)$, we get $\psp(\bopi^1_n)$ by asymmetric game. So for each $x$, $C_x$ has the psp, so $\forall x (\aleph_1^{M_x} < \aleph_1)$. Also, if $\alpha < \aleph_1$ then there is some $x \in \wf_\alpha$. But then $x\in M_x$, which implies $\wf_\alpha\cap M_x \ne \emptyset$, and hence $\aleph_1^{M_x} > \alpha$. So $M_x$ has lots of cardinals that are not cardinals in $V$.

    To see this, consider $\alpha \coloneqq \aleph_1^{M_z}$. Find $x$ such that $||x|| = \alpha$. Then $\aleph_1^{M_x} > \alpha$, but then by the above argument (independent of $z$) we have that $\aleph_1^{M_z} < \aleph_1^{M_x} < \aleph_1$. This argument shows that for each $\alpha < \aleph_1$, there is some $x$ such that $\alpha < \aleph_1^{M_x} < \aleph_1$. In particular, the ordinals that look like $\aleph_1$ in these models are unbounded in $\aleph_1$.

    \underline{Claim}: $M\models \aleph_1^V$ inaccessible.

    \underline{Proof of Claim}: Since $M\subset M_x$, we have $\aleph_1^M \le \aleph_1^{M_x} < \aleph_1^V$. Regularity is preserved by inner models, hence $M$ believes $\aleph_1^V$ is regular.

    Since $M \models \textrm{GCH}$, strong limit is equivalent to limit. So we just need to show that if $\kappa < \aleph_1$ cardinal, $\lambda$ is an ordinal such that $M\models \lambda = \kappa^+$ (cardinal successor), then $\lambda < \aleph_1$. Once this is shown, then $\aleph_1^V$ is a limit in $M$, hence strong limit, and also regular hence inaccessible.

    Let $x \in \wf_\kappa$, which exists in $V$ since $\kappa < \aleph_1$. As we had before, $x \in M_x \implies \aleph_1^{M_x} > \kappa$. Since $M\subset M_x$, $\aleph_1^{M_x}$ is a cardinal in $M$, and by assumption is the smallest cardinal $>\kappa$ in $M$.

    This implies $\lambda \le \aleph_1^{M_x} < \aleph_1$. Hence $\lambda < \aleph_1$. \qedsymbol

    This completes the proof of the claim, and hence the theorem.
\end{proof}

\underline{Next}: We move up from inaccessible cardinals to \it{measurable cardinals}. The remainder of the course will focus on the relationship between determinacy and large cardinals at this level. We will prove the following:

\begin{remark*}[Theorem 1]
    If there is a measurable cardinal, then $\Det(\bopi^1_1)$.
\end{remark*}
\begin{remark*}[Theorem 2]
    If $\zf + \ad$, $\aleph_1$ is a measurable cardinal.
\end{remark*}

\marginpar{Lecture 17}

\subsection*{Measurable Cardinals}

Let $\kappa$ be an uncountable cardinal. We say that $F\subset \P(\kappa)$ is a \undf{filter on $\kappa$} if:
\begin{enumerate}[label = (\alph*)]
    \item $\kappa\in F,\emptyset\not\in F$
    \item $A,B\in F \implies A\cap B \in F$
    \item $A\in F, B\supseteq A\implies B \in F$
\end{enumerate}
A filter $F$ is called an \undf{ultrafilter} if for every $A\subset \kappa$, either $A \in F$ or $\kappa\backslash A \ in F$.

A filter $F$ is called \undf{$\lambda$-complete} if for every $\gamma < \lambda$ and every family $\{A_\alpha ;\alpha < \gamma\}\subset F$, we have $\bigcap_{\alpha < \gamma} A_\alpha \in F$, \it{i.e.} closed under intersections of length less than $\lambda$.

\begin{remark*}
    If $\lambda = \aleph_0$, then this means closure under finite intersections (which follows anyway from (b)).

    If $\lambda = \aleph_1$, this means closure under countable intersections, often called \undf{$\sigma$-completeness}
\end{remark*}

If $\alpha \in \kappa$, then $U_\alpha \coloneqq \{X\subset \kappa : \alpha \in X\}$ is an ultrafilter wchich is $\lambda$-complete for all $\lambda$. We call these \undf{principal ultrafilters}, and those not of this form we call \undf{non-principal ultrafilters}.

So, if $U$ is a non-principal ultrafilter, then for all $\alpha \in \kappa$, $\{\alpha\}\not\in U$. Therefore, if it is $\lambda$-complete and $A\subset \kappa$ with $|A| < \lambda$, then $A \not\in U$, since it is a length $< \lambda$ union of things not in $U$, and hence is not in $U$ by a recharacterisation of $\lambda$-completeness.

\begin{remark*}[Fact]
    $\zfc$ proves that the set of ultrafilters on $\kappa$ has cardinality $2^{2^\kappa}$. Of these, exactly $\kappa$ many are principal.
\end{remark*}

\begin{defin*}[Measurable Cardinal]
    A cardinal $\kappa$ is called \undf{measurable} if there is a $\kappa$-complete non-principal ultrafilter on $\kappa$. [In particular, for such a $U$, all elements of $U$ have cardinality $\kappa$.]
\end{defin*}

At first glance, this is a rather combinatorial property, and it is not obvious at all that this is in fact a large cardinal property.

\begin{theorem*} [ZFC]
    Every measurable cardinal is inaccessible.
\end{theorem*}
We highlight ZFC because we will later see that $\zf + \ad \implies \aleph_1$ is measurable, and clearly $\aleph_1$ is not inaccessible. So this proof needs AC.
\begin{proof}
    We need to show regular \& strong limit. Fix $U$ a $\kappa$-complete non-principal ultrafilter on $\kappa$.

    \underline{Regular}: Suppose not. So there is some sequence $(\gamma_\alpha; \alpha < \lambda)$ with $\lambda < \kappa$ and $\gamma_\alpha < \kappa$ such that $\kappa = \bigcup_{\alpha < \lambda}\gamma_\alpha$. By choice of $U$, $\gamma_\alpha \not\in U$ since it is an initial segment and $\kappa$ is a cardinal (hence initial ordinal) so $|\gamma_\alpha| < \kappa$.

    By $\kappa$-completeness, $\bigcup_{\alpha < \lambda}\gamma_\alpha \not\in U$, contradicting the filter property $\kappa \in U$.

    Note that so far this is a ZF theorem.

    \underline{Strong Limit}: We need to show that if $\lambda < \kappa$, then $2^\lambda \not\ge \kappa$; equivalently in ZFC, we show $2^\lambda < \kappa$. We do this by assuming that there is an injection $f : \kappa \ra 2^\lambda$ and derive a contradiction. Note that we can identify $2^\lambda$ with $S = \{g; g : \lambda \ra 2\}$. So wlog say $f : \kappa \xhookrightarrow{} S$. So for $\alpha \in \kappa$, $f(\alpha): \lambda \ra 2$.

    For fixed $\gamma < \lambda$, we consider $A^0_\gamma = \{\alpha; f(\alpha)(\gamma) = 0\}$ and $A^1_\gamma = \{\alpha ; f(\alpha)(\gamma) = 1\}$. Clearly $A^0_\gamma \cup A^1_\gamma = \kappa$, and  $A^0_\gamma \cap A^1_\gamma = \emptyset$. Thus exactly one of the two is in $U$.

    Let $i_\gamma \in \{0,1\}$ be such that $A^{i_\gamma}_\gamma \in U$. Then the family $A^{i_\gamma}_\gamma;\gamma \in \lambda\}\subset U$, and thus by $\kappa$-completeness we have $\bigcap_{\gamma < \lambda}A^{i_\gamma}_\gamma \in U$.

    We now claim that $\left|\bigcap_{\gamma < \lambda}A^{i_\gamma}_\gamma \right| \le 1$, in contradiction to the non-principality of $U$.

    Suppose $\alpha \in \bigcap_{\gamma < \lambda}A^{i_\gamma}_\gamma$. Consider $f(\alpha)$. Then $f(\alpha)(\gamma) = i_\gamma$, so there is only one function $f$ such that for all $\alpha,\alpha' \in \bigcap_{\gamma < \lambda}A^{i_\gamma}_\gamma$, $f = f(\alpha) = f(\alpha')$. But $f$ is injective, so there can only be one such element; hence $\left|\bigcap_{\gamma < \lambda}A^{i_\gamma}_\gamma\right| \le 1$.
\end{proof}

\begin{remark*}
    The use of AC is in the setup of the proof, which requires that $2^\lambda$ is wellorderable.
\end{remark*}

\begin{defin*}[Erd\H{o}s-Rado Arrow Notation]
    Let $\kappa,\lambda,\mu$ be cardinals and $n \in \N$. We write $[X]^n$ for the set of $n$-element subsets of $X$. A function $c : [X^n] \ra \mu$ is called a \undf{$\mu$-colouring}.

    If $c$ is a $\mu$-colouring, we call $H\subset X$ \undf{$c$-homogeneous} or \undf{$c$-monochromatic} if there is some $ \alpha \in \mu$ such that $\forall s \in [H]^n$, $c(s) = \alpha$.

    Now we write $$\kappa \rightarrow (\lambda)^n_\mu$$ (pronounced $\kappa$ ``arrows'' $(\lambda)^n_\mu$) for ``every $\mu$-colouring of $[\kappa]^n$ has a homogeneous subset $H$ of cardinality $\lambda$''.
\end{defin*}

\begin{defin*}[Weakly Compact Cardinal]
    A cardinal $\kappa$ is called \undf{weakly compact} if $\kappa \rightarrow (\kappa)^2_2$.
\end{defin*}
\begin{remark*}[Facts]\ 
    \begin{enumerate}[label = (\arabic*)]
        \item Every weakly compact cardinal is inaccessible [ES\#3].
        \item Every measurable cardinal is weakly compact [proof next time].
    \end{enumerate}
\end{remark*}

\begin{defin*}
    We write $[X]^\lom$ for the set of all finite subsets of $X$. A function $c : [X]^\lom \ra\mu$ is called a \undf{$\mu$-colouring}.

    If $n \in \N$, $H\subset X$ we call $H$ \undf{$n$-$c$-homogeneous} if there is $\alpha \in \mu$ such that for all $s \in [H]^n$, $c(s) = \alpha$.

    We then write $$\kappa \rightarrow (\lambda)^\lom_\mu$$ if every $\mu$-colouring has an $n$-$c$-homogeneous set of size $\lambda$ for every $n \in \N$.
\end{defin*}

\begin{theorem*}[Rowbottom's Theorem]
    If $\kappa$ is measurable and there is a countable set $$\{c_k;k \in \N\}$$ of $\gamma$-colourings (where $\gamma < \kappa$), then there is a set $H$ of cardinality $\kappa$ [actually in the ultrafilter $U$] such that $H$ is $n$-$c_k$-homogeneous for all $n,k\in \N$ simultaneously.
\end{theorem*}
This theorem will again be discussed on ES\#3.

\marginpar{Lecture 18}

Proof of Fact (1) is on Example Sheet \#3 (29). We're going to see a weak version of Fact (1).

\begin{remark*}[Proposition (ZFC)]
    $\aleph_1$ is not weakly compact.
\end{remark*}
\begin{proof}
    By AC, we have suc $C\subset \wf$ with $|C| = \aleph_1$. So write $$ C = \{x_\alpha ; \alpha < \om_1\}$$ with $||x_\alpha || = \alpha$. Consider the lexicographic order on $\om^\om$:
    \begin{align*}
        x <_L y \iff x\restriction n = y \restriction n\textrm{ and }x(n) < y(n)\textrm{ [$n$ unique]}
    \end{align*}

    If $(y_\alpha; \alpha < \gamma)$ is any increasing sequence in the order $<_L$, then we can isolate the elements of the sequence by the following basic open sets:
    \begin{align*}
        y_\alpha \restriction n = y_{\alpha+1}\restriction n\\
        y_\alpha(n) < y_{\alpha+1}(n)
    \end{align*}
    Take $s_\alpha \coloneqq y_\alpha \restriction n+1$. Then $[s_\alpha]\ni y_\alpha$ but for no other element $y_\beta$ do we have $y_\beta \in [s_\alpha]$. Since there aer only countably many basic open sets, we see that $\gamma$ must be a countable ordinal.

    Define $c : [\aleph_1]^2 \ra 2$ by
    \begin{align*}
        c(\alpha,\beta) = \left\lbrace \begin{array}{cl}1 & \textrm{ if }<_L,<\textrm{ agree on }\alpha,\beta\\ 0 & \textrm{ o/w} \end{array}\right.
    \end{align*}
    If $H$ has cardinality $\aleph_1$ and is $c$-homogeneous for colour $1$, then it defines a $<_L$-increasing sequence in $C$; otherwise, if it's $c$-homogeneous for colour $0$, then it defines a $<_L$-decreasing sequence [note that the above argument works for $>$ as well as $<$]. Both are a contradiction.
\end{proof}

Now (2): Every measurable cardinal is weakly compact.

\begin{remark*}
    Example (30) gives a proof of (2) without additional assumptions. We are going to see a slightly different proof that makes another assumption.
\end{remark*}

\begin{defin*}
    If $X_\alpha \subset \kappa$ are subsets of $\kappa$ for $\alpha < \kappa$, we call the set $$
    \tri_{\alpha < \kappa}X_\alpha \coloneqq \{\gamma < \kappa; \gamma \in \bigcap_{\alpha <\gamma}X_\alpha$$ the \undf{diagonal intersection}.
\end{defin*}
\begin{defin*}[Normal Ultrafilter]
    An ultrafilter $U$ is called \undf{normal} if it is closed under diagonal intersections.
\end{defin*}
\begin{remark*}[Proposition]
    \emph{
        If $U$ is an ultrafilter on $\kappa$ such that all elements of $U$ have size $\kappa$ and $U$ is normal, then $U$ is $\kappa$-complete.
    }
\end{remark*}
\begin{proof}
    Let $X_\alpha (\alpha < \lambda)$ be in $U$. Want to show that $\bigcap_{\alpha < \lambda}X_\alpha \in U$. We define:
    \begin{align*}
        Y_\alpha \coloneqq \left\lbrace \begin{array}{cl}X_\alpha & \alpha < \lambda \\ \kappa & \alpha \ge \lambda \end{array}\right.
    \end{align*}
    Every $Y_\alpha \in U$, so $\tri_{\alpha < \kappa}Y_\alpha \in U$ by normality. Since $\lambda < \kappa$, we have $\lambda \not\in U$, so $\kappa\backslash \lambda \in U$, so $Y\coloneqq (\tri_{\alpha < \lambda}Y_\alpha)\backslash \lambda \in U$.

    Let $\eta \in Y$. Then $\eta > \lambda$, and $\eta \in \bigcap_{\alpha < \eta}Y_\alpha \subset \bigcap_{\alpha < \lambda}Y_\alpha = \bigcap_{\alpha < \lambda}X_\alpha$. So $Y\subset \bigcap_{\alpha < \lambda}X_\alpha$, and hence $\bigcap_{\alpha < \lambda}X_\alpha \in U$.
\end{proof}
\begin{remark*}[Fact (ZFC)]
    If $\kappa$ is measurable, then there is a normal ultrafilter on $\kappa$. [Proof omitted.]
\end{remark*}
\begin{theorem*}[ZFC]
    Measurable cardinals are weakly compact.
\end{theorem*}
\begin{proof}
    Let $U$ be a normal $\kappa$-complete ultrafilter on $\kappa$. Let $c : [\kappa]^2 \ra 2$ be any $2$-colouring. If $\alpha \in \kappa$, we write
    \begin{align*}
        c_\alpha(\beta) \coloneqq \left\lbrace \begin{array}{cl}c(\alpha,\beta) & \alpha \ne \beta\\ 0 & \alpha = \beta \end{array}\right.
    \end{align*}
    Then we define:
    \begin{align*}
        X^0_\alpha &\coloneqq \{\beta ; c_\alpha(\beta) = 0\}\\
        X^1_\alpha &\coloneqq \{\beta ; c_\alpha(\beta) = 1\}
    \end{align*}
    There is $i_\alpha \in \{0,1\}$ such that $X^{i_\alpha}_\alpha \in U$. Then let $I_0 \coloneqq \{\alpha; i_\alpha = 0\}$, and $I_1 = \{\alpha; i_\alpha = 1\}$. Then exactly one of $I_0$,$I_1 \in U$; wlog $I_0 \in U$.

    Define:
    \begin{align*}
        X_\alpha \coloneqq \left\lbrace \begin{array}{cl}X^0_\alpha & \textrm{ if }\alpha \in I_0 \\ \kappa & \textrm{ o/w} \end{array}\right.
    \end{align*}
    By assumption, all $X_\alpha \in U$. Thus $\tri_{\alpha < \kappa}X_\alpha \in U$. And $H\coloneqq I_0 \cap \tri_{\alpha < \kappa}X_\alpha \in U$. If we can show that $H$ is $c$-homogeneous for colour $0$, we are done [since $|H| = \kappa$].

    Let $\alpha < \beta$ , $\alpha,\beta \in H$. Hence $\alpha,\beta \in I_0$, and $X_\alpha = X^0_\alpha,X_\beta = X^0_\beta$. Moreover, $\beta \in \tri_{\delta < \kappa}X_\delta = \{\gamma;\gamma \in \bigcap_{\delta < \gamma}X_\delta\}$, and so $\beta \in \bigcap_{\delta < \beta}X_\delta$. This means that $c_\alpha(\beta) = 0$, which by definition means $c(\alpha,\beta) = 0$.

    Hence for arbitrary $\alpha,\beta\in H$, $(\alpha,\beta)$ has colour zero as required.
\end{proof}

\subsection*{Tree Representations}

We proved that:
\begin{enumerate}[label = (\Roman*)]
    \item $C$ closed $\iff$ there is a tree $T$ on $\omega$ such that $C = [T]$
    \item $A$ analytic $\iff$ there is $T$ tree on $\om\times\om$ such that $A = p[T]$.
\end{enumerate}
We also saw that tree representations of type (I) are important for determinacy proofs:

Gale-Stewart (even without AC): if $T$ is a tree on a wellordered set $X$, then $A\subset X^\om$ with $A = [T]$ is determined.

\underline{Question}: Can we lift a determinacy argument for tree representations of type (I) to type (II)?

\begin{defin*}
    Let $\kappa$ be a cardinal, and $T$ be a tree on $\kappa \times \omega$. [Note that $\kappa\times \om$ is wellordered.] We define $p[T] \coloneqq \{x \in \om^\om; \exists y \in \kappa^\om (y,x) \in [T]\}$. 
\end{defin*}
\begin{remark*}
    This is the same as before, but with general $\kappa$ instead of just $\kappa = \omega$. Indeed, if $\kappa = \aleph_0$, then thi sis exactly the analytic sets [follows from our tree representation of type (II)].
\end{remark*}
\begin{defin*}
    If $\kappa$ is a cardinal and $A\subset \om^\om$, we say $A$ is \undf{$\kappa$-Suslin} if there is a tree $T$ on $\kappa \times \om$ such that $A = p[T]$.
\end{defin*}
By the remark, being $\aleph_0$-Suslin is equivalent to being analytic.

\underline{Hope}: Prove a Gale-Stewart type theorem for $\kappa$-Suslin sets.

How might this work? The technique here is that of\dots

\subsection*{Auxiliary Games}
\begin{defin*}[Auxiliary Game]
    If $A$ is $\kappa$-Suslin, say $A = p[T]$ for some $T$ on $\kappa \times \omega$, we define the \undf{auxiliary game} $G_\textrm{aux}(T)$ as follows:
    \begin{center}
        \begin{tabular}{c|ccccccc}
            I & $\alpha_0,x_0$ & & $\alpha_1,x_2$ & & $\alpha_2,x_4 $& & $\dots $\\ \hline
            II & & $x_1$ & & $x_3 $& &$ x_5$ & $\dots$ 
        \end{tabular}
    \end{center}
    with $\alpha_i \in \kappa,x_i \in \om$. We write $y(i) = \alpha_i,x(i) =x_i$. Then we have $y  \in \kappa^\om,x \in \om^\om$. Player I wins if $(y,x) \in [T]$.
\end{defin*}

We consider the relationship between the games $G(A)$ and $G_\aux(T)$.

Suppose player I wins $G_\aux(T)$ by $\sigma$. Then when playing $G(A)$, player I secretly plays another game of $G_\aux(A)$ and uses $\sigma$ in that game to produce their move back in $G(A)$. This produces a run $x$ such that $(y,x) \in [T]$ for some $y\in \kappa^\om$. Hence this produces a winning strategy $\sigma^\ast$ for $G(A)$.

What about the other direction? Say, if player II wins in $G_\aux(T)$? Not quite as simple.

\marginpar{Lecture 19}

We are looking to show $\Det(\bopi^1_1)$.

\underline{Rough idea}: Fix $\kappa$ measurable. Then:
\begin{enumerate}[label = (\alph*)]
    \item show that every $\bopi^1_1$ set is $\kappa$-Suslin
    \item Use measurability to prove the translation for player II
    \item Use determinacy of $G_\aux (T)$ to get determinacy of $G(A)$
\end{enumerate}
Is it possible that this is an accurate description? The answer is no: if every set is $\kappa$-Suslin (for $\kappa$ measurable) is determined, then this proof would give much more, namely AD. This cannot be the case.

Indeed, suppose that if $A$ is $\kappa$-Suslin and $\kappa$ is measurable then $A$ is determined ($\ast$). Observe that if $\lambda < \kappa$ and $A$ is $\lambda$-Suslin then $A$ is $\kappa$-Suslin. Also, every set $A$ is $2^{\aleph_0}$-Suslin [ES\#3].

If $\kappa$ is measurable, then $2^{\aleph_0} < \kappa$, so every set is $\kappa$-Suslin. So if ($\ast$) is true in this abstract form, then every set is determined. We conclude: it can't be just the fact that $A$ is $\kappa$-Suslin, but we must require some properties of the tree $T$ that witnesses $A = p[T]$.

We will use our structure theory for $\bopi^1_1$ sets in order to define a tree on $\kappa\times \omega$ for $\bopi^1_1$ sets.

\begin{remark*}[Roadmap]
    If $A \in \bopi^1_1$, there is a tree $T$ on $\om\times\om$ such that:
    \begin{align*}
        x \in A &\iff T_x \textrm{ is wellfounded}\\
        &\iff (T_x,\supsetneq)\textrm{ is wellfounded}\\
        &\iff \textrm{ there is an order-preserving map from }(T_x,\supsetneq)\textrm{ into }(\om_1,<)
    \end{align*}
    If we code $T_x$ by some bijection $\om^\lom \xleftrightarrow{} \om$ then the o.p. map is essentially an element of $\om_1^\om$.

    \underline{FIND}: tree $\hat{T}\subset (\om_1\times\om)^\lom$ such that if $(y,x)\in [\hat{T}]$, then $y$ is an order-preserving map from $T_x$ into $\om_1$ [up to coding]. Then: $x \in A \iff \exists y((y,x)\in[\hat{T}])$. We can give an informal definition of $\hat{T}$: $(u,s)\in \hat{T}$ where $v \in \om_1^\lom,s\in\om^\lom$ if and only if $u$ is consistent with being an initial segment of an order-preserving map from $T_x$ into $\om_1$. [where $x$ is any extension of $s$].
\end{remark*}

\begin{theorem*}[Shoenfield]
    Every $\bopi^1_1$ set is $\kappa$-Suslin (for every $\kappa \ge \om_1$).
\end{theorem*}
\begin{proof}
    Fix $A$ $\bopi^1_1$, and fix $T$ such that $x \in A \iff T_x$ is wellfounded. Also fix $i \mapsto s_i$ a bijection between $\om$ and $\om^\lom$, with the property that if $s_i\subset s_j$ then $i\le j$. [This in particular implies that $\lh(s_i)\le i$.] Recall that $T_x = \{t \in \om^\lom; (t,x\restriction \lh(t))\in T\}$. Define $T_s \coloneqq \{t \in \om^\lom; (t,s\restriction \lh(t))\in T\}$. $T_s$ is then a tree that has finitely many levels; if $t\in T_s$ then $\lh(t)\le \lh(s)$. But of course $T_s$ could still be an infinite set.

    Write $K_x \coloneqq \{i \in \om; s_i \in T_x\}$, and $K_s \coloneqq \{i\in \om; i\le \lh(s)\textrm{ and }s_i \in T_s\}$. Then $K_s$ is \it{finite}! Note that we have $T_x = \{s_i;i\in K_x\}$, but in general $T_s\supsetneqq \{s_i;i\in K_s\}$, but $K_x = \bigcup_{n\in\N}K_{x\restriction n}$.

    Observe that $M \coloneqq \{p;p\textrm{ is a partial function }\om\ra\kappa\textrm{ with finite domain}\}$ has cardinality $\kappa$, so it is sufficient to provide a tree $\hat{T}$ on $M\times \om$ with the desired properties (rather than on $\kappa\times\om$). We can define a linear order on $\kappa^\lom$ that extends the order $\supsetneqq$, known as the \undf{Kleene-Brouwer order}:
    \begin{align*}
        s\lkb t :\iff& t\supsetneqq s\textrm{ or there is }i \in \dom(s)\cap\dom(t)\textrm{ such that }\\
        &s(i)\ne t(i);\textrm{ let }i\textrm{ be least such AND }s(i) < t(i)
    \end{align*}
    It is strict extension if it applies and lexicographic otherwise. Intuitively, consider $x_s : \om\ra \kappa$ defined by
    \begin{align*}
        x_s(i) \coloneqq \left\lbrace \begin{array}{cl}s(i) & i\in\dom(s)\\ \infty &\textrm{o/w} \end{array}\right.
    \end{align*}
    and order these lexicographically. Clearly, $\lkb$ extends $\supsetneqq$ and is total.

    \underline{Fact}: If $T$ is a tree then $(T,\supsetneqq)$ is wellfounded $\iff$ $(T,\lkb)$ is a wellorder [ES\#3].

    Let's put all this together. Define $\hat{T} \subset (M\times\om)^\lom$. If $u \in M^\lom$ and $s \in \om^\lom$, we say that $u$ is \undf{coherent with $s$} if:
    \begin{enumerate}[label = (\arabic*)]
        \item $\lh(u)\le\lh(s)$
        \item $i < \lh(u)\implies \dom(u(i)) = K_{s\restriction i}$
        \item $i < \lh(u) \implies u(i) : (K_{s\restriction i},\lkb)\ra \kappa$ is order-preserving
        \item $i\le j \implies u(i)\subset u(j)$
    \end{enumerate}
    Then $\hat{T}\coloneqq \{(s,u);u\textrm{ is coherent with }s\}$ is the \undf{Shoenfield tree}.

    \underline{Claim}: $A = p[\hat{T}]$.

    \underline{$\subset$}: Suppose $x \in A$. By assumption, $T_x$is wellfounded, so $(T_x,\lkb)$ is a wellorder so there is an order-preserving function $g: (T_x,\lkb)\ra(\kappa,K)$. Translate into $K_x$ by $h:(K_x,<_x)\ra(\kappa,<)$ where $i<_xj :\iff s_i \lkb s_j$. This is order-preserving.

    Define $u(i)\coloneqq h\restriction K_{x\restriction i}$. Then clearly conditions (2), (3), and (4) are true. So for every $i$, $(u\restriction i,x\restriction i)\in \hat{T}$. So $(u,x)\in [\hat{T}]$. So $x \in p[\hat{T}]$.

    \marginpar{Lecture 20}

    \underline{$\supseteq$}: Let $x \in p[\hat{T}]$. We have the characterisation:
    \begin{align*}
        x \in p[\hat{T}] \iff & \textrm{ there is }u\textrm{ s.t. }(u,x)\in[\hat{T}]\\
        \iff&\textrm{ there is }u\textrm{ s.t. }\forall n\ (u\restriction n,x\restriction n)\in\hat{T}\\
        \iff&\textrm{ there is }u\textrm{ s.t. }\forall n\ u\restriction n\textrm{ is coherent with }x\restriction n
    \end{align*}
    This means $u(i) : K_{x\restriction i}\ra \kappa$ order-preserving (properties (1)/(2)), and $u(i)\subset u(j)$ for $i\le j$ (property (3)).

    By (3), define $\hat{u} : K_x \ra \kappa$ by $\hat{u} = \bigcup_{i\in\N}u(i)$. If $\hat{u}$ is order-preserving from $(K_x,\lkb)$ into $(\kappa,<)$, then $T_x$ is wellfounded and thus $x \in A$. So, we're left to show that $|hat{u}$ is order-preserving.

    Suppose not. Then there aer $i,j\in K_x$ such that $s_i \lkb s_j$ but $\hat{u}(i)\not<\hat{u}(j)$. [Since $K_x = \bigcup_{n\in\N}K_{x\restriction n}$, there is $n\in\N$ such that $i,j \in K_{x\restriction n}$.] But $u(n)$ was order-preserving, so $s_i \lkb s_j \iff u(n)(i) < u(n)(j)$, but by definition of $\hat{u}$ this is a contradiction.

    Thus $\hat{u}$ is order-preserving, and so $x \in A$.
\end{proof}

\begin{theorem*}[Martin, 1969/70]
    If there is a measurable cardinal, then all $\bopi^1_1$ sets are determined.
\end{theorem*}
\begin{defin*}
    Recall: $\kappa$ is measurable $\iff$ there is $U$ a $\kappa$-complete, non-principal ultrafilter on $\kappa$.
\end{defin*}
\begin{theorem*}[ZFC, proof omitted]
    If $\kappa$ is measurable, then there is a normal $\kappa$-complete non-principal ultrafilter on $\kappa$.
\end{theorem*}

\begin{theorem*}[Rowbottom, ES\#3]
    If $U$ is normal on $\kappa$ and $\gamma < \kappa$ and $c_n : [\kappa]^\lom \ra \gamma$ a colouring for each $n$, then there is a set $H\in U$ such that for all $n,k\in\N$, $c_n\restriction [H]^k$ is constant.
\end{theorem*}

We say $\kappa$ \undf{satisfies Rowbottom's Theorem} if for all $\gamma < \kappa$, $c_n:[\kappa]^\lom \ra \gamma$ there is $H$ with $|H| = \kappa$ such that for all $n,k\in\N$, $c_n\restriction [H]^k$ is constant.

\begin{remark*}[Summary]
    ZFC implies that measurable cardinals satisfy Rowbottom's Theorem.
\end{remark*}

\begin{theorem*}[Martin]
    If $\kappa$ satisfies Rowbottom's Theorem, then all $\bopi^1_1$ sets are determined.
\end{theorem*}
\begin{proof}
    Fix $A\in \bopi^1_1$. By Shoenfield, we have tree $\hat{T}$ on $M\times\om$ such that $A = p[\hat{T}]$.

    We know that if you compare $G(A)$ and $G_\aux(\hat{T})$, we get:
    \begin{itemize}
        \item $G_\aux(\hat{T})$ is determined [Gale-Stewart]
        \item winning strategies of player I transfer from $G_\aux(\hat{T})$ to $G(A)$
    \end{itemize}
    So we still need to show that if player II has a winning strategy in $G_\aux(\hat{T})$, then player II has a winning strategy in $G(A)$ ($\ast$).

    Our game looks like:
    \begin{center}
        \begin{tabular}{c|ccccccc}
            I & $u_0,x_0$ & & $u_1,x_2$ & & $u_2,x_4 $& & $\dots $\\ \hline
            II & & $x_1$ & & $x_3 $& &$ x_5$ & $\dots$ 
        \end{tabular}
    \end{center}

    We can assume that $u_0$ is played such that player I does not lose immediately. So $u:K_{x\restriction 0} \ra \kappa$ order-preserving, and $K_{x\restriction 0} = K_\emptyset = \{i\le 0; s_i \in T_{x\restriction 0}\} = \{0\}$. So $u_0$ just picks out a point in $\kappa$.

    Going one step further, we have $u_1 : K_{x\restriction 1} \ra \kappa$, and $K_{x\restriction 1} = K_{\langle x_0 \rangle} = \{i \le 1; s_i \in T_{x\restriction 1}\}$, which is either the set $\{0\}$ or $\{0,1\}$; the latter occurs $\iff$ $s_1 \in T_{x_0}$. So we can identify $u_1$ \it{either} with a single element of $\kappa$, \it{or} a $2$-element subset of $\kappa$, where the lower element is the image of the element that is least with respect to $\lkb$.

    Let's formalise this. Let $k_s \coloneqq |K_s|$ and $Q \in [\kappa]^{k_s}$. Then $Q$ determines a unique order-preserving map
    \begin{align*}
        w_Q :(K_s,\lkb)\ra (Q,<)
    \end{align*}
    Define $u^{s,Q}(i) \coloneqq w\restriction K_{s\restriction i}$. Then $u^{s,Q}$ is coherent with $s$. This allows us to tranaslate a position in the original game
    \gamec{x}{x}
    into a (legal) position in the auxiliary game:
    \begin{center}
        \begin{tabular}{c|ccccccc}
            I & $u^{s,Q}(0),x_0$ & & $u^{s,Q}(1),x_2$ & & $u^{s,Q}(2),x_4 $& & $\dots $\\ \hline
            II & & $x_1$ & & $x_3 $& &$ x_5$ & $\dots$ 
        \end{tabular}
    \end{center}
    We call this entire position $s^Q_\ast$, and it is a position in $M^\lom \times \om^\lom$.

    Define a colouring $c_s:[\kappa]^{k_s} \ra \om$ by, for $Q \in [\kappa]^{k_s}$, $c_s(Q) = \tau(s^Q_\ast)$, where $\tau$ is our winning strategy for player II in $G_\aux(\hat{T})$. This is a family of countably many colourings, so by Rowbottom we find $H\subset \kappa$, $|H| = \kappa$ such that $c_s\restriction [H]^{k_s}$ is constant. [\it{i.e.} if $Q,Q'\in [H]^{k_s}$, then $c_s(Q) = c_s(Q')$, so $\tau(s^Q_\ast) = \tau(s^{Q'}_\ast)$. In words, if player II fills in the gaps by \it{any} $Q \in [H]^{k_s}$, then the answer of $\tau$ does not depend on the precise choice of $Q$.]

    So, let $Q_{H,s} \coloneqq$ the first $k_s$ many elements of $H$. Define strategy $\tau_H$ in the game $G(A)$ for player II:
    \begin{center}
        \begin{tabular}{c|ccccccc}
            I & $u^{s,Q_{H,s}}(0),x_0$ & & $u^{s,Q_{H,s}}(1),x_2$ & & $u^{s,Q_{H,s}}(2),x_4 $& & $\dots $\\ \hline
            II & & $x_1$ & & $x_3 $& &$ x_5$ & $\dots$ 
        \end{tabular}
    \end{center}
    by having player II fill in the auxiliary moves for player I as above, where $\tau_H(s)\coloneqq \tau(s^{Q_{H,s}}_\ast)$.

    \underline{Claim}: If $\tau$ was winning in $G_\aux(T)$, then $\tau_H$ is winning in $G(A)$.

    \marginpar{Lecture 21}

    \underline{Proof of Claim}: Suppose not. So there is a $\sigma$ for player I such that $x \coloneqq \sigma \ast \tau_H \in A \iff T_x$ is wellfounded $\iff$ ther is an order-preserving map $g : (K_x,<_x) \ra (H,<)$ [since $H$ is uncountable].

    Consider $u_i \coloneqq g\restriction K_{x\restriction i}$. Then
    
    \begin{center}
        \begin{tabular}{c|ccccccc}
            I & $u_0,x_0$ & & $u_1,x_2$ & & $u_2,x_4 $& & $\dots $\\ \hline
            II & & $x_1$ & & $x_3 $& &$ x_5$ & $\dots$ 
        \end{tabular}
    \end{center}

    is a run of $G_\aux(\hat{T})$, producing $(g,x)\in [\hat{T}]$. In particular, this run is a win for player I in the auxiliary game. We'll show that this is a run according to the strategy $\tau$ - but $\tau$ was a winning strategy for player II in $G_\aux(\hat{T})$, contradiction.

    Let $s = (x_0,x_2,\dots,x_{2n}) = x\restriction 2n+1$, and write $s^g_\ast$ for $(u_0,x_0,x_1,u_1,x_2,x_3,u_2,x_4,x_5,\dots,u_n,x_{2n})$. We need to show that $\tau(s^g_\ast) = x_{2n+1}$. We have:
    \begin{align*}
        x_{2n+1} &= \tau_H(x\restriction 2n+1)\textrm{ by chioce of }x\\
        &= \tau(s^{Q_{s,H}}_\ast)\textrm{ by definition of }\tau_H\\
        &= \tau(s^g_\ast)
    \end{align*}
    by the fact that range$(g)\subset H$, and $H$ is homogeneous.

    This proves the claim, and hence the theorem.
\end{proof}

\begin{remark*}
    Note that we did not need that $H$ has size $\kappa$, only that it is uncountable. We don't need the full strength of Rowbottom's Theorem, but only:
    
    ``For every countable collection $c_n:[\kappa]^\lom \ra \om$ there is an uncountable simultaneously homogeneous set $H$''.

    \underline{Compare}: $\om_1$-Erd\H{o}s Cardinals.

    This is part of the story as to why we can't have a converse to Martin's Theorem.

    The precise strength of analytic determinacy is called \undf{zero sharp}. Zero Sharp is an unusual large cardinal axiom which doesn't talk about large cardinals, but rather about a real number which codes truth functions for inner models.

    For large cardinal notions of this type, we can have real equivalences:

    \begin{theorem*}[ZFC]
        $\det(\bosig^1_1)\iff$ for all $x \in \om^\om$, the object $x^{\#}$ exists.
    \end{theorem*}
\end{remark*}

Our final goal for this course is to prove:

\begin{theorem*}[Solovay]
    If $\ad$ holds, then $\aleph_1$ is a measurable cardinal.
\end{theorem*}

[This implies that there is an inner model $M$ such that $M\models \zfc +$ there is a measurable cardinal. This requires techniques we didn't do in this course.]

\underline{Note}: ``being measurable'' only implies ``being large'' if it implies inaccessibility. But recall the proof of ``measurable $\implies$ inaccessible'' - this required that $2^\lambda$ [$\lambda < \kappa$] is wellorderable. In our setting above $2^{\aleph_0}$ is clearly not wellorerable (since if it is, we can find a non-determined set).

Working in ZFC + AD means that we have to be very cautious about almost everything that we have done so far. The most important fragment of AC that we used in our account of descriptive set theory was $\ac_\om(\R)$, equivalently $\ac_\om(\om^\om)$. In fact, everything we've done on Borel and projective sets [with the exception of Martin's Borel determinacy] can be done in ZF + $\ac_\om(\R)$.

Recall ES\#1 (4): $\zf + \ac_\om(\R)\implies \aleph_1$ is regular, and the proof of $\bosig^0_{\om_1} = \bopi^0_{\om_1}$ used exactly that.

\begin{remark*}[Proposition]
    \emph{
        $\ad\implies\ac_\om(\om^\om)$
    }
\end{remark*}
\begin{proof}
    Suppose $(X_n;n\in\N)$ is a sequence of non-empty subsets of $\om^\om$. Consider the game:
    \gamec{x}{x}
    and say that player II wins if $x_\II\in X_{x_0}$.

    Player I cannot have a winning strategy; the $x_i;i\ge 1$ have no effect, so winning strategy would essentially just be a natural number $x_0$. If $y \in X_{x_0}$ and $\tau_y$ is the blindfolded strategy ``play $y$'', then $x_0 \ast \tau_y$ is a win for player II.

    So, by AD, player II has a winning strategy $\tau$. Consider $(k\ast \tau)_\II\in X_k$. Then $k\mapsto (k\ast\tau)_\II$ is a choice function for $(X_n;n\in\N)$.
\end{proof}
\begin{remark*}
    This proof generalises to $\ad_M \implies \ac_M(M^\om)$, where $\ad_M$ is the assertion that for all $A \subset M^\om$, the game $G(A)$ is determined.

    In particular, $\ad_\R \implies \ac_\R(\R)$. $\ac_\R(\R)$ is the uniformisation principle, ES\#1 (5).
\end{remark*}

\begin{remark*}
    What about ultrafilters?

    If $X$ is any set and $a \in X$, then $\{A\subset X;a \in A\}$ is a principal (and therefore $\kappa$-complete for all $\kappa$) ultrafilter - this is true in ZF without choice.

    But the proof that arbitrary filterson $X$ can be extended to ultrafilters is an application of Zorn's Lemma, so not in general a ZF-theorem.

    \underline{Question}: Is there a non-principal ultrafilter on $\N$?

    The answer is in fact no. The theorem we will prove next time is:

    \begin{theorem*}
        $\ad\implies$ there are no non-principal ultrafilters on $\N$.
    \end{theorem*}

    This implies that every ultrafilter is $\aleph_1$-complete. So we are left with showing that there exists a non-principal ultrafilter on $\aleph_1$.
\end{remark*}

\marginpar{Lecture 22}

\underline{Zorn's Lemma}
\begin{itemize}
    \item If $F$ contains the complements of the singletons, then $U\supseteq F$ ultra cannot be principal
    \item So if $F\coloneqq\{A\subset X; X\backslash A\textrm{ is finite}\}$ is the \undf{Frechet filter (cofinite filter)}, then $U$ is a non-principal ultrafilter
\end{itemize}

Suppose $\pi:X\ra Y$ any map and $F$ is a filter on $X$, we can define $\pi_\ast F\coloneqq \{A\subset Y; f^{-1}[A]\in F\}$, the \undf{image filter}. Indeed, it is a filter:

\begin{itemize}
    \item $f^{-1}[\emptyset] = \emptyset \not\in F$
    \item $f^{-1}[Y] = X\in F$
    \item $f^{-1}[A\cap B] = f^{-1}[A] \cap f^{-1}[B]$
    \item $A\subset B\implies f^{-1}[A]\subset f^{-1}[B]$
\end{itemize}

Furthermore:
\begin{itemize}
    \item If $F$ was ultra, then $\pi_\ast F$ is ultra since $f^{-1}[X\backslash A] = Y\backslash f^{-1}[A]$
    \item If $F$ was $\lambda$-complete, then so is $\pi_\ast F$, since $f^{-1}[\bigcap_{\alpha < \gamma}X_\alpha] = \bigcap_{\alpha < \gamma}f^{-1}[X_\alpha]$
\end{itemize}

In general, non-principality is \it{not} preserved by images:

If $\pi$ is a constant function $\pi(x) = a\in Y$, all $x \in X$, then $\pi_\ast F = \{A\subset Y;\pi^{-1}[A]\in F\}$, and $\pi^{-1}[A]$ is either $X$ or $\emptyset$, depending on whether or not $a \in A$, so $\pi_\ast F = \{A\subset Y;a \in A\}$.

\begin{remark*}
    The main obstacle later is to show that non-principality of images of ultrafilters.
\end{remark*}

\begin{remark*}[Lemma]
    \emph{
        If there is an ultrafilter $U$ on $X$ that is not $\aleph_1$-complete, then there is a non-principal ultrafilter on $\N$.
    }
\end{remark*}
\begin{proof}
    By assumption, $U$ is not $\aleph_1$-complete, so we can write $X = \bigcup_{n\in\N}x_n$ such that the $X_n$ are pairwise disjoint and $X_n \not\in U$ for all $n\in\N$.

    We then take $\pi : X \ra \N$ by $x\mapsto n$ if $n\in X_n$. By our general theory of image filters, we get that $\pi_\ast U$ is an ultrafilter on $\N$.

    \underline{Claim}: $\pi_\ast U$ is non-principal.

    \underline{Proof of Claim}: Suppose not. Then there exists $\{k\}\in \pi_\ast U$. So $\pi^{-1}[\{k\}] \in U$, hence $X_k \in U$, contradiction.
\end{proof}
\begin{remark*}[Corollay]
    \emph{
        ZF + there is no non-principal ultrafilter on $\N$ $\implies$ every ultrafilter is $\aleph_1$-complete.
    }
\end{remark*}
We can combine this with our next theorem:
\begin{theorem*}
    AD $\implies$ there is no non-principal ultrafilter on $\N$.
\end{theorem*}
From which we can draw the conclusion that:
\begin{remark*}[Corollary]
    \emph{
        AD $\implies$ every ultrafilter is $\aleph_1$-complete.
    }
\end{remark*}
\begin{proof}
    We will be using the technique of \undf{strategy stealing}: assume I/II has a winning strategy, and use it to define a winning strategy for II/I.

    So assume $U$ is a non-principal ultrafilter on $\N$ and define a non-determined game as follows:

    We call the game $G_U$. Moves are $s \in [\N]^\lom$ [use definable bijection between $\N$ and $[\N]^\lom$ to see that $\ad$ implies the determinacy of games like this].

    \gamed{s}

    We have an additional rule: $s_n$ has to be disjoint from $\bigcup_{i < n}s_i$. If you are the first one to break the rule, you lose.
    
    The game produces sets $A_\I \coloneqq \bigcup_{i\in\N}s_{2i}$ and $A_\II \coloneqq \bigcup_{i\in\N}s_{2i+1}$. If no breaks the rule, then $A_\I \cap A_\II = \emptyset$ and say player I wins if $A_\I \in U$. Note that if $\II$ wins, it means that $\N\backslash A_\I \in U$, but not necessarily that $A_\II \in U$.

    \underline{Case 1}: Suppose player II wins $G_U$ by a strategy $\tau$. Let player I steal that strategy. We again use the method of auxiliary games.

    An initial attempt might be for player I to pretend that II has played $\emptyset$ in the back room, and play $\tau$ against that, so $\tau(\emptyset) = s_0$, and so on. Then in the original game we have sets $A_\I$, $A_\II$ and in the back room we have $A_\I^\ast$ and $A_\II^\ast$. Then we have:

    $A_\I = \bigcup_{i\in\N}s_{2i} = \bigcup_{i\in\N}s_{2i} = A^\ast_\II$, and $A_\II = \bigcup_{i\in\N}s_{2i+1} = \emptyset \cup \bigcup_{i\in\N}s_{2i+1} = A^\ast_\I$. Since $\tau$ is winning for $\II$, we have that $A_\I^\ast \not\in U$, but we do not in general have that $A^\ast_\II \in U$.

    Second attempt: pretend II has played $\emptyset$, and $\tau$ gives $\tau(\emptyset) = s_0$. Player I then plays $s_0^\ast = s_0 \cup\{1\}$. Player II responds with $s_1$, and then we find $s_2 = \tau(\emptyset,s_0,s_1)$, and define $s_2^\ast$ as:

    \begin{align*}
        s_2^\ast \coloneqq \left\lbrace \begin{array}{cl}s_2\backslash \{0\} & 1 \in s_0 \cup s_1\\ (s_2\backslash \{0\})\cup \{1\} & 1\not\in s_0\cup s_1\\ \end{array}\right.
    \end{align*}
    and in general, we have:
    \begin{align*}
        s_{2n}^\ast \coloneqq \left\lbrace \begin{array}{ll}s_{2n}\backslash \{0,\dots,n-1\} & n \in \bigcup_{k<2n}s_k\\ (s_{2n}\backslash \{0,\dots,n-1\})\cup \{n\} & n\not\in \bigcup_{k<2n}s_k\\ \end{array}\right.
    \end{align*}

    Now $A_\I \cup A_\II = \N$, so $A_\I = \N\backslash A_\II$. Again, since $\tau$ is winning, $A^\ast_\I \not\in U$, but $A^\ast_\I = A_\II$. Together these imply that $A_I \in U$. Hence this is a winning strategy for player I.

    To summarise: if player II has a winning strategy, then we can modify it to a winning strategy for player I. So II does not have a winning strategy. Note even in ZFC player II cannot win this game.

    \underline{Case 2}: If $\sigma$ is a winning strategy for player I in $G_U$, construct a winning strategy for II. Once more, we use the method of auxiliary games.

    Suppose player I leads with $s_0$. In the back room, $\sigma$ tells II to play $\sigma(\emptyset) = s_1$ (which is independent of $s_0$). But this might overlap, so we define $s_1^\ast = s_1\backslash s_0$ and II plays that against I in the real game. II gets back the move $s_2$ from I. We then have $\sigma(s_1,s_2) = s_3$, and define $\sigma^\ast_3 \coloneqq s_3 \backslash s_0$. Indeed, we in general have $s^\ast_{2i+1}\coloneqq s_{2i+1}\backslash s_0$.

    Then $A_\II = \bigcup_{i\in\N}s_{2i+1}^\ast = \left(\bigcup_{i\in\N}s_{2i+1}\right)\backslash s_0 = A^\ast_\I \backslash s_0$. Since $\tau$ is winning, $A^\ast_\I \in U$. By non-principality, of $U$, $A_\II \in U$. So $A_\I\not\in U$. Thus II was in $G_U$.
\end{proof}

\marginpar{Lecture 23}

Still need to show that there is a non-principal ultrafilter on $\aleph_1$.

\underline{Basic idea}: Have an ultrafilter $U$ on $X$ and function $f:X\ra\aleph_1$ and consider $f_\ast U$, which is an ultrafilter on $\aleph_1$. Filter properties get transferred \it{except} for non-principality, so we will need to do extra work to show this.

Our first idea is to consider $f : \wf \ra \aleph_1$, $x\mapsto ||x||$. If we could use $\ad$ to define a game on $\wf$ such that it defines an ultrafilter $U$ on $\wf$ and $f_\ast U$ is non-principal, we would be done.

This is Solovay's original idea, and will be proven on ES\#4.

We're instead going to see a slightly different proof, due to Tony Martin.

Consider $V_{\om+1}$, the $(\om+1)^\th$ level of the von Neumann hierarchy: $\P V_\om$. Elements of $V_{\om+1}$ are:
\begin{enumerate}
    \item If $x \in \om^\om$, then $x \in V_{\om+1}$ since elements of $x$ are pairs $(n,m)\in \om\times\om$, which set theoretically is $\{\{n\},\{n,m\}\}$ which has rank $\le \max(n+2,m+2)\in\N$. Hence if $p \in x$, then $p \in V_\om$ so $x \in \P V_\om = V_{\om+1}$.
    \item Similarly $\om^\lom$, but also \it{strategies}. A strategy $\sigma : \om^\lom \ra \om$ has elements $p \in \sigma$ of the form $p = (s,u)$, with $s,u \in V_\om$. So by the same argument $p \in V_\om$ and hence $\sigma \in \P V_\om = V_{\om+1}$.
\end{enumerate}

\begin{defin*}
    Let $x,y\in V_{\om+1}$. Define $x\led y$ if there is a formula $\phi$ such that $(V_{\om+1},\in)\models \phi(w,y) \iff w \in x$. That is to say, \undf{$x$ is definable from $y$}.
\end{defin*}
\begin{remark*}[Properties]\ 
    \begin{enumerate}
        \item $\led$ is reflexive: $x\led x$ by $\phi(w,z)\coloneqq w\in z$
        \item $\led$ is transitive: $x\led y$ and $y\led z$, witnessed by formulae $\phi,\psi$ respectively.
        
        Then $w \in y \iff V_{\om+1}\models \phi(w,y)$, and $w \in y \iff V_{\om+1}\models \psi(w,z)$.

        Our formula is then $\exists u\forall v(v\in u \leftrightarrow \psi(v,z))\land \phi(w,u)$. This formula defines $x$ from $z$. Hence $x\led z$.

        \item If $x \in V_{\om+1}$, then $\{y \in V_{\om+1};y\led x\}$ is countable, since there are only countably many formulae witnessing $\led$.
        \item This means that $\led$ has no largest element.
        \item If $x$ is definable without parameters, \it{i.e.} $w \in x \iff V_{\om+1} \models \phi(w)$, then for any $y$, $x\led y$. Thus these objects are all minimal in $\led$.
        \item $\led$ is \it{not} antisymmetric: if $x,y$ are both definable without parameters, but $x\ne y$, then by 5. $x\led y$ and $y\led x$ but $x\ne y$. For example, $\emptyset,\{\emptyset\}\in V_{\om+1}$:
        \begin{itemize}
            \item $w \in \emptyset \iff V_{\om+1}\models w\ne w$
            \item $w \in \{\emptyset\} \iff V_{\om+1}\models \forall v(v\not\in w)$
        \end{itemize}
    \end{enumerate}
\end{remark*}

\begin{defin*}[Preorder]
    A relation $\le$ on $X$ is called a \undf{preorder} if $\le$ is reflexive and transitive.

    If $\le$ is a preorder, then $x \equiv y :\iff (x\le y)\land (y\le x)$ is an equivalence relation. Consider $Q \coloneqq X/\equiv$ and define $\le$ on $Q$ by $[x]_\equiv \le [y]_\equiv :\iff x\le y$.

    We need to check that this is well-defined:

    If $x\equiv x',y\equiv y',x\le y$, then we have $x'\le x \le y\le y'$ so by transitivity $x'\le y'$.
\end{defin*}


Applied to $\led$, we have that $(V_{\om+1}/\equiv_D,\led)$ is a partial order. We saw that $\om^\om \subset V_{\om+1}$, so restricting to Baire space we have $\mathcal{D}_D \coloneqq (\om^\om/\equiv_D,\led)$, the partical order of \undf{definability degrees}.

[$x\led y \iff $ there is $\phi$ such that $w\in x\iff V_{\om+1}\models \phi(w,y)$. If $x,y\in\om^\om$, this makes sense.]

If $x,y\in\om^\om$, we define $x\ast y(k)$ by:
\begin{align*}
    x\ast y(k)\coloneqq \left\lbrace \begin{array}{cl}x(n) & 2n = k\\ y(n) & 2n+1 = k \end{array}\right.
\end{align*}
If $\sigma_x,\sigma_y$ are the blindfolded strategies for $x,y$, then $\sigma_x\ast \sigma_y = x\ast y$. Also recall the inverses of interleaving, $x\mapsto x_\I$ and $x\mapsto x_\II$. For every $x$, $x_\I \led x$ and $x_\II \led x$ since they are definable subsequences.

Thus $x\led x\ast y$ and $y\led x\ast y$. In fact we will see that $x\ast y$ is the least upper bound of $x$ and $y$ in $\led$.

\begin{thmenv*}[Proposition]
If $x,y\in \om^\om$ and $z$ is such that $x\led z$ and $y\led z$ then $x\ast y\led z$.
\end{thmenv*}
\begin{proof}
    We have:
    \begin{enumerate}[label=(\arabic*)]
        \item $w\in x \iff V_{\om+1}\models \phi(w,z)$
        \item $w \in y \iff V_{\om+1}\models \psi(w,z)$
    \end{enumerate}
    Now we have:
    \begin{align*}
        w = (n,\ell) \in x\ast y \iff& [\exists n,\ell [\phi((n,\ell),z)\land w = (2n,\ell)]]\\
        \lor& [\exists n,\ell [\psi((n,\ell),z)\land w = (2n+1,\ell)]]
    \end{align*}
    and the RHS is the formula we require.
\end{proof}

The operation $x,y\mapsto x\ast y$ is also known as the \undf{Turing Join} of $x$ and $y$.

Consider the structure of $(\D_D,\led)$. It has a least element (the equivalence class of definable sets), but no largest element: For some $x$ in the order we have that $\{y;y\led x\}$ is countable. On the other end, we have $\{y; x\led y\}$ - what do we know about this? It isn't necessarily the complement, but we can deduce some information.

Consider $\{y\in \om^\om;x\led y\}$, the \undf{cone of $x$}. If $z\ne z' \in \om^\om$, we get that $x\led x\ast z$ and $x\led x\ast z'$ and $x\ast z \ne x \ast z'$, so $|\cone(x)| = 2^{\aleph_0}$.

This idea gives rise to the following definition:

\begin{defin*}[Cone Filter]
    We define the \undf{cone filter $F_D$} on $\om^\om$ by $A \in F_D$ $\iff$ there is an $x \in \om^\om$ such that $\cone(x) \subset A$.

    Let's check the filter properties.

    \begin{itemize}
        \item Clearly $\emptyset \not\in F_D$.
        \item Clearly $\om^\om \in F_D$.
        \item If $A \in F_D$ and $B \supseteq A$ then $B \in F_D$.
        \item Let $A,B \in F_D$. Then there are $x,y$ such that $\cone(x) \subset A$ and $\cone(y) \subset B$. Define $z \coloneqq x \ast y$.
        
        \underline{Claim}: $\cone(z)\subset A\cap B$.

        \underline{Proof of Claim}: Let $w \in \om^\om$.
        \begin{align*}
            z\led w \implies &x\led z \led w\\
            \implies & w \in \cone(x)\\
            \implies &w \in A\\
            \textrm{and } z\led w \implies&y\led z \led w\\
            \implies & w \in \cone(y)\\
            \implies & w \in B
        \end{align*}
        So $w \in \cone(z) \implies w \in A \cap B$. So $\cone(z) \subset A\cap B$.
    \end{itemize}

    \textbf{Summary}: $F_D$ is a filter on $\om^\om$.
\end{defin*}

\marginpar{Lecture 24}

If $Z\subset \D_D$, then $\bigcup Z \subset \om^\om$. The set $\bigcup Z$ is \undf{$\equiv_D$-invariant}, \it{i.e.} if $x \in \bigcup Z$ and $y \equiv_D x$ then $y \in \bigcup Z$. Being $\equiv_D$-invariant just means being a union of $\equiv_D$-equivalence classes.

The operation $Z \mapsto \bigcup Z$ can be reversed (on the $\equiv_D$-invariant sets) by $A\mapsto \{d\in \D_D;d\subset A\}$. Thus subsets of $\D_D$ are $\equiv_D$-invariant sets in disguise; we will go back and forth between these two notions, so it is important to understand this similarity.

\subsubsection*{Relationship Between Strategies \& $ \bm{\D_D}$}

Take your favourite definable bijection $\pi : \om\ra \om^\lom$. Then we can code strategies. If $\sigma$ is a strategy, then write $\cd(\sigma) : \om \ra \om$, $n\mapsto \sigma(\pi(n))$. Then $\cd(\sigma)\in\om^\om$ and we have that $\sigma \equiv_D \cd(\sigma)$. If we play $\sigma$ against $\tau$, we get $\sigma \ast \tau \led \cd(\sigma)\ast\cd(\tau)$ since the latter contains all the information about the former.

For example, $\sigma \ast x$: this is the unique element $z\in \om^\om$ such that for all $k$, $z(2k+1) = x(k)$ and $z(2k) = \cd(\sigma)(\pi^{-1}(z\restriction 2k))$. This formula then witnesses that $\sigma \ast x \led \cd(\sigma) \ast x$. [In general, $\sigma \ast x \not\equiv_D \cd(\sigma)\ast x$.]

\begin{thmenv*}[Martin's Lemma]
    Suppose $A$ is $\equiv_D$-invariant. Then:
    \begin{enumerate}[label = (\roman*)]
        \item if player I wins $G(A)$, then $A \in F_D$
        \item if player II wins $G(A)$, then $\om^\om \backslash A \in F_D$
    \end{enumerate}
\end{thmenv*}

\begin{thmenv*}[Corollary]
    $\ad\implies$ the filter on $\D_D$ defined by $Z \in U_M \iff \bigcup Z \in F_D$ (the \undf{Martin Measure}) is an ultrafilter.
\end{thmenv*}

So we are left to prove Martin's Lemma.

\begin{proof}
    We're just going to see the case of player I. Please check that the case for player II is the same argument.
    
    Suppose $A\subset \om^\om$ is $\equiv_D$-invariant and $\sigma$ is winning for player I in $G(A)$. Let $c\coloneqq \cd(\sigma)$ and let's \underline{claim}: $\cone(c)\subset A$.

    \underline{Proof of claim}: Take any $x \in \cone(c) = \{y \in \om^\om ; c \led y\}$, so $c\led x$. Note that whenever $a\led b$, then $a\ast b \equiv_D b$ [since we proved that $a \ast b$ is the least upper bound]. So we have that $c\led x \equiv_D c\ast x =\cd(\sigma)\ast x$. We had $\sigma\ast x \led \cd(\sigma)\ast x = c\ast x$. But clearly $x\led \sigma\ast x$, so we have equality throughout, and in particular $\sigma \ast x \equiv_D x$. But $\sigma$ was winning, so $\sigma \ast x \in A$, and $\sigma \ast x \equiv_D x$, so $x \in A$ also as $A$ was $\equiv_D$-invariant. \qedsymbol

    This concludes the claim and hence the proof.
\end{proof}

We are now poised to tackle:

\begin{theorem*}[Solovay]
    If $\ad$ holds, then $\aleph_1$ is a measurable cardinal.
\end{theorem*}
\begin{proof}
    We have established that $U_M$ is an ultrafilter on $\D_D$.
    
    \underline{Idea}: give a function $f:\D_D \ra \aleph_1$ such that $f_\ast U_M$ is a non-principal ultrafilter on $\aleph_1$. It is clear that no matter what $f$ is, $f_\ast U_M$ is going to be an $\aleph_1$-complete ultrafilter. So the crucial part is to find $f$ such that $f_\ast U_M$ is non-principal.

    \underline{Important}: if $d = [x] \in \D_D$, then $d$ is a countable set of elements of $\om^\om$. We make the definition $W_x \coloneqq \{\alpha; \exists y(y\led x \land y \in \wf_\alpha)\}$. This is a countable set. Clearly if $x\equiv_D x'$, then $W_x = W_{x'}$. So let $W_D \coloneqq W_x$ for any $x \in d$. By $\ac_\om(\R)$, $\aleph_1$ is regular, thus $\alpha_d \coloneqq \sup W_d < \om_1$. Define similarly $\alpha_x \coloneqq \sup W_x$.

    Define $f : \D_D \ra \aleph_1$, $d\mapsto \alpha_d$. Then $f_\ast U_M$ is an $\aleph_1$-complete ultrafilter on $\aleph_1$.

    \underline{Claim}: $f_\ast U_M$ is not principal.

    \underline{Proof of Claim}: Suppose it is. Then $\{\gamma\} \in f_\ast U_M$ for some $\gamma$. We have:
    \begin{align*}
        \{\gamma\} \in f_\ast U_M &\iff \{d \in \D_D;\alpha_d = \gamma\} \in U_M\\
        &\iff \{x \in \om^\om;\alpha_x = \gamma\}\in F_D\\
        &\iff \textrm{ there is }z\textrm{ such that }\cone(z)\subset \{x\in\om^\om;\alpha_x = \gamma\}
    \end{align*}
    Consider some $y \in \wf_{\gamma+1}$. Clearly $y \not\in \{x \in \om^\om;\alpha_x = \gamma\}$. Now $z\ast y \in \cone(z)$, but $y \led z\ast y$, so $\gamma+1 \in W_{z\ast y}$, so $\alpha_{z\ast y} \ne \gamma$. This is a contradiction.
\end{proof}

\end{document}