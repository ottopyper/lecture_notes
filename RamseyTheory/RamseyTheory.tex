\documentclass[10pt]{article}

\usepackage{amsmath}
\usepackage{amssymb}
\usepackage{amsthm}
\usepackage{graphicx}
\usepackage{parskip}
\usepackage{xcolor}
\usepackage{pagecolor}
\usepackage[margin=1.2in]{geometry}
\usepackage{enumerate}
\usepackage{enumitem}
\usepackage{tikz}
\newcommand*\circled[1]{%
   \tikz[baseline=(C.base)]\node[draw,circle,inner sep=1.2pt,line width=0.2mm,](C) {#1};}
\newcommand*\Myitem{%
   \stepcounter{enumi}\item[\circled{\theenumi}]}

\usepackage[utf8]{inputenc}
\usepackage[english]{babel}

\usepackage{mathtools}
\DeclarePairedDelimiter\bra{\langle}{\rvert}
\DeclarePairedDelimiter\ket{\lvert}{\rangle}
\DeclarePairedDelimiterX\braket[2]{\langle}{\rangle}{#1 \delimsize\vert #2}

\definecolor{thmcolour}{rgb}{0,0,0}
\definecolor{defcolour}{rgb}{0,0,0}
\definecolor{textcolour}{rgb}{0,0,0}
\definecolor{backgroundcolour}{rgb}{1,1,1}

\pagecolor{backgroundcolour}
\color{textcolour}

\newtheoremstyle{custhm}
{%space above
}{%space below
}{%body font
\color{thmcolour}\em
}{%indent amount
-0em
}{%head font
\bfseries\color{thmcolour}
}{%head punct
}{%after head space
1em
}{%head spec
\thmname{#1}\if\relax\detokenize{#2}\relax:\else\thmnumber{ #2}:\fi\if\relax\detokenize{#3}\relax\else\thmnote{ (#3)}\fi
}

\newtheoremstyle{ex}
{%space above
}{%space below
}{%body font
\color{thmcolour}
}{%indent amount
-0em
}{%head font
\bfseries\color{thmcolour}
}{%head punct
}{%after head space
1em
}{%head spec
\thmname{#1}\if\relax\detokenize{#2}\relax:\else\thmnumber{ #2}:\fi\if\relax\detokenize{#3}\relax\else\thmnote{(#3)}\fi
}

\newtheoremstyle{remark}
{%space above
}{%space below
}{% body font
}{%indent amount
-0em
}{%head font
\bfseries
}{%head punct
}{%after head space
1em
}{%head spec
\if\relax\detokenize{#3}\relax\thmname{#1}:\else\thmname{#3}:\fi
}

\newtheoremstyle{numremark}
{%space above
}{%space below
}{% body font
}{%indent amount
-0em
}{%head font
\bfseries
}{%head punct
}{%after head space
1em
}{%head spec
\thmname{#1}\thmnumber{ #2}:
}

\newtheoremstyle{cusdef}
{%space above
}{%space below
}{%body font
\color{defcolour}
}{%indent amount
-0em
}{%head font
\bfseries\color{defcolour}
}{%head punct
}{%after head space
1em
}{%head spec
%if numbered, include number
%if named, include name
\thmname{#1}\if\relax\detokenize{#2}\relax:\else\thmnumber{ #2}:\fi\if\relax\detokenize{#3}\relax\else\thmnote{ (#3)}\fi
}

\theoremstyle{custhm}
\newtheorem{theorem}{Theorem}[section]
\theoremstyle{cusdef}
\newtheorem{defin}[theorem]{Definition}
\theoremstyle{custhm}
\newtheorem{lemma}[theorem]{Lemma}
\theoremstyle{custhm}
\newtheorem{cor}[theorem]{Corollary}

\theoremstyle{custhm}
\newtheorem{prop}[theorem]{Proposition}

\theoremstyle{ex}
\newtheorem{ex}[theorem]{Example}

\theoremstyle{custhm}
\newtheorem*{theorem*}{Theorem}

\theoremstyle{cusdef}
\newtheorem*{defin*}{Definition}

\theoremstyle{remark}
\newtheorem*{remark*}{Remark}

\theoremstyle{remark}
\newtheorem{remark}[theorem]{Remark}

\theoremstyle{numremark}
\newtheorem{numremark}[theorem]{Remark}

\setcounter{section}{-1}

%\marginpar{to describe which lecture it is}

\newcommand{\N}{\mathbb{N}}
\newcommand{\Z}{\mathbb{Z}}
\newcommand{\Q}{\mathbb{Q}}
\newcommand{\R}{\mathbb{R}}
\newcommand{\C}{\mathbb{C}}
\newcommand{\e}{\mathrm{e}}
\newcommand{\ra}{\rightarrow}
\newcommand{\lef}{\left(}
\newcommand{\res}{\right)}
\newcommand{\ie}{\textit{i.e.}}
\newcommand{\eps}{\varepsilon}
\newcommand{\E}{\mathbb{E}}
\newcommand{\suminf}{\sum_{n=0}^{\infty}}
\newcommand{\suminfa}[1]{\sum_{#1=0}^{\infty}}
\renewcommand{\P}{\mathbb{P}}
\newcommand{\undf}[1]{\textit{\textbf{#1}}}
\renewcommand{\L}{\mathcal{L}}
\renewcommand{\it}[1]{\textit{#1}}
\newcommand{\M}{\mathcal{M}}
\renewcommand{\phi}{\varphi}
\newcommand{\proves}{\vdash}
\newcommand{\lra}{\leftrightarrow}

\renewcommand{\bar}{\overline}
\renewcommand{\O}{\mathcal{O}}


\newcommand{\ac}[1]{\mathcal{#1}}
\newcommand{\A}{\mathcal{A}}


\renewcommand{\subset}{\subseteq}

\renewcommand{\th}{\textrm{th}}

\usepackage{graphicx}

\newcommand{\hj}{\textrm{HJ}}
\newcommand{\wnfc}{\textrm{W}\N\textrm{FC}}
\newcommand{\fs}{\textrm{FS}}
\newcommand{\fp}{\textrm{FP}}
\newcommand{\F}{\mathcal{F}}
\newcommand{\U}{\mathcal{U}}
\newcommand{\V}{\mathcal{V}}
\newcommand{\fu}{\forall_\U}
\newcommand{\for}[1]{\forall_{#1}}

%\usepackage[lite]{mtpro2}

\title{Ramsey Theory}
\author{Lectures by Imre Leader}
\date{}

\begin{document}
\maketitle
\section{Introduction}
Ramsey Theory is all about the following question:

\begin{center}
    \underline{\textbf{Can we find some order in (enough) disorder?}}    
\end{center}

In a sense, the entire course is about answering this question in different settings.

Chapter 1: Monochromatic systems. Abstract and concrete.

Chapter 2: Partition Regular Equations. More concrete; looking at the naturals with addition and multiplication, and asking about order/disorder there.

Chapter 3: Infinite Ramsey Theory. Abstract; taken to a limit (countable only though).

Prerequisites: None (other than some basic concepts of topology \it{e.g.} open/closed/compact sets).

Literature: In theory, this course is self-contained. But if you would like a different viewpoint/some reinforcement, consider:
\begin{itemize}
    \item Bollobas, ``Combinatorics'', C.U.P. 1986 (For Ch. 3)
    \item Graham, Rothschild, Spencer, ``Ramsey Theory'', Wiley 1990 (For Ch. 1\&2)
\end{itemize}

Example Sheets: this a 16 lecture course, so there are 3 sheets and 3 classes.

\clearpage
\section{Chapter 1}
\subsection{Monochromatic Systems}

This is the introductory chapter from which everything else will flow.

In this course, we take $\N = \{1,2,3,\dots\}$, and write $[n] = \{1,2,\dots,n\}$. For a set $X$ and $r\in \N$, we write $X^{(r)} = \{A\subset X : |A| = r\}$; this is the collection of all $r$-sets in $X$.

\underline{Queston}: Suppose that we are given a $2$-colouring of $\N^{(2)}$, \it{i.e.} a $c:\N^{(2)}\ra\{1,2\}$. Can we find an infinite subset $M\subset \N$ such that $M$ is monochrome, \it{i.e.} $c$ is constant on $M^{(2)}$.

Let's try some examples to get a feel for it.

\begin{remark*}[Examples]\ 
    \begin{enumerate}
        \item Colour $ij$ red if $i + j$ is even, and blue if $i+j$ is odd.
        
        Here we can do it, rather easily - take $M = \{2,4,6,8,\dots\}$, or any subset thereof, or all the odd numbers, \it{etc...}.

        \item Colour $ij$ red if $\max\{n:2^n|i+j\}$ even, and blue otherwise.
        
        This is also a yes, \it{e.g.} $M = \{4^0,4^1,4^2,\dots\}$.

        \item Colour $ij$ red if $i+j$ has an even number of (distinct) prime factors, and blue otherwise.
        
        This is also a yes, but it is not obvious how...
    \end{enumerate}
\end{remark*}

\begin{theorem}[Ramsey's Theorem]
    Let $c$ be a $2$-colouring of $\N^{(2)}$. Then there exists an infinite monochromatic $M\subset \N$.
\end{theorem}
\begin{proof}
    Pick any $a_1 \in \N$. There are infinitely many edges out of $a_1$, so infinitely many have the same colour; say all edges from $a_1$ to infinite set $B_1$ have colour $c_1$.

    Now within $B_2$, we take a point $a_2$ and find an finite $B_2\subset B_1\backslash\{a_2\}$ such that all edges from $a_2$ to $B_2$ are the same colour - these may be red or blue.

    Repeat this process again, within $B_3$, and repeat ad infinitum.

    This gives us an infinite sequence $(a_i)_{i=1}^{\infty}$ of points, and infinite sequence of colours $c_i$ such that edge $a_i$ to $a_j$ with $i<j$ has colour $c_i$.

    Take a constant subsequence of $c_i$, say $c_i:i\in I$. Then $M = \{a_{i}:i\in I\}$ is monochromatic.
\end{proof}
\begin{remark*}[Remarks]\ 
    \begin{enumerate}
        \item This is sometimes called a `2-pass' proof, because we had to do the whole induction, and then go over it again to finish it off.
        \item In example 3, no such example is known!
        \item The same proof shows that whenever $\N^{(2)}$ is $k$-coloured (\it{i.e.} $c:\N^{(2)}\ra [k]$), there still exists an infinite monochromatic set.
        
        Alternatively, we could deduce this from Theorem 1 plus induction. We could view the colours as `1' and `2 or 3 or $\dots$ or $k$' and apply Theorem 1; if we get an infinite $1$-set, then done, and if we get an infinite `2 or $\dots$ or $k$'-set then done by induction on $k$.

        \item An infinite monochromatic set is \undf{more} than having arbitrarily large finite monochromatic sets, \it{e.g.} take disjoint blue $K_2$, $K_3$, $K_4$, and so on, then connect everything remaining with red edges. While this (of course) does not contradict Ramsey's Theorem, we clearly have arbitrarily large blue sets (the $K_n$s), but there is no \it{infinite} blue set.
    \end{enumerate}
\end{remark*}

\begin{remark*}[Example]
    Any sequence $x_1,x_2,\dots$ in $\R$ (or any totally ordered set) has a monotone subsequence.

    Indeed, $2$-colour $\N^{(2)}$ by giving $ij$ ($i<j$) colour \it{up} if $x_i < x_j$, and \it{down} if $x_i > x_j$. Then apply Theorem 1.
\end{remark*}

\reversemarginpar{Lecture 2}

What if we coloured $\N^{(r)}$, say for $r = 3,4,\dots$. Given a $2$-colouring $c:\N^{(r)}\ra \{1,2\}$, must there be an infinite monochromatic set?

For instance, $r = 3$: colour $ijk$ ($i < j < k$) red if $i | j+k$, and blue otherwise. Here we can do this easily - just take $M = \{2^0,2^1,2^2,\dots\}$.

\begin{theorem}[Ramsey for $r$-sets]
    Let $r \in \N$. Then whenever $\N^{(r)}$ is $2$-coloured, there exists an infinite monochromatic set.
\end{theorem}
\begin{proof}
    In the previous proof, when we picked $a_1$ and looked at the lines out of it to other points, we in fact used the $r = 1$ case on the colouring induced on the singletons in the neighbourhood of $a_1$. Armed with these ideas, the proof here will act in exactly the same way.

    $r = 1$ is trivial (just infinite pigeonhole), or if you prefer $r = 2$ is Theorem 1. We apply induction on $r$. Thus suppose the result holds for $r - 1$. Given $c:\N^{(r)}\ra \{1,2\}$, we look at the induced colouring.

    Pick $a_1 \in \N$, and look at $(\N\backslash\{a_1\})^{(r-1)}$. This has an induced colouring $d$ given by $d(F) = c(F\cup\{a_1\})$. Now by induction there is an infinite $B_1\subset \N\backslash\{a_1\}$ such that all the $r-1$-sets have the same colour according to $d$, \it{i.e.} $c(F\cup\{a_1\}) = c_1$ for all $F\subset B^{(r-1)}$, for some colour $c_1 \in [r]$.

    Repeating, we have $a_1 \in B_1$, and infinite $B_2\subset B_1\backslash \{a_2\}$ such that all $F\cup \{a_2\}$, $F\in B_2^{(r-1)}$, have the same colour $c_2$.

    We keep going to infinity, giving us an infinite sequence of distinct points $a_1,a_2,\dots$ and colours $c_1,c_2,\dots$ such that $c(a_{i_1},a_{i_2},\dots,a_{i_r})=c_{i_1}$ ($i_1 < i_2 < \dots < i_r$). There is then an infinite index set $I$ such that $c_i:i\in I$ is constant, and so $M = \{a_i:i\in I\}$ is monochrome.
\end{proof}

\begin{remark*}[Example]
    We saw from Theorem 1 that given points $(1,x_1),(2,x_2),\dots$ in $\R^2$, there exists a subsequence such that the induced (piecewise-linear) function is monotone.

    Functions can have other properties that are stricter; for instance, we could ask for the function to be convex/concave - in fact we can ensure that this is the case.

    On the surface, this seems really hard - but not for us, with $r$-set Ramsey.

    Indeed, just $2$-colour $\N^{(3)}$ by giving $ijk$ colour \it{convex} if they form a convex shape, and \it{concave} (otherwise) if they from a concave shape (any three points must fall into one of the two categories).
    
    We get an infinite monochrome subsequence; the induced function is either convex or concave for any of the three points; and so the overall function is convex/concave.
\end{remark*}

How long does the proof take in the $r = 3$ case? Well, to find each $a_i$ we need to do an infinite two-pass proof (of the $r = 2$ case). So this happens an infinite number of times, and then theres another pass at the end. Essentially, it takes a very long time.

Surprisingly, the infinite version of Ramsey \it{implies} the finite version.

\begin{theorem}[Finite Ramsey]
    For all $m, r$ there exists an $n$ such that whenever $[n]^{(r)}$ is $2$-coloured, there exists a monochromatic $m$-set.
\end{theorem}
\begin{proof}
    Suppose not. We will show that there ia a $2$-colouring of $\N^{(r)}$ without a monochromatic $m$-set, (massively) contradicting Theorem 2.

    For each $n\ge r$, have a $2$-colouring $c_n$ of $[n]^{(r)}$ with no monochromatic $m$-set. [We want to take their union to get a bad colouring of the whole of $\N^{(r)}$, but the problem is that the colours aren't necessarily \it{nested}, \it{i.e.} any two agree where both defined.]

    There are only finitely many ways to $2$-colour $[r]^{(r)}$ (two, in fact), so infinitely many of the $c_n$ agree on $[r]^{(r)}$: say $c_n\restriction [r]^{(r)} = d_r$ for all $n\in A_1$, some $d_n:[r]^{(r)}:\{1,2\}$.

    There are only finitely many ways to $2$-colour $[r+1]^{(r)}$, so infintely many of the $c_n:n\in A_1$ agree on $[r+1]^{r+1}$: say $c_n\restriction [r+1]^{(r)} = d_{r+1}$ for all $n \in A_2$, for some $d_{r+1}:[r+1]^{(r)}\ra \{1,2\}$.

    Continue inductively. We obtain colourings $d_n:[n]^{(r)}\ra\{1,2\}$ for each $n\ge r$ such that
    \begin{enumerate}[label=\arabic*)]
        \item $d_n$ has no monochromatic $m$-set, since $d_n = c_n'\restriction [n]^{(r)}$ for some $n' \ge n$
        \item $d_{n'}\restriction [n]^{(r)} = d_n$ for all $n' \ge n$ by construction of the $d_n$s
    \end{enumerate}

    Now put $c(F) = d_n(F)$, any $n\ge \max F$ (for each $r$-set $F$). We can say \it{any}, because all the colourings agree.

    This is well-defined (by 2)), and has no monochromatic $m$-set (by 1)). Massive contradiction.
\end{proof}

\begin{remark*}[Remarks]
    This is called a `compactness' argument, similar to the proof of Bolzano-Weierstrass in IA Analysis. What we are essentially doing is showing that if the space $\{1,2\}^{\N}$ of $2$-colourings, with the product topology (\it{i.e.} induced from the metric $d(f,g) = 1/(\min \{n:f(n)\ne g(n)\})$) is (sequentially) compact.

    \underline{Note}: This proof also gives \it{no} information on how large $n = n(m,r)$ can be (such proofs do exists though, \it{c.f.} IID Graph Theory).
\end{remark*}

What if we colour $\N^{(2)}$ with infinitely many colours, \it{i.e.} have $c:\N^{(2)}\ra X$, some set $X$?

Of course, we do not get an infinite monochromatic set, since \it{e.g.} we can just colour each edge with a unique colour. But we can ask a slightly different question...

Do we always get an infinte ste $m$ such that $c\restriction m^{(2)}$ is either constant or injective?

Sadly, the answer is still no. We can achieve this by colouring $ij$ ($i<j$) with colour $i$. This clearly is neither injective nor constant on any infinite subset.

Another option  here is that we can colour $ij$ with colour $j$ ($i<j$) - then each colour class is finite, instead of infinite. As it happens, at least one of these four situations must arise:

\begin{theorem}
    Let $c$ be a colouring of $\N^{(2)}$ with an arbitrary set of colours. Then there exists an infinite $M\subset \N$ such that one of the following holds:
    \begin{enumerate}[label=\roman*)]
        \item $c$ is constant on $M^{(2)}$
        \item $c$ is injective on $M^{(2)}$
        \item $\forall ij,kl \in M^{(2)}$, $c(ij) = c(kl)\iff i = k$
        \item $\forall ij,kl \in M^{(2)}$, $c(ij) = c(kl)\iff j = l$
    \end{enumerate}
\end{theorem}

[It is worth remarking that this trivially implies Theorem 1: if there are only finitely many colours, then cases ii), iii), iv) cannot happen.]

\begin{proof}
    There is a superbly nice idea here. The act of comparing pairs of edges is the same as asking a question about the 4-set of the vertices, and we can use Ramsey's Theorem for $r = 4$ to help us here.

    Define a $2$-colouring of $\N^{(4)}$ by giving $ijkl$ colour \it{same} if $c(ij) = c(kl)$, and \it{dif} if $c(ij)\ne c(kl)$. Note that, as ever, the notation implies $i<j<k<l$.

    By Ramsey for $4$-sets, ther exists an infinite monochromatic $M_1$ for this colouring. If the colour is \it{same}, then we have case i). Indeed, given $ij,kl\in M_1^{(2)}$, choose $m < n$ in $M_1$ with $m > i,j,k,l$. Then $c(ij) = c(mn)$, and $c(kl) = c(mn)$ (this deals with the case $j= k$, and other anomalies). So we may otherwise assume that $M_1$ has colour diff.

    Now, $2$-colour $M_1^{(4)}$ by giving $ijkl$ colour \it{same} if $c(jk) = c(il)$, and \it{diff} if not.

    Again by $Ramsey$-4, we have infinite $M_2\subset \M_1$ monochromatic for this colouring. Note that we cannot have $M_2$ be the colour \it{same}, as otherwise pick $i < j < k < l < m < n \in M_2$. We would then have $c(jk) = c(in) = c(lm)$, contradicting $j,k,l,m \in M_1$. Thus $M_2$ is colour \it{diff}.

    The final type of edge pairs in a $4$-set could be interlocking:

    $2$-colour $M_2^{(4)}$ by giving $ijkl$ colour \it{same} if $c(ik) = c(jl)$, and \it{diff} if not. Again we obtain an infinite monochromatic $M_3 \subset M_2$. Once again, we cannot have $M_3$ colour \it{same}, else we choose $i < j < k < l < m < n \in M_3$, and we have $c(im) = c(kn) = c(jl)$, contradicing the fact that $i,j,l,m \in M_2$. Thus $M_3$ is colour \it{diff}.

    So now we know that any two edges in $M_3$ have different colours if they are not adjacent; we now deal with the adjacent case.

    $2$-colour $M_3^{(3)}$ by giving $ijk$ colour \it{same} if $c(ij) = c(jk)$, and \it{diff} if not. Have an infinite monochromatic $M_4 \subset M_3$. We cannot have $M_4$ colour \it{same}, else pick $i < j < k < l \in M_4$, and we have $c(ij) = c(jk) = c(kl)$, contradicting the above. So $M_4$ is colour \it{diff}.

    There are other ways that edge pairs can be adjacent, and these account for the various cases in the Theorem.

    $2$-colour $M_4^{(3)}$ by giving $ijk$ colour \it{left-same} if $c(ij) = c(ik)$, and \it{left-diff} if not. We obtain infinite monochromatic $M_5\subset M_4$.

    Finally, $2$-colour $M_5^{(3)}$ by giving $ijk$ colour \it{right-same} if $c(jk) = c(ik)$, and \it{right-diff} otherwise.

    We obtain infinite monochromatic $M_6\subset M_5$.
\begin{center}
    \includegraphics[scale=0.8]{together.jpg}
\end{center}

    \underline{If $M_6$ is \it{left-diff} and \it{right-diff}}: All pairs of edges have a different colour; this is case 2.

    \underline{If $M_6$ is \it{left-same} and \it{right-diff}}: If edges meet at the left they are the same; at the right they are different; this is case 3.

    \underline{If $M_6$ is \it{left-diff} and \it{right-same}}: Similarly, this is case 4.

    \underline{If $M_6$ is \it{left-same} and \it{right-same}}: Pick $i < j < k \in M_6$. Then $c(ij) = c(ik) = c(jk)$, contradiction.
\end{proof}

\begin{remark*}[Remarks]\ 
    \begin{enumerate}
        \item We could use just one colouring, by colouring a $4$-set $ijkl$ with the partition of $[4]^{(2)}$ induced by $c$ on $\{i,j,k,l\}$. The number of colours is then the number such partitions, which is ${4\choose 2} = 6$. We didn't do this because it seems a bit magical and out of the blue, and slightly obscures what's going on. There are also lots of symbols, making it a bit unpleasant.
        
        \item Similarly, if $c:\N^{(r)}\ra X$ is an arbitrary colouring, we get an infinite $M\subset \N$ and a set $I\subset [r]$ such that $\forall x_1,\dots,x_r,y_1,\dots,y_r\in M^{(r)}$ we have $c(x_1,\dots,x_r) = c(y_1,\dots,y_r) \iff x_i=y_i\ \forall i\in I$, where $I$ is an index set. These $2^r$ coloursing are called the \undf{canonical} colourings of the $r$-sets.
        
        For instance, $ = 2$:
        \begin{itemize}
            \item Case i) corresponds to $I = \emptyset$
            \item Case ii) corersponds to $I = \{1,2\}$
            \item Case iii) corresponds to $I = \{1\}$
            \item Case iv) corresponds to $I = \{2\}$
        \end{itemize}
    \end{enumerate}    
\end{remark*}

\subsection{Van der Waerden's Theorem}

If we $2$-colour $\N$, can we find $3$ consecutive poitns of the same colour?

\underline{Answer}: of course not; just colour $\N$ alternately.

What about $3$ equally spaced points, \it{i.e.} $(a,a+d,a+2d)$? This is not obviously false. If it were true though, that would be nice - we would have found some order amongst the disorder. Could we find more, perhaps $4$ points, or even $m$?

\underline{Aim}: For every $m$, whenever $\N$ is $2$-coloured there exists a monochromatic artihmetic progression of length $m$.

Just to be clear, by length we mean the number of terms \it{i.e.} a sequence $\{a, a + 2,\dots, a+(m-1)d\}$ has length $m$.

This is Van der Waerden's Theorem, and it is very hard to solve.

By our usual compactness argument, this is the same as:

\underline{Aim'}: $(\forall m)(\exists N)$ such that whenever $[n]$ is $2$-coloured, there exists a monochromatic AP of length $m$.

Indeed, if this is false then there is an $m$ such that for every $n \ge m$ there is a colouring $c_n$ of $[n]$ with no monochromatic AP of length $m$. We want to combine these into one big colouring of $\N$, but we can't yet.

But infinitely many $c_n$ agree on $[m]$, and, of those, infinitely many agree on $[m+1]$, and so on.... Put together those (nested) restrictions to obtain a $2$-colouring of $\N$ with no monochromatic AP of length $m$. Contradiction.

In proving this (Aim'), one key idea is to generalise: we in fact show that $\forall m,k\ \exists n$ such that whenever $[n]$ is $k$-coloured there exists a monochromatic AP of length $m$.

\underline{Note}: proving a \it{stronger} result might be \it{easier}, \it{e.g.} in a proof by induction.

Another key idea: given APs $A_1,\dots,A_r$, each of length $m$ - so $A_i = \{a_i,a_i+d_i,\dots,a_i+(m-1)d_i\}$ - we say they are \undf{focused} at $f$ if $a_i + md_i = f$ for each $i$. \it{E.g.} $\{1,4\}$ and $\{5,6\}$ are focused at $7$.

If in addition each $A_i$ is monochromatic (for a given colouring), with no two the same colour, we say they are \undf{colour-focused}. Why do we care?

In a $k$-colouring, if we have APs $A_1,\dots,A_k$, each of length $m-1$, that are colour-focused, then we actually have a monochromatic AP of length $m$, by asking ``what colour is the focus?''

Write $W(m,k)$ for the least $n$ (if it exists) such that whenever $n$ is $k$-coloured, there exists a monochromatic AP of length $m$.

\begin{prop}
    $\forall k, \exists n$ such that whenever $[n]$ is $k$-coloured there exists a monochromatic AP of length $3$.
\end{prop}
\underline{Note}: This will be contained in Theorem 6; it is included here for clarity.


\begin{proof}
    \underline{Claim}: $\forall r \le k, \exists n$ such that whenever $[n]$ is $k$-coloured, we have either:
    \begin{itemize}
        \item a monochromatic AP of length 3, or
        \item $r$ colour-focused APs of length 2
    \end{itemize}
    Given this: put $r = k$ and look at the focus. Whatever colour it is, we get a monochromatic AP of length 3.
    
    \underline{Proof of Claim}: Induction on $r$. $r = 1$ is trivial (can take $n = k+1$).

    Given $n$ suitable for $r - 1$, we'll show that $(k^{2n}+1)\cdot 2n$ is suitable for $r$. So given a $k$-colouring of $[(k^{2n}+1)2n]$, with no mono AP of length 3:

    Break up $[(k^{2n}+1)2k]$ into blocks of length $2n$, say $B_1,B_2,\dots,B_{k^{2n}+1}$, where $B_i = [(i-1)2n+1,i2n]$ (square brackets denote interval with endpoints).

    Inside any interval of length $2n$, we have $r-1$ colour-focused APs of length 2 (by choice of $n$), together with their focus (as length = $2n$).

    Now, the number of ways to $k$-colour a block is $k^{2n}$, and since we have $k^{2n}+1$ it must be the case that some two are identically coloured, say $B_s$ and $B_{s+t}$. Inside $B_s$, we have $r-1$ colour-focused APs of length $2$, say $\{a_1,a_1+d_1\},\dots,\{a_{r-1},a_{r-1}+d_{r-1}\}$ focused at $f$.

    But now the APs $\{a_1,a_1+d_1+2nt\},\dots,\{a_{r-1},a_{r-1}+d_{r-1}+2nt\}$ are colour-focused at $f + 4nt$, and $\{f,f+2nt\}$ is also focused there, giving $r$ colour-focused APs of length $2$. So we have finished the induction, so proved the claim, so finished the proof.
\end{proof}
\begin{remark*}[Remarks]\ 
    \begin{enumerate}
        \item The idea of looking at the number of ways to colour a block is called a \undf{product argument}.
        
        \item The proof shows that $$W(3,k) \le \left. k^{k^{.^{.^{k^{4k}}}}}\right\rbrace k$$
        
        So \it{e.g.} $W(3,3)\le 3^{3^{12}}$. This is a `tower-type' bound.
    \end{enumerate}
\end{remark*}

We are now better-equipped to tackle the full theorem:

\begin{theorem}[Van der Waerden's Theorem]
    $\forall m,k,\exists n$ such that whenever $[n]$ is $k$-coloured there exists a monochromatic AP of length $m$.
\end{theorem}
\begin{proof}
    We induct on $m$. $m=1$ is trival (or $m = 2$ is pigeonhole, or $m = 3$ is Prop 1.5).

    So we may assume that that $W(m-1,k)$ exists for every $k$.

    \underline{Claim}: for every $r\le k$, there exists $n$ such that whenever $[n]$ is $k$-coloured, we have either:
    \begin{itemize}
        \item a monochromatic AP of length $m$, or
        \item $r$ colour-focused APs of length $m-1$
    \end{itemize}
    [Given this, put $ r = k$ and look at the focus to get a mono AP of length $m$.]

    \underline{Proof of Claim}: Induction on $r$: $r = 1$ (take $n = W(m-1,k)$).
    Given $n$ suitable for $r-1$, we'll show that $W(m-1,k^{2n})\cdot 2n$ is suitable for $r$.

    So, we are given a $k$-colouring of $[W(k-1,k^{2n})2n]$, with no mono AP of length $m$:

    Break up $[W(m-1,k^{2n})]$ into blocks of length $2n$, say $B_1,B_2,\dots,B_{W(m-1,k^{2n})}$, where $B_i = [(i-1)2n+1,i2n]$.

    As in the proof before, we need three equally spaced identical blocks - but this is much harder to get.

    The number of ways to $k$-colour a block is $k^{2n}$, so since we have $W(m-1,k^{2n})$ \it{blocks}, we must have (by definition of $W(m-1,k^{2n})$) some  $m-1$ equally spaced blockes that are coloured identically. Say $B_s, B_{s+t},\dots,B_{s+(m-2)t}$.

    Now, inside $B_s$ we have $r-1$ colour-focused APs of length $m-1$ (by definition of $n$), together with their focus (as length = $2n$): say $A_1,\dots,A_{r-1}$ focused at $f$, where $A_i$ has first term $a_i$ and common difference $d_i$. Then the APs $A_1',\dots,A_{r-1}'$, where $A_i'$ has first term $a_i$ and common difference $d_i + 2nt$ are colour-focused at $f + (m-1)2n$, and also $\{f,f+2nt,f+2(2nt),\dots,f+(m-2)2nt$ is monochromatic of a different colour to the $A_i'$s. This gives $r$ colour-focused APs of length $m-1$.

    This completes the induction, which proves the claim, which completes the outer induction on $m$ and hence concludes the proof.
\end{proof}

\underline{Note}: for these proofs, focus on the pictures (even though I haven't drawn them here)!

\marginpar{Lecture 5}

\subsection*{The Ackermann/Grzegorczyk Hierarchy}

\begin{defin*} The \undf{Ackermann} or \undf{Grzegorcyzk} \undf{Hierarchy} is the sequence of funcgtions $f_1,f_2,\dots$ (each $\N\ra\N$) given by:
    \begin{itemize}
        \item $f_1(x) = 2x$
        \item $f_{n+1}(x) = f_n^{(x)}(1) = \underbrace{f_n(f_n(\dots f_n}_{x\textrm{ times}}(1)\dots))$
    \end{itemize}
\end{defin*}

Let's explore these a bit.
\begin{itemize}
    \item $f_2(x) = 2^x$
    \item $f_3(x) = \left. 2^{2^{.^{.^{2}}}} \right\rbrace x$
    \item $f_4(1) = 2$, $f_4(2) = 2^2 = 4$, $f_4(3) = 2^{2^{2^2}} = 65536$, $f_4(4) = \left. 2^{2^{.^{.^{2}}}} \right\rbrace 65536$ \it{etc}...
\end{itemize}

We say $f:\N\ra \N$ is of \undf{type $n$} if there exist $c,d > 0$ such that $f_n(cx) \le f(x) \le f_n(dx)$ for all $x$. So our upper bound for $W(3,k)$ was a function of $k$ of type 3 (note that even though it is a tower of $k$s, not $2$s, the height of the tower is far more significant). Our bound on $W(m,k)$ ($m$ fixed) is of type $m$.

In fact, our bound on $W(m) = W(m,2)$, as a function of $m$, grows faster than every $f_n$! [This means that $W(m)$ is \it{not} primitive recursive.] This is often a feature of such `double inductions', and for a long time it was thought that perhaps $W(m)$ really does grow this fast.

However, Shelah (1987) found a proof of VdW using induction only on $m$. His proof gives that $W(m,k) \le f_4(m+k)$. This isn't bad, but $f_4$ is still a pretty big function. Graham offered \$1000 for a bound on $W(m)$ that was $f_3(m)$. Gowers (1998) showed $W(m) \le 2^{2^{2^{2^{2^{m+9}}}}}$ - `almost' of type $2$ - a huge improvement on type 3.

What about a lower bound? It is known that $W(m) \ge \frac{2^m}{8m}$; this is comparatively extremely small.

\begin{cor}
    Whenever $\N$ is finitely coloured, some colour class arbitrarily long APs. \qedsymbol
\end{cor}
\begin{remark*}
    We \it{cannot} guarantee an infinite AP - \it{e.g.} R-BB-RRR-BBBB....
    
    Alternatively, list all infinite APs as $A_1,A_2,\dots$ (which we can do since they are countable). Pick distinct $x_1,y_1\in A_1$ and make $x_1$ red and $y_1$ blue. Now pick distinct $x_2,y_2 \in A_2 \backslash \{x_1,y_1\}$ and make $x_2$ red, $y_2$ blue. Continue. Then we wipe out all the APs.

    The idea here is that there simply aren't enough APs.
\end{remark*}

\begin{theorem}[Strengthened Van der Waerden Theorem]
     For all $m$, whenever $\N$ is finitely coloured there exists an AP of length $m$ that, together with its common difference, is monochromatic.
\end{theorem}
\begin{proof}
    We will prove this with VdW itself (this is the sign of a good theorem).

    We induct on $k$, the number of colours. $k = 1$ is trivial.

    Given $n$ suitable for $k-1$ (\it{i.e.} $n$ such that whenever $[n]$ is $k-1$-coloured, there exists a monochromatic AP+CD of length $m$), we'll show $W(n(m-1)+1,k)$ suitable for $k$.

    Given a $k$-colouring of $[W(n(m-1)+1,k)]$, we have a mono AP of length $n(m-1)+1$; say $\{a,a+d,\dots,a+2d,\dots,a+n(m-1)d\}$ is red.

    Now if $d$ is red then we are done by $\{a,a+d,\dots,a+(m-1)d\}\cup\{d\}$ is monochromatic. Similarly, if any $rd$, $1\le r \le n$ is red, then we are done since $\{a,a+rd,\dots, a + (m-1)rd\}\cup\{rd\}$ is monochromatic.

    Thus none of $\{d, 2d, \dots, nd\}$ is red, \it{i.e.} $\{d,2d,\dots,nd\}$ is $(k-1)$-coloured. So, by definition of $n$, we are done.
\end{proof}

What kind of bounds do we get here? Since we are iterating on the left hand side, this actually grows vastly more quickly than anything we have considered before; it's so mind bogglingly large that it essentially isn't worth thinking about.

\begin{remark*}[Remarks]\ 
    \begin{enumerate}
        \item From now on, we don't care aboud bounds.
        \item The case $m = 2$ is \it{Schur's Theorem}: Whenever $\N$ is finitely coloured, there exists monochromatic $x,y,z$ with $x + y = z$.
        \item Can also prove Schur direcly from Ramsey. Indeed, given a $k$-colouring $c$ of $\N$, induce a colouring of $[n]^{(2)}$ ($n$ large enough) by $d(ij) = c(j-i)$.
        
        By Ramsey, we can find a monochromatic triangle, say $ijk$. So $c(j-i) = c(k-j) = c(j-i)$, and $(j-i) + (k-j) = (k-i)$, so done.
    \end{enumerate}
\end{remark*}

\subsection{The Hales-Jewett Theorem}

You may have noticed that we haven't taken full advantage of the structure of the natural numbers - all we needed was some idea of equal spacing, and we didn't care so much about the additive structure. It turns out that we have a theorem that describes the same phenomenon but with all the clutter removed.

\begin{defin*}[Combinatorial Line]
Let $X$ be a finite set. A subset $L$ of $X^n$ (`the $n$-dimensional cube on alphabet $X$') is called a \undf{line} or \undf{combinatorial line} if there exists non-empty $I\subset [n]$ and $a_i \in X$, each $i \in [n] - I$ such that $$L = \{(x_1,\dots,x_n)\in X^n: x_i = a_i\ \forall i \not\in I, x_i = x_j\ \forall i,j\in I\}$$
We call $I$ the set of \undf{active coordinates}
\end{defin*}

For example, in $[3]^{(2)}$, lines can be:
\begin{itemize}
    \item vertical lines, where $I = \{2\}$
    \item horizontal lines, where $I = \{1\}$
    \item one diagonal line, where $I = \{1,2\}$ (the main diagonal only)
\end{itemize}
In $[3]^{3}$, we could have
\begin{itemize}
    \item $\{(1,2,1),(1,2,2),(1,2,3)\}$ with $I = \{3\}$, or
    \item $\{(1,3,1),(2,3,2),(3,3,3)\}$ with $I = \{1,3\}$, \it{etc}...
\end{itemize}

Note that the definition of `line' is invariant under reorderings of $x$ - so \it{e.g.} $\{1,3),(2,2),(3,1)\}$ is \it{not} a line in $[3]^{2}$.

\begin{theorem}[The Hales-Jewett Theorem]
    For all $m,k$ there exists $n$ such that whenever $[m]^{n}$ is $k$-coloured, there exists a monochromatic line.
\end{theorem}
\begin{remark*}[Remarks]\ 
    \begin{enumerate}
        \item The least such $n$ (if it exists) is denoted $\hj(m,k)$.
        \item A game of $m$-in-a-row Noughts \& Crosses, played in enough dimensions, cannot end in a draw! [Exercise: show that it is a first-player win (optional).]
        \item $\hj\implies$VdW immediately: just map $X^n$ linearly to $\N$ (note: perhaps requries positive coefficients).
        
        Indeed, given a $k$-colouring $c$ of $\N$, induce a $k$-colouring $d$ of $[m]^n$ ($n$ large) by $d((x_1,\dots,x_n)) = c(x_1+\dots+x_n)$. By assumption we have a monochromatic line $L$ in $[m]^{n}$, which corresponds to a monochromatic AP of length $m$ in $\N$ (with common difference = \# active coords of $L$).

        Thus we can view HJ as an `abstract version' of VdW.
    \end{enumerate}
\end{remark*}

Before we begin, we also need a few definitions:

For a line $L$ in $[m]^{(n)}$, write $L^-$ for its first point and $L^+$ for its last point (in the ordering on $[m]^{(n)}$) given by $x \le y$ if $x_i \le y_i$ for all $i$)

Say lines $L_1,\dots,L_r$ are \undf{focused} at $f$ if $L_i^+ = f$ for all $i$, and they are \undf{colour-focused} (for a given colouring) if in addition each $L_i - \{L_i^+\}$ is monochromatic, no two of the same colour.

\begin{proof}
   Induction on $m$. $m = 1$ is trivial.

   Given $m > 1$, we may assume $\hj(m-1,k)$ exists for all $k$.

   \underline{Claim}: for all $r\le k$, there exists an $n$ such that whenever $[m]^{(n)}$ is $k$-coloured, there exists either:
   \begin{itemize}
       \item a monochromatic line, or
       \item $r$ colour-focused lines
   \end{itemize}
   [Then done: put $r = k$ and look a the focus.]

   \underline{Proof of Claim}: Induction on $r$.

   $r = 1$ is done - take $n = \hj(m-1,k)$.

   Given $n$ suitable for $m-1$, we'll show $n + \hj(m-1,k^{m^n}) = : n + n'$ is suitable for $r$.

   So, given a $k$-colouring $c$ of $[m]^{n+n'}$ with no monochromatic line: View $[m]^{n+n'}$ as $[m]^{n}\times [m]^{n'}$. There are $k^{m^n}$ ways to colour $[m]^{n}\times [m]^{n'}$. At each point of $[m]^{n'}$ we have one of $k^{m^n}$ ``patterns''.

   So by choice of $n'$, we have a line $L$ in $[m]^{n'}$ such that for all $a \in [m]^{n}$, for all $b,b' \in L - \{L^+\}$ we have $c(a,b) = c(a,b') =: c'(a)$, say. By definition of $n$, $c'$ has $r-1$ colour-focused lines, say $L_1,\dots,L_{r-1}$ with active coordinate sets $J_1,\dots,J_{r-1}$ respectively, focused at $f$. Let $L$ have active coord set $I$. Then the lines $L_1',\dots,L_{r-1}'$, where $L_i'$ starts at $(L_i^-,L^-)$ and has active coord set $J_i\cup I$, are colour-focused at $(f,L^+)$. Also the line starting $(f,L^-)$ with active coord set $I$ is monochromatic (apart from final point), of a different colour to the $L_i'$s, giving $r$ colour-focused lines.

   This completes the induction, hence the proof of the claim, has the whole proof.
\end{proof}

What is a line though? If you think about it, it's just a one-dimensional subspace. This begs the question, can we perhaps get a monochromatic \it{two}-dimesional subspace?

A \undf{$d$-parameter set} or \undf{$d$-dimensional subspace} of $X^n$ is a set $S \subset X^n$ such that there exist disjoint, non-empty $I_1,\dots,I_d\subset [n]$ and $a_i \in X$, each $i \in [n]-(I_1\cup\dots\cup I_d)$ such that
\begin{align*}
    S = \{x \in X^n : x_i = a_i\ \forall i\not\in I_1\cup\dots\cup I_d,\textrm{ and }x_i = x_j\ \forall i,j\textrm{ with }i,j\in I_k\textrm{ for some }k\}
\end{align*}
For instance, in $X^3$:
\begin{itemize}
    \item $\{(x,y,2):x,y\in X\}$ is a $2$-parameter set: $I_1 = \{1\}$, $I_2 = \{2\}$
    \item $\{(x,y,x):x,y\in X\}$ is a $2$-parameter set: $I_1 = \{3\}$, $I_2 = \{2\}$
\end{itemize}

This is in fact true.

\begin{theorem}[Extended Hales-Jewett Theorem]
    For all $m,k,d$ ther exists $m$ such that whenever $[m]^{n}$ is $k$-coloured, there exists a monochromatic $d$-parameter set.
\end{theorem}
\begin{proof}
    View $X^{dn}$ (the $dn$-dimensional cube over alphabet $X$) as $(X^d)^n$, the $n$-dimensional cube on alphabet $X^d$. Clearly any line in $(X^d)^n$ (alphabet $X^d$) corresponds to a $d$-parameter set in $X^{dn}$ (alphabet $X$). So we are done - we can take $n = d\cdot \hj(m^d,k)$.
\end{proof}

Let $S$ be a finite subset of $\N^d$. A \undf{homothetic copy} of $A$ is any set of the form $a + \lambda S$, where $a \in \N^d$ and $\lambda \in S$.

For instance, in $\N^1$, a homothetic copy of $\{1,2,\dots,m\}$ is precisely an AP of length $m$.

Can we guarantee a homothetic copy of $S$ that is monochromatic? For instance, in $\N^2$, can we find a monochromatic square?

\begin{theorem}[Gallai's Theorem]
    Let $S$ be a finite subset of $\N^d$. Then, whenever $\N^d$ is fiinitely coloured, there exists a monochromatic homothetic copy of $S$.
\end{theorem}
\begin{proof}
    Let $S = \{S(1),\dots,S(m)\}$. Given a $k$-colouring $c$ of $\N^d$, induce a $k$-colouring $c'$ of $[m]^n$ ($n$ large) by: $c'(x_1,x_2,\dots,x_d) = c(S(x_1)+\dots + S(x_d))$.

    Then we have a monochromatic line $L$ for $c'$ ($n$ large), say with active coord set $I$. But now, taking $S(x_1)+\dots + S(x_d)$ for each $x \in L$, we have a monochromatic, homothetic copy of $S$ (with $\lambda = |I|$).
\end{proof}
\begin{remark*}[Remarks]\ 
    \begin{enumerate}
        \item Suppose $S = \{(0,0),(1,0),(0,1),(1,1)\} \subset \N^2$. Then we get a monochromatic square. Could we instead have applied Extended HJ, $d = 2$, on alphabet of size $2$?
        
        The answer is \it{no}: this would only give a monochromatic rectangle.

        \item Or, can prove Gallai by product arguments and focusing (don't try it, symbol overload, but similar argument).
    \end{enumerate}
\end{remark*}

\marginpar{Lecture 7}

\section{Partition Regular Equations}

This follows on from the concrete elements of Chapter 1.

\subsection*{Rado's Theorem}

Schur says: $\wnfc$ (``whenever the naturals are finitely coloured''), there exists monochromatic $x,y,z$ with $x + y = z$.

Strengthened VDW says: $\wnfc$ there exists monochromatic $x_1,x_2,y_1,y_2,\dots,y_m$ such that $y_1 = x_1 + x_2, y_2 = x_1 + 2x_2,\dots,y_m = x_1 + mx_2$.

Let $A$ be an $m\times n$ matrix with rational entries. We say $A$ is \undf{partition regular} if $\wnfc$ there exists a monochromatic $x \in \N^n$ with $Ax = 0$.

\begin{remark*}[Examples]\ 
    \begin{enumerate}
        \item Schur states that $(1,1,-1)$ is PR, as this asserts that $\wnfc$ there exists monochromatic $x,y,z$ such that $(1,1,-1) (x,y,z)^T = 0$
        \item Strengthened VDW says:
        \begin{align*}
            \left(\begin{array}{cccccccc}
                1 & 1 & -1 & 0 & 0 & 0 & \dots & 0\\
                1 & 2 & 0 & -1 & 0 & 0 & \dots & 0\\
                1 & 3 & 0 & 0 & -1 & 0 & \dots & 0\\
                \vdots & & & & & & \ddots & \\
                1 & m & 0 & 0 & 0 & 0 &\dots & -1\\
            \end{array}\right)
        \end{align*}
        is partition regular.

        \item How about some other examples, say $(2,3,-5)$? This is partition regular. That statement asserts that $\wnfc$ we can find $x,y,z$ monochromatic such that $2x + 3y - 5z = 0$. Indeed, this is easy; just take $x = y = z$.
        
        \item But what about $(2,3,-6)$? This doesn't have an immediate solution like the above.
    \end{enumerate}
\end{remark*}

\begin{remark*}[Notes]\ 
    \begin{enumerate}
        \item We have that $A$ is PR $\iff$ $\lambda A$ PR, for any $\lambda \in \Q \backslash \{0\}$. So we could assume that all entries of $A$ are integers. We will find that this is sometimes helpful.
        \item We can also say `the system of equations $Ax = 0$ is PR'.
        \item Not all matrices are PR. For example, $(1,-2)$ is not PR. Indeed, if it is PR then this asserts that $\wnfc$ there exists $x\in \N$ with $x$ and $2x$ the same colour; this is clearly false. For instance, $2$-colour $\N$ by $c(x) = \max\{n:2^n |x\} \mod 2$.
    \end{enumerate}
\end{remark*}

\begin{defin*}
    Let $A$ as before be an $m\times n$ rational matrix, with columns $c^{(1)},\dots,c^{(n)}\in\Q^m$:
    \begin{align*}
        A = \left( \begin{array}{cccc}
            \uparrow & \uparrow & & \uparrow\\
            c^{(1)} & c^{(2)} & \dots & c^{(n)}\\
            \downarrow & \downarrow & & \downarrow
        \end{array}\right)
    \end{align*}

    Say $A$ has the \undf{columns property} if there exists a partition $B_1\cup\dots\cup B_r$ of $[n]$ such that:
    \begin{itemize}
        \item $\sum_{i\in B_1}c^{(i)} = \underline{0}$
        \item $\sum_{i\in B_s}c^{(i)} \in \langle c^{(i)} : i \in B_1\cup\dots \cup B_{s-1}\rangle$, for each $s = 2,\dots,r$ ($\langle \rangle$ denotes linear span, say over $\R$).
    \end{itemize}
\end{defin*}

\begin{remark*}[Examples]\ 
    \begin{enumerate}
        \item $(1,1,-1)$ has the columns property; $B_1 = \{1,3\}$, $B_2 = \{2\}$.
        \item $(2,3,-5)$: $B_1 = \{1,2,3\}$, so this has CP.
        \item $(1,-2)$ does \it{not} have CP since no subset of the columns sums to zero.
        [And $(1,-\lambda)$ has CP $\iff \lambda = 1$. Also, $(1, -\lambda)$ is PR $\iff \lambda = 1$]
        \item Recall our VdW matrix:
        \begin{align*}
            \left(\begin{array}{cccccccc}
                1 & 1 & -1 & 0 & 0 & 0 & \dots & 0\\
                1 & 2 & 0 & -1 & 0 & 0 & \dots & 0\\
                1 & 3 & 0 & 0 & -1 & 0 & \dots & 0\\
                \vdots & & & & & & \ddots & \\
                1 & m & 0 & 0 & 0 & 0 &\dots & -1\\
            \end{array}\right)
        \end{align*}
        Does this have CP? Notice that $B_1 = \{1,3,\dots,n+2\}$ is a good start, and then $B_2 = \{2\}$ finishes it off since the $B_1$-columns span the whole space.
    \end{enumerate}
\end{remark*}

\begin{remark*}
    Could also have used linear span over $\Q$ (if a rational vector is a real linear combination of other rational vectors, then it is also a rational L.C. of them). It doesn't much matter which we use.
\end{remark*}

Of course, it is looking a lot like we might have CP$\iff$PR, and indeed this is Rado's Theorem; our next aim in the course.

\begin{remark*}[Remarks]\ 
    \begin{enumerate}
        \item This us a finite-time check for PR, which is a significant improvement over the previous infinite-time check we had (go through all the colourings).
        \item This theorem has a very unique property in mathematics, namely that \it{neither} direction of this is obvious.
    \end{enumerate}
\end{remark*}

For clarity, we will start with Rado for a \it{single} equation, \it{i.e.} $m = 1$. Note that $(a_1,\dots,a_n)$ has CP $\iff$ some (non-empty) subset of the (non-zero) $a_i$ sums to zero, or all the $a_i = 0$.

In other words, our task is, given non-zero $a_1,\dots,a_n\in \Q$: show that the equation $a_1x_1+\dots + a_nx_n = 0$ is PR $\iff$ $\sum_{i\in I}a_i = 0$ for some non-empty $I\subset [n]$.

\underline{Note}: Still, in this case, it is completely non-obvious in either direction. Which direction is harder? $\implies$ is going infinite to finite; we might expect to find a clever colouring that proves the RHS must hold. On the other hand, $\impliedby$ appears harder, as we go from finite to infinite.

Fix a prime $p$. Write $d(x)$ for the last non-zero digit in the base $p$ expansion of $x$. \it{I.e.} if $x = d_rp^r + d_{r-1}p^{r-1}+\dots + d_1p^1+d_0$, $(0\le d_i < p\ \forall i)$, then $d(x) = d_{L(X)}$, where $L(x) = \min\{i;d_i\ne 0\}$.

\it{E.g.} if $x$ is $10002070430000$ in base $p$, then $L(x) = 4$ and $d(x) = 3$.

This is a colouring of $\N$ with $p-1$ colours.

\begin{prop}
    Let $a_1,\dots,a_n\in \A\backslash \{0\}$ such that $(a_1,\dots,a_n)$ PR. Then $\sum_{i\in I}a_i = 0$ for some $\emptyset\ne I\subset [n]$.
\end{prop}
\begin{proof}
    Multiplying up, we may assume wlog that $\forall i (a_i \in \Z)$. Fix a large prime $p$ - say $p > \sum |a_i|$ - and $(p-1)$-colour $\N$ as above.

    We then have monochromatic $x_1,\dots,x_n$ with $a_1x_1+\dots + a_nx_n = 0$ - say $x_1,\dots,x_n$ all have colour $d$. Let $L = \min\{L(x_1),\dots,L(x_n)\}$, and put $I = \{i:L(x_i) = L\}$. Considering $\sum a_i x_i$ evaluated in base $p$, we have $\sum_{i\in I}a_i d \equiv 0 \mod p$.

    Thus $\sum_{i\in I}a_i \equiv 0\mod p$, since $p$ prime. But since $p$ was large, we in fact have $\sum_{i\in I}a_i = 0$.
\end{proof}

\begin{remark*}[Remarks]\ 
    \begin{enumerate}
        \item Instead of picking one cunning prime, we can run this for all $p$ and obtain $I_p\subset [n]$ such that $\sum_{i\in I_p} \equiv 0\mod p$. Since there are only finitely many subsets, there must be some $I$ such that $\sum_{i\in I}a_i\equiv 0\mod p$ for infinitely many $p$. So $\sum_{i\in I}a_i = 0$. In the proof of the whole theorem, we will have to use this approach rather than the `pick a good $p$ approach'.
        \item We looked at the \it{end} in base $p$; can also do the \it{start} in base $p$. But this is much harder and more fiddly.
        \item Apart from the above, no other proof of Proposition 2.1 is known! This is because it is so difficult to find colourings that `mesh' well with addition.
    \end{enumerate}
\end{remark*}

\underline{Other direction}: Start with the first non-trivial case, namely $(1,\lambda,-1)$.

\begin{lemma}
    Let $\lambda \in \Q$. $\wnfc$ there exists monochromatic $x,y,z$ with $ x + \lambda y = z$.
\end{lemma}
\begin{proof}
    This is trivial if $\lambda = 0$, and if $\lambda < 0$, rewrite as $z - \lambda y = x$. So wlog assume $\lambda > 0$. Write $\lambda = r/s$, for $r,s\in\N$.

    \underline{Task}: $\forall k\exists n$ such that whenever $[n]J$ is $k$-coloured there exists monochromatic $x,y,z$ with $ x + (r/s)y = z$. We proceed by induction on $k$.

    $k = 1$: trivial; $n = \max (s,r+1), x = 1, y=s,z=r+1$.

    Given $n$ suitable for $k-1$, we'll show that $sW(nr+1,k)$ is suitable for $k$.

    Indeed, given a $k$-colouring of $[sW(nr+1,k)]$: inside $[W(nr+1,k)]$ we have a monochromatic AP of length $nr+1$, say $a,a+d,a+2d,\dots,a+(nr)d$ are all red. Now if any $isd$, $1\le i\le n$ is red, then we are done: take $x = a, y = isd, z = a + ird$. Note that we may have $s > r$, which is why we took $s$ copies of $W(nr+1,k)$ - just to be sure all the above actually have colours. So wlog $\{sd,2sd,\dots,nsd\}$ is $(k-1)$-coloured, so done by choice of $n$.
\end{proof}

\begin{remark*}[Remarks]\ 
    \begin{enumerate}
        \item Very similar to proof of strengthened VdW.
        \item For general $\lambda$, seems not to follow just from Ramsey's Theorem - unlike the case $\lambda = 1$.
    \end{enumerate}
\end{remark*}

\begin{theorem}[Rado for a single equation]
    Let $a_1,\dots,a_n \in \Q\backslash \{0\}$. Then $(a_1,\dots,a_n)$ PR $\iff$ $\sum_{i\in I}a_i = 0$ for some non-empty $I\subset [n]$.    
\end{theorem}
\begin{proof}
    \underline{$\implies$}: Prop 2.1.

    \underline{$\impliedby$}: Fix some $i_0 \in I$. For suitable $x,y,z$ we'll put
    \begin{align*}
        x_i = \left\lbrace \begin{array}{cc} x\ &\textrm{ if }i = i_0\\ z\ &\textrm{ if }i\in I \backslash \{i_0\}\\ y\ &\textrm{ if }i\not\in I \end{array}\right.
    \end{align*}

So need $x,y,z$ monochromatic and:
\begin{align*}
    a_{i_0}x + \left(\sum_{i\in I\backslash \{i_0\}}a_i\right)z + \left(\sum_{i\not\in I}a_i\right) y = 0\\
    \iff a_{i_0}x - a_{i_0}z + \left(\sum_{i\not\in I}a_i\right)y = 0\\
    \iff x - z + \frac{1}{a_{i_0}}\left(\sum_{i\not\in I}a_i\right)y = 0 
\end{align*}
So done by Lemma 2.2.
\end{proof}

There's an interesting open problem here:

\begin{remark*}[Rado's Boundedness Conjecture (1930)]
    \emph{
        Suppose that the $m\times n$ matrix $A$ is not PR. So, for some $k$ there exists a bad (no monochromatic solution) $k$-colouring of $\N$. Can we bound $k$, in terms of $m,n$?
    }
    \emph{
        In other words, is there a function $k = k(m,n)$ such that for any $m\times n$ matrix $A$, we have: PR for $k(m,n)$ colours $\implies$ PR for any number of colours?
    }
\end{remark*}

This is known for $1\times 3$ - but even here it is a very hard problem. It turns out that (Fox \& Kleitman, 2006) 24 colours suffice. But this is open even for $1\times 4$.

\begin{prop}
    Let $A$ be an $m\times n$ rational matrix that is PR. Then $A$ has CP.
\end{prop}
\begin{proof}
    wlog all entries of $A$ are integers. Let columns of $A$ be $c^{(1)},\dots,c^{(n)}\in \Z^m$.

    Given a prime $p$, colour $\N$ by giving $x \in \N$ the colour $d(x)$ as before (the last non-zero digit of $x$ when written in base $p$). Then we have a monochromatic solution $x_1,\dots,x_n \in \N$ with $Ax = 0$, \it{i.e.} $\sum x_ic^{(i)} = 0$: say all $x_i$ have colour $d$.

    Partition $[n]$ as $B_1\cup\dots\cup B_r$, where:
    \begin{itemize}
        \item $i,j \in B_s \iff L(x_i) = L(x_j)$ (for any $s$)
        \item $i \in B_s, j \in B_t \iff L(x_i) < L(x_j)$ (for any $s < t$)
    \end{itemize}

    Infinitely many primes $p$ given the same parition - say for all $p \in P$. Fix $p \in P$ and consider $\sum x_ic^{(i)} = 0$, calculated in base $p$. We have:
    \begin{itemize}
        \item $\sum_{i \in B_1}dc^{(i)} \equiv 0\mod p$ (where $u\equiv v \mod p$ means $u_i \equiv v_i \mod p$ for each $i$).
        \item For each $s \ge 2$: $$p^t\sum_{i \in B_s}dc^{(i)} + \sum_{i \in B_1\cup\dots\cup B_{s-1}}x_ic^{(i)}\equiv 0\mod p^{t+1}$$ for some $t$.
    \end{itemize}

    From the first of these, we get that $\sum_{i\in B_1}c^{(i)} \equiv 0 \mod p$ (since $d$ is invertible). But this holds for all $p \in P$, $P$ infinite, whence $\sum_{i \in B_1}c^{(i)} = 0$. A good start.

    From the second bullet point, for each $s \ge 2$: $$p^t\sum_{i \in B_s}c^{(i)} + \sum_{i \in B_1\cup\dots B_{s-1}}(d^{-1}x_i)c^{(i)} \equiv 0 \mod p^{t+1}$$ where $d^{-1}$ is the inverse of $d \mod p^{t+1}$.

    \underline{Claim}: $\sum_{i \in B_s}c^{(i)}\in \langle c^{(i)}:i\in B_1\cup\dots B_{s-1}\rangle$.

    \underline{Proof of Claim}: Suppose not. Then there exists $u \in \Z^m$ with $u\cdot c^{(i)} = 0$ for all $i \in B_1\cup\dots B_{s-1}$, but $u\cdot \sum_{i \in B_s}c^{(i)}\ne 0$.

    Dot our equation with $u$:
    \begin{align*}
        p^t u\cdot \sum_{i \in B_s}c^{(i)} + 0 \equiv 0\mod p^{t+1}
    \end{align*}

    So $u\cdot \sum_{i \in B_s}c^{(i)} \equiv 0 \mod p$. This holds for all $p \in P$, so $u\cdot \sum_{i \in B_s}c^{(i)} = 0$, contradiction.\qedsymbol

    This concludes the proof of the claim, and hence the whole proposition.
\end{proof}

\begin{defin*}[$(m,p,c)$-set]
    Let $m,p,c \in \N$. A subset $S \subset \N$ is called an \undf{$(m,c,p)$-set} if $\exists x_1,\dots,x_n \in \N$ (the `generators' of $s$) such that:
    \begin{align*}
        S = \left\lbrace \sum_{i=1}^{n}\lambda_i x_i : \exists j\textrm{ with }\lambda_i = 0\ \forall i < j, \lambda _j = c, \lambda_i \in [-p,p]\ \forall i > j\right\rbrace
    \end{align*}

    So $S$ consists of all:
    \begin{align*}
        cx_1 + \lambda_2 x_x + \lambda_3 x_3+\dots + \lambda_m x_m&:\lambda_i\in[-p,p]\ \forall i\\
        cx_2 + \lambda_3x_3 +\dots + \lambda_m x_m &: \lambda_i \in [-p,p]\ \forall i\\
        cx_{m-1} + \lambda_m x_m &: \lambda_m \in [-p,p]\\
        cx_m &
    \end{align*}
    These are called the \undf{rows} of $S$. So it is like `iterated AP + CD'.
\end{defin*}

\begin{remark*}[Examples]\ 
    \begin{enumerate}
        \item A $(2,p,1)$-set is $$\{x_1-px_2,x_1-(p-1)x_2,\dots,x_1+px_2,x_2\}$$ \it{i.e.} an AP of length $2p+1$ together with its CD.
        \item A $(2,p,3)$-set is $$\{3x_1-px_2,3x_1-(p-1)x_2,\dots,3x_1+px_2,3x_2\}$$ \it{i.e.} an AP of length $2p+1$ with middle term a multiple of $3$, together with $3$ times its CD.
    \end{enumerate}
\end{remark*}

\begin{theorem}
    Let $m,p,c\in\N$. Then $\wnfc$ there exists a monochromatic $(m,p,c)$-set.
\end{theorem}
\begin{proof}
    Fix $k$, the number of colours.

    \underline{Claim}: $\forall M\exists n$ such that whenever $[n]$ is $k$-coloured, there exists an $(m,p,c)$-set with each row monochromatic.

    [Then done: put $M = km$ and observe that some $m$ rows must be the same colour - yielding a mono $(m,p,c)$-set.]

    \underline{Proof of Claim}: Induction on $M$: $M = 1$ is trivial, since the $(m,p,c)$ set is a single point.

    Given $M > 1$: consider a $k$-colouring of $[n]$, $n$ large (this will mean `large enough for what is chosen later'). Inside $\{c,2c,3c,\dots,\lfloor n/c \rfloor c\}$, we have an AP of length $2d+1$ ($d$ large), say: $$A = \{cx_1 - da, cx_1 - (d-1)a,\dots,cx_1,\dots,cx_1+da\}$$

    Let $t = \lfloor d/(mp)\rfloor$, and consider $\{a,2a,3a,\dots,ta\}$. If $t$ large enough, this set contains (induction hypothesis) an $(M-1,p,c)$-set with all rows monochromatic: say on generators $x_2,\dots,x_M$.

    Then $cx_1 + \lambda_2x_2 + \dots + \lambda_m x_m\in A$ for any $lambda_2,\dots,\lambda_m$ with $|\lambda_i|\le p$ for each $i$, and so the $(M,p,c)$-set on generators $x_1,\dots,x_m$ has all rows monochromatic.
\end{proof}

The special case $(m,1,1)$ immediately gives the following.

For $x_1,\dots,x_m \in \N$, write $\fs(x_1,\dots,x_m)$ for $\{\sum_{i\in I}x_i:\emptyset\ne I\subset [m]\}$.

\begin{cor}[Finite Sums Theorem/ Folkman's Theorem/ Sanders' Theorem]\ \\
    $\forall m$, $\wnfc$ there exists $x_1,\dots,x_m$ such that $\fs(x_1,\dots,x_m)$ monochromatic. \qedsymbol
\end{cor}

\begin{remark*}[Remarks]
    \begin{enumerate}
        \item Case $m = 2$ is Schur's Theorem.
        \item Hence, also $\wnfc$ there exists $x_1,\dots,x_m$ such that FP$(x_1,\dots,x_m)$ is monochromatic (where $\fp(x_1,\dots,x_m) = \{\prod_{i\in I}x_i : \emptyset \ne I\subset [m]\}$) - just look at $\{2^1,2^2,2^3,\dots\}$ and apply finite sums theorem.
        \item How about $\fs(x_1,\dots,x_m)\cup\fp(x_1,\dots,x_m)$? This is currently unknown, even for the first case $m = 2$: can we always find $x,y,x+y,xy$ monochromatic?
        
        The reason why this is difficult is becaue addition and multiplication don't really mesh well together. In our proofs of things like VdW, we often use `scaled up' versions of the induction hypothesis. We can scale up w.r.t. addition and scale up w.r.t multiplication, but not w.r.t. both at the same time (in any obvious way).

        It is known that $\wnfc$ there exists $x,y$ with $x+y,xy$ of the same colour (Moreira, 2017) (apart from $x = y = 2$).
    \end{enumerate}
\end{remark*}

\begin{prop}
    Let the matrix $A$ have the columns property. Then there exists $m,p,c \in \N$ such that every $(m,p,c)$-set contains a solution of $Ax = 0$, \it{i.e.} each entry of $x$ is in the $(m,p,c)$-set.
\end{prop}
\begin{proof}
    Let the columns of $A$ be $c^{(1)},\dots,c^{(n)}$. Have a partition $B_1\cup\dots\cup B_r$ of $[n]$ such that $\forall s$, $\sum_{i\in B_s}c^{(i)} \in \langle c^{(i)}: i \in B_1\cup\dots\cup B_{s-1}\rangle$: say $\sum_{i \in B_s}c^{(i)} = \sum_{i \in B_1\cup\dots\cup B_{s-1}}q_{is}c^{(i)}$, for some $q_{is} \in \Q$.

    Thus $\forall s$ have $\sum_{i=1}^{n}d_{is}c^{(i)} = 0$, where:
    \begin{align*}
        d_{is} = \left\lbrace \begin{array}{cl} 0 & \textrm{ if }i\not\in B_1\cup\dots\cup B_s\\
            1 & \textrm{ if }i\in B_s\\
            -q_{is}&\textrm{ if }i \in B_1\cup\dots\cup B_{s-1} 
        \end{array}\right.
    \end{align*}

    Given $x_1,\dots,x_r \in \N$, define $y_1,\dots,y_n$ by: $y_i = \sum_{s}d_{is}x_s$. Then 
    \begin{align*}
        A_y = \sum_{i}c^{(i)}y_i &= \sum_{i}c^{(i)}\sum_{j}d_{is}x_s\\
        &= \sum_{s}x_s\sum_{i}d_{is}c^{(i)}\\
        &=0
    \end{align*}

    So done: take $m =r$, $c = $ LCM of denominators of the $q_{is}$, $p = c\times$ max numerator of the $q_{is}$.
\end{proof}

\begin{center}
    \includegraphics[scale=0.8]{together.jpg}
\end{center}

\begin{theorem}[Rado's Theorem]
    Let $A$ be a matrix with rational entries. Then $A$ is PR $\iff$ $A$ has CP.
\end{theorem}
\begin{proof} We have proven both directions already:

    \underline{$\implies$}: Prop 2.4.

    \underline{$\impliedby$}: Theorem 2.5 \& Prop 2.7.
\end{proof}

\begin{remark*}[Remarks]\ 
    \item Having proved Rado, results like VdW, Schur, finite sums \it{etc}... are just trivial CP checks.
    \item From the proof of Rado, we have: if matrix $A$ is PR for each `last digit base $p$'-colouring, then $A$ is PR for all colourings. No direct proof (\it{i.e.} not via Rado) is known.    
\end{remark*}

\begin{theorem}[Consistency Theorem]
    Let $A,B$ be partition regular matrices. Then $\left(\begin{array}{cc}A & 0\\ 0 & B\end{array}\right)$ is PR.

    [In other words, if we can always solve $Ax = 0$ in one colour class and $By = 0$ in one colour class, then we can solve them both in the same colour class.]
\end{theorem}
\begin{proof}
    Trivial by CP: if $A$ has CP and $B$ has CP, then $\left(\begin{array}{cc}A & 0\\ 0 & B\end{array}\right)$ has CP.
\end{proof}
\begin{remark*}[Remarks]\ 
    \begin{itemize}
        \item Highly non-obvious from definition of PR.
        \item This \it{can} be proved directly - harder.
    \end{itemize}
\end{remark*}

\begin{theorem}
    $\wnfc$ there exists a colour class containing a solution to \textbf{every} PR sytem of equations.
\end{theorem}
\begin{proof}
    Suppose not: then we have a partition $D_1\cup\dots D_k$ of $\N$, such that for every $i$ there exists a PR matrix $A_i$ such that $A_ix = 0$ has no solution for $x$ in $D_i$.

    But now let $A = \textrm{diag}(A_1,A_2,\dots,A_k)$. Then $A$ is PR (by Consistency Theorem), but no $D_i$ contains a solution of $Ax = 0$. Contradiction.
\end{proof}

\begin{remark*}[Rado's Conjecture (1933)]
    Say $D \subset \N$ is \undf{partition regular} if $D$ contains a solution to every PR system of equations. So Theorem 2.10 says: if $\N = D_1\cup\dots\cup D_k$, then some $D_i$ is PR

    Rado then asked:
    
    \emph{
        If $D$ is PR and $D = D_1\cup\dots\cup D_k$, must some $D_i$ be PR?
    }

    Proved by Deuber (1975). He introduced $(m,p,c)$-sets, and he proved a fact we know: $D$ is PR $\iff$ for every $m,p,c$, $D$ contains an $(m,p,c)$-set. He showed that $\forall m,p,c,k,\ \exists n,q,d$ such that whenever we $k$-colour an $(n,q,d)$-set there exists a monochromatic $(m,p,c)$-set, thus establishing Rado's Conjecture. This last fact is similar to our proof of Theorem 2.5, but replacing VdW by extended HJ.
\end{remark*}

\subsection*{Ultrafilters}

For a sequence $x_1,x_2,\dots \in \N$, we write $$\fs(x_1,x_2,\dots) \coloneqq \left\lbrace\sum_{i\in I}s_i : I\subset \N, |I| < \infty, I\ne \emptyset\right\rbrace$$

\underline{Aim}: Prove \textbf{Hinderman's Theorem}: $\wnfc$ there exists $x_1,x_2,\dots$ with $\fs(x_1,x_2,\dots)$ monochromatic.

\underline{Idea}: A \it{filter} is a notion of `large' for subset of $\N$; an \it{ultrafilter} is a more refined such notion.

\begin{defin*}[Filter]
    A \undf{filter} is a non-empty family $\F \subset \P(\N)$ such that:
    \begin{enumerate}[label = \roman*)]
        \item $\emptyset \not\in \F$
        \item if $A\in \F$ and $A\subset B$ then $B \in \F$ (``closed under supersets'')
        \item if $A \in \F$ and $B \in \F$ then $A\cap B \in \F$ (``closed under finite intersections'')
    \end{enumerate}
\end{defin*}
\begin{remark*}[Examples]\ 
    \begin{enumerate}[label = \arabic*)]
        \item $\{A\subset \N : 1 \in A\}$
        \item $\{A\subset \N : 1,2\in A\}$
        \item \it{not} $\{A\subset \N : |A| = \aleph_0\}$ (can find two such sets with finite intersection)
        \item $\{A\subset \N: |A^c| < \aleph_0\}$ - this is the \undf{cofinite filter}
        \item $\{A\subset \N: |E\backslash A| < \aleph_0\}$, where $E$ is the set of all even numbers
    \end{enumerate}
\end{remark*}

\begin{defin*}[Ultrafilter]
    An \undf{ultrafilter} is a maximal filter, \it{i.e.} it is contained in no other filter.
\end{defin*}
\begin{remark*}[Examples]\
    Let's take another look at the above examples. 
    \begin{enumerate}[label = \arabic*)]
        \item Yes, and for each $x \in \N$, we have $\{A\subset \N : x \in A\}$ the \undf{principal ultrafilter} at $x$, written $\tilde{x}$.
        \item No, it is contained in 1)
        \item Not a filter
        \item No, it is contained in 5)
        \item Also no, contained in $\{A\subset \N : |M\backslash A| < \aleph_0\}$, where $M$ is the set of all multiples of 4.
    \end{enumerate}
\end{remark*}

\begin{prop}
    A filter $\F$ is an ultrafilter iff for every $A \subset \N$, either $A \in \F$ or $A^c\in \F$.
\end{prop}
\begin{proof}
    \underline{$\impliedby$}: Trivial. Cannot add $A\subset \N$ if $\F$ already contains $A^c$.

    \underline{$\implies$}: Suppose $A,A^c \not\in \F$. Then we must have some $B \in \F$ such that $B\cap A = \emptyset$, since otherwise $\{C\subset \N : C\supseteq A\cap B,\textrm{ some }B\in \F\}$ is a filter that extends $F$. But then $B\cap A = \emptyset$, so $B\subset A^c$, whence $A^c \in \F$. Contradiction.
\end{proof}

This makes ultrafilters feel a little more tangible.

\begin{remark*}
    Hence also: for an ultrafilter $\U$, if $A \in \U$ and $A = B\cup C$ then $B \in \U$ or $C \in \U$.

    Indeed, if $B,C\not\in\U$, then $B^c,C^c\in \U$, whence $A^c = B^c \cap C^c \in \U$. Contradiction.
\end{remark*}

\begin{theorem}
    Every filter is contained in an ultrafilter.
\end{theorem}
\begin{proof}
    For a filter $\F_0$, we seek a maximal element of $X = \{\F\textrm{ filter}: \F\supseteq \F_0\}$. By Zorn, it is enough to check that any non-empty chain $\{\F_i:i\in I\}$ in $X$ has an upper bound in $X$:

    Let $\F = \bigcup_{i\in I}\F_i$. Trivially $\F_0 \subset \F$; need to show $\F$ is a filter:
    \begin{enumerate}[label=\roman*)]
        \item $\emptyset\not\in \F$, as $\emptyset\not\in \F_i$ for all $i$.
        \item Given $A \in \F$, $B\supseteq A$: have $A \in \F_i$ for some $i$, so $B \in \F_i$.
        \item Given $A,B\in \F$, have $A \in \F_i,B\in \F_j$, where wlog $\F_i\supset \F_j$. So $A\cap B \in \F_j$.
    \end{enumerate}
    So every chain has an upper bound; by Zorn $X$ has a maximal filter extending $\F_0$.
\end{proof}

\begin{remark*}[Remarks]\ 
    \begin{enumerate}
        \item So there exists a non-principal ultrafilter - just take any ultrafilter extending the cofinite filter.
        \item Conversely, if ultrafilter $\U$ is non-principal, then it must extend the cofinite filter.
        
        Indeed, suppose not. Then $A\in \U$ for some finite $A$. Whence $\{x\} \in \U$ for some $x \in A$ (by repeated use of the `$B\cup C$' remark).

        \item Some form of AC \it{is needed} to get a non-principal ultrafilter. Note that this means that if you think you've written down an explicit non-principal ultrafilter, you are wrong (normally you've messed up the intersection property).
    \end{enumerate}
\end{remark*}

Write $\beta\N$ for the set of all ultrafilters on $\N$. We can define a topology on $\beta\N$ by taking as basic open sets the sets $C_A = \{\U\in \beta\N : A\in \U\}$, for each $A\subset \N$.

This really is a base for a topology - just need $\bigcup_{A}C_A = \beta\N$, and $C_A\cap C_B$ is open. Certainly $\bigcup_A C_A = \beta\N$, and also $C_A \cap C_B = C_{A\cap B}$ (as $A\cap B \in \U \iff A,B\in \U$).

Thus the open sets are all sets $\bigcup_{i\in I}C_{A_i} = \{\U : A_i \in U,\textrm{ some }i\}$.

The basic closed sets are $\beta\N\backslash C_A = C_{A^c}$. Thus the closed sets are all sets of the form $\bigcap C_{A_i} = \{\U:\ \forall i(A_i\in \U)\}$.

We can view $\N$ as a subset of $\beta\N$, by identifying $n\in \N$ with $\tilde{n} \in \beta\N$. Note that each $\tilde{n}$ is isolated, since $\{\tilde{n}\} = C_{\{n\}}$. Also, $\N$ is dense in $\beta\N$ - indeed, $\tilde{n} \in C_A$ for every $n \in A$.

\begin{theorem}
    $\beta\N$ is a compact Hausdorff space.
\end{theorem}
\begin{proof}
    \underline{Hausdorff}: Given $\U\ne \V$, there exists some $A \in \U$ with $A\not\in \V$. So $A^c \in \V$. Thus $\U \in C_A, \V \in C_{A^c}$, and $C_A\cap C_{A^c} = \emptyset$.

    \underline{Compact}: Need to show that if $F_i:i\in I$ are closed sets with the finite intersection property (every finite intersection is non-empty) then $\bigcap_{i\in I}F_i\ne\emptyset$.

    wlog each $F_i$ is a basic closed set; say $F_i = C_{A_i}$. Note that the sets $A_i$ themselves have the finite intersection property, as $C_{A_{i_1}}\cap\dots\cap C_{A_{i_n}} = C_{A_{i_1}\cap\dots\cap A_{i_n}}$, whence $A_{i_1}\cap\dots\cap A_{i_n}\ne \emptyset$.

    Define $\F = \{A\subset\N : A\supseteq A_{i_1}\cap\dots\cap A_{i_n},\textrm{ for some }i_1,\dots,i_n\in I\}$. Then $\F$ is a filter; let $\U \in \beta\N$ extend $\F$. So for all $i$, $A_i\in \U$, \it{i.e.} $\U \in C_{A_i}$ for all $i$.
\end{proof}

So this is a weird space, but it does at least have the warm and reassuring properties of Hausdorfness and compactness.

\begin{remark*}[Remarks]\ 
    \begin{enumerate}
        \item \it{or}, we can view an ultrafilter as a function $\P(\N) \ra \{0,1\}$, \it{i.e.} as a point in $\{0,1\}^{\P(\N)}$. So $\beta\N \subset \{0,1\}^{\P(\N)}$. Can check that our topology on $\beta\N$ is (the restriction of) the product topology, and $\beta\N$ is closed in $\{0,1\}^{\P(\N)}$ - whence $\beta\N$ is compact, by Tychonoff.
        \item Why is $\beta\N$ interesting? It is the biggest compact Hausdorff space in which $\N$ is dense. More precisely, for any compact, Hausdorff $X$ and function $f:\N\ra X$, there exists a unique continuous $\tilde{f}:\beta\N \ra X$ extending $f$. We say that $\beta\N$ it the \it{Stone-Cech compactification} of $\N$.
    \end{enumerate}
\end{remark*}

\begin{defin*}[Ultrafilter Quantifiers]
    For ultrafilter $\U$ and statement $p(x)$, write $\forall_\U xp(x)$, (read as `for most $x$' or `for $\U$-most $x$') if $\{x : p(x)\}\in \U$.
\end{defin*}
\begin{remark*}[Examples]\ 
    \begin{itemize}
        \item For $\U$ non-principal, have $\forall_\U x : x > 17$.
        \item For $\U$ principal at $7$, $\forall_\U xp(x) \iff p(7)$.
    \end{itemize}
\end{remark*}

Ultrafilter quantifiers behave `perfectly' with respect to logical connectives, as follows:

\begin{prop}
    Let $\U$ be an ultrafilter, and $p(x),q(x)$ statements.
    \begin{enumerate}[label = \roman*)]
        \item $\forall_\U x (p(x)\land q(x)) \iff (\forall_\U x p(x))\land (\forall_\U x q(x))$
        \item $\forall_\U x (p(x) \lor q(x)) \iff (\forall_\U xp(x))\lor (\forall_\U xq(x))$
        \item $\neg \forall_\U xp(x) \iff \forall_\U x \neg p(x)$
    \end{enumerate}
\end{prop}
\begin{proof}
    Let $A = \{x : p(x)\}$, and $B = \{x : q(x)\}$.

    So i) says: $A\cap B \in \U \iff A \in \U \land B \in \U$, which is true.

    ii) says: $A\cup B \in \U \iff A\in \U \lor B \in \U$, also true.

    iii) says: $A\not \in \U \iff A^c \in \U$, again true. So done.
\end{proof}

There is a small price to be paid here.

\begin{remark*}
    Not true that $\fu x\for{\V}y p(x,y) \iff \for{\V}y \fu xp(x,y)$, even when $\U = \V$.

    For example, let $\U$ be non-principal. Then $\fu x\fu y x < y$, since for every $x\in \N$ we have $\fu y x < y$. But $\fu y \fu x x <y$ is \it{false}, since there is no $y$ for which $\fu x (x<y)$.
\end{remark*}

\begin{defin*}[Addition on $\beta\N$]
    For $\U,\V\in \beta\N$, let $\U + \V \coloneqq \{A\subset\N : \fu x\for{\V}y (x+y\in A)\}$.

    Equivalently, $\U+\V = \{A\subset\N : \{x\in \N : \{y \in \N : x + y \in A\} \in \V\}\in \U\}$. This is an absolute nightmare without the quantifiers, which is why we use them.
\end{defin*}

\begin{remark*}[Example] $\tilde{m} + \tilde{n} = \widetilde{m+n}$.  
\end{remark*}

Note that $\U + \V$ \it{is} an ultrafilter:
\begin{itemize}
    \item $\emptyset\not\in \U + \V$
    \item if $A \in \U + \V$ and $B \supseteq A$ then $B \in \U + \V$
    \item if $A,B\in \U + \V$ then $\fu x \for{\V} y (x + y \in A)$ and $\fu x \for{\V} y (x + y \in B)$, \it{i.e.} $\fu x\for{\V} y (x + y \in A \cap B)$ by Prop 2.14(i) twice. So $A\cap B \in \U + \V$.
    \item if $A\not\in \U + \V$ then $\neg \fu x\for{\V} y (x + y \in A)$, \it{i.e.} $\fu x \for{\V} y (x + y \in A^c)$ by Prop 2.14(iii) twice. So $A^c \in \U + \V$.
\end{itemize}

\end{document}