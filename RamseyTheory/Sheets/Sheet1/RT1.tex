\documentclass[10pt]{article}

\usepackage{amsmath}
\usepackage{amssymb}
\usepackage{amsthm}
\usepackage{graphicx}
\usepackage{parskip}
\usepackage{xcolor}
\usepackage{pagecolor}
\usepackage[margin=1.2in]{geometry}
\usepackage{enumerate}
\usepackage{enumitem}
\usepackage{tikz}
\newcommand*\circled[1]{%
   \tikz[baseline=(C.base)]\node[draw,circle,inner sep=1.2pt,line width=0.2mm,](C) {#1};}
\newcommand*\Myitem{%
   \stepcounter{enumi}\item[\circled{\theenumi}]}

\usepackage[utf8]{inputenc}
\usepackage[english]{babel}

\usepackage{mathtools}
\DeclarePairedDelimiter\bra{\langle}{\rvert}
\DeclarePairedDelimiter\ket{\lvert}{\rangle}
\DeclarePairedDelimiterX\braket[2]{\langle}{\rangle}{#1 \delimsize\vert #2}

\definecolor{thmcolour}{rgb}{0,0,0}
\definecolor{defcolour}{rgb}{0,0,0}
\definecolor{textcolour}{rgb}{0,0,0}
\definecolor{backgroundcolour}{rgb}{1,1,1}

\pagecolor{backgroundcolour}
\color{textcolour}

\newtheoremstyle{custhm}
{%space above
}{%space below
}{%body font
\color{thmcolour}\em
}{%indent amount
-0em
}{%head font
\bfseries\color{thmcolour}
}{%head punct
}{%after head space
1em
}{%head spec
\thmname{#1}\if\relax\detokenize{#2}\relax:\else\thmnumber{ #2}:\fi\if\relax\detokenize{#3}\relax\else\thmnote{ (#3)}\fi
}

\newtheoremstyle{ex}
{%space above
}{%space below
}{%body font
\color{thmcolour}
}{%indent amount
-0em
}{%head font
\bfseries\color{thmcolour}
}{%head punct
}{%after head space
1em
}{%head spec
\thmname{#1}\if\relax\detokenize{#2}\relax:\else\thmnumber{ #2}:\fi\if\relax\detokenize{#3}\relax\else\thmnote{(#3)}\fi
}

\newtheoremstyle{remark}
{%space above
}{%space below
}{% body font
}{%indent amount
-0em
}{%head font
\bfseries
}{%head punct
}{%after head space
1em
}{%head spec
\if\relax\detokenize{#3}\relax\thmname{#1}:\else\thmname{#3}:\fi
}

\newtheoremstyle{numremark}
{%space above
}{%space below
}{% body font
}{%indent amount
-0em
}{%head font
\bfseries
}{%head punct
}{%after head space
1em
}{%head spec
\thmname{#1}\thmnumber{ #2}:
}

\newtheoremstyle{cusdef}
{%space above
}{%space below
}{%body font
\color{defcolour}
}{%indent amount
-0em
}{%head font
\bfseries\color{defcolour}
}{%head punct
}{%after head space
1em
}{%head spec
%if numbered, include number
%if named, include name
\thmname{#1}\if\relax\detokenize{#2}\relax:\else\thmnumber{ #2}:\fi\if\relax\detokenize{#3}\relax\else\thmnote{ (#3)}\fi
}

\theoremstyle{custhm}
\newtheorem{theorem}{Theorem}[section]
\theoremstyle{cusdef}
\newtheorem{defin}[theorem]{Definition}
\theoremstyle{custhm}
\newtheorem{lemma}[theorem]{Lemma}
\theoremstyle{custhm}
\newtheorem{cor}[theorem]{Corollary}

\theoremstyle{custhm}
\newtheorem{prop}[theorem]{Proposition}

\theoremstyle{ex}
\newtheorem{ex}[theorem]{Example}

\theoremstyle{custhm}
\newtheorem*{theorem*}{Theorem}

\theoremstyle{cusdef}
\newtheorem*{defin*}{Definition}

\theoremstyle{remark}
\newtheorem*{remark*}{Remark}

\theoremstyle{remark}
\newtheorem{remark}[theorem]{Remark}

\theoremstyle{numremark}
\newtheorem{numremark}[theorem]{Remark}

\setcounter{section}{-1}

%\marginpar{to describe which lecture it is}

\newcommand{\N}{\mathbb{N}}
\newcommand{\Z}{\mathbb{Z}}
\newcommand{\Q}{\mathbb{Q}}
\newcommand{\R}{\mathbb{R}}
\newcommand{\C}{\mathbb{C}}
\newcommand{\e}{\mathrm{e}}
\newcommand{\ra}{\rightarrow}
\newcommand{\lef}{\left(}
\newcommand{\res}{\right)}
\newcommand{\ie}{\textit{i.e.}}
\newcommand{\eps}{\varepsilon}
\newcommand{\E}{\mathbb{E}}
\newcommand{\suminf}{\sum_{n=0}^{\infty}}
\newcommand{\suminfa}[1]{\sum_{#1=0}^{\infty}}
\renewcommand{\P}{\mathbb{P}}
\newcommand{\undf}[1]{\textit{\textbf{#1}}}
\renewcommand{\L}{\mathcal{L}}
\renewcommand{\it}[1]{\textit{#1}}
\newcommand{\M}{\mathcal{M}}
\renewcommand{\phi}{\varphi}
\newcommand{\proves}{\vdash}
\newcommand{\lra}{\leftrightarrow}

\renewcommand{\bar}{\overline}
\renewcommand{\O}{\mathcal{O}}


\newcommand{\ac}[1]{\mathcal{#1}}
\newcommand{\A}{\mathcal{A}}


\renewcommand{\subset}{\subseteq}

\renewcommand{\th}{\textrm{th}}

\usepackage{graphicx}

\newcommand{\hj}{\textrm{HJ}}

\title{Ramsey Theory: Example Sheet 1}
\author{Otto Pyper}
\date{}

\begin{document}
\maketitle

1. Let $f : [n] \ra [m]\cup\{\ast\}$ such that $f^{-1}(\{\ast\})\ne\emptyset$. We can identify $f$ with a combinatorial line by $I = \{i : f(i) = \ast\}$ and for $i\not\in I$, $x_i = f(i)$; similarly, given a combintorial line $L$ there is a unique such $f$ that represents it.

So the number of combinatorial lines in $[m]^n$ is just the number of such $f$s, which equals ($\#\{f:[n]\ra [m]\cup\{\ast\}\} - \# \{f:[n]\ra [m]\}) = (m+1)^n - m^n$.\ \\

2. Induction on $k$. Base case: $\hj(2,1) = 1$ since the single combinatorial line must be monochrome, and the empty set has a $1$-colouring with no monochromatic line.

Now assume that $\hj(2,k-1) = k-1$. Observe that $(0,0,\dots,0)\in [2]^k$ forms a combinatorial line with any other point, so $[2]^k \backslash \{(0,\dots,0)\}$ must be $k-1$-coloured. Hence $[2]^{k-1}\times\{1\}$ is $(k-1)$-coloured. The colouring on $[2]^{k-1}\times \{1\}$ induces a colouring on $[2]^{k-1}$, which has a monochromatic combinatorial line $L = \{\underline{x}_1,\underline{x}_2\}$, and so $\{\underline{x}_11,\underline{x}_21\}$ is a monochromatic combinatorial line in $[2]^{k}$. Hence $\hj(2,k) \le k$.

Moreover, $\hj(2,k) > k-1$; we can $k$-colour $[2]^{k-1}$ by $c(x_1,\dots,x_{k-1}) = x_1 + \dots + x_{k-1}$, where the colours are the set $\{0,\dots,k-1\}$. This contains no monochromatic line. Hence $\hj(2,k) = k$ for all $k$.

3. $2$-colour $A^{(4)}$ by giving $B\subset A$ the colour red if any of the points lies within the triangle formed by the other three, and blue otherwise. This is unambiguous since no three points are colinear.

Then by Ramsey's Theorem there exists a monochromatic set $C$ of size $k = 2021$. $C$ is then clearly convex, but this is surprisingly difficult to actually prove.

4. Suppose that this is true, and let $c_n : \N^{(n)}\ra [2]$ be the $2$-colouring whereby $A$ is red if $n \in A$, and blue otherwise. If $M \subset \N$ is infinite and monochromatic under $c_n$, then it cannot contain $n$ since then its colour must be red, but there of course exists $A\subset M\backslash \{n\}$ with $|A| = n$, and then $c_n(A)$ is blue.

So let $c = \bigcup c_n$ be the $2$-colouring of $\bigcup \N^{(n)}$. Then there exists an infinite $M\subset \N$ such that, for each $r$, $c$ is constant on $M^{(r)}$. But for each $r$, $c\restriction M^{(r)} = c_r$, and since $c_r$ is constant on $M$ we conclude that $r \not \in M$. But $r$ was arbitrary, so $M$ is empty, contradiction.

10. With choice: false. Wellorder $\R$ ($<_\alpha$) and colour $\{a,b\}$ red if their order in the wellorder agrees with their natural order in $\R$. A monochromatic subset is then $M = \{a_i : i \in I\}$ with $a_i < a_j$ for $i < j$ such that either of the following holds:
\begin{itemize}
    \item for all $i < j$, $a_i <_\alpha a_j$, or
    \item for all $i < j$, $a_j <_\alpha a_j$
\end{itemize}
But then if $I$ is uncountable, then in the first case $M$ is well-ordered, and in the second case $-M$ is wellordered; in either case we have an uncountable wellordered subset of $\R$. Contradiction.

Without choice: it is consistent in ZF that there are countable pairwise disjiont sets $A_n$ such that $\bigcup_{n\in\N}A_n$ is uncountable. Then $2$-colour the edges of $\bigcup A_n$ by making $\{a,b\}$ red if they lie in the same $A_n$, and blue otherwise. We end up with a countable infinite of red islands, with the edges between all islands being blue. This has no uncountable monochromatic subset.

\end{document}