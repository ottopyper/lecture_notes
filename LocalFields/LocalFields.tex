\documentclass[]{article}


\usepackage{amsmath}
\usepackage{amssymb}
\usepackage{amsthm}
\usepackage{graphicx}
\usepackage{parskip}
\usepackage{xcolor}
\usepackage{pagecolor}
\usepackage[margin=1.2in]{geometry}
\usepackage{enumitem}
\usepackage{stmaryrd}
\usepackage{centernot}

\usepackage[utf8]{inputenc}
\usepackage[english]{babel}

\usepackage{mathtools}
\DeclarePairedDelimiter\bra{\langle}{\rvert}
\DeclarePairedDelimiter\ket{\lvert}{\rangle}
\DeclarePairedDelimiterX\braket[2]{\langle}{\rangle}{#1 \delimsize\vert #2}


\makeatletter
\newcommand{\xMapsto}[2][]{\ext@arrow 0599{\Mapstofill@}{#1}{#2}}
\def\Mapstofill@{\arrowfill@{\Mapstochar\Relbar}\Relbar\Rightarrow}
\makeatother

\definecolor{thmcolour}{rgb}{0,0,0}
\definecolor{defcolour}{rgb}{0,0,0}
\definecolor{textcolour}{rgb}{0,0,0}
\definecolor{backgroundcolour}{rgb}{1,1,1}

\pagecolor{backgroundcolour}
\color{textcolour}

\newtheoremstyle{custhm}
{%space above
1em
}{%space below
1em
}{%body font
\color{thmcolour}\itshape
}{%indent amount
-0em
}{%head font
\bfseries\color{thmcolour}
}{%head punct
}{%after head space
1em
}{%head spec
\thmname{#1}
\if\relax\detokenize{#2}\relax:
\else\thmnumber{ #2}:\fi
\if\relax\detokenize{#3}\relax
\else\thmnote{ (#3)}\fi
}

\newtheoremstyle{remark}
{%space above
}{%space below
}{% body font
}{%indent amount
-0em
}{%head font
\bfseries
}{%head punct
}{%after head space
0em
}{%head spec
\if\relax\detokenize{#3}\relax \thmname{#1}:
\else \thmname{#3}:
\fi
}

\newtheoremstyle{cusdef}
{%space above
1em
}{%space below
1em
}{%body font
\color{defcolour}
}{%indent amount
-0em
}{%head font
\bfseries\color{defcolour}
}{%head punct
}{%after head space
1em
}{%head spec

%if numbered, include number
%if named, include name

\thmname{#1}
\if\relax\detokenize{#2}\relax:
\else\thmnumber{ #2}:\fi
\if\relax\detokenize{#3}\relax
\else\thmnote{ (#3)}\fi
}

\theoremstyle{custhm}
\newtheorem{theorem}{Theorem}[section]
\theoremstyle{cusdef}
\newtheorem{defin}[theorem]{Definition}
\theoremstyle{custhm}
\newtheorem{lemma}[theorem]{Lemma}
\theoremstyle{custhm}
\newtheorem{cor}[theorem]{Corollary}

\theoremstyle{custhm}
\newtheorem{prop}[theorem]{Proposition}

\theoremstyle{custhm}
\newtheorem*{theorem*}{Theorem}

\theoremstyle{cusdef}
\newtheorem*{defin*}{Definition}

\theoremstyle{remark}
\newtheorem*{remark*}{Remark}


%\marginpar{to describe which lecture it is}

\newcommand{\N}{\mathbb{N}}
\newcommand{\Z}{\mathbb{Z}}
\newcommand{\Q}{\mathbb{Q}}
\newcommand{\R}{\mathbb{R}}
\newcommand{\C}{\mathbb{C}}
\newcommand{\e}{\mathrm{e}}
\newcommand{\ra}{\rightarrow}
\newcommand{\lef}{\left(}
\newcommand{\res}{\right)}
\newcommand{\ie}{\textit{i.e.}}
\newcommand{\eps}{\varepsilon}
\newcommand{\E}{\mathbb{E}}
\newcommand{\suminf}{\sum_{n=0}^{\infty}}
\newcommand{\suminfa}[1]{\sum_{#1=0}^{\infty}}
\renewcommand{\P}{\mathbb{P}}
\newcommand{\undf}[1]{\textit{\textbf{#1}}}
\renewcommand{\L}{\mathcal{L}}
\renewcommand{\it}[1]{\textit{#1}}
\newcommand{\M}{\mathcal{M}}
\renewcommand{\phi}{\varphi}
\newcommand{\proves}{\vdash}
\newcommand{\lra}{\leftrightarrow}
\renewcommand{\value}{|\cdot|}
\newcommand{\val}[1]{\left|#1\right|}
\newcommand{\valk}{(K,|\cdot|)}
\renewcommand{\bar}{\overline}
\renewcommand{\O}{\mathcal{O}}

\renewcommand{\lnot}{\neg}
\newcommand{\false}{\bot}
\newcommand{\true}{\top}

\title{Local Fields}
\author{Lectures by Rong Zhou}
\date{}

\begin{document}

\maketitle
\clearpage
\tableofcontents
\clearpage

\marginpar{Lecture 1}
\section{Basic Theory}

A motivating question for an algebraic number theorist is `how can we find solutions to Diophantine equations?' \ie $f(x_1,\dots,x_r)\in \Z[x_1,\dots,d_r],\ f = 0$.

In general, this is very difficult (\it{e.g.} Fermat's Last Theorem). However, a more approachable problem might be to solve $f(x_1,\dots,x_r)\equiv 0 \mod p$. From here, we might be able to gain insight into solving $f(x_1,\dots,x_r)\equiv 0\mod p^n$ for each $n\in\N$.

Local fields gives us a way to package all of this information together.

\subsection{Absolute Values}

\begin{defin}[Absolute Value]
	Let $K$ be a field. An \undf{absolute value} on $K$ is a function $|\cdot|:K\ra\R_{\ge0}$ such that:
	\begin{enumerate}[label = (\roman*)]
		\item $|x| = 0$ iff $x= 0$
		\item $|xy| = |x||y|$ for all $x,y\in K$
		\item $|x+y| \le |x| + |y|$ for all $x,y\in K$ - this is known as the \undf{triangle inequality}.
	\end{enumerate}

We say $(K,|\cdot|)$ is a \undf{valued field}.
\end{defin}

\undf{Examples}:
\begin{itemize}
	\item $K = \R$ or $\C$ with $|\cdot|$ the usual absolute value. We write $|\cdot|_{\infty}$ for this absolute value
	\item $K$ is any field. The \undf{trivial absolute value} on $K$ is defined by \[|x| = \left\{\begin{array}{cc}0&\ \ \textrm{ if }x = 0\\ 1&\ \ \textrm{ if }x\ne 0\end{array}\right.
	\]
	We will ignore this case in this course as it is of no interest to us. However, the \it{most} interesting example is:
	\item $K = \Q$, $p$ prime. For $0\ne x\in \Q$, write $x = p^n\frac{a}{b}$, where $a,b\in\Z$, $(a,p) = 1$ and $(b,p) = 1$.
	
	The \undf{$p$-adic absolute value} is defined to be
	\[
	|x|_p = \left\{\begin{array}{cc}0 &\ \ x = 0\\ p^{-n}&\ \ x = p^n\frac{a}{b}\end{array}    \right.
	\]
	
	The intuition behind this is that if you consider, say, just the integers, then the $p$-adic value of $m$ will be very small if $m$ is highly divisible by $p$. That is to say it encodes information about how divisible $m$ is by $p$.
	
	We check the axioms to be sure this is indeed an absolute value. (i) is clear.
	
	Write $y = p^m\frac{c}{d}$.
	
	Then (ii): $|xy|_p = \left|p^{m+n}\frac{ac}{bd}\right|_p = p^{-m-n} = |x|_p|y|_p$.
	
	For (iii): wlog. $m\ge n$. Then 
	\begin{align*}
	|x+y|_p &= \left|p^n\left(\frac{ad+p^{m-n}bc}{bd}\right)\right|_p\\
	&= |p^n|_p\left|\frac{ad+p^{m-n}bc}{bd}\right|_p
	\end{align*}

	In this second term, the denominator is not divisible by $p$, so the absolute value of the term on the rate is $\le 1$. Hence
	\begin{align*}
		|x+y|_p \le p^{-n} = \max(|x|_p,|y|_p)
	\end{align*}

	We remark that the above is known as the \undf{ultrametric inequality}, which is stronger than the triangle inequality.
\end{itemize}

An absolute value on $K$ induces a metric on $K$ by $d(x,y) = |x-y|$, and thus induces a topology on $K$.

\undf{Exercise}: prove that $+,\cdot$ are continuous with respect to this topology.

\begin{defin}[Equivalent Absolute Values]
Let $|\cdot|,|\cdot|'$ be absolute values on a field $K$. We say $|\cdot|,|\cdot|'$ are \undf{equivalent} if they induce the same topology.
\end{defin}

We will see that if two absolute values are equivalent, then they determine each other and give us the same theory. Accordingly, equivalence of absolute values is indeed an equivalence relation; such an equivalence class is called a \undf{place}.

\begin{prop}
	Let $|\cdot|,\ |\cdot|'$ be non-trivial absolute values on $K$. Then the following are equivalent:
	\begin{enumerate}[label = (\roman*)]
		\item $|\cdot|$, $\cdot|'$ are equivalent
		\item $|x| < 1$ iff $|x|' < 1$ for all $x \in K$
		\item $\exists c\in\R_{>0}$ such that $|x|^c = |x|'$ for all $x\in K$
	\end{enumerate}
\end{prop}

\begin{proof}
$(i)\implies (ii)$:
\begin{align*}
|x| < 1 &\iff x^n \ra 0\ \textrm{w.r.t. }|\cdot|\\
&\iff x^n \ra 0\ \textrm{w.r.t. }|\cdot|'\\
&\iff |x|' < 1
\end{align*}

$(ii)\implies (iii)$: Let $a\in K^{\times}$ such that $|a| < 1$ (this exsts since $|\cdot|$ is non-trivial). We need to show that
\begin{align*}
	\forall x\in K^\times:\ \frac{\log|x|}{\log|a|} = \frac{\log|x|'}{\log|a|'}
\end{align*}

We proceed by contradiction. Assume that $\log|x|/\log|a| < \log|x|'/\log|a|'$. Choose $m,n\in\Z$ such that $\log|x|/\log|a| < m/n < \log|x|'/\log|a|'$.

then we have that $n\log|x| < m\log|a|$, but also that $n\log|x|' > m\log|a|'$. Hence $|x^n/a^m| < 1$ and $|x^n/a^m|' > 1\  \bot$.

Similarly for the case with the inequality reversed.

$(iii)\implies (iv)$ is in fact clear, because if (iii) holds then any open ball with respect to one topology will also be an open ball in the other topology. Hence the topology generated by these absolute values is the same.
\end{proof}

So it suffices to work only with equivalence classes of absolute values. In this course, we are mainly interested in the following types of absolute values:

\begin{defin}[Non-Archimedean AV]
An absolute value on $K$ is said to be \undf{non-archimedean} if it satisfies the ultrametric inequality $|x+y|\le \max(|x|,|y|)$.
	
If $|\cdot|$ is not non-archimedean, then it is archimedean.
\end{defin}

For example:
\begin{itemize}
	\item $|\cdot|_{\infty}$ on $\R$ is archimedean
	\item $|\cdot|_p$ is a non-archimedean absolute value on $\Q$
\end{itemize}

It turns out that non-archimedean absolute values give rise to some rather interesting properties:

\begin{lemma}[All triangles are isosceles]
Let $(K,|\cdot|)$ be a non-archimedean valued field, and let $x,y\in K$. If $|x| < |y|$, then $|x-y| = |y|$
\end{lemma}
\begin{proof}
\textbf{Fact}: $|1| = |-1| = 1$, and $|-y| = |y|$. These results are left as an exercise.

Observe that $|x-y| \le \max(|x|,|y|) = |y|$, and moreover $|y|\le \max(|x|,|x-y|)$, so by assumption $|y|\le |x-y|$. Hence we have equality.
\end{proof}

While this property is unusual and might appear to make some things more difficult to reason about, there are in fact some properties of this topology that make our lives easier - for example, convergence.

\begin{prop}
	Let $(K,|\cdot|)$ be non-archimedean and $(x_n)_{n=1}^{\infty}$ a sequence in $K$. If $|x_n - x_{n+1}|\ra 0$, then $(x_n)_{n=1}^{\infty}$ is Cauchy.
	
	In particular, if $K$ is in addition complete, then $(x_n)_{n=1}^{\infty}$ converges.
\end{prop}
\begin{proof}
	For $\eps > 0$, choose $N$ such that $|x_n - x_{n+1}| < \eps$ for all $n > N$. Then for $N < n < m$:
	\begin{align*}
		|x_n - x_m| &= |(x_n-x_{n+1}) + (x_{n+1}-x_{n+2})+\dots+(x_{m-1}-x_m)|\\
		&<\eps
	\end{align*}
So $(x_n)_{n=1}^{\infty}$ is Cauchy.
\end{proof}

\textbf{Example}: $p = 5$, construct sequence $(x_n)_{n=1}^{\infty}$ such that:
\begin{enumerate}[label = (\roman*)]
	\item $x_n^2 + 1 \equiv 0 \mod 5^n$
	\item $x_n \equiv x_{n+1} \mod 5^n$
\end{enumerate}
as follows.

Take $x_1 = 2$. Suppose we have constructed $x_n$. Let $x_n^2 + 1 = a\cdot5^n$ and set $x_{n+1} = x_n + b\cdot5^n$. Then
\begin{align*}
	x_{n+1}^2 + 1 &= x_n^2 + 2b\cdot5^nx_n + b^2\cdot5^{2n} + 1\\
	&=a5^n + 2b5^nx_n + b^25^{2n}
\end{align*}
We remark that the final term is already $\equiv 0 \mod 5^{n+1}$ as $n > 1$, hence we need only choose $b$ such that $a + 2bx_n \equiv 0\mod 5$, which is always possible since both $2$ and $x_n$ are units $\mod 5$. Then $x_{n+1}^2 + 1 \equiv 0 \mod 5^{n+1}$ as desired.

So we have constructed a sequence satisfying these two properties. The second property tells us that the 5-adic values of the differences between successive terms tends to zero as $n\ra\infty$. Hence $(x_n)_{n=1}^{\infty}$ is Cauchy. Does the limit exist?

Suppose $x_n\ra \ell\in\Q$. Then $x_n^\ra\ell^2$. But (i) tells us that $x_n^2\ra-1$, so $\ell^2 = -1\ \bot$.

Thus $(\Q,|\cdot|_5)$ is \it{not} complete.

\begin{defin}[$p$-adic numbers $\Q_p$]
The \undf{$p$-adic numbers $\Q_p$} is the completion of $\Q$ with respect to $|\cdot|_p$.
\end{defin}

This gives us analogy with $\R$; $\R$ is the completion of $\Q$ under $|\cdot|_{\infty}$, whereas $\Q_p$ is the completion under $|\cdot|_p$.

The $p$-adic numbers are the prototypical example of a local field. They also have a field structure (\it{c.f.} sheet 1), and as we have seen are strictly larger than the rationals. The completion for the reals is much more geometric, whereas the completion for the $p$-adics contains more interesting \it{arithmetic} information.

\marginpar{Lecture 2}
Let $\valk$ be a non-archimedean valued field.

For $x\in K$ and $r\in \R_{>0}$, define the \undf{open/closed balls}:
\begin{align*}
	B(x,r) &= \{y\in K: \val{x-y} < r\}\\
	\bar{B}(x,r)&=\{y\in K:|x-y|\le r\}
\end{align*}

\begin{lemma}
	Let $\valk$ be non-archimedean.
	\begin{enumerate}[label = (\roman*)]
		\item If $z\in B(x,r)$, then $B(z,r) = B(x,r)$. In other words, open balls don't have centres.
		\item If $z\in \bar{B}(x,r)$, then $\bar{B}(z,r) = \bar{B}(x,r)$ - ditto for closed balls.
		\item $B(x,r)$ is closed
		\item $\bar{B}(x,r)$ is open
	\end{enumerate}
\end{lemma}
These statements are unusual from someone who has an archimedean perspective, since they are obviously not true in \it{e.g.} the reals. This suggests that we will make heavy use of the ultrametric inequality in the proof.
\begin{proof}
(i) Let $y\in B(x,r)$. Then
\begin{align*}
	\val{x-y} < r\implies \val{z-y} &= \val{(z-x) + (x-y)}\\
	&\le \max(|z-x|,|x-y|)\\
	&< r
\end{align*}
Thus $B(x,r)\subset B(z,r)$, and the reverse inclusion follows by symmetry.

(ii) Follows in the same way as (i), just with $\le$ instead of $<$.

(iii) We show that for any point not in the ball, there exists a ball around it that does not intersect with $B(x,r)$.

Let $y\not\in B(x,r)$. If $z\in B(x,r)\cap B(y,r)$, then $B(x,r) = B(z,r) = B(y,r)$. But then $y\in B(x,r)\ \bot$. Hence $B(x,r)\cap B(y,r) = \emptyset$, and so $B(y,r)$ is the ball we desire.

(iv) again follows similarly. We show that every point inside $\cap{B}(x,r)$, there is an open neighbourhood contained within that contains it.

If $z\in \bar{B}(x,r)$, then $B(z,r) \subset \bar{B}(z,r) = \bar{B}(x,r)$.
\end{proof}

\section{Valuation Rings}

\begin{defin}[Valuation]
	Let $K$ be a field. A \undf{valuation} on $K$ is a function $v:K^\times \ra \R$ such that:
	\begin{enumerate}[label = (\roman*)]
		\item $v(xy) = v(x) + v(y)$
		\item $v(x+y) \ge \min(v(x),v(y))$
	\end{enumerate}
\end{defin}
What's the point of this definition? Well, this valuation actually determines a non-archimedean absolute value.

Fix $0 < \alpha < 1$. If $v$ is a valuation on $K$, then
\begin{align*}
	|x| = \left\{ \begin{array}{cc}\alpha^{v(x)}\quad &x\ne 0\\ 0\quad&x=0\end{array}\right.
\end{align*}
determines a non-archimedean absolute value.

Conversely, a non-archimedean absolute value determines a valuation $v(x) = \log_\alpha|x|$.

So these determine each other; what is the point of a valuation then? Well, it turns out that thinking about valuations is in some cases a little bit more useful. We will use the concept of a valuation to define something called a discrete valuation, and this will make our lives easier. Moreover, the concept of a valuation is more amenable to generalisation than an absolute value.

We can in fact define a more general version of a valuation, whereby we replace the reals in the definition with a totally ordered group. Using this we can define geometric objects called \undf{adic spaces}. These are super useful nowadays due to some recent work from Peter Schultzer. He invented \undf{perfectoid spaces}, but this is getting a little beyond the course now.

\begin{remark*}[Remarks]
\begin{itemize}
	\item We ignore the trivial valuation $v(x) = 0$ for all $x\in K^\times$. This corresponds with the trivial absolute value in the above way.
	\item We say $v_1,\ v_2$ are \undf{equivalent} if $\exists c \in \R_{>0}$ such that $v_1(x) = cv_2(x)\ \forall x \in K^\times$. This notion of equivalence of course corresponds with the notion of equivalence of absolute values - so again we can reduce to considering just equivalence classes of valuations.
\end{itemize}
\end{remark*}
\begin{remark*}[Examples]
	\begin{itemize}
		\item Let $K = \Q$, $v_p(x) = -\log_p|x|_p$ is the $p$-adic valuation. This is obtained in the above way by taking $\alpha = 1/p$. So the valuation of $x$ is just the power of $p$ appearing in its prime factorisation.
		\item Let $k$ be a field, and define $K = k(t) = \textrm{Frac}(k[t])$ its rational function field. We can then define a valuation $v\left(t^nf(t)/g(t)\right) = n$, where $f,g\in k[t]$, $f(0),g(0)\ne 0$ (note that we can always write any rational function in this way). This is known as the \undf{$t$-adic valuation}
		
		There is an important analogy here with $K$ and the rational numbers. $\Q$ and $K$ are the prototypical examples of a \undf{global field}. A lot of modern number theory in fact leverages this analogy to derive results about $\Q$ by studying $K$.
		
		\item $K = k((t)) = \textrm{Frac}(k[[t]]) = \{ \sum_{i=n}^{\infty}a_it^i : a_i \in k, n\in \Z \}$ is the field of formal Laurent series over $k$. We then have the valuation $v\left(\sum_{i}a_it^i\right) = \min\{i:a_i\ne 0\}$. This is also known as the $t$-adic valuation on $k$.
		
		The reason for this is that example three is in fact the completion of example two under the topology induced in two.
	\end{itemize}
\end{remark*}

\begin{defin}[Valuation Ring]
	Let $\valk$ be a non-archimedean valued field. The \undf{valuation ring} of $K$ is defined to be
\begin{align*}
	\O_K &= \{x\in K: |x|\le 1\}\\
	(&= \bar{B}(0,1))\\
	 (&=  x\in K^\times:v(x)\ge 0 \cup\{0\})
	\end{align*}
\end{defin}
\begin{prop} Let $\valk$ be as usual.
\begin{enumerate}[label = (\roman*)]
	\item $\O_K$ is an open subring of $K$ - this is a very special, non-archimedean property.
	\item The subsets $\{x\in K : |x| \le r\}$ and $\{x \in K : |x| < r\}$ for $\le 1$ are open ideals in $\O_K$
	\item $\O_K^\times = \{x\in K: |x| = 1\}$
\end{enumerate}
\end{prop}
\begin{proof}
Last lecture, we say $|1| = 1$, so $1\in \O_K$, and $|0| = 0$ so $0\in \O_K$. Moreover, $|-1| = |1|\implies |-x| = |x|$, so if $x\in \O_K$ then $-x \in \O_K$.

If $x,y\in \O_K$, then $|x+y|\le \max(|x|,|y|)\le 1$, which implies $x+y\in \O_K$ also.

If $x,y\in \O_K$, then $|xy| = |x||y|\le 1$, so again $xy\in \O_K$.

Thus $\O_K$ is a ring. Then since $\O_K = \bar{B}(0,1)$, it is open.

(ii) Is very similar to (i) (exercise).

(iii) Note that $|x||x^{-1}| = |xx^{-1}| = 1$. Thus $|x| = 1 \iff |x^{-1}| = 1\iff x,x^{-1} \in \O_K \iff x \in 
_k^\times$
\end{proof}

The point here is that there some very nice algebraic structure going on here.

\begin{remark*}[Notation]
\begin{itemize}
	\item $m\coloneqq \{x\in \O_K : |x| < 1\}$ - this is a maximal ideal of $\O_K$
	\item $k\coloneqq \O_K/m$ is the \undf{residue field}
\end{itemize}
\end{remark*}
\begin{cor}
$\O_K$ is a local ring with a unique maximal ideal, and this ideal is $m$.
\end{cor}
Recall that a ring with a maximal ideal is a local ring iff everything outside the maximal ideal is a unit. From Prop 2.3 we see that if an element of $\O_K$ doesn't lie in the maximal ideal $m$, then it has to have absolute value $1$, and is hence a unit.

\begin{remark*}[Examples]
We exhibit some example of valuation rings:
\begin{itemize}
	\item $K = k((t))$. Then $\O_K = k[[t]]$, $m = (t)$. The residue field is just $k$.
	\item $K = \Q$ with $|\cdot|_p$. Then $\O_K = \Z_{(p)}$, $m = p\Z_{(p)}$, $k = \mathbb{F}_p$
\end{itemize}
\end{remark*}

We have arrived at an important definition:

\begin{defin}[Discrete Valuation]
Let $v:K^\times \ra\R$ be a valuation. If $v(K^\times)\simeq \Z$, we say $v$ is a \undf{discrete valuation}, and $K$ is said to be a discretely valued field. An element $\pi \in \O_K$ is a \undf{uniformizer} if $v(\pi) > 0$ and $v(\pi)$ generates $v(K^\times)$ - \ie an element with minimal positive valuation. These of course only exist when the valuation is discrete.
\end{defin}
\begin{remark*}[Examples]
In this course we are mainly going to be focused on the cases when the valuation is discrete - such fields turn out to have some very nice properties. The valuations we have met so far indeed happen to be discrete.
\begin{itemize}
	\item $K = \Q$, the $p$-adic valuation
	\item $K = k(t)$, the $t$-adic valuation
\end{itemize}
\end{remark*}
\begin{remark*}
	If $v$ is a discrete valuation, we can replace it with an equivalent one such that $v(K^\times) = \Z\subset \R$. Such $v$ are called \undf{normalized valuations}. For such a valuation with normalizer $\pi$, we will then have $v(\pi) = 1$.
\end{remark*}

\begin{lemma}
	Let $v$ be a valuation on $K$. TFAE:
	\begin{enumerate}[label = (\roman*)]
		\item $v$ is discrete
		\item $\O_K$ is a PID
		\item $\O_K$ is Noetherian
		\item $m$ is principal
	\end{enumerate}
\end{lemma}
(ii) is the strongest condition, clearly implying (iii) and (iv). However, the equivalence of all the statements is quite cool, because it tells us that if our valuation is discrete then $\O_K$ satisfies some very nice properties.

\begin{proof}
(i)$\implies$ (ii) We need to show every non-zero ideal is principal. Let $I\subset \O_K$ be a non-zero ideal. Let $x\in I$ scuh that $v(d) = \min \{v(a):a\in I\}$ which exists since $v$ is discrete.  Then $x\O_K = \{a\in \O_K : v(a)\ge v(x)\}\subset I$, and hence $x\O_K = I$ by the definition of $x$. If we have some $y\in I\backslash x\O_K$, then $v(y) < v(x)$.

(ii)$\implies$(iii) is clear, since all PIDs are finitely generated (by one element).

(iii)$\implies$ (iv): Write $m = x_1\O_K + \dots+x_n\O_K$, and wlog. $v(x_1)\le v(x_2)\le \dots \le v(x_n)$. Then by a similar argument to (i)$\implies$(ii) we have that $m = x_1\O_K$.

(iv)$\implies$(i): Let $m = \pi\O_K$ for some $\pi \in \O_K$ and let $c = v(\pi)$. Then if $v(x) > 0$, $x\in m$ and hence $v(x)\ge c$. Thus $v(K^\times)\cap(0,c) = \emptyset$. Since $v(K^\times)$ is a subgroup of $(\R,+)$, this can only happen if $v(K^\times) = c\Z$.
\end{proof}
\begin{lemma}
	Let $v$ be a discrete valuation on $K$ and $\pi \in \O_K$ a uniformizer. $\forall x \in K^\times$, $\exists n\in \Z$ and $\exists u\in \O_K^\times$ such that $x = \pi^n u$. In particular, $K = \O_K[1/x]$, for any $x\in m$ and hence $k = \textrm{Frac}(\O_K)$.
\end{lemma}
This says that the arithmetic of the field is determined by the arithmetic of the valuation field.
\begin{proof}
For $x \in K^\times$, let $n$ be s.t. $v(x) = v(\pi^n) = nv(\pi)$. Then $v(x\pi^{-n}) = 0$, so $u = x\pi^{-n}\in \O_K^\times$.

Then if you invert any element of the maximal ideal, you also have to invert the uniformizer.
\end{proof}
\begin{defin}[Discrete Valuation Ring]
A rgint $R$ is called a \undf{discrete valuation ring} (DVR) if it is a PID with exactly one non-zero prime ideal (that is necessarily maximal).
\end{defin}
Note that while this name contains the term `valuation', despite its definition lacking any explicit reference. This is now cleared up:

\begin{lemma}\ 
\begin{enumerate}[label = (\roman*)]
	\item Let $v$ be a discrete valuation on $K$. Then $\O_K$ is a DVR.
	\item Let $R$ be a DVR. Then there exists a valuation $v$ on $K\coloneqq \textrm{Frac}(R)$ such that $R = \O_K$.
\end{enumerate}
\end{lemma}
\begin{proof}
(i) $\O_K$ is a PID by Lemma 2.6.

Let $0\ne I \subset \O_K$ be an ideal, so $I = (x)$. If we write $x = \pi^nu$ for $\pi$ a uniformizer, then $(x)$ is prime iff $n = 1$, and so $I = (\pi) = m$. This is because if $n > 1$ we can break up $x$ into a product of two elements, each containing a power of the uniformizer. So $(x)$ is the product of two ideals and is not prime. So we must have $n = 1$, and this is clearly sufficient.

(ii)Let $R$ be a DVR with maximal ideal $m$. Then $m = (\pi)$ for some $\pi \in R$. By unique factorisation of PIDs, we may write any $x\in R\backslash \{0\}$ uniquely as $\pi^n u$, with $n\ge 0$ and $u\in R^\times$. Then any $y\in K^\times$ can be written uniquely as $\pi^mu$, $u\in R^\times$, $m\in \Z$.

We then define $v(\pi^mu) = m$; it is easy to check that this is a well-defined valuation and $\O_K = R$.
\end{proof}

\begin{remark*}[Examples]
We revisit our previous examples.
\begin{itemize}
	\item $\Z_{(p)}$ is a DVR, which is the valuation ring of $|\cdot|_p$ on $\Q$.
	\item $k[[t]]$, the ring of formal power series is a DVR, the valuation ring for the $t$-adic absolute value on $k((t))$.
	\item We also have a \it{non}-example. Let $K = k(t)$ be the rational function field, and define $K' = K(t^{1/2},t^{1/4},t^{1/8},\dots)$. It can be shown that the $t$-adic valuation in fact extends to this larger field, and then that $v(t^{1/2^n}) = 1/2^n$, so the valuation cannot be discrete. The idea here is that we've found an infinite sequence of elements with positive valuation tending to zero.
\end{itemize}
\end{remark*}

\marginpar{Lecture 3}

\section{The $p$-adic numbers}

This is, in some sense, the prototypical example of a discretely valued field, and from the point of view of number theory it is the most important example.

\begin{remark*}[Recall]
$\Q_p$ is defined to be the completion of $\Q$ with respect to the metric induced by $\value_p$. \it{c.f.} Example Sheet 1 for the proof that $\Q_p$ is a field.

$\value_p$ extends to $\Q_p$ and the associated valuation is discrete, \it{i.e.} $\Q_p$ is a discretely valued field.
\end{remark*}

\begin{defin}[The $p$-adic integers $\Z_p$]
The ring of $p$-adic integers $\Z_p$ is the valuation ring $\Z_p = \{x\in\Q_p : |x|_p \le 1\}$.
\end{defin}
\begin{remark*}[Facts]
\begin{itemize}
	\item $\Z_p$ is a DVR with maximal ideal $p\Z_p$; the maximal ideal is always generated by the uniformiser, and $p$ is an element of $\Z_p$ with normalized valuation $1$, so it is a uniformiser.
	\item The non-zero ideals in $\Z_p$ are $p^n\Z_p$, $n\in\N$.
\end{itemize}
\end{remark*}
\begin{prop}
$\Z_p$ is the closure of $\Z$ inside $\Q_p$. In particular, $\Z_p$ is the completion of $\Z$ with respect to $\value_p$.
\end{prop}
In other words, $\Z_p$ is to $\Q_p$ what $\Z$ is to $\Q$.
\begin{proof}
We need to show that $\Z$ is dense in $\Z_p$.

$\Q$ is dense in $\Q_p$. Since $\Z_p \subset \Q_p$ is open, $\Z_p \cap \Q$ is dense in $\Z_p$. But $\Z_p \cap \Q =\{x\in \Q : |x|_p \le 1 \} = \{\frac{a}{b}\in \Q : p\not|\ b\} = \Z_{(p)}$, the localisation at $p$. $\Z_{(p)}$ is clearly dense in $\Z_p$, so it suffices to show that $\Z$ is dense in $\Z_{(p)}$.

Let $a/b\in \Z_{(p)}, a,b\in\Z,p\not|\ b$. For $n\in\N$, choose $y_n\in \Z$ such that $by_n \equiv a\mod p^n$. Then $y_n\ra a/b$ as $n\ra\infty$, so $\Z$ is dense in $\Z_{(p)}$ and hence in $\Z_p$.

For the final part, this is also clear: $\Z$ is dense in $\Z_p$, and $\Z_p$ is complete.
\end{proof}

This proposition tells us that $\Z_p$ can be constructed via a very natural analytic property from $\Z$. However, there is also an algebraic construction.

\subsection*{Digression on Inverse Limits}

Let $(A_n){n=1}^{\infty}$ be a sequence of sets/groups/rings together with homomorphisms $\phi_n : A_{n+1}\ra A_n$ (\it{transition maps}). Then the \undf{inverse limit} of $(A_n)_{n=1}^{\infty}$ is the set/group/ring:
\begin{align*}
\varprojlim_nA_n = \{(a_n)_{n=1}^{\infty}\in \prod_{n=1}^{\infty}A_n : \phi_n(a_{n+1}) = a_n\}
\end{align*}
In other words, we have the sequence of sets/groups etc... (objects!)
\begin{align*}
A_{n+1}\ra A_n\ra A_{n-1}\\
a_{n+1}\xmapsto{\phi_n}a_n\xmapsto{\phi_{n-1}}a_{n-1}
\end{align*}
\underline{Fact}: If $A_n$ is a group/ring/\it{etc}..., then $\varprojlim_n A_n$ is a group/ring also.

Let $\theta_m : \varprojlim_nA_n\ra A_m$ denote the natural projection. The important thing about the inverse limit is that it satisfies the following \undf{universal property}:

\begin{prop}
Let $((A_n)_{n=1}^{\infty},(\phi_n)_{n=1}^{\infty})$ be as above. Then for any set/group/ring (object!) $B$ together with homomorphisms (morphisms!) $\psi_n:B\ra A_n$ such that $\phi_n\circ\psi_{n+1} = \psi_n$ (\it{i.e.} diagram $B\ra A_{n+1}\ra A_n$ commutes for all $n$), there is a unique homomorphism $\psi: B\ra\varprojlim_n A_n$ such that $\theta_n\circ\psi = \psi_n$.
\end{prop}

The point is that the universal limit completely characterises the objects involved. This is obviously really helpful, since we can just study one object rather than infinitely many at the same time. Obviously \textrm{c.f.} Category Theory.

\begin{proof}
Define $\psi : B\ra \prod_{n=1}^{\infty}A_n$ by $b\mapsto \prod_{n=1}^{\infty}\psi_n(b)$. Then $\psi_n = \phi_n\circ \psi_{n+1} \implies \psi(b)\in \varprojlim_n A_n$. The map is clearly unique since it is determined by $\psi_n  = \phi_n \circ \psi_{n+1}$, and is a homomorphism of rings (check).
\end{proof}

\begin{defin}[$I$-adic completion]
Let $R$ be a ring and $I\subset R$ an ideal. The $I$-adic completion of $R$ is the ring $\hat{R}\coloneqq \varprojlim_n R/I^n$, where $\phi_n:R/I^{n+1}\ra R/I^n$ is the natural projection.

Note there is a natural map $i:R\ra\hat{R}$ by the universal property. We say that $R$ is $I$-adically complete if $i$ is an isomorphism. 
\end{defin}
\undf{Fact}: $\ker (i:R\ra\hat{R}) = \bigcap_{n=1}^{\infty}I^n$.

We now apply this to the context of valued fields.

Let $\valk$ be a non-archimedean valued field, and $\pi \in \O_K$ such that $|\pi| < 1$.

\begin{prop}
Assume $K$ is complete. Then
\begin{enumerate}[label = (\roman*)]
	\item Then $\O_k \cong \varprojlim \O_K / \pi^n\O_k$ (\it{i.e.} $\O_K$ is $\pi$-adically complete)
	\item If in addition $K$ is discretely valued and $\pi$ is a uniformiser, then every element $x\in \O_K$ can be written uniquely as $x = \sum_{i=0}^{\infty}a_i\pi^i,\ a_i\in A$, where $A$ is a set of coset representatives for $k\coloneqq \O_k/\pi\O_K$.
	Moreover, any series $\sum_{i=0}^{\infty}a_i\pi^i$ converges to an element in $\O_K$.
\end{enumerate}
\end{prop}
\begin{proof}(i) We have $i:\O_K\ra\varprojlim_n \O_K/\pi^n\O_K$. Since $\bigcap_{n=1}^{\infty}\pi^n\O_K = \{0\}$, $i$ is injective. Let $(x_n)_{n=1}^{\infty}\in \varprojlim_n \O_K/\pi^n\O_K$ and for each $n$, choose $y_n\in \O_K$ a lift of $x_n\in \O_K/\pi^n\O_K$. Let $v$ be the valuation on $K$ normalised such that $v(\pi) = 1$, then $v(y_n - y_{n+1})\ge n$, which follows from $y_n-y_{n+1}\in \pi^n\O_K$. This implies that $(y_n)_{n=1}^{\infty}$ is a Cauchy sequence in $\O_K$. But $\O_K$ is complete, since $\O_K \subset K$ is closed. Hence $y_n\ra y\in \O_K$, and $y$ maps to $(x_n)_{n=1}^{\infty}$. So $i$ is surjective.
	
(ii) Let $x\in \O_K$. Choose $a_i$ inductively. Choose $a_0$ such that $a_0 \equiv x \mod \pi\O_K$, which is possible because $a_0\in A$ is a set of coset representatives for the residue field.

Suppose we have chosen $a_1,\dots,a_k$ such that $\sum_{i=0}^{k}a_i\pi^i \equiv x\mod \pi^{k+1}$. Then $a_i\pi^i - x = c\pi^{k+1}$ for $c\in \O_K$. Then choose $a_{k+1}\equiv c\mod \pi \O_K$. Then $\sum_{i=0}^{k+1}a_i \equiv x\mod \pi^{k+2}\O_K$. Altogether this implies that $\sum_{i=0}^{\infty}a_i = x$.

For uniqueness, suppose that $\sum_{i=0}^{\infty}a_i\pi^i = \sum_{i=0}^{\infty}b_i\pi^i\in \O_K$. Then let $n$ be minimal such that $a_n \ne b_n$. Then $\sum_{i=0}^{\infty}a_i\pi^i\not\equiv\sum_{i=0}^{\infty}b_i\pi^i \mod\pi^{n+1}\ \bot$.

Moreover is also clear, since any series of that form defines a Cauchy sequence and hence converges in $\O_K$ because $\O_K$ is complete.
\end{proof}

\underline{Warning}: If $\valk$ is not discretely valued, then $\O_K$ is not necessarily $m$-adically complete.

\begin{cor}
If $K$ is as in part (ii) of Prop. 3.5, then every $x\in K$ can be written uniquely as $\sum_{i=n}^{\infty}a_i\pi^i,\ a_i\in A$. Conversely, any such expression defines an element of $K$.
\end{cor}
\begin{proof}
Follows from 3.5 using the fact that $K = \O_K[1/\pi]$.
\end{proof}
\begin{cor}
\begin{enumerate}[label=(\roman*)]
	\item $\Z_p\cong\varprojlim_n\Z/p^n\Z$
	\item Every element of $\Q_p$ can be written uniquely as $\sum_{i=n}^{\infty}a_ip^i,\ a_i\in \{0,1,\dots,p-1\}$.
\end{enumerate}
\end{cor}
\begin{proof}
By Prop 3.5, it suffices to show that
\[
\Z_p/p^n\Z_p \cong \Z/p^n\Z
\]
Let $f_n:\Z\ra\Z_p/p^n\Z_p$ be the natural map. We have $\ker (f_n) = \{x\in \Z : |x|_p \le p^{-n}\} = p^n\Z$. This implies $\Z/p^n\Z \ra \Z_p/p^n\Z_p$ is injective.

For surjectivity, let $\bar{c}\in \Z_p/p^n\Z_p$, and $c\in \Z_p$ a lift. Since $\Z$ is dense in $\Z_p$, we can choose $x\in\Z$ such that $x\in c + p^n\Z_p$ - this is a closed ball, and is hence open in $\Z_p$ so we find the element by density. So $f_n(x) = \bar{c}$, and thus $\Z/p^n\Z\ra\Z_p/p^n\Z_p$ is surjective.

(ii) Follows directly from Corollary 3.6, noting that $\Z_p/p\Z_p\cong \mathbb{F}_p$.
\end{proof}

So we have an algebraic construction for $\Z_p$ as well as the analytic one. Moreover, we can define an algebraic notion of the topology that coincides with the analytic one.

\begin{remark*}[Examples]\ 
\begin{itemize}
\item For instance,
\begin{align*}
\frac{1}{1-p} = 1 + p + p^2 +p^3+\dots \in \Q_p
\end{align*}
\item $K = k((t))$ with the $t$-adic valuation. Then
\begin{align*}
\O_K = k[[t]] = \varprojlim_n k[[t]]/(t^n)
\end{align*}
Moreover, $\O_K$ is the $t$-adic completion of $k[t]$.
\end{itemize}
\end{remark*}

\marginpar{Lecture 4}

\section{Complete Valued Fields}

``This is where the fun begins.''

\subsection{Hensel's Lemma}

One of the main questions in algebraic number theory is about finding solutions to equations, specifically over the rational numbers. As we noted, it's hard to do this in general, but much easier to do as a congruence modulo $p$.

Something a little harder, and thus in between these two problems, is to solve Diophantine equations over the $p$-adic numbers. It is this problem that Hensel's Lemma helps us with. In particular, for complete valued fields there is a nice way to produce solutions in $\O_K$ to certain equations from solutions modulo $m$.

\begin{theorem}[Hensel's Lemma v.1]
Let $\valk$ be a complete discretely valued field. Let $f(x)\in \O_K[x]$ and assume $\exists a\in \O_K$ such that $|f(a)| < |f'(a)|^2$, where $f'$ is the formal derivative of $f$.

Then there exists a unique $x\in \O_K$ such that $f(x) = 0$ and $|x-a| < |f'(a)|$.
\end{theorem}
\begin{proof}
Let $\pi\in \O_K$ be a uniformizer and let $r = v(f'(a))$. The idea is that we will use an iteration to find better and better approximations to a solution. More precisely, a Cauchy sequence whose limit actually gives us a solution. We construct this sequence as follows.

Let $(x_n)_{n=1}^{\infty}$ be a sequence in $\O_K$ such that
\begin{enumerate}[label=(\roman*)]
	\item $f(x_n)\equiv 0\mod \pi^{n+2r}$
	\item $x_{n+1}\equiv x_n \mod \pi^{n+r}$
\end{enumerate}

Take $x_1 \equiv a$; then $f(x_1)\equiv 0 \mod \pi^{1+2r}$.

Suppose we have constructed $x_1,\dots,x_n$ satisfying (i) and (ii). Then define $x_{n+1}\coloneqq x_n - f(x_n)/f'(x_n)$. Since $x_n \equiv x_1 \mod \pi^{r+1}$, $v(f'(x_n)) = r$ and hence $f(x_n)/f'(x_n) \equiv 0\mod \pi^{n+r}$ by (i). It then follows that $x_{n+1}\equiv x_n \mod \pi^{n+r}$, so (ii) holds.

To show that (i) holds, note that for $x,y$ indeterminates, $f(x+y) = f_0(x) + f_1(x)y + f_2(x)y^2+\dots$, where $f_i(x)\in \O_K[x]$ and $f_0(x) = f(x)$, $f_1(x) = f'(x)$. Thus $f(x_{n+1}) = f(x_n) + f'(x_n)c + f_2(x_n)c^2+\dots$, where $c = -f(x_n)/f'(x_n)$.

Since $c \equiv 0 \mod \pi^{n+r}$ and $v(f_i(x_n)) \ge 0$ for any $i$, we have that $f(x_{n+1}) \equiv f(x_n) + f'(x_n)c \mod \pi^{n+2r+1}$ - \it{i.e.} all the third order and higher terms are killed by $c$. Hence $f(x_{n+1}) \equiv 0 \mod \pi^{n+2r+1}$, and in particular (i) holds.

Property (ii) clearly implies that $(x_n)_{n=1}^{\infty}$ is Cauchy, so let $x\in \O_K$ be such that $x_n\ra x$. Then $f(x) = \lim_{x\ra\infty}f(x_n) = 0$ by (i). Moreover, (ii) implies that $a = x_1 \equiv x_n \mod \pi^{r+1}$ for all $n$, which also implies that $a\equiv x\mod \pi^{r+1}$, and so $|x-a| < |f'(a)|$. This proves the existence.

For uniqueness, suppose for contradiction that there exists another solution $x'$ with $f(x') = 0$, $|x'-a| < |f'(a)|$. Set $\delta = x' - x\neq 0$ by assumption. Then $|x'-a| < |f'(a)|$, $|x-a|<|f'(a)|$ and the ultrametric inequatliy together imply that $|\delta| = |x - x'| < |f'(a)| = |f'(x)|$.

But $0 = f(x') = f(x+\delta) = f(x) + f'(x) + \delta + \underbrace{\dots\dots}_{|\ \cdot\ |\le |\delta|^2}$. Hence $|f'(x)\delta| \le |\delta|^2$, so $|f'(x)| \le |\delta|\ \bot$
\end{proof}
\begin{cor}
Let $\valk$ be a complete discretely valued field. Let $f(x)\in \O_K[x]$ and $\bar{c}\in k \coloneqq \O_k/m$ a simple root of $\bar{f}(x) \coloneqq f(x) \mod m\in k[x]$. Then there exists a unique $x\in \O_K$ such that $f(x) = 0$, $x\equiv \bar{c}\mod m$.
\end{cor}
\begin{proof}
Apply Theorem 4.1 to a lift $c\in \O_K$ of $\bar{c}$. Then $|f(c)| < |f'(c)|^2 = 1$ since $\bar{c}$ is a simple root.
\end{proof}
This statement is seemingly weaker, but it is in fact equivalent to Hensel's Lemma, and in the opinion of the lecturer is much more palatable. We will use it to solve some problems:
\begin{remark*}[Example]
$f(x) = x^2 - 2$ has a simple root $\mod 7$. Thus $\sqrt{2} \in \Z_7 \subset \Q_7$.
\end{remark*} 
\begin{cor}\ 
\begin{align*}
\Q_p^\times/\left(\Q_p^\times\right)^2 \cong \left\lbrace \begin{array}{cc} \left(\Z/2\Z\right)^2 &\quad\textrm{if }p > 2\\ \left(\Z/2\Z\right)^3&\quad\textrm{if }p = 2\end{array}\right.
\end{align*}
\end{cor}
\begin{proof}
Case $p > 2$. Let $b\in \Z_p^\times$. Applying Corollary 4.2 to $f(x) = x^2 - b$, we find that $b\in (\Z_p^\times)^2$ iff $b\in \left(\mathbb{F}_p^\times\right)^2$. So Hensel's Lemma tells us that if the equation has a solution modulo $p$, then the solution itself lies in $(\Z_p^\times)^2$.

Thus $\Z_p^\times/(\Z_p^\times)^2 \cong \mathbb{F}_p^\times/(\mathbb{F}_p^\times)^2 \cong \Z/2\Z$, since $\mathbb{F}_p^\times \cong \Z/(p-1)\Z$.

We have an isomorphism $\Q_p^\times \cong \Z_p^\times \times \Z$ given by $(u,p)\mapsto up^n$. Thus $\Q_p^\times/(\Q_p^\times)^2\cong (\Z/2\Z)^2$.

Case $p = 2$. Let $b \in \Z_2^\times$. Consider $f(x) = x^2 - b$. $f'(x) = 2x \equiv 0\mod 2$, so $f$ can never have a simple root; we need the `stronger' version of Hensel.

Let $b\equiv 1\mod 8$. Then $|f(1)|_1 \le 2^{-3} < |f'(1)|_2^2 = 2^{-2}$. Hensel's Lemma then says that $f(x)$ has a root in $\Z_2$, and so $b\in(\Z_p^\times)^2$ iff $b\equiv 1\mod 8$. Thus $\Z_2^\times/(\Z_2^\times)^2 \cong (\Z/8\Z)^2 \cong (\Z/2\Z)^2$.

Again using that $\Q_2^\times \cong \Z_2^\times \times \Z$, we find that $\Q_2^\times/(\Q_2^\times)^2 \cong (\Z/2\Z)^3$.
\end{proof}
\begin{remark*}
The proof of Hensel's Lemma uses the iteration $x_{n+1} = x_n - f(x_n)/f'(x_n)$, which looks identical to the Newton-Raphson method, and in fact in a strong sense is the non-archimedean analogue of this result.
\end{remark*}

For later applications, we will require a slightly different version of Hensel's Lemma:

\begin{theorem}[Hensel's Lemma v.2]
Let $\valk$ be a complete discretely valued field, and $f(x)\in \O_K[x]$. Suppose $\bar{f}(x)\coloneqq f(x)\mod m\in k[x]$ factorises as $\bar{f}(x) = \bar{g}(x)\bar{h}(x)$ in $k[x]$, with $\bar{g}(x),\bar{h}(x)$ coprime.

Then there is a factorisation $f(x) = g(x)h(x)$ in $\O_K[x]$, with $\bar{g}(x) = g(x) \mod m$, $\bar{h}(x) = h(x)\mod m$, and $\deg \bar{g} = \deg g$.
\end{theorem}
\begin{proof}
Example Sheet 1.
\end{proof}
\begin{cor}
Let $f(x) = a_nx^n + \dots + a_0 \in K[x]$ with $a_0,a_n\ne 0$. If $f(x)$ is irreducible, then $|a_i| \le \max (|a_0|,|a_n|)$ for all $i$.
\end{cor}
\begin{proof}
Upon scaling, we may assume $f(x)\in \O_K[x]$ with $\max_i (|a_i|) = 1$. Thus we need to show that $\max(|a_0|,|a_n|) = 1$. If not, let $r$ be minimal such that $|a_r| = 1$; then $0 < r < n$. Thus we have
\[
\bar{f}(x) = x^r(a_r + \dots + a_n x^{n-r})\mod m
\]
Then Theorem 4.5 implies $f(x) = g(x)h(x)$, $0 < \deg g < n$.
\end{proof}

\section{Perfect Lifts}

Recall: Lecture 3. Every element of $x\in \Q_p$ can be written as $x = \sum_{i=n}^{\infty}a_ip^i$, $a_i\in \{0,1,\dots,p-1\}\eqqcolon A$. However, this is not the best choice of representatives. One we that we can see this is to consider that the map $\mathbb{F}_p \ra A\subset\Z_p$ doesn't respect any algebraic structure.

It turns out there is a natural choice of coset reps in many cases which does respect some algebraic structure.

\begin{defin}[Perfect Ring]
A ring of characteristic $p$ is a \undf{perfect ring} if th Frobenius map $x\mapsto x^p$ is an automorphism of $R$. A field of characteristic $p$ is a perfect field if it is perfect as a ring.
\end{defin}
\begin{remark*}
Since char $R = p$, $(x+y)^p = x^p + y^p$, so the Frobenius map is a ring homomorphism.
\end{remark*}



\end{document}
