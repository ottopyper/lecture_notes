\documentclass[]{article}


\usepackage{amsmath}
\usepackage{amssymb}
\usepackage{amsthm}
\usepackage{graphicx}
\usepackage{parskip}
\usepackage{xcolor}
\usepackage{pagecolor}
\usepackage[margin=1.2in]{geometry}
\usepackage{enumerate}


\usepackage[utf8]{inputenc}
\usepackage[english]{babel}

\usepackage{mathtools}
\DeclarePairedDelimiter\bra{\langle}{\rvert}
\DeclarePairedDelimiter\ket{\lvert}{\rangle}
\DeclarePairedDelimiterX\braket[2]{\langle}{\rangle}{#1 \delimsize\vert #2}

\definecolor{thmcolour}{rgb}{0,0,0}
\definecolor{defcolour}{rgb}{0,0,0}
\definecolor{textcolour}{rgb}{0,0,0}
\definecolor{backgroundcolour}{rgb}{1,1,1}

\pagecolor{backgroundcolour}
\color{textcolour}

\newtheoremstyle{custhm}
{%space above
1em
}{%space below
1em
}{%body font
\color{thmcolour}
}{%indent amount
-0em
}{%head font
\bfseries\color{thmcolour}
}{%head punct
}{%after head space
1em
}{%head spec
\thmname{#1}
\if\relax\detokenize{#2}\relax:
\else\thmnumber{ #2}:\fi
\if\relax\detokenize{#3}\relax
\else\thmnote{ (#3)}\fi
}

\newtheoremstyle{remark}
{%space above
}{%space below
}{% body font
}{%indent amount
-0em
}{%head font
\bfseries
}{%head punct
}{%after head space
0em
}{%head spec
\if\relax\detokenize{#3}\relax \thmname{#1}:
\else \thmname{#3}:
\fi
}

\newtheoremstyle{cusdef}
{%space above
1em
}{%space below
1em
}{%body font
\color{defcolour}
}{%indent amount
-0em
}{%head font
\bfseries\color{defcolour}
}{%head punct
}{%after head space
1em
}{%head spec

%if numbered, include number
%if named, include name

\thmname{#1}
\if\relax\detokenize{#2}\relax:
\else\thmnumber{ #2}:\fi
\if\relax\detokenize{#3}\relax
\else\thmnote{ (#3)}\fi
}

\theoremstyle{custhm}
\newtheorem{theorem}{Theorem}[section]
\theoremstyle{cusdef}
\newtheorem{defin}[theorem]{Definition}
\theoremstyle{custhm}
\newtheorem{lemma}[theorem]{Lemma}
\theoremstyle{custhm}
\newtheorem{cor}[theorem]{Corollary}

\theoremstyle{custhm}
\newtheorem{prop}[theorem]{Proposition}

\theoremstyle{custhm}
\newtheorem*{theorem*}{Theorem}

\theoremstyle{cusdef}
\newtheorem*{defin*}{Definition}

\theoremstyle{remark}
\newtheorem*{remark*}{Remark}


%\marginpar{to describe which lecture it is}

\newcommand{\N}{\mathbb{N}}
\newcommand{\Z}{\mathbb{Z}}
\newcommand{\Q}{\mathbb{Q}}
\newcommand{\R}{\mathbb{R}}
\newcommand{\C}{\mathbb{C}}
\newcommand{\e}{\mathrm{e}}
\newcommand{\ra}{\rightarrow}
\newcommand{\lef}{\left(}
\newcommand{\res}{\right)}
\newcommand{\ie}{\textit{i.e.}}
\newcommand{\eps}{\varepsilon}
\newcommand{\E}{\mathbb{E}}
\newcommand{\suminf}{\sum_{n=0}^{\infty}}
\newcommand{\suminfa}[1]{\sum_{#1=0}^{\infty}}
\renewcommand{\P}{\mathbb{P}}
\newcommand{\undf}[1]{\textit{\textbf{#1}}}
\renewcommand{\L}{\mathcal{L}}
\renewcommand{\it}[1]{\textit{#1}}
\newcommand{\M}{\mathcal{M}}
\renewcommand{\phi}{\varphi}
\newcommand{\proves}{\vdash}
\newcommand{\lra}{\leftrightarrow}
\renewcommand{\value}{|\cdot|}
\newcommand{\val}[1]{\left|#1\right|}
\newcommand{\valk}{(K,|\cdot|)}
\renewcommand{\bar}{\overline}
\renewcommand{\O}{\mathcal{O}}

\renewcommand{\lnot}{\neg}
\newcommand{\false}{\bot}
\newcommand{\true}{\top}

\title{Topics in Combinatorics Sheet 1}
\author{Otto Pyper}
\date{}

\begin{document}

\maketitle
\clearpage

\textbf{1}. We can count the LHS differently. For $b\in B$, let $b_x = \{x:b\in A+x\} = \{b-a:a\in A\}$, so $|b_x| = |A|$. Then $|(A+x)\cap B| = \sum_{b\in B} \mathbb{I}_{x\in b_x}$, and hence
\begin{align*}
\sum_{x\in \Z_n} |(A+x)\cap B| &= \sum_{x\in \Z_n}\sum_{b\in B}\mathbb{I}_{x\in b_x}\\
&=\sum_{b\in B}\sum_{x\in \Z_n}\mathbb{I}_{x\in b_x}\\
&=\sum_{b\in B}|b_x| = \sum_{b\in B}|A|\\
&=|A||B|
\end{align*}

Hence for some $x$, $|(A+x)\cap B| \ge |A||B|/n$, otherwise the sum over all $x$ is too small.

Given $|A|$, $|B|$ and $n$, we have for all $x$ that $|(A+x)\cap B| \le \max_x |(A+x)\cap B|$, and so $|A||B| \le n\max_x |(A+x)\cap B|$, and we have the lower bound $\max \ge |A||B|/n$, or the ceiling thereof if $|A||B|/n$ is not an integer.

In the case where $|A||B|/n$ is an integer and the bound is attainable, we must then have $|(A+x)\cap B| = |A||B|/n$ for all $x$, else the sum is again too small.

If $|A| | n$, it is very helpful to find $A \le \Z_n$, as then $\{(A+x):0\le x < n/|A|\}$ partitions $\Z_n$ into disjoint cosets, so for $B$ we can freely pick $k$ elements from each coset, for any fixed $1\le k\le |A|$.

In general we do not have this niceness, but it may still be the case that $|A| | n$, in which case for any given $|B|$ we can minimise $\max_x|(A+x)\cap B|$ by taking $A$ a subgroup as above and then choosing $B$ as evenly as possible from the cosets, so that for any $x,y$ we have $||(A+x)\cap B| - |(A+y)\cap B|| \le 1$, which is best possible (with all $=0$ iff $n$ divides $|A||B|$).

In even further generality, making no assumptions about the sizes $|A|,|B|$, we might still hope the above is possible - that for any $x,y$ we have $||(A+x)\cap B| - |(A+y)\cap B|| \le 1$.

The plan will be to space $A$ `as evenly as possible' across $\Z_n$, and take $B$ to be a consecutive set of $|B|$ integers. Henceforth write $|A| = a$, $|B| = b$. As there is some non-empty intersection, we also assume wlog that $0\in A\cap B$. Let $|A\cap B| = c$, and we assume further that this is the maximum size. Then for any $x$, we need $\lfloor ab/n\rfloor = c-1\le |(A+x)\cap B| \le c$. If there are $y$ intersections of size $c-1$, and $n-y$ of size $c$, then $y(c-1) + (n-y)c = ab$, so $y = nc - ab = n - ab - \lfloor ab/n\rfloor$.

A set $X$ is spaced `as evenly as possible' if for all $x,y\in X$, $|x - y| \le 1$ for an appropriate norm.

We will space $A$ as evenly as possible. We find a consecutive pair of integers $s,s+1$ such that there exist $p,q\in \N\ge0$ such that $ps+q(s+1) = n$ and $p+q = a$. In particular, $as = n-q$ and $a(s+1) = n+p$, so $s\le n/a \le s+1$. So $s = \lfloor n/a \rfloor$, $q = n - a\lfloor n/a\rfloor$.

My gut says this should be possible if we space $A$ as evenly as possible, and then space the larger gaps as evenly as possible, and so on...

\textbf{2}. Choose a random subset $V\subset G$ of vertices, where for $x\in G$ we have $x\in V$ with probability $p$. Then for any $v,w\in G$, $\P[v\in V,w\in W, vw\in E(G)]$ = $p(1-p)m/\binom{n}{2}$. Hence
\begin{align*}
\E[\#\textrm{edges from }V\textrm{ to }W] &= \sum_{v\ne w\in V}p(1-p)\frac{m}{\binom{n}{2}}\\
&= n(n-1)p(1-p)\frac{m}{\binom{n}{2}}\\
&= \frac{m}{2}
\end{align*}

Hence there exists some $V$ for which at least half of the edges are between $V$ and $W$.

\textbf{3}. (i) Presumably the $\eps_i$ are chosen from $\{-1,1\}$ each with probability $1/2$, so that $\E[\eps_i] = 0$, and $\E[\eps_i^2] = 1$ (if $>0$, the first result does not hold). Let $X = \sum_i a_i\eps_i$. Then $\E\sum_i a_i\eps_i = \sum_ia_i\E[\eps_i] = 0$ by linearity of expectation. Similarly, since $\E[\eps_i\eps_j] = \E[\eps_i]\E[\eps_j] = 0$ for $i\ne j$ by independence, we have that $\E[(\sum_i a_i\eps_i)^2] = \sum_i a_i^2\E[\eps_i^2] = \sum_i a_i^2$.

(ii) We can expand $X^{2k}$ and take expectation.
\begin{align*}
X^{2k} = \sum_{i_1+\dots+i_n = 2k}\frac{(2k)!}{i_1!\cdots i_n!}(a_1\eps_1)^{i_1}\cdots (a_n\eps_n)^{i_n}
\end{align*}
When taking expectation, we remark that if $i_j$ is odd for any $j$, then $\E[\eps_j^{i_j}]=0$ so the expectation of the entire summand is zero. Thus every $i_j$ is even, which we may write as $i_j = 2m_j$. Hence
\begin{align*}
\E[X^{2k}] = \sum_{m_1+\dots+m_n = k} \frac{(2k)!}{(2m_1)!\cdots(2m_n)!}a_1^{2m_1}\cdots a_n^{2m_n}
\end{align*}
This expression is very similar to the expansion of Var$[X]^{k}$, the difference being the factorial coefficient in each summand:
\begin{align*}
(\textrm{Var}[X])^{k} = \sum_{m_1+\dots+m_n = k} \frac{k!}{m_1!\dots m_n!} a_1^{2m_1}\dots a_n^{2m_n}
\end{align*}

If we write $S = \{(m_1,\dots,m_n): \sum m_j = k\}$, we can express these sums as $\sum_{s\in S} A_s p_s$ and $\sum_{s\in S}B_s p_s$ respectively. Let $t$ be such that $A_t/B_t$ is maximised. Then for all $s$, we have $p_sA_sB_t \le p_sB_sA_t$, and hence (summing over $s$) we have that
\begin{align*}
\frac{\sum_{s\in S}A_s p_s}{\sum_{s\in S}B_sp_s} \le \frac{A_t}{B_t}
\end{align*}
And the constant $A_t/B_t$ is a quotient of multinomial coefficients, which is dependent only on $k$.


\textbf{4}. We choose a random antisymmetric relation on $[N]$, by saying that for $i < j$, $(i,j)\in R$ with probability $p$, and $(j,i)\in R$ with probability $1-p$. That is to say we go through each subset $\{i,j\}$, and flip a coin with probability $p$ of landing heads. If heads, we put $(i,j)\in R$. If tails, we put $(j,i)\in R$.

Then define $X_{S}$ to be the indicator function of the event `there exists $x\in [N]$ with $xRs$ for all $s\in S$', and then define $X = \sum_{S\in [N]^{(k)}} X_{S}$. We will take $\E[X]$, so we need to know $\E [X_{S}]$.

I'm going to write down a big formula and explain it later.
\begin{align*}
\P[X_S = 0] &= \prod_{m=0}^{k}\left[(1-p^{k-m}(1-p)^{m})^{s_{m+1}-s_m-1}\right]
\end{align*}
The idea here is that we have a $k$-set $S = \{s_1<s_2<\dots<s_k\}$, and we've defined $s_0 = 0$, $s_{k+1} = N+1$. We then take the product over all $x\in [N]\backslash S$ that we do not have $xRs$ for all $s\in S$. This exact probability depends only on where $x$ is relative to the $s_i$; so if it is between $s_m$ and $s_{m+1}$, then the probability $x$ relates to all of them is $(1-p)^{m}p^{k-m}$, so $1 -$ that is the probability this doesn't happen.

Taking the product over all $s_m < x < s_{m+1}$ gives the probability that these all fail, hence the exponent $s_{m+1}-s_m - 1$ in the multiplicand. Then taking the product over all such $m$ gives the probability we have success for no $x \in [N]\backslash S$. We will henceforth denote this product as $p_S$.

So now we have
\begin{align*}
\E[X] &= \sum_{S\in [N]^{(k)}}(1-p_S)\\
&=\binom{N}{k} - \sum_{S\in [N]^{(k)}}p_S
\end{align*}
and the idea is to appropriately choose $p$ and $N$ so that the latter sum is less than 1. Then since $X$ takes integer values, there must be some relation for which $X = \binom{N}{k}$, {\it i.e.} every $k$-set is related to by some $x$.

While it could have been helpful, it turns out that we didn't need $p$ in full generality, and could have just used $p = 1/2$. This hugely simplifies the situation, as we then just have $p_S = (1-2^{-k})^{N-k}$. The latter sum is then $\binom{N}{k}(1-2^{-k})^{N-k}$, which is exponentially decaying in $N$ and so is indeed eventually small enough.

This also gives us an upper bound on how large $N$ needs to be, as
\begin{align*}
\binom{N}{k}(1-2^{-k})^{N-k}&<1\\
\impliedby \left(\frac{eN}{k}\right)^k\left(1-2^{-k}\right)^{N-k} &<1\\
\iff N^k(1-2^{-k})^N &< \left(\frac{k}{e}\right)^k(1-2^{-k})^k = c_k\\
\iff k\log N + N\log(1-2^{-k}) &< \log c_k\\
\iff \frac{N}{\log N} &> -\frac{k}{\log(1-2^{-k})} + \frac{\log c_k}{\log N\log (1-2^{-k})}
\end{align*}
For any fixed $k$, the latter term vanishes as $N\ra \infty$, so we can ignore this term (it is also negative, so we will obtain a stronger condition).

For any $\eps > 0$, we have $N^\eps > \log N$ for sufficiently large $N$, and hence $N/\log N < N^{1-\eps}$. Thus, for any $\eps > 0$ and sufficiently large $k,N$, and noting that $\log(1-2^{-k})\approx -2^{-k}$, we have the sufficient condition for a solution:
\begin{align*}
|X| \ge k^{\frac{1}{1-\eps}}2^{k/(1-\eps)}
\end{align*}
or, for instance, a uniform bound $|X|\ge k^22^{2k}$ implies such an antisymmetric relation exists.

\textbf{5}. Colour each member of $X$ red with probability $p$ and blue with probability $1-p$. Let $X_i$ be the indicator of the event that $A_i$ is not monochrome, and let $X = \sum_{i=1}^{r}X_i$.

$\P[X_i = 1] = 1 - (1/2)^m - (1-1/2)^m = 1 - 2^{1-m}$, since $X_i = 1$ iff the elements of $A_i$ are not all red and not all blue. Hence:
\begin{align*}
\E[X] &= \sum_{i=1}^{r} \P[X_i = 1]\\
&= r(1-2^{1-m}) = r - 2^{1-m}r
\end{align*}
So if $r < 2^{m-1}$ then $2^{1-m}r < 1$, and hence $\E[X] > r -1$, so there exists some colouring for which $X \ge r$; but $X \le r$, so there exists a colouring for which every $A_i$ contains at least one red element and at least one blue element.

Let $R(m)$ be the least $r$ such that there exist sets $A_1,\dots,A_r$ of size $m$ such that for every red-blue colouring there is some $i$ with $A_i$ monochrome. We can bound $R(m)$ recursively.

Given $R(m)$, construct $R(m)$ $m+1$-sets by extending the $A_1,\dots,A_{R(m)}$ $m$-sets.

Pick a set of $B = \{b_i:1\le i\le m+1\}$ such that $B\cap A_i = \emptyset$ for all $i$, and then take the sets $C_{i,j} = A_i\cup\{b_j\}$, altogether along with $B$. Then in any red-blue colouring there must be some $i$ such that $A_i\subset C_{i,j}$ is monochrome, say blue. So then we either have a monochrome set, or every $b_j$ has been coloured red - in which case $B$ is monochrome.

Therefore $R(m+1)\le R(m)(m+1)+1$. Thus:
\begin{align*}
R(m) &\le 1 + R(m-1)m\\
&\le 1 + m + R(m-2)(m-1)m\\
&\le \sum_{j=0}^{k}\frac{m!}{(m-j)!}+R(m-(k+1))(m-k)(m-(k-1))\dots(m)\\
&\le \sum_{j=0}^{m-2} + R(1)m!= \sum_{j=0}^{m-1}\frac{m!}{(m-j)!} =m!\sum_{j = 0}^{m-1}\frac{1}{(m-j)!}\\
&=m!\sum_{\ell = 1}^{m}\frac{1}{\ell!} \le m!(-1 + \sum_{\ell = 0}^{\infty}\frac{1}{\ell!})\\
&\le m!(e-1)
\end{align*}
So we have an upper bound $R(m)\le m!(e-1)$ for each $m$.

\textbf{6}.
\end{document}