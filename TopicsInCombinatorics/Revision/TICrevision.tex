\documentclass{article}


\usepackage{amsmath}
\usepackage{amssymb}
\usepackage{parskip}

\newcommand{\mc}[1]{\mathcal{#1}}
\renewcommand{\rm}[1]{\mathrm{#1}}
\newcommand{\bb}[1]{\mathbb{#1}}
\newcommand{\e}{\rm{e}}
\newcommand{\eps}{\varepsilon}
\newcommand{\R}{\bb{R}}
\newcommand{\Z}{\bb{Z}}
\newcommand{\N}{\mathbb{N}}
\renewcommand{\P}{\mathbb{P}}

\title{Topics in Combinatorics Revision Questions}

\author{Otto Pyper}
\date{}

\begin{document}
    
\maketitle

\begin{enumerate}
    \item \textbf{Prove that if $X$ is a random variable, then $\P[X\ge EX] > 0$ and $\P[X\le EX] > 0$}
    \item State and prove a proposition about average degrees in bipartite graphs.
    \item What is this method called, roughly speaking?
    \item State and prove another proposition about average degrees, concerning an inequality instead of an equality.
    \item Show that for a planar graph, $E\ge 3F/2$.
    \item Define $\partial_s\mc{A}$, in context. What is this called?
    \item State and prove a lower bound on $|\partial_S\mc{A}|$, representing it in two different ways.
    \item State Sperner's Theorem.
    \item \textbf{Prove Sperner's Theorem.}
    \item State the Erdos-Ko-Rado Theorem.
    \item \textbf{Prove the Erdos-Ko-Rado Theorem.}
    \begin{enumerate}
        \item Show that, for a random ordering, at most $k$ intervals can be part of an intersecting family.
        \item What is the expected total number of such intervals belonging to the family?
        \item In the equality case, how are we able to still have $k$ intervals?
        \item Construct a helpful cylcic order to show any $A$ of size $k$ is contained in $\mc{A}$.
    \end{enumerate}
    \item Define the crossing number of $G$.
    \item Show that a planar graph with $n$ vertices has at most $3n-6$ edges.
    \item Prove that a graph with $n$ vertices and $m$ edges has crossing number at least $m - 3n$.
    \item Let $G$ be a graph drawn in the plane with $n$ vertices and $m$ edges, with $m\ge 6n$. $G$ must have at least how many crossings?
    \item \textbf{Prove this.}
    \item State the Szemeredi-Trotter Theorem
    \item \textbf{Prove the Szemeredi-Trotter Theorem.}
    \begin{enumerate}
        \item What is an appropriate way to convert the points/lines into a graph?
        \item Put two different bounds on the number of crossings.
    \end{enumerate}
    \item Give three examples of graphs demonstrating the S-T bound (up to a constant).
    \item State Stirling's Formula. [optional]
    \item State a gentler upper/lower bound on $n!$.
    \item Prove this.
    \item State bounds on $2^{-n}{n\choose n/2}$.
    \item Prove these bounds.
    \item Give an example of an event that occurs with probability $2^{-n}{n\choose n/2}$.
    \item Give a bound on ${n\choose m}$, useful for when $m \ll  n$.
    \item Improve this bound slightly. In what scenario is this often useful?
    \item State the quotient of consecutive binomial coefficients.
    \item Bound $\sum_{k=0}^{m}{n\choose k}$ in the case $m =\alpha n$, $\alpha < 1/2$.
    \item State an example of the \textit{concentration of measure phenomenon}.
    \item Let $m = (1/2 - \theta)n$ with $0 < \theta \le 1/2$. Then $2^{-n}{n\choose m}\le \e^{-\theta^2n/2}$.
    \item \textbf{Prove that if $X_1,\dots,X_n$ are independent random variables of mean zero taking values in $[-1,1]$, and $X = \sum X_i$, then $\P[X\ge \eps n] \le \e^{-\eps^2n/4}$.}
    \begin{enumerate}
        \item What is the exponential moment?
        \item Markov\dots
        \item Optimise over something?
    \end{enumerate}
    \item What similar result do we immediately get by switching signs?
    \item Let $m = (1/2 - \eps)n$. Show that $2^{-n}\sum_{k=0}^{m}{n\choose k} \le \e^{-\eps^2n}$.
    \item What is a general question regarding well-separated sets.
    \item What is the critical threshold for changing behaviour at intersections of size $\alpha n$, and why?
    \item \textbf{Prove that if $\alpha > 1/4$, there can be exponentially many subsets of $[n]$ of size $n/2$ intersecting in no more than $\alpha n$.}
    \begin{enumerate}
        \item Let $A$ be a random set of size $n/2$. What is a good probability to estimate?
        \item Then how many bad intersections can we have?
        \item How can we mitigate this?
    \end{enumerate}
    \item What is the characteristic function of a set?
    \item What space does it live in?
    \item What is an oftentimes more useful function with which to associate a set?
    \item Why is it thusly named?
    \item Prove that if $A,B\subset [n]$ have size $n/2$ and $f_A,f_B$ are their balanced functions, then $\langle f_A,f_B\rangle = |A\cap B| - n/4$.
    \item Use this to prove that if $A_1,\dots,A_m\subset [n]$ intersect in at most $(1/4 - \delta)n$, then $m \le 1 + \delta/4$.
    \item State and prove a more general theorem, of which the above is a special case.
    \item \textbf{Prove that if $x_1,\dots,x_m$ are non-zero vectors in $\R^n$ such that $\langle x_i,x_j\rangle\le 0$ for every $i\ne j$ then $m\le 2n$.}
    \begin{enumerate}
        \item Induction on $n$.
        \item How can we reduce to $n-1$-dimensional space?
    \end{enumerate}
    \item \textbf{Furthermore, prove that if $m = 2n$ then there is an orthonormal basis $a_1,\dots,a_n$ such that each $x_i$ is a multiple of some $a_j$ (so we have exactly one positive multiple and one negative multiple of each $a_i$).}
    \item Relate the above back to the problem of finding families of sets with intersections in at most $n/4$.
    \item Define a Hadamard matrix.
    \item Define the Walsh matrices $W_m$.
    \item State three basic facts about $W_m$.
    \item How do we find the desired set system from the $W_m$?
    \item Define the Paley matrix $P_n$.
    \item Prove that for every prime $p$ and every $d\not\equiv 0 \mod p$ we have that $\sum_{x\in\Z_p} \left( \frac{x}{p}\right) \left(\frac{x+d}{p}\right) = -1$
    \item Hence show $P_n$ has orthogonal rows.
    \item What is the smallest $n$, a multiple of 4, for which no Hadamard matrix is known?
    \item Prove that if $|A| = n$, then $|A+A| \ge 2n-1$. When do we have equality?
    \item Demonstrate a similar result for $|A.A|$.
    \item State a famous conjecture of Erdos and Szemeredi.
    \item Define $\rho_A^+(x)$ and $\rho_A^\times(x)$.
    \item Define the multiplicative energy and additive energy of $A$.
    \item What is the quotient set of $A$?
    \item State and prove a lemma about multliplicative energy.
    \item Prove that $\sum_{i=1}^{n} |a_i|^2\ge n^{-1}(\sum_i|a_i|)^2$.
    \item Prove it again, in a different way.
    \item Show that the multiplicative energy of $A$ is at least $|A|^4/|A.A|$.
\end{enumerate}
\end{document}