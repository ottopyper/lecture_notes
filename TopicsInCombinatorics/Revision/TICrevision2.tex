\documentclass[10pt]{article}


\usepackage{amsmath}
\usepackage{amssymb}
\usepackage{parskip}

\newcommand{\mc}[1]{\mathcal{#1}}
\renewcommand{\rm}[1]{\mathrm{#1}}
\newcommand{\bb}[1]{\mathbb{#1}}
\newcommand{\e}{\rm{e}}
\newcommand{\eps}{\varepsilon}
\newcommand{\R}{\bb{R}}
\newcommand{\Z}{\bb{Z}}
\newcommand{\N}{\mathbb{N}}
\renewcommand{\P}{\mathbb{P}}
\newcommand{\ra}{\to}

\title{Topics in Combinatorics Revision Questions}

\author{Otto Pyper}
\date{}

\begin{document}
    
\maketitle

\begin{enumerate}
    \item State Jensen's inequality.
    \item \textbf{Prove that if $X$ is a random variable, then $\P[X\ge EX] > 0$ and $\P[X\le EX] > 0$}
    \item State and prove a proposition about average degrees in bipartite graphs.
    \item What is this method called, roughly speaking?
    \item State and prove another proposition about average degrees, concerning an inequality instead of an equality.
    \item Show that for a planar graph, $E\ge 3F/2$.
    \item Define $\partial_s\mc{A}$, in context. What is this called?
    \item State and prove a lower bound on $|\partial_S\mc{A}|$, representing it in two different ways.
    \item State Sperner's Theorem.
    \item \textbf{Prove Sperner's Theorem.}
    \item State the Erdos-Ko-Rado Theorem.
    \item \textbf{Prove the Erdos-Ko-Rado Theorem.}
    \begin{enumerate}
        \item Show that, for a random ordering, at most $k$ intervals can be part of an intersecting family.
        \item What is the expected total number of such intervals belonging to the family?
        \item In the equality case, how are we able to still have $k$ intervals?
        \item Construct a helpful cylcic order to show any $A$ of size $k$ is contained in $\mc{A}$.
    \end{enumerate}
    \item Define the crossing number of $G$.
    \item Show that a planar graph with $n$ vertices has at most $3n-6$ edges.
    \item Prove that a graph with $n$ vertices and $m$ edges has crossing number at least $m - 3n$.
    \item Let $G$ be a graph drawn in the plane with $n$ vertices and $m$ edges, with $m\ge 6n$. $G$ must have at least how many crossings?
    \item \textbf{Prove this.}
    \item State the Szemeredi-Trotter Theorem
    \item \textbf{Prove the Szemeredi-Trotter Theorem.}
    \begin{enumerate}
        \item What is an appropriate way to convert the points/lines into a graph?
        \item Put two different bounds on the number of crossings.
    \end{enumerate}
    \item Give three examples of graphs demonstrating the S-T bound (up to a constant).
    \item State Stirling's Formula. [optional]
    \item State a gentler upper/lower bound on $n!$.
    \item Prove this.
    \item State bounds on $2^{-n}{n\choose n/2}$.
    \item Prove these bounds.
    \item Give an example of an event that occurs with probability $2^{-n}{n\choose n/2}$.
    \item Give a bound on ${n\choose m}$, useful for when $m \ll  n$.
    \item Improve this bound slightly. In what scenario is this often useful?
    \item State the quotient of consecutive binomial coefficients.
    \item Bound $\sum_{k=0}^{m}{n\choose k}$ in the case $m =\alpha n$, $\alpha < 1/2$.
    \item State an example of the \textit{concentration of measure phenomenon}.
    \item Let $m = (1/2 - \theta)n$ with $0 < \theta \le 1/2$. Then $2^{-n}{n\choose m}\le \e^{-\theta^2n/2}$.
    \item \textbf{Prove that if $X_1,\dots,X_n$ are independent random variables of mean zero taking values in $[-1,1]$, and $X = \sum X_i$, then $\P[X\ge \eps n] \le \e^{-\eps^2n/4}$.}
    \begin{enumerate}
        \item What is the exponential moment?
        \item Markov\dots
        \item Optimise over something?
    \end{enumerate}
    \item What similar result do we immediately get by switching signs?
    \item Let $m = (1/2 - \eps)n$. Show that $2^{-n}\sum_{k=0}^{m}{n\choose k} \le \e^{-\eps^2n}$.
    \item What is a general question regarding well-separated sets.
    \item What is the critical threshold for changing behaviour at intersections of size $\alpha n$, and why?
    \item \textbf{Prove that if $\alpha > 1/4$, there can be exponentially many subsets of $[n]$ of size $n/2$ intersecting in no more than $\alpha n$.}
    \begin{enumerate}
        \item Let $A$ be a random set of size $n/2$. What is a good probability to estimate?
        \item Then how many bad intersections can we have?
        \item How can we mitigate this?
    \end{enumerate}
    \item What is the characteristic function of a set?
    \item What space does it live in?
    \item What is an oftentimes more useful function with which to associate a set?
    \item Why is it thusly named?
    \item Prove that if $A,B\subset [n]$ have size $n/2$ and $f_A,f_B$ are their balanced functions, then $\langle f_A,f_B\rangle = |A\cap B| - n/4$.
    \item Use this to prove that if $A_1,\dots,A_m\subset [n]$ intersect in at most $(1/4 - \delta)n$, then $m \le 1 + \delta/4$.
    \item State and prove a more general theorem, of which the above is a special case.
    \item \textbf{Prove that if $x_1,\dots,x_m$ are non-zero vectors in $\R^n$ such that $\langle x_i,x_j\rangle\le 0$ for every $i\ne j$ then $m\le 2n$.}
    \begin{enumerate}
        \item Induction on $n$.
        \item How can we reduce to $n-1$-dimensional space?
    \end{enumerate}
    \item \textbf{Furthermore, prove that if $m = 2n$ then there is an orthonormal basis $a_1,\dots,a_n$ such that each $x_i$ is a multiple of some $a_j$ (so we have exactly one positive multiple and one negative multiple of each $a_i$).}
    \item Relate the above back to the problem of finding families of sets with intersections in at most $n/4$.
    \item Define a Hadamard matrix.
    \item Define the Walsh matrices $W_m$.
    \item State three basic facts about $W_m$.
    \item How do we find the desired set system from the $W_m$?
    \item Define the Paley matrix $P_n$.
    \item Prove that for every prime $p$ and every $d\not\equiv 0 \mod p$ we have that $\sum_{x\in\Z_p} \left( \frac{x}{p}\right) \left(\frac{x+d}{p}\right) = -1$
    \item Hence show $P_n$ has orthogonal rows.
    \item What is the smallest $n$, a multiple of 4, for which no Hadamard matrix is known?
    \item Prove that if $|A| = n$, then $|A+A| \ge 2n-1$. When do we have equality?
    \item Demonstrate a similar result for $|A.A|$.
    \item State a famous conjecture of Erdos and Szemeredi.
    \item Define $\rho_A^+(x)$ and $\rho_A^\times(x)$.
    \item Define the multiplicative energy and additive energy of $A$.
    \item What is the quotient set of $A$?
    \item State and prove a lemma about multliplicative energy (a reformulation).
    \item Prove that $\sum_{i=1}^{n} |a_i|^2\ge n^{-1}(\sum_i|a_i|)^2$.
    \item Prove it again, in a different way.
    \item Show that the multiplicative energy of $A$ is at least $|A|^4/|A.A|$.
    \item \textbf{Prove that $\sum_{m\in A/A}\rho^\div_A(m^{-1})^2 \le 2|A+A|^2 \lceil \log |A| \rceil$.}
    \begin{enumerate}
        \item What is dyadic decomposition?
        \item Think a bit about what $\rho^\div_A(m^{-1})^2$ means.
    \end{enumerate}
    \item Hence, prove Solymosi's Theorem.
    \item Define the Kneser Graph $G_{n,k}$.
    \item Give a $k+2$-colouring of $G_{n,k}$.
    \item State the Borsuk-Ulam Theorem.
    \item State an equivalent formulation of the Borsuk-Ulam Theorem.
    \item Prove this equivalence.
    \item State yet another variant of Borsuk-Ulam.
    \item Show that it is implied by the above.
    \item State the mixed version.
    \item Show that the open sets version implies the mixed version.
    \item \textbf{Let $\delta > 0$. Prove that the graph on $S^d$ defined by joining $u,v$ iff $\langle u,v\rangle < -1 + \delta$ has chromatic number at least $d + 2$.}
    \item Show that this bound is sharp for sufficiently small $\delta$.
    \item If $\delta$ is small, what does this say about where a vertex is located relative to its neighbourhood? What can we conclude about odd cycles in this graph?
    \item \textbf{Prove that the chromatic number of the Kneser Graph $G_{n,k}$ is $k + 2$.}
    \begin{enumerate}
        \item How might we represent this graph on the sphere?
        \item What is a good choice of a base set of $2n+k$ points? What key property should they have?
        \item Find a good choice of $k + 2$ open/closed sets to apply B-U to.
    \end{enumerate}
    \item Define what it means for permutation $\sigma$ to contain permutation $\pi$.
    \item Give a construction of a general 2413-avoiding permutaiton.
    \item Define what it means in terms of adjacency matrices of bipartite graphs for one to contain/avoid another.
    \item State the conjecture solved by Marcus \& Tardos.
    \item Define an ordered quotient of an $n\times n$ $01$-matrix.
    \item Let $P$ be a permutation matrix and let $A$ be a $P$-avoiding $01$-matrix. Show that every ordered quotient of $A$ avoids $P$.
    \item Suppose $k^2 | n$. Define a block of $A$.
    \item What does it mean for a block to be wide?
    \item What does it mean for a block to be tall?
    \item Roughly speaking, what is the structure of the following proof?
    \item Show that if $A$ does not contain the $k\times k$ $P$, then for each $j$ the number of wide blocks with columns in $C_j$ is at most $(k-1){k^2\choose k}$ and similarly for each $i$ the number of tall blocks with rows in $R_i$ is at most $(k-1){k^2\choose k}$.
    \item Prove that, for $P$ a $k\times k$ permutation matrix, and $k^2 | n$, then $f(n)$, the largest number of non-zero entires in any $01$-matrix avoiding $P$, satisfies \[f(n) \le 2k^2 n(k-1){k^2\choose k}+(k-1)^2f(n/k^2)\]
    \begin{enumerate}
        \item How many $1$s can we see in wide blocks or tall blocks?
        \item Where does the $f(n/k^2)$ term come from?
    \end{enumerate}
    \item Now prove the theorem of Marcus \& Tardos.
    \item For the $C$ in question, how does it vary with $k$?
    \item State the Stanley-Wilf conjecture.
    \item Prove it, using M-T.
    \item State the Khinchin axioms for entropy.
    \begin{enumerate}
        \item State axiom 0.
        \item State axiom 1.
        \item State axiom 2.
        \item State axiom 3.
        \item State axiom 4.
        \item State axiom 5.
    \end{enumerate}
    \item Prove that if $X,Y$ are independent, then $H[Y|X] = H[Y]$ and $H[X,Y] = H[X] + H[Y]$.
    \item Prove that if $X$ takes just one value, then $H[X] = 0$.
    \item Let $A\subset B$, $X$ uniform on $A$, $Y$ uniform on $B$. Then $H[X]\le H[Y]$, with equality iff $A = B$.
    \item Let $X$ be a random variable and $Y = f(X)$, some function $X$. Then $H[Y] \le H[X]$.
    \item Prove that $H[X] \ge 0$ for every discrete random variable $X$ taking values on a finite set $A$.
    \item Prove that if $X$ takes at least two values with non-zero probability, then $H[X] > 0$.
    \item State and prove the chain rule for entropy.
    \item Describe the process of expressing the P3 bound using entropy.
    \item Show that, in the distribution defined in lectures, the individual edges of the random P3 are uniformly distributed.
    \item Show that a bipartite graph with density $\alpha$ must contain at least $\alpha^3 |A|^2|B|^2$ labelled P3s, and show this is sharp.
    \item State Sidorenko's Conjecture.
    \item Write down the formula for entropy.
    \item What is the entropy of a uniformly random variable?
    \item Prove that the formula for entropy satisfies the Khinchin axioms (in particular, maximality and additivity).
    \item Prove that the Khinchin axioms uniquely determine the formula.
    \begin{enumerate}
        \item Prove that if $X$ is uniform on a set of size $2^k$, then $H[X] = k$.
        \item Prove that if $X$ is uniformly distributed on a set of size $n$, then $H[X] = \log n$.
        \item Now show that $H[X] = \sum_{x\in A} p_a\log(1/p_a)$.
    \end{enumerate}
    \item Define the permanent of a square matrix $A$.
    \item What is the complexity class of the problem of calculating the permanent?
    \item If $A$ is a bipartite adjacency matrix, what does $\rm{per}(A)$ represent?
    \item State Bregman's Theorem.
    \item Demonstrate that the bound given is sharp.
    \item \textbf{Prove the theorem.}
    \begin{enumerate}
        \item Define a random variable whose entropy we aim to bound.
        \item Write down the upper bound that we are aiming for.
        \item Bound $H[\sigma(x_1)]$.
        \item Bound $H[\sigma(x_k)|\sigma(x_1),\dots,\sigma(x_{k-1})]$.
        \item For any fixed $\sigma$, find the distribution of $d^\sigma_{k-1}(x)$.
        \item Take expectations and complete the proof.
    \end{enumerate}
    \item Let $X,Y$ be discrete random variables. Prove that $H[X,Y] \le H[X] + H[Y]$.
    \begin{enumerate}
        \item Prove this using the formula.
        \item Prove this using the axioms.
        \begin{enumerate}
            \item First uniform.
            \item Then rational.
            \item Then real.
        \end{enumerate}
    \end{enumerate}
    \item Deduce a more general subadditivity formula.
    \item State Shearer's Lemma.
    \item Prove Shearer's Lemma.
    \item Let $G$ be a graph with $m$ edges and $t$ triangles. Prove that $t\le (2m)^{3/2}/6$.
    \item Prove that a $\triangle$-intersecting family $\mc{G}$ of graphs on vertex set $[n]$ has size at most $2^{n\choose k}/4$.
    \item State the theorem arising from Dvir's solution to the Kakeya problem for finite fields.
    \item Let $A\subset \bb{F}_p^n$ be a set of size ${n+d\choose d}$. Then there exists a non-zero polynomial $P(x_1,\dots,x_n)$ of degree $d$ that vanishes on $A$.
    \begin{enumerate}
        \item What is another way of representing a polynomial of degree $d$ in variables $x_1,\dots,x_n$?
        \item What is an equivalent notion now to vanishing on $A$?
    \end{enumerate}
    \item In particular this is true when?
    \item State three sufficient conditions (involving a set $A$, a degree $d$, and a polynomial $f$) for the existence of a non-zero degree $d$ polynomial that vanishes everywhere on $\bb{F}_p^n$.
    \item Let $f$ be a non-zero polynomial on $\bb{F}_p^n$ of degree less than $p$. Show that $f$ is not identically zero.
    \item Now prove Dvir's result.
    \item Let $f$ be a non-zero polynomial of degree at most $d$ on $\bb{F}_p^n$. Prove that $f$ has at most $dp^{n-1}$ roots. [Schwarz-Zippel Lemma.]
    \item Name a real-world application of the Schwarz-Zippel Lemma.
    \item State Alon's combinatorial Nullstellensatz.
    \item Prove Alon's combinatorial Nullstellensatz.
    \item State the Cauchy-Davenport Theorem.
    \item Prove the Cauchy-Davenport Theorem.
    \item Define $A\dot{+}B$.
    \item State a result of da Silva and Hamidoune, which is a variant of C-D T.
    \item Prove this.
    \item Show that for when $|A| =|B|$ this result is sharp.
    \item State a theorem due to Roy Meshulam.
    \item State a closely related theorem due to Roth.
    \item Define the rank of $f:X\times Y \ra \bb{F}$.
    \item Define the slice rank of $f : X\times Y\times Z \ra \bb{F}$.
    \item Let $X$ be a finite set, let $A\subset X$, $\bb{F}$ a field, $f:X^3 \ra \bb{F}$ a function such that $f(x,y,z)\ne 0$ iff $x = y = z$ and $x \in A$. Show that the slice rank of $f$ is $A$.
    \item Connect the slice rank to the cap-set problem.
    \item Show that the slice rank of the polynomial $P(x,y,z) = \prod_{i=1}^{n}(1-(x_i+y_i+z_i)^2)$ is at most $3M$, where $M$ is the number of $012$-sequences of length $n$ that sum to at most $2n/3$.
    \item Obtain an upper bound on the above $M$.
    \item Define the Hamming cube $Q^n$.
    \item State the sensitivty conjecture (-adjacent problem) that Hao Huang proved.
    \item Define a helpful class of matrices $A_n$: $n\in\N$.
    \item Show that $A_n$ is symmetric.
    \item Show that the rows of $A_n$ are orthogonal.
    \item What are the possible entries of $A_n$?
    \item How many non-zero entries are there in each row/column of $A_n$?
    \item If the entries are indexed with $01$ sequences, what is the value of $(A_n)_{xy}$ in terms of $x$ and $y$?
    \item How does $A_n$ then relate to $Q^n$?
    \item Describe the assignment of signs to the adjacency matrix of $Q^n$ inductively.
    \item What is $A_n^2$? Prove this.
    \item Prove Huang's Theorem.
    \item Suppose $X\subset \R^n$ and the members of $X$ are all pairwise the same distance. How large can $X$ be?
    \item Let $a_1,\dots,a_m \in \R^n$ be such that the number of distinct distances $d(a_i,a_j)$ with $i\ne j$ is at most $2$. Then $m \le (n+1)(n+4)/2$.
    \item Suppose $\mc{A}$ is a family of subsets of $[n]$, all of even size, all with even intersections. How big can $|\mc{A}|$ be?
    \item Now what happens if we demand the intersections to be of odd size instead?
    \item In particular, let $\mc{A}$ be a family of subsets of $[n]$ all of even size, such that any two distinct sets have odd intersection. Prove that $|\mc{A}| \le n$ if $n$ odd, and $|\mc{A}| \le n-1$ if $n$ is even.
    \item Now prove the even-even case is $|\mc{A}| \le 2^{\lfloor n/2\rfloor}$.
    \item Let $\mc{A}$ be a family of subsets of $[n]$ such that the size of every $A \in \mc{A}$ is a multiple of $p$, but no two distinct sets in $\mc{A}$ have intersection of size a multiple of $p$. Show that $|\mc{A}| \le {n\choose 0} + {n\choose 1} + \dots + {n\choose p-1}$.
    \item Let $p$ be an odd prime and let $n = 4p$. Show that the largest measure of a set $X$ of unit vectors in $\R^{n}$ that contains no pair of orthogonal vectors is exponentially small.
    \item State Borsuk's (disproved) conjecture.
    \item Let $n = 4p$ for a prime $p$. Show that $\R^{n^2}$ contains a set $X$ of size ${n\choose 2p}$ such that every subset of $X$ of smaller diameter has size at most $\sum_{m=0}^{p-1}{n\choose m}$.
\end{enumerate}
\end{document}