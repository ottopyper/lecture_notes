\documentclass[]{article}


\usepackage{amsmath}
\usepackage{amssymb}
\usepackage{amsthm}
\usepackage{graphicx}
\usepackage{parskip}
\usepackage{xcolor}
\usepackage{pagecolor}
\usepackage[margin=1.2in]{geometry}
\usepackage{enumerate}


\usepackage[utf8]{inputenc}
\usepackage[english]{babel}

\usepackage{mathtools}
\DeclarePairedDelimiter\bra{\langle}{\rvert}
\DeclarePairedDelimiter\ket{\lvert}{\rangle}
\DeclarePairedDelimiterX\braket[2]{\langle}{\rangle}{#1 \delimsize\vert #2}

\definecolor{thmcolour}{rgb}{0,0,0}
\definecolor{defcolour}{rgb}{0,0,0}
\definecolor{textcolour}{rgb}{0,0,0}
\definecolor{backgroundcolour}{rgb}{1,1,1}

\pagecolor{backgroundcolour}
\color{textcolour}

\newtheoremstyle{custhm}
{%space above
1em
}{%space below
1em
}{%body font
\color{thmcolour}\itshape
}{%indent amount
-0em
}{%head font
\bfseries\color{thmcolour}
}{%head punct
}{%after head space
1em
}{%head spec
\thmname{#1}\if\relax\detokenize{#2}\relax:
\else\thmnumber{ #2}:\fi
\if\relax\detokenize{#3}\relax
\else\thmnote{ (#3)}\fi
}

\newtheoremstyle{remark}
{%space above
}{%space below
}{% body font
}{%indent amount
-0em
}{%head font
\bfseries
}{%head punct
}{%after head space
0em
}{%head spec
\if\relax\detokenize{#3}\relax \thmname{#1}:
\else \thmname{#3}:
\fi
}

\newtheoremstyle{cusdef}
{%space above
1em
}{%space below
1em
}{%body font
\color{defcolour}
}{%indent amount
-0em
}{%head font
\bfseries\color{defcolour}
}{%head punct
}{%after head space
1em
}{%head spec
%if numbered, include number
%if named, include name
\thmname{#1}\if\relax\detokenize{#2}\relax:\else\thmnumber{ #2}:\fi\if\relax\detokenize{#3}\relax\else\thmnote{ (#3)}\fi
}

\theoremstyle{custhm}
\newtheorem{theorem}{Theorem}[section]
\theoremstyle{cusdef}
\newtheorem{defin}[theorem]{Definition}
\theoremstyle{custhm}
\newtheorem{lemma}[theorem]{Lemma}
\theoremstyle{custhm}
\newtheorem{cor}[theorem]{Corollary}

\theoremstyle{custhm}
\newtheorem{prop}[theorem]{Proposition}

\theoremstyle{custhm}
\newtheorem*{theorem*}{Theorem}

\theoremstyle{cusdef}
\newtheorem*{defin*}{Definition}

\theoremstyle{remark}
\newtheorem*{remark*}{Remark}


%\marginpar{to describe which lecture it is}

\newcommand{\N}{\mathbb{N}}
\newcommand{\Z}{\mathbb{Z}}
\newcommand{\Q}{\mathbb{Q}}
\newcommand{\R}{\mathbb{R}}
\newcommand{\C}{\mathbb{C}}
\newcommand{\e}{\mathrm{e}}
\newcommand{\ra}{\rightarrow}
\newcommand{\lef}{\left(}
\newcommand{\res}{\right)}
\newcommand{\ie}{\textit{i.e. }}
\newcommand{\eps}{\varepsilon}
\newcommand{\E}{\mathbb{E}}
\newcommand{\suminf}{\sum_{n=0}^{\infty}}
\newcommand{\suminfa}[1]{\sum_{#1=0}^{\infty}}
\renewcommand{\P}{\mathbb{P}}
\newcommand{\undf}[1]{\textit{\textbf{#1}}}
\renewcommand{\L}{\mathcal{L}}
\renewcommand{\it}[1]{\textit{#1}}
\newcommand{\M}{\mathcal{M}}
\renewcommand{\phi}{\varphi}
\newcommand{\proves}{\vdash}
\newcommand{\lra}{\leftrightarrow}
\renewcommand{\value}{|\cdot|}
\newcommand{\val}[1]{\left|#1\right|}
\newcommand{\valk}{(K,|\cdot|)}
\renewcommand{\bar}{\overline}
\renewcommand{\O}{\mathcal{O}}
\newcommand{\A}{\mathcal{A}}

\renewcommand{\lnot}{\neg}
\newcommand{\false}{\bot}
\newcommand{\true}{\top}


%opening
\title{Topics In Combinatorics}
\author{}
\date{}

\begin{document}

\maketitle

\marginpar{Lecture 1}

\section{Averaging Arguments}

If you've got a random variable that takes real variables, then with positive probability it will be at least as big as its average, and similarly at least as small:

\begin{theorem}
	Let $X$ be a random variable. Then $\P[X\ge \E X] > 0$.
\end{theorem}

When $X$ is discrete, this result is almost trivial, but in the general (continuous) case it isn't \textit{quite} as trivial.

\begin{proof}
	Suppose that $\P[X\ge \E X] = 0$. The tempting idea here is to say that then $X$ is always strictly less than the average, so when you take the average it's still strictly less than the average - we need to be careful about making this work:
	
	Define $P_n = \P[\E X - \frac{1}{n} < X \le \E X - \frac{1}{n+1}]$ - with $P_0$ denoting $-\infty$ on the left.
	
	It is then the case that $\sum_{n=0}^{\infty}P_n = 1$, so $\exists n: P_n > 0$. But then\[ \E X\le \sum_{n=0}^{\infty}P_n\lef\E X - \frac{1}{n+1}\res = \E X - \suminf \frac{P_n}{n+1} < \E X \quad\bot \]
	
	Similarly with the other case.
\end{proof}

We won't use this case much, but it's fun to see!

The really surprising thing is that this extremely basic fact is also extremely useful. The way we use it (in the discrete case) is to simply deduce that such an event is possible.

\textbf{Question.} How many edges does an icosahedron have?

Perhaps this seems a little tedious - but there is a trick we can use.

We know the icosahedron has 20 faces, and that these faces are triangles. We then reason that each face has three edges, and each edge is part of two faces. That is to say, $2E = 3F$, so $E = 3F/2 = 30$.

The idea here is that both $2E$ and $3F$ are counting something, namely edge-face pairs.

Consider another example - let $G$ be a bipartite graph, with vertex sets $X$ and $Y$. Let's suppose that we have the regularity conditions:
\begin{itemize}
	\item $(\forall x\in X)d(x) = d_1$
	\item $(\forall y\in Y)d(y) = d_2$
\end{itemize}

Then counting the edges from the perspectives of $X$ and $Y$, we have that $|E(G)| = d_1|X| = d_2|Y|$. This is simply an abstraction of the above result.

Moreover, if we instead have
\begin{itemize}
	\item $(\forall x\in X)d(x) \le d_1$
	\item $(\forall y\in Y)d(y) \ge d_2$
\end{itemize}

Then $d_2|Y| \le |E(G)| \le d_1|X|$, so $|Y| \le d_1|X|/d_2$. We can apply this in many ways, such as:

Let $[n] := \{1,2,\cdots,n\}$, and $[n]^{(r)}$ be the collection of subsets of $[n]$ or size $r$.

Let $\mathcal{A} \subset [n]^{(r)}$ (this is an \it{$r$-uniform hypergraph}), and let $s > r$. The \undf{$s$-upper shadow} of $\mathcal{A}$ is
\[
\partial^s\mathcal{A} = \{B\in[n]^{(s)}: \exists A\in \mathcal{A} \textrm{ s.t. } A\subset B\}
\]
Then join $A\in \mathcal{A}$ to $B\in \mathcal{B} = \partial^s\mathcal{A}$ iff $A\subset B$. We then have:
\begin{align*}
	d(A) &= {n-r \choose s-r}\quad \forall A\in \mathcal{A}\\
	d(B) &\le {s \choose r}\quad \forall B\in \mathcal{B}
\end{align*}
Then $\mathcal{A} {n-r \choose s-r} = E \le {s \choose r}|\mathcal{B}|$. Hence $|\mathcal{B}| \le |\mathcal{A}|{n-r \choose s-r}/{s \choose r} = |\mathcal{A}| {n \choose s}/{n \choose r}$.

We remark that this is not the tightest known bound - for an improvement see \it{Kruskal-Katona}.

We have a few more results on graphs.

Recall that a \undf{tree} is a connected, acyclic graph. A basic fact about trees on $n$ vertices is that they have $n-1$ edges. We will see why this is true now.

Firstly, we note that every tree has a vertex of degree one (a \it{leaf}); if not, then we start at a vertex $v_1$, find a neighbour $v_2$, then a neighbour of $v_2$ that is $v_3$, and keep going until we run out of new vertices. Then we must return to a previous $v_i$, forming a cycle.

Then to see that every tree has $n-1$ edges, we consider a tree on $n$ vertices and remove a leaf. This cannot disconnect the graph and cannot create a cycle, so we are left with a tree of degree $n-1$, which by induction has $n-2$ edges. Thus the original tree has $n-1$ edges.

\begin{theorem*}[Euler's Formula]
If $G$ is a connected planar graph with $V$ vertices, $E$ edges and $F$ faces, then $V - E + F = 2$.
\end{theorem*}
Note that a $\it{face}$ of a graph is just a component of the complement of the drawing of the graph in the plane. We prove this through a slightly unusual induction.

\begin{proof}
Base case: $G$ is a tree. Then we have $V-1$ edges, and $G$ is acyclic so we have only one face. Hence $V - E + F = V - (V-1) + 1 = 2$.

If $G$ is not a tree, then $G$ contains a cycle. We then remove an edge from a cycle. The graph remains connected, $V$ stays the same, $E$ decreases by 1, and $F$ decreases by 1 since we have joined two components of the plane. Hence we obtain smaller $G'$ with $V' - E' + F' = V - E + F$. Continue until we are left with a tree.
\end{proof}

\begin{remark*}[Corollary]
$E \le 3V - 6$. This is since each edge is in $\le 2$ faces, and each face contains $\ge 3 $ edges. So $2E \ge 3F$, and hence $2 = V - E + F \le V  - E/3$. Hence $E \le 3V - 6$.
\end{remark*}

\marginpar{Lecture 2}

\section{Intersecting Families}

Suppose we have some $\A\subset \mathcal{P}[n]$ with $A,B\in \A\implies A\cap B \ne \emptyset$. How big can $\A$ be?

Note that if $A\in \A$ then $A^{c}\not\in \A$, hence $|\A|\le 2^{n-1}$. Moreover, we can take $\A = \{A:1\in A\}$, so then $|\A| = 2^{n-1}$ and equality is achieved. So the problem looks pretty solved, but we can also ask what happens in the equality case - if we have an intersecting family of size $2^{n-1}$, must it be of the above form? The answer is no:

If $n$ is odd, we can take $\A = \{A\subset [n]:|A| > n/2\}$. Then for each $A\in [n]$, either $A\in \A$ or $A^c \in \A$, so $|\A| = 2^{n-1}$ and $\A$ is not of the above form.

If instead $n$ is even, we can take all sets of size $ > n/2$, and from the sets of size $n/2$ take exactly one from each complementary pair $(A,A^c)$. This gives us exactly half of $\mathcal{P}[n]$, and it is still an intersecting family since the only way we could have an empty intersection among the size $n/2$ sets would be with a complementary pair.

Another example is $\A = \{A\subset [n] : |A\cap\{1,2,3\}|\ge 2\}$. Since any two pairs of size 2 subsets of $\{1,2,3\}$ intersect non-trivially, this is indeed an intersecting family. Moreover if $A\not\in\A$ then $A$ intersects $\{1,2,3\}$ at most once, so its complement does so at least twice and $A^c\in\A$ so this is indeed half of all the sets.

More generally, we could take some intersecting family that we want, in $\mathcal{P}[m]\subset[n]$, and extend all subsets in that family to get a maximal family that has half of the sets.

Hopefully it is now clear that this is no unique way in which $\A$ can be an intersecting family.

What if instead all sets have size $k$?

Let $\A\subset [n]^{(k)}$ be an intersecting family. How big can $\A$ be?

Boring case: when $k > n/2$, then any two sets intersect, so we can take all ${n \choose k}$ of them.

Mildly interesting case: if $k = n/2$, then since we can't pick any two from an intersecting pair we can only choose at most half of them - but as we have seen this is sufficient to create an intersecting family, so we can have $\frac{1}{2}{n\choose k}$.

Interesting case: what if $k < n/2$. First we will find a \it{lower} bound, then an upper bound.

If we take all sets containing $x$, then $|\A| = {n-1\choose k-1}$ - so this is certainly attainable.

\begin{theorem*}[Erdos-Ko-Rado Theorem]
IF $k < n/2$ and $\A$ is an intersecting family of $k$-subsets of $[n]$, then $|\A| \le {n-1\choose k-1}$.

Moreover, if $|\A| = {n-1 \choose k-1}$, then $\exists x$ such that $\A = \{A\in [n]^{(k)}:x\in A\}$.
\end{theorem*}

The proof given here is not the original one, but is a much simpler one due to Katona.

\begin{proof}
Let $\A$ be an intersecting family of $k$-sets in $[n]$. Pick a random $k$-set as follows.

First, choose a random cyclic order $x_1,\dots,x_n$ of $\{1,\dots,n\}$ - this is a permutation whereby the indices are given modulo $n$, \ie$x_1 < x_2 < \dots < x_n < x_1$.

We then define an interval or arc in this order to be a collection of elements that are next to each other. Pick a random interval of length $k$ in this cyclic order.

The probability that this interval lies in $\A$ is simply $|\A|/{n\choose k}$, since the $k$-sets chosen in this way are uniformly distributed.

\reversemarginpar{BUT} How many intervals can be in $\A$? Consider one such interval $x_1|x_2|x_3|\dots|x_k \in \A$. Then any other interval in $\A$ must start before $x_1$ and end inside the interval, or start after $x_k$ and end inside the interval. However, given a particular starting line as drawn above, we cannot have both the interval going left from it and going right from it as this pair is not intersecting. Since all the possible intersecting intervals arise in this way, we can have at most $k-1$ additional intervals, \ie the number of dividing lines. So including the original interval we have in total $\le k$ of these intervals.

Moreover, for the equality case, if we want all possible intervals we have to have, for each dividing line, either the interval ending on the left or starting on the right. In fact, if we have one going left from a given dividing line, then we can't have one going right from the next line, since $n > 2k$ (the intervals can't meet round the back).

So we have intervals to the right up to a certain point, and then left from that point onwards: $|\ra|\ra\dots |\ra x \leftarrow |\leftarrow|\dots\leftarrow |$. And we see that all the intervals contain the shown $x$ where directions switch. This is what we want for this specific cyclic order, but we are not yet done.

So we can only have $k$ intervals in any given cyclic order, and there are only $n$ cyclic intervals of length $k$.

Note that we initially calculated the probability by first choosing a random cyclic order, and then choosing a random interval from it. If we do this the other way round, then given a cyclic order the probability we end up in the set is $\le k /n$, but the cyclic orders are uniform across the $k$-sets.

So, given a set $A = \{x_1,\dots,x_k\}$, we have $\P(A\in \A) = \sum_{\sigma}\P(A\in \A|\sigma)\P(\sigma)$, where the sum is over all possible cyclic orders containing $x_1<x_2<\dots < x_k$. But for any $\sigma$, $\P(A\in \A|\sigma) \le k/n$ and hence $\P(A\in\A)\le k/n$.

Thus $|\A| \le \frac{k}{n}{n\choose k} = {n-1\choose k-1}$.

For the uniqueness part, we remark that for every cyclic ordering of $[n]$, the intervals must be of the form all containing a single fixed element. Suppose we look at a cyclic ordering and have an interval in $\A$, say $x_1\dots x_k$, and another interval $x_0 < x_1 < \dots < x_{k-1}$ that is \it{not} in $\A$. Then the inflection point in the first interval must be $x_k$, and we contain all intervals beginning after $x_1$ and passing through $x_k$.

So if we take an arbitrary cyclic order, and choose $x_1 < x_2 < \dots < x_{2k-1}$ such that all the intervals of length $k$ within $\A$ are contained in $[x_1,x_{2k-1}]$ and we take $x_0 < x_1$, then $[x_0,x_{k-1}]\not\in\A$, but $[x_j,x_{j+k-1}]\in\A$ for each $j = 1,\dots,k$.

Now what we want to prove is that if we have any set containing $x_k$, then it must be in $\A$, \ie claim $B = \{y_1,\dots,y_{k-1},x_k\}\in\A$.

wlog $y_1,\dots,y_r\in\{x_1,\dots,x_{k-1}\}$ and $y_{r+1},\dots,y_{k-1}\not\in\{x_1,\dots,x_{k-1}\}$. We construct a new cyclic order starting with $x_0$ as above: $x_0 < z_1 < z_2 < \dots < z_s < y_1 < \dots < y_r < x_{k} < y_{r+1}<\dots <y_{k-1}<\dots$, where the elements $z_i$ are precisely the elements $\{x_1,\dots,x_{k-1}\}\backslash \{y_1,\dots,y_r\}$.

Then in this new order the interval $[u,x_k]$ is the same interval as what we had before that did not belong to $x_k$, but $[z_1\dots x_k]$ is simply a reordering of $[x_1,x_k]$ as before, so we have the same situation as before; two adjacent intervals, one not in $\A$ and one in $\A$. Hence all of the $k$-intervals containing $x_k$ in this order are in $\A$ - and this includes the interval $[y_1,\dots,y_r,x_k,y_{r+1},\dots,y_{k-1}]$. Hence the arbitrary $B\supset \{x_k\}$ is in $\A$.
\end{proof}

We now consider a different constraint on a set system - no set in the system is contained within any other; these are sometimes called \it{antichains}.

Before that though, we discuss the discrete cube.

We can think about the discrete cube in different ways; \it{e.g.} as $\mathcal{P}X$ where $|X| = n$, or $\{0,1\}^n$.

For the case $n = 3$, this is of course the standard 3D cube we are all familiar with. We can label the vertices of the cube with binary sequences of length 3, or with subsets of $\{1,2,3\}$, in layers (by drawing the cube appropriately) such that each layer correspond to the possible sizes of sets.

If we have $\A \subset \mathcal{P}[n]$, we define $\A_k = \{A\in\A:|A| = k\}$ to be the \it{$k^{\textrm{th}}$ layer} of $\A$.

\begin{theorem*}[Sperner's Theorem]
Let $\A \subset \mathcal{P}[n]$. If we have $A,B\in\A\implies A\not\subsetneq B$ (\textrm{i.e.} not a proper subset), then
\[
|\A| \le \binom{n}{\lfloor n/2 \rfloor}
\]
\end{theorem*}
This proof has a remarkably short and simple proof, though it omits some further information about this situation.
\begin{proof}
Pick a random maximal chain $\mathcal{C} = \{\emptyset, \{x_1\},\{x_1,x_2\},\dots,\{x_1,\dots,x_n\}\}$. How many elements of $\A$ do we expect this chain to contain?

One answer is that we of course can have at most one, since for any pair in the chain one contains the other.

Let $Asubset [n]$ with $|A| = k$ and consider the probabilty that $A$ belongs to this chain. Then if it belongs to the chain $\mathcal{C}$, then it is the unique set of size $k$ within it. But the chains containing any $k$-set are distributed uniformly, so $\P[A\in\mathcal{C}] = 1/{n\choose k}$.

So the expected number of elements of $\A$ in $\mathcal{C}$ by linearity of expectation is
\[
\sum_{A\in\A}\frac{1}{{n\choose |A|}} = \sum_{k=0}^{n} \frac{|\A_k|}{{n\choose k}}
\]
As we noted before, this cannot exceed 1, hence
\begin{align*}
	\sum_{k=0}^{n} \frac{|\A_k|}{\binom{n}{k}}&\le 1\\
	\therefore \sum_{k=0}^{n}|\A_k| = |\A| &\le \binom{n}{\lfloor n/2\rfloor}
\end{align*}

We will deal with the equality case as well. If $n$ is even, then when we multiplied through by $\binom{n}{n/2}$ we must have maintained equality, and so $|\A_k| = 0$ for $k \ne n/2$. Hence $\A \subset [n]^{(n/2)}$.

Similarly if $n$ is odd, then setting $m = (n-1)/2$ we have
\[
\A\subset [n]^{(m)} \cup[n]^{(m+1)}
\]
We will show that we must simply have $\A = \A_m$ or $\A_{m+1}$.

We must have $\A_{m+1}\cap\partial \A_m = \emptyset$, so if $|\A_{m+1}| + |\A_m| = \binom{n}{m} = \binom{n}{m+1}$ then $|\partial^{m+1}\A_m|\le |\A_m|$. Each $A\in \A_m$ is contained in $m+1$ sets in $[n]^{(m+1)}$ and hence in $\partial \A_m$. Each $B\in \partial A_m$ contains at most $m+1$ sets in $\A_m$, so $|\partial \A_m| \ge |\A_m|$ by our earlier double counting argument, with equality if and only if every set in $\partial \A_m$ contains $m+1$ sets in $\A_m$.

So if $A \in \A_m$, then adding any element and removing another gives another set in $\A_m$. Using this process we can turn any $m$-set into any other, hence either $\A_m = \emptyset$ or $\A_m = [n]^{(m)}$.
\end{proof}

\begin{remark*}
The inequality
\[
\sum_{k=0}^{n} \frac{|\A_k|}{\binom{n}{k}} \le 1
\]
is known as the LYM inequality, after Lubell, Yamamoto and Meshalkin, since it was discovered independently by these three people.
\end{remark*}

\marginpar{Lecture 3}

\section{Szemeredi-Trotter Theorem}

To describe this theorem, we first define an \undf{incidence}: if we have some points and lines in the plane, an incidence is simply a point on a line.

The proof we will give is a cuter one that is due to Szekely.

\begin{defin*}[Crossing Number]
The crossing number of a graph is the smallest number of crossings (pairs of crossing edges) you can have when you draw the graph in the plane.

If $G$ is planar then its crossing number is of course zero.
\end{defin*}

Recall that a planar graph with $n$ vertices has at most $3n-6$ edges (for $n\ge 3$ - this breaks down for $n = 2$ in the obvious way).

We can say a bit about crossing numbers just using the above fact; if we have a graph $G$ with $e(G) = m > 3n-6$ then $G$ has a crossing. So we could find a crossing, and remove an edge from the pair. Then either there are no more crossings, or $m - 1 > 3n-6$ and $G$ has at least two crossings.

This simple argument gives us that $G$ has at least $m - 3n + 6$ crossings. The important bit to remember is that it is $\ge m - 3n$. However, we can say a whole lot more than that using an averaging argument.

Now let $G$ be a graph with $m$ edges and $n$ vertices (we will have $m >> n$). Suppose we pick a random induced subgraph $H$ of $G$ by picking each vertex independently with probability $p$, and suppose $G$ has $t$ crossings. Then the expected number of vertices of $H$ is $pn$, the expected number of edges is $p^2 m$, and the expected number of crossings we count by considering how edge pairs survive; this requires all four vertices involved to be chosen, which happens with probability $p^4$. Hence we expect $H$ to have $p^4t$.

It then follows that $p^4t \ge p^2m - 3pn$ (by expectation), otherwise there exists a graph where LHS - RHS < 0, which is not possible. Then choose $p$ to be $6n/m$ (for which we require $m\ge 6n$), so that $p^4t \ge 3pn$, and thus $t \ge 3n/p^3 = 3nm^3/216n^3 = m^3/72n^2$.

Hence if $m \ge 6n$ then $\#$ crossings $\ge m^3/72n^2$.

This is much better; if we have $m = O(n^2)$, then we have a lower bound of the form $O(n^4)$.

This crossing number inequality is precisely the ingredient that we need in order to prove the Szemeredi-Trotter Theoerem. Before that though, we need to turn a system of lines and points in the plane into a graph. The simple way in which we do this is to say that if we have two points that are joined by a line segment, then we regard that line segment as an edge. We note that it does not need to be planar.

Let the lines be $L_1,\dots,L_m$, and the points $x_1,\dots,x_n$. Define $r_i>0 = \#$ of points in line $L_i$. The $\#$ of incidences is then $\sum_{i=1}^{m}r_i = I$, the $\#$ of edges is $\sum_{i=1}^{m}(r_i - 1) = I - m$, since we do not use the parts of the line that stretch away to infinity. The $\#$ of vertices is $n$.

If $I - m \ge 6n$, then the crossing number inequality tells us that $\#$ crossings $\ge (I-m)^3/72n^2$. Moreover, we have the trivial upper bound that the number of crossings is at most the number of pairs of lines, hence $\#$ crossings $\le \binom{m}{2}\le m^2/2$. Hence
\begin{align*}
\frac{(I-m)^3}{72n^2} &\le \frac{m^2}{2}\\
\therefore I &\le (36m^2n^2)^{1/3} + m
\end{align*}
In particular, $I \le C\max\{m,n,m^{2/3}n^{2/3}\}$ for some constant $C$. So we have

\begin{theorem}[Szemeredi-Trotter Theorem]
	\[
	I\le C\max\{m,n,m^{2/3}n^{2/3}\}
	\]
\end{theorem}

There are a lot of occasions when working on a combinatorics problem that we require some kind of bounds on the parameters involved. We will now cover some fairly ubiquitous bounds that we will find very helpful.

The first bound we take a look at is a bound on $n!$. The main trick here is to consider
\[
\log n! = \log 1 + \log 2 + \dots + \log n
\]
which we bound by sandwiching it between a pair of integrals. In particular, since $\log x$ is concave we have
\[
\log n! \le \int_{1}^{n+1}\log x\ \textrm{d}x = [x\log x - x]_{1}^{n+1}
\]
Hence
\begin{align*}
\log n! &\le (n+1)\log(n+1) - n\\
\implies n! &\le (n+1)^{n+1}\e^{-n}= n\cdot n^n\cdot\left(1+\frac{1}{n}\right)^{n+1}\e^{-n}\\
&\le (n+1)n^n\e^{-n} = (n+1)\left(\frac{n}{e}\right)^n
\end{align*}
Similarly, if we take the limit to $n$ instead of $n+1$, we have
\begin{align*}
\log n! &\ge \int_{1}^{n}\log x\ \textrm{d}x = [x\log x - x]_{1}^{n} = n\log n - n + 1\\
\therefore n! &\ge n^n\e^{-n}\cdot \e \ge \left(\frac{n}{e}\right)^n 
\end{align*}
The takeaway here is that $(n/e)^n$ is often a good approximation for $n!$ - but not always. Consider
\[
\binom{n}{\frac{n}{2}}  = \frac {n!}{\left(\frac{n}{2}\right)!\left(\frac{n}{2}\right)!}\le 2^n
\]
and this isn't very good. But we can do better:
\begin{align*}
2^{-n}\binom{n}{n/2} &= \frac{1\cdot 2\cdots n}{2\cdot 4\cdots n\cdot 2\cdot 4\cdots n} = \frac{1\cdot3\cdots (n-1)}{2\cdot 4\cdots n}\\
\implies [\ \cdot\ ]^2 &= \left(\frac{1}{2}\right)^2\left(\frac{3}{4}\right)^2\cdots\left(\frac{n-1}{n}\right)^2\\
&\le \left(\frac{1}{2}\cdot\frac{2}{3}\right)\left(\frac{3}{4}\cdot\frac{5}{6}\right)\cdots\left(\frac{n-1}{n}\cdot\frac{n}{n+1}\right)\\
&\le \frac{1}{n+1}
\end{align*}
But also,
\begin{align*}
\left(\frac{1}{2}\right)^2\left(\frac{3}{4}\right)^2\cdots\left(\frac{n-1}{n}\right)^2 &\ge \left(\frac{1}{2}\cdot\frac{1}{2}\right)\left(\frac{2}{3}\cdot\frac{3}{4}\right)\cdots\left(\frac{n-2}{n-1}\cdot\frac{n-1}{n}\right)\\
&= \frac{1}{2n}
\end{align*}
Putting this all together, we have thus proved that
\[
\frac{1}{\sqrt{2n}} \le 2^{-n}\binom{n}{\frac{n}{2}} \le \frac{1}{\sqrt{n}}
\]
This isn't particularly surprising, if we consider an r.v. $X = X_1+\dots + X_n$, with $X_i = \pm 1$ with probability $1/2$. Then
\[
2^{-n}\binom{n}{n/2} = \P[X = 0]
\]
and the standard deviation of $X$ is $\sqrt{n}$, so the distribution is clustered about $0$ (roughly normally) contained mostly within a range $\sqrt{n}$, and the middle `slice' of this is one of the discrete values. This isn't rigorous, but it is just a sanity check.

Another very simple estimate is
\begin{align*}
\binom{n}{m}&\le n^m
\end{align*}
This is almost embarrassingly simple, but it is often very helpful. We can also improve the bound using our earlier work, to conclude
\[
\binom{n}{m} \le \left(\frac{\e n}{m}\right)^m
\]
However, even the above bound stops being useful when $m \approx n/2$... we'll deal with this in a second. For now, consider
\begin{align*}
\frac{\binom{n}{k-1}}{\binom{n}{k}} = \frac{k}{n-k+1}\le \frac{k}{n-k}
\end{align*}
We can use this. Let $\theta > 0$ and assume everything that needs to be an integer is. Then
\begin{align*}
\binom{n}{\left(\frac{1}{2} - \theta\right)n} \le \left(\frac{\frac{1}{2} - \frac{\theta}{2}}{\frac{1}{2}+\frac{\theta}{2}}\right)^{\frac{\theta n}{2}}\binom{n}{\left(\frac{1}{2} - \frac{\theta}{2}\right)n}
\end{align*}
This looks like a complicated mess, but we get there simply by `walking' up the $\theta n/2$ binomial coefficients between the two shown, picking up the upper bound on the ratio each time. Moreover, notice that
\begin{align*}
\left(\frac{1-\theta}{1+\theta}\right)^{\frac{\theta n}{2}} \le (1-\theta)^{\frac{\theta n}{2}} \le \e^{-\frac{\theta^2 n}{2}}
\end{align*}
(in general, $(1-x)^t \le \e^{-tx}$). Hence
\begin{align*}
\binom{n}{\left(\frac{1}{2}-\theta\right)n} \le \e^{-\theta^2 n/2}\cdot 2^{-n}
\end{align*}
We obtain a similar bound using a sum of independent random variables. Let $X_1\dots,X_n$ be independent taking values in $[-1,1]$, all of mean zero, and take $X = \sum X_i$. We consider $\P[X\ge \eps n]$, for some $\eps > 0$. We will take \undf{exponential moments}, and then use \undf{Markov's Inequality}. Consider
\begin{align*}
\E \e^{\lambda X} &= \E \e^{\lambda \sum X_i}\\
&=\E \prod \e^{\lambda X_i} = \prod \E\e^{\lambda X_i}
\end{align*}
We can commute $\prod$ and $\E$ by the independence of the $X_i$. We have that
\begin{align*}
\E \e^{\lambda X_i} &= \E\left[1 + \lambda X_i + \frac{\lambda^2X_i^2}{2} + \frac{\lambda^3X_i^3}{3!}+\cdots\right]\\
&\le 1 + \frac{\lambda^2}{2!} + \frac{\lambda^3}{3!}+\cdots\\
&\le 1 + \lambda^2 \le \e^{\lambda^2}\textrm{ for when }\lambda \le 1
\end{align*}
Hence $\E \e^{\lambda X} \le \e^{n\lambda^2}$. Thus
\begin{align*}
\P[X\ge \eps n] = \P[\e^{\lambda X}\ge \e^{\lambda \eps n}] \le \e^{n\lambda^2 - \lambda \eps n}
\end{align*}
by Markov.

We are now free to choose $0 < \lambda < 1$; pick $\lambda = \eps/2$, minimising the RHS, to get $\e^{-\eps^2 n/4}$. We also note that by looking at $-X$ instead of $X$, we have
\begin{align*}
\P[X\le -\eps n] \le \e^{-\eps^2n/4}
\end{align*}
We now consider the following
\begin{align*}
2^{-n}\left(\binom{n}{0} +\binom{n}{1}+\cdots+\binom{n}{\left(\frac{1}{2}-\theta\right)n}\right) = \P[X \le -2\theta n]
\end{align*}
which holds because, if the $X_i$ were only to take values $0$ or $1$ with probability 1/2, the LHS is the probability that their sum  is not larger than $(1/2 - \theta)n$ by conditioning over the acceptable outcomes. Transforming to let the $X_i$ take values $\{-1,1\}$ then gives the desired equality. Then, by the above, we have:
\begin{align*}
2^{-n}\left(\binom{n}{0} +\binom{n}{1}+\cdots+\binom{n}{\left(\frac{1}{2}-\theta\right)n}\right) \le \e^{-\theta^2 n}
\end{align*}
A final remark is that it is useful to have a better bound in some cases:
\[
\binom{n}{\alpha n} \sim 2^{H(\alpha)n}
\]
where
\[
H(\alpha) = \alpha \log \frac{1}{\alpha} + (1-\alpha)\log \frac{1}{1-\alpha}
\]
is the \undf{entropy function}.

\section{Well-Separated Set Systems}

We shall take a collection $\A\subset [n]^{(n/2)}$, with the condition that $A\ne B\in \A\implies |A\cap B|\le \alpha n$.

Since we expect that $|A\cap B| = n/4$, it makes a huge difference how large $\alpha$ is relative to $1/4$.

If $\alpha > 1/4$, write $\alpha = \frac{1}{4}+\eps$; {\it i.e.} the intersection sizes shouldn't be too much larger than if we just chose them at random.

Let $A \in [n]^{(n/2)}$, and choose $B\in [n]^{(n/2)}$ at random. What is $\P[|A\cap B| >(1/4+\eps)n]$?

Let $m = (1/4 - \eps)n$; then $|A\cap B| \ge (1/4 + \eps)n \iff |A^c\cap B| \le m$.

The total number of such $B$ is just $\binom{n}{n/2}$, and we want to count those which intersect as desired:
\begin{align*}
 &=\binom{n}{n/2}^{-1}\sum_{r=0}^{m}\binom{n/2}{r}\binom{n/2}{n/2-r}\\
 &\le \frac{2\sqrt{n}}{2^n}\frac{2^{n/2}}{\sqrt{n/2}}\sum_{r=0}^{m}\binom{n/2}{r}\\
 &\le 4 \cdot 2^{-n/2}\sum_{r=0}^{m}\binom{n/2}{r}
\end{align*}
We then use the above bound derived earlier, to obtain ($n\mapsto n/2$, $\eps \mapsto 2\eps$):
\begin{align*}
\le 4\cdot \e^{-2\eps^2 n}
\end{align*}

Now pick $A_1,\dots,A_N$ independently at random from $[n]^{(n/2)}$. Call a pair $(i,j)$ {\it bad} if $i\ne j$ and $|A_i\cap A_j| \ge (1/4+\eps)n$.

Then $\E$\# bad pairs $\le \binom{N}{2}\cdot 4\cdot \e^{-2\eps^2 n}$. So we can find $A_1,\dots,A_N$ such that the number of bad pairs is also bounded by that amount.

If \# bad pairs is $\le N/2$, then we can throw away a set from each bad pair and still have $\ge N/2$ sets, which will then be well-separated as desired. For this to work, we have a sufficient condition
\begin{align*}
	\frac{N^2}{2}\cdot 4\e^{-2\eps^2n}&\le \frac{N}{2}\\
\iff N &\le \frac{1}{4}\e^{2\eps^2 n}
\end{align*}

This technique is good, but we could instead have simply chosen our sets greedily:

Pick $A_1$, find $A_2$ good, find $A_3$ good with both $A_1,A_2$ {\it etc...}, until we get stuck at some $A_N$. Each $A_i$ rules out at most $4\e^{-2\eps^2n}\cdot 2^n$ sets, so we will not get stuck unless $N\cdot 4\e^{-2\eps^2 n} \ge 1$, so we can have $N = \frac{1}{4}\e^{2\eps^2n}$; hence the greedy algorithm works just as well here (though this is not always the case).

What if instead we have $\alpha < \frac{1}{4}$? We now turn to a ubiquitous technique in combinatorics:

\undf{Think of sets as vectors!}

There are many ways in which we can do this; perhaps the most obvious way is to replace it by its indicator sequence.

In this case, we have $A \in [n]^{(n/2)}$; a rather nice way of converting it into a vector is to identify it with the function $f_A: [n]\ra\R$, defined by
\begin{align*}
	f_A(x) = \left\lbrace \begin{array}{c}\ \ 1\quad x\in A\\ -1\quad x\not\in A\end{array}\right.
\end{align*}

If $|A\cap B| \le (1/4 - \eps)n$, then $\langle f_A,f_B\rangle \le -4\eps n$ by drawing a Venn diagram and bounding the various regions appropriately. If we then normalise by letting $g_A = n^{-1/2}f_A$, then $||g_A||_2 = 1$, and $\langle g_A,g_B\rangle \le -4\eps$.

So we can solve a new problem: how many unit vectors in $\R^n$ can there be if any two have inner product $\le -\delta$, for some $\delta > 0$.

Let $u_1,\dots,u_m$ be unit vectors, and $\langle u_i,u_j\rangle \le -\delta$ when $i\ne j$. Consider:
\begin{align*}
0\le ||\sum_{i=1}^{m}u_i||^2_2 &= \sum_{i,j=1}^{m}\langle u_i, u_j\rangle\\
&\le m - m(m-1)\delta\\
\implies m &\le 1 + \frac{1}{\delta}
\end{align*}
Surprisingly, this bound is independent of the dimension, which is a huge contrast from the exponentially-many sets that we had before. In particular, back in the original problem this gives us at most $1 + 1/4\eps$ sets.

This argument generalises very easily. Suppose that each set has size $\theta n$, and $|A \cap B|\le (\theta^2 - \eps)n$. Then again we get a bound that is independent of $n$.

It is worth remarking that the $1 + 1/\delta$ bound is best possible when $\delta = 1/k$ for some $k \in \N$. To achieve this we take the $k+1$ vertices of a regular simplex in $k$-dimensions. To prove this is rather tricky, unless you have a trick:

We are trying to find a simplex with $k+1$ vertices in $\R^k$; the trick is to look in the subsapce $\{x\in \R^{k+1}:x_1+\dots+x_{k+1} = 0\}$. We then find vertices of the form $(-1/k,-1/k,\dots,-1/k,1,-1/k,\dots,-1/k) = v_i$ for the $1$ in the $i^{\textrm{th}}$ place.

Then for $i\ne j$, we have $\langle v_i,v_j\rangle = \frac{k-1}{k^2} - \frac{2}{k} = -\frac{1}{k^2} - \frac{1}{k}$. If $i=j$, then we have $||v_i||^2 = k/k^2 + 1 = 1 + 1/k$. So to normalise, we set $u_i = v_i/\sqrt{1+1/k}$, and this cancels out appropriately to give $\langle u_i, u_j\rangle = - 1/k$, as desired.

\end{document}