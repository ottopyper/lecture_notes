\documentclass[]{article}

\usepackage{amsmath}
\usepackage{amssymb}
\usepackage{amsthm}
\usepackage{graphicx}
\usepackage{parskip}
\usepackage{xcolor}
\usepackage{pagecolor}
\usepackage[margin=1.2in]{geometry}
\usepackage{enumerate}
\usepackage{enumitem}
\usepackage{tikz}
\newcommand*\circled[1]{%
   \tikz[baseline=(C.base)]\node[draw,circle,inner sep=1.2pt,line width=0.2mm,](C) {#1};}
\newcommand*\Myitem{%
   \stepcounter{enumi}\item[\circled{\theenumi}]}

\usepackage[utf8]{inputenc}
\usepackage[english]{babel}

\usepackage{mathtools}
\DeclarePairedDelimiter\bra{\langle}{\rvert}
\DeclarePairedDelimiter\ket{\lvert}{\rangle}
\DeclarePairedDelimiterX\braket[2]{\langle}{\rangle}{#1 \delimsize\vert #2}

\definecolor{thmcolour}{rgb}{0,0,0}
\definecolor{defcolour}{rgb}{0,0,0}
\definecolor{textcolour}{rgb}{0,0,0}
\definecolor{backgroundcolour}{rgb}{1,1,1}

\pagecolor{backgroundcolour}
\color{textcolour}

\newtheoremstyle{custhm}
{%space above
}{%space below
}{%body font
\color{thmcolour}\em
}{%indent amount
-0em
}{%head font
\bfseries\color{thmcolour}
}{%head punct
}{%after head space
1em
}{%head spec
\thmname{#1}\if\relax\detokenize{#2}\relax:\else\thmnumber{ #2}:\fi\if\relax\detokenize{#3}\relax\else\thmnote{ (#3)}\fi
}

\newtheoremstyle{ex}
{%space above
}{%space below
}{%body font
\color{thmcolour}
}{%indent amount
-0em
}{%head font
\bfseries\color{thmcolour}
}{%head punct
}{%after head space
1em
}{%head spec
\thmname{#1}\if\relax\detokenize{#2}\relax:\else\thmnumber{ #2}:\fi\if\relax\detokenize{#3}\relax\else\thmnote{(#3)}\fi
}

\newtheoremstyle{remark}
{%space above
}{%space below
}{% body font
}{%indent amount
-0em
}{%head font
\bfseries
}{%head punct
}{%after head space
1em
}{%head spec
\if\relax\detokenize{#3}\relax\thmname{#1}:\else\thmname{#3}:\fi
}

\newtheoremstyle{numremark}
{%space above
}{%space below
}{% body font
}{%indent amount
-0em
}{%head font
\bfseries
}{%head punct
}{%after head space
1em
}{%head spec
\thmname{#1}\thmnumber{ #2}:
}

\newtheoremstyle{cusdef}
{%space above
}{%space below
}{%body font
\color{defcolour}
}{%indent amount
-0em
}{%head font
\bfseries\color{defcolour}
}{%head punct
}{%after head space
1em
}{%head spec
%if numbered, include number
%if named, include name
\thmname{#1}\if\relax\detokenize{#2}\relax:\else\thmnumber{ #2}:\fi\if\relax\detokenize{#3}\relax\else\thmnote{ (#3)}\fi
}

\theoremstyle{custhm}
\newtheorem{theorem}{Theorem}[section]
\theoremstyle{cusdef}
\newtheorem{defin}[theorem]{Definition}
\theoremstyle{custhm}
\newtheorem{lemma}[theorem]{Lemma}
\theoremstyle{custhm}
\newtheorem{cor}[theorem]{Corollary}

\theoremstyle{custhm}
\newtheorem{prop}[theorem]{Proposition}

\theoremstyle{ex}
\newtheorem{ex}[theorem]{Example}

\theoremstyle{custhm}
\newtheorem*{theorem*}{Theorem}

\theoremstyle{cusdef}
\newtheorem*{defin*}{Definition}

\theoremstyle{remark}
\newtheorem*{remark*}{Remark}

\theoremstyle{remark}
\newtheorem{remark}[theorem]{Remark}

\theoremstyle{numremark}
\newtheorem{numremark}[theorem]{Remark}

\setcounter{section}{-1}

%\marginpar{to describe which lecture it is}

\newcommand{\N}{\mathbb{N}}
\newcommand{\Z}{\mathbb{Z}}
\newcommand{\Q}{\mathbb{Q}}
\newcommand{\R}{\mathbb{R}}
\newcommand{\C}{\mathbb{C}}
\newcommand{\e}{\mathrm{e}}
\newcommand{\ra}{\rightarrow}
\newcommand{\lef}{\left(}
\newcommand{\res}{\right)}
\newcommand{\ie}{\textit{i.e.}}
\newcommand{\eps}{\varepsilon}
\newcommand{\E}{\mathbb{E}}
\newcommand{\suminf}{\sum_{n=0}^{\infty}}
\newcommand{\suminfa}[1]{\sum_{#1=0}^{\infty}}
\renewcommand{\P}{\mathbb{P}}
\newcommand{\undf}[1]{\textit{\textbf{#1}}}
\renewcommand{\L}{\mathcal{L}}
\renewcommand{\it}[1]{\textit{#1}}
\newcommand{\M}{\mathcal{M}}
\renewcommand{\phi}{\varphi}
\newcommand{\proves}{\vdash}
\newcommand{\lra}{\leftrightarrow}

\renewcommand{\bar}{\overline}
\renewcommand{\O}{\mathcal{O}}


\newcommand{\ac}[1]{\mathcal{#1}}
\newcommand{\A}{\mathcal{A}}


\renewcommand{\subset}{\subseteq}

\renewcommand{\th}{\textrm{th}}



\date{}
\author{Otto Pyper}
\title{QC revision questions}

\begin{document}

\maketitle

\begin{enumerate}
    \item What is the time complexity of the fastest classical factoring algorithm?
    \item What is the time complexity of Shor's algorithm?
    \item Describe the Quantum Periodicity Determination Problem.
    \item How is the quantum orcale represented as a gate?
    \item What is the query complexity of an algorithm?
    \item Fastest possible classical periodicity algorithm?
    \item Describe the Quantum Period Finding Algorithm.
    \item How does QFT act on a state $\ket{x}$?
    \item State the Coprimality Theorem.
    \item State the `Probability Lemma'.
    \item QFT maps which basis to the standard basis? Describe the states.
    \item Describe the eigenvalues of the above basis states.
    \item What is $[QFT]_{k\ell}$?
    \item Describe the Hidden Subgroup Problem.
    \item What time complexity do we aim for in the HSP?
    \item What form of a solution do is acceptable for HSP?
    \item Express periodicity as an HSP.
    \item Express the Discrete Logarithm Problem as an HSP.
    \item Describe the Graph Isomorphism Problem.
    \item Express the Graph Isomorphism Problem as a non-Abelian HSP.
    \item Who found a quasi-polynomial time classical algorithm for GI, when, and what is its runtime?
    \item Describe another problem that can be rephrased as an HSP.
    \item What is a representation of $G$? What property does a represenation have when $G$ is abelian?
    \item Prove that any value $\chi(g)$ is a $|G|^{\textrm{th}}$ root of unity.
    \item State (and prove *) Schur's Lemma (Orthogonality).
    \item Enumerate the different representations of $G$.
    \item What is the trivial irrep?
    \item What are the shift operators?
    \item What is the state $\ket{\chi_k}$?
    \item How are these states acted on by shift operators? Prove this.
    \item What is QFT? [Again.]
    \item What is $[QFT^{-1}]_{gk}$?
    \item What is $[QFT]_{kg}$?
    \item What is $QFT \ket{G}$?
    \item What is QFT on $G = \Z_M$?
    \item Describe the Quantum Algorithm for Finite Abelian HSP.
    \item What is the output for the above algorithm?
    \item How do we use said output to determine the hidden subgroup?
    \item For non-abelian $G$, what is the problem with the QFT construction?
    \item What is an irreducible representation for a non-abelian group $G$?
    \item What is a complete set of irreps?
    \item State the generalisation of the previous representation theorem for non-abelian groups.
    \item How is QFT defined on such a $G$?
    \item Why does the same algorithm not work for non-abelian HSP?
    \item How can we modify it to obtain \textit{some} information about $K$?
    \item How efficiently must we be able to implement QFT to use it here?
    \item Under what circumstances does efficient implementation of QFT suffice to solve HSP?
    \item For general non-abelian HSP, how many random coset states suffice to determine $K$?
    \item Why is this not enough to solve HSP?
    \item What is the Phase Estimation problem?
    \item What extra gates do we need for PE? How do they act?
    \item Given $U$ as a formula or circuit distribution, how can we implement C-$U$?
    \item What further information do we need to control $U$ if it is given as a black box?
    \item Why is this further information necessary?
    \item Given this information, draw a diagram to implement C-$U$.
    \item Now what gate do we actually need?
    \item Construct this gate with a diagram.
    \item How does this gate act on $\ket{\xi} = \ket{v_\phi}$?
    \item Describe (with the aid of a diagram) the Quantum Phase Estimation Algorithm.
    \item State and prove Theorem (PE).
    \item How many lines do we need to calculate $\phi$ to accuracy $m$ bits with probability $1-\eta$?
    \item How does implementing $C-U^{2^k}$ impact the algorithm?
    \item What happens if you do PE to an arbitrary state $\ket{\xi}$?
    \item What is the precision issue in the above process?
    \item What is the reflection operator $I_{\ket{\alpha}}$?
    \item How does $I_{\ket{\alpha}}$ interact with unitaries?
    \item How does this generalise to a $k$-dimensional subspace $A\subset \mathcal{H}_d$ with onb $\ket{a_i}:i\le k$?
    \item That is, define $P_A$ and $I_A$.
    \item Describe the context for using Grover's Algorithm. Which problems are these closely related to?
    \item Write down the Grover iteration operator and describe its terms.
    \item State Grover's Theorem (1996).
    \item Describe Grover's Algorithm.
    \item Approximately how many iterations are required in the algorithm?
    \item What sort of speed-up does Grover's Algorithm give on the classical case?
    \item Write down the generalisation of the Grover operator used in AA.
    \item State and prove the Amplitude Amplification Theorem.
    \item Describe the Amplitude Amplification Algorithm, including the number of iterations required.
    \item What is the approximate accuracy of the AA process?
    \item Implementation of which gates are sufficient for AA?
    \item What conditions are sufficient to be able to compute $I_G$? Prove this.
    \item How is $I_{\ket{\psi}}$ implemented?
    \item How does AA affect the distribution of $\ket{\psi}$ restricted to the good subspace?
    \item What is the above particularly useful for?
    \item How can AA be made exact?
    \item Describe how AA solves Grover search with one or more `good' items.
    \item Describe how AA gives a square-root speedup of general quantum algorithms.
    \item Describe the use of both AA \& PE in Quantum Counting.
    \item Write down the time-independent Schrodinger equation (in units where $\bar{h} =1 $) and its solution.
    \item What is the Hamiltonian Simulation problem?
    \item What the the operator norm/spectral norm of operator $A$?
    \item What properties does it have?
    \item What is meant by `$\tilde{U}$ approxiamtes $U$ to within $\eps$'?
    \item What constraints are we aiming for in HamSim?
    \item Define a $k$-local (Hamiltonian) operator.
    \item Why are we able to work better with $k$-local Hamiltonians?
    \item Write down the Ising Model operator.
    \item Write down the Heisenberg Model operator.
    \item State the Solovay-Kitaev Theorem.
    \item State and prove the lemma describing error accumulation under the operator norm.
    \item Prove that for any $k$-local $H$ with commuting $H_j$s, $\e^{-iHt}$ can be efficiently approximated.
    \item State and prove the Lie-Trotter Product Formula.
    \item What is the overall circuit size for $k$-local HamSim?
    \item How is this changed if we instead want to use a standard universal set?
    \item What levels of complexity in $t$ can be achieved by refining Lie-Trotter?
    \item What is HHL used for?
    \item What type of solution do we aim to output?
    \item What are some common applications of HHL?
    \item What is the best known classical runtime for solving systems of linear equations?
    \item What is the condition number of a matrix?
    \item State an intermediate issue immediately faced by trying to compute properties of large systems.
    \item State three conditions on a matrix $A$ necessary for applying HHL, defining any terms.
    \item Define `row-sparse' and `row-$s$-sparse'.
    \item Define `row-computable'.
    \item State the Hamiltonian Simulation Property.
    \item Give an example of a class of matrices which are row-sparse \& row-computable.
    \item State the conditions on $b$ required for HHL.
    \item State the conditions required on $M$ for efficient computation of $x^\dagger Mx$.
    \item How do you get around one of the above conditions failing?
    \item What is the best-known classical runtime achieving the same output as HHL?
    \item What is the runtime of HHL to produce output state within $\eps$ of $\ket{\hat{x}}$?
    \item In the regime with $\eps = 1/(\textrm{poly}(\log N))$, how does HHL compare to classical runtime?
    \item Describe the HHL algorithm, including how AA can be used to improve runtime.
    \item State the Chernoff-Hoeffding bound.
    \item How can PE be modified to give improved accuracy?
    \item What is the worst source of accuracy in HHL?
    \item Which method can be used to replace this? What else can this method be used for?
    \item Define weak simulation of QC.
    \item Define strong simulation of QC.
    \item Demonstrate that efficient simulation is possible if we have a product state promise.
    \item Define the $n$-qubit Pauli group, $\mathcal{P}_n$.
    \item Define a Clifford operation, and the Clifford group.
    \item How does the Clifford group relate to the Pauli group?
    \item State some important applications of the Pauli group \& Clifford group.
    \item Give five examples of Clifford operators.
    \item State a theorem characterising all Clifford circuits.
    \item Define `Clifford computation'.
    \item State the Gottesman-Knill Theorem (variant).
    \item An alternative proof of GK uses which formalism? What was it introduced for?
    \item Prove the GK Theorem.
    \item Define Adaptive and Non-adaptive Clifford circuits.
    \item State a theorem about adaptive/non-adaptive Clifford circuits.
    \item Prove (roughly) the non-adaptive case (demonstrate the critical idea).
    \item Define the $T$-gate.
    \item State a fact about $T$-gates and Clifford circuits.
    \item Define a `magic state' $\ket{A}$.
    \item Implement a $T$-gate using an adaptive Clifford circuit and $\ket{A}$.
    \item Explain how this proves the seconnd part of the theorem.
    \item Give a full characterisation of adaptive/non-adaptive Clifford circuits.
    \item State an ingredient that elevates classically limited power to full quantum power.
    \item Why is this potentially alarming?
\end{enumerate}


\end{document}