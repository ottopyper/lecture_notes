\documentclass[]{article}


\usepackage{amsmath}
\usepackage{amssymb}
\usepackage{amsthm}
\usepackage{graphicx}
\usepackage{parskip}
\usepackage{xcolor}
\usepackage{pagecolor}
\usepackage[margin=1.2in]{geometry}
\usepackage{enumerate}
\usepackage{mathabx}

\usepackage[utf8]{inputenc}
\usepackage[english]{babel}

\usepackage{mathtools}
\DeclarePairedDelimiter\bra{\langle}{\rvert}
\DeclarePairedDelimiter\ket{\lvert}{\rangle}
\DeclarePairedDelimiterX\braket[2]{\langle}{\rangle}{#1 \delimsize\vert #2}

\definecolor{thmcolour}{rgb}{0,0,0}
\definecolor{defcolour}{rgb}{0,0,0}
\definecolor{textcolour}{rgb}{0,0,0}
\definecolor{backgroundcolour}{rgb}{1,1,1}

\pagecolor{backgroundcolour}
\color{textcolour}

\newtheoremstyle{custhm}
{%space above
1em
}{%space below
1em
}{%body font
\color{thmcolour}
}{%indent amount
-0em
}{%head font
\bfseries\color{thmcolour}
}{%head punct
}{%after head space
1em
}{%head spec
\thmname{#1}
\if\relax\detokenize{#2}\relax:
\else\thmnumber{ #2}:\fi
\if\relax\detokenize{#3}\relax
\else\thmnote{ (#3)}\fi
}

\newtheoremstyle{remark}
{%space above
}{%space below
}{% body font
}{%indent amount
-0em
}{%head font
\bfseries
}{%head punct
}{%after head space
0em
}{%head spec
\if\relax\detokenize{#3}\relax \thmname{#1}:
\else \thmname{#3}:
\fi
}

\newtheoremstyle{cusdef}
{%space above
1em
}{%space below
1em
}{%body font
\color{defcolour}
}{%indent amount
-0em
}{%head font
\bfseries\color{defcolour}
}{%head punct
}{%after head space
1em
}{%head spec

%if numbered, include number
%if named, include name

\thmname{#1}
\if\relax\detokenize{#2}\relax:
\else\thmnumber{ #2}:\fi
\if\relax\detokenize{#3}\relax
\else\thmnote{ (#3)}\fi
}

\theoremstyle{custhm}
\newtheorem{theorem}{Theorem}[section]
\theoremstyle{cusdef}
\newtheorem{defin}[theorem]{Definition}
\theoremstyle{custhm}
\newtheorem{lemma}[theorem]{Lemma}
\theoremstyle{custhm}
\newtheorem{cor}[theorem]{Corollary}

\theoremstyle{custhm}
\newtheorem{prop}[theorem]{Proposition}

\theoremstyle{custhm}
\newtheorem*{theorem*}{Theorem}

\theoremstyle{cusdef}
\newtheorem*{defin*}{Definition}

\theoremstyle{remark}
\newtheorem*{remark*}{Remark}


%\marginpar{to describe which lecture it is}

\newcommand{\N}{\mathbb{N}}
\newcommand{\Z}{\mathbb{Z}}
\newcommand{\Q}{\mathbb{Q}}
\newcommand{\R}{\mathbb{R}}
\newcommand{\C}{\mathbb{C}}
\newcommand{\e}{\mathrm{e}}
\newcommand{\ra}{\rightarrow}
\newcommand{\lef}{\left(}
\newcommand{\res}{\right)}
\newcommand{\ie}{\textit{i.e.}}
\newcommand{\eps}{\varepsilon}
\newcommand{\E}{\mathbb{E}}
\newcommand{\suminf}{\sum_{n=0}^{\infty}}
\newcommand{\suminfa}[1]{\sum_{#1=0}^{\infty}}
\renewcommand{\P}{\mathbb{P}}
\newcommand{\undf}[1]{\textit{\textbf{#1}}}
\renewcommand{\L}{\mathcal{L}}
\renewcommand{\it}[1]{\textit{#1}}
\newcommand{\M}{\mathcal{M}}
\renewcommand{\phi}{\varphi}
\newcommand{\proves}{\vdash}
\newcommand{\lra}{\leftrightarrow}
\renewcommand{\value}{|\cdot|}
\newcommand{\val}[1]{\left|#1\right|}
\newcommand{\valk}{(K,|\cdot|)}
\renewcommand{\bar}{\overline}
\renewcommand{\O}{\mathcal{O}}

\renewcommand{\lnot}{\neg}
\newcommand{\false}{\bot}
\newcommand{\true}{\top}

\title{Category Theory: Sheet 1}
\author{Otto Pyper}
\date{}

\begin{document}

\maketitle
\clearpage

\textbf{1.16}. We have, for any $a\in L$, that $a \land 0 = 0$, $a\land 1 = a$, $a\lor 0 = a$. Hence we have identity morphisms $1_{n} = \delta_{ij}$, such that for any $A:n\ra m$ we have that $(A1_{n})_{ij} = \bigvee_{k=1}^{n}(A_{ik}\land \delta_{kj}) = A_{ij}\lor \bigvee_{k=1}^{n}(0) = A_{ij}$, and similarly for the identity on the right.

For $A:k\ra m$ and $B:m\ra n$, the composite $BA$ is given by $(BA){ij} = \bigvee_{k=1}^{m}(B_{ik}\land A_{kj})$, so $BA : k\ra n$ and this composition is associative since we may distributive the meets over the joins. Thus these form a category $\textbf{Mat}_L$.

We then define a functor $F:\textbf{Mat}_L\ra \textbf{Rel}_f$, by $m\mapsto [m]$ and $A\mapsto \{(i,j):A_{ji} = 1\}$. This defines a functor since $F1_n = \{(i,i):i\in[n]\}$ is the identity relation, and $FBFA = \{(i,j):\exists \ell\in[m]\textrm{ such that }(i,\ell)\in FA\textrm{ and }(\ell,j)\in FB \} = \{(i,j):\exists \ell \in [m]\textrm{ such that }B_{j\ell} =1\textrm{ and }A_{\ell i} = 1\} = \{(i,j):(BA)_{ji} = 1\} = FBA$.

This is essentially surjective since the isomorphism classes of objects in $\textbf{Rel}_f$ are all sets of fixed size (any two such sets are isomorphic by a diagonal relation between the two), it is faithful since the data of $FA$ describe precisely the value of $A_{ij}$ for each $i,j$, and every relation $R:[m]\ra[n]$ is the image under $F$ of the matrix $A:m\ra n$, $A_{ji} = \textrm{Id}_{(i,j)\in R}$.

So $F$ is fully faithful and essentially surjective, and hence $\textbf{Mat}_L\equiv \textbf{Rel}_f$.

\textbf{1.17}. To avoid confusion, let $\underline{e}:e\in \mathcal{E}$ be the objects of $\mathcal{C}[\check{\mathcal{E}}]$ and $\tilde{f}$ the morphisms between them.

(i) First note that $dfe = f\implies dfee = dfe = fe$, and similarly $dfe = df $. So $dfe = f = df = fe$, and the other direction is clear. Composition is already defined for such $f$ as in $\mathcal{C}$, so we need only show that this composition preserves the desired property.

Indeed if $df = fe = f$ and $cg = gd = g$, then $cgf = gf = gfe$, so $\tilde{gf} : \underline{e}\ra\underline{c}$ as required.

The identity $1_{\underline{e}}$ is simply $\tilde{e}$, which clearly satisfies $ee = ee = e$, and for any $\tilde{f}$ we have $df = fe = e$, and so $d(fe) = (fe)e = fe$, and similarly for the identity on the left.

So these form a category $\mathcal{C}[\check{\mathcal{E}}]$.

(ii) We will let $I: \mathcal{C} \ra \mathcal{C}[\check{\mathcal{E}}]$ by $A\mapsto \underline{1_A}$ and $f:A\ra B$ is mapped to $\tilde{f}$, which is a morphism in the target category since $1_Bf = f1_A = f$. This is faithful since if $\tilde{f} = \tilde{g}$ then $df = fe = f = g = ge = dg$, and so $I$ must be full since every morphism in $\mathcal{C}[\check{\mathcal{E}}]$ comes from a morphism in $\mathcal{C}$.


$\implies$ Suppose we can factor $T = \widehat{T}I$ for some $\widehat{T}$. Given $A\in \C$, and idempotent $e:A\ra A$, consider the objects $\underline{1_A},\underline{e} \in \mathcal{C}[\check{\mathcal{E}}]$. There are morphisms $e_1 : \underline{1_A}\ra \underline{e}$ and $e_2 : \underline{e}\ra \underline{1_A}$, both of which are really just the morphism $e$ but have bit written with subscripts to highlight that they are not the same morphism in $\mathcal{C}[\check{\mathcal{E}}]$, since they do not agree on domain/codomain.

The composite $e_1e_2$ is then the morphism $e:\underline{e}\ra\underline{e}$, which is the identity $1_{\underline{e}}$, and the composite $e_2e_1$ is the morphism $\tilde{e}:\underline{1_A}\ra\underline{1_A} = Ie$. So $Ie$ is a split idempotent in $\mathcal{C}[\check{\mathcal{E}}]$, and hence so is $Te = \widehat{T}Ie$ in $\mathcal{D}$ since $\widehat{T}$ preserves identities.

$\impliedby$ Suppose we have $\tilde{f}:\underline{e}\ra\underline{d}$ in $\mathcal{C}[\check{\mathcal{E}}]$. Say $Te = g_2g_1$, where $g_1:TA\ra C$ and $Td = h_2h_1$, where $h_1:TB \ra D$ split as described.

We then define $\widehat{T}\underline{e} = C$, and similarly $\widehat{T}\underline{d} = D$, and so on for all objects. We then define $\widehat{T}\tilde{f} = h_1Tfg_2$. This preserves identities since if $f = 1_{\underline{e}} = e$, we have $\underline{e} = \underline{d}$ so $h_1 = g_1$ and $Tf = Te = g_2g_1$, so all in all $\widehat{T}\tilde{f} = g_1(g_2g_1)g_2 = (g_1g_2)(g_1g_2) = 1_{C}$.

Moreover $\widehat{T}$ is functorial since (extending the names of objects/morphisms canonincally), $\widehat{T}\tilde{g}\widehat{T}\tilde{f} = (i_1Tgh_2)(h_1Tfg_2) = i_1Tg(h_2h_1)Tfg_2 = i_1 TgTdTfg_2 = i_1T(gf)Teg_2 = i_1T(gf)g_2g_1g_2 = i_1T(gf)g_2 = \widehat{T}\tilde{gf}$.

(iii) The clear object function of our functor will be $T\mapsto \widehat{T}$. Now suppose we have $\alpha:T\ra S$ a natural transformation. We define its image $\widehat{\alpha}$ by its components, given by $\widehat{\alpha}_{\underline{e}}: \widehat{T}\underline{e}\ra\widehat{S}\underline{e}$ by $\widehat{\alpha}_{\underline{e}} = \widehat{S}e_1\alpha_A\widehat{T}e_2$.

We then have naturality: $\widehat{S}f\widehat{S}e_1\alpha_A\widehat{T}e_2 = \hat{S}d_1Sf\hat{S}e_2\widehat{S}e_1\alpha_A\widehat{T}e_2 = \widehat{S}d_1 Sf Se\alpha_A\widehat{T}e_2 = \widehat{S}d_1Sf\alpha_A\widehat{T}e_2 = \widehat{S}d_1\alpha_BTf\widehat{T}e_2 = \widehat{S}d_1\alpha_B \widehat{T}d_2\widehat{T}f = \widehat{\alpha}_{\underline{d}}\widehat{T}f$, as required.

Functoriality also follows: $\widehat{\beta}_{\underline{e}}\widehat{\alpha}_{\underline{e}} = \widehat{R}e_1\beta_A\widehat{S}e_2\widehat{S}e_1\alpha_A\widehat{T}e_2 = \widehat{R}e_1\beta_ASe\alpha_A\widehat{T}e_2 = \widehat{R}e_1\beta_A\alpha_ATe\widehat{T}e_2 = \widehat{R} e_1\beta_A\alpha_A \widehat{T}(e_2e_1e_2) = \widehat{R}e_1\beta_A\alpha_A\widehat{T}e_2 = \widehat{\beta\alpha}_{\underline{e}}$.

So we have defined this `hat' functor between $[\mathcal{C},\mathcal{D}]$ and $[\widehat{\mathcal{C}},\mathcal{D}]$. We need to show it is part of an equivalence.

Essential surjectivity: suppose we have $S\in [\widehat{C},D]$. Then $SI$ is a functor $\mathcal{C}\ra\mathcal{D}$, so we might expect it to be a good candidate to have $\widehat{SI}$ isomorphic to $S$. Note that they aren't necessarily equal, since the functor $SI$ might factorise in different ways, and not in exactly the same way as the definition above.

However, the choice in factorisation arises due to the fact that there may be multiple different ways in which an idempotent $SIe$ can split in $\mathcal{D}$. We must have $S1_A = \widehat{SI}1_A = SIA$, and then we find an isomorphism $Se_1\widehat{SI}e_2 : \widehat{SI}\underline{e}\ra S\underline{e}$ - this has inverse $\widehat{SI}e_1Se_2$.

We use these isomorphisms to define a natural transformation, and hence natural isomorphism, between $S$ and $\widehat{SI}$. This satisfies the naturality properties $SfSe_1\widehat{SI}e_2 =Sd_1Sd_2SfSe_1\widehat{SI}e_2 = Sd_1S(d_2fe_1)\widehat{SI}e_2 = Sd_1\widehat{SI}(d_2fe_1)\widehat{SI}e_2 = Sd_1\widehat{SI}d_2 \widehat{SI}f$. So $\widehat{\cdot}$ is essentially surjective.

We are actually done now; we have functors $F:[\mathcal{C},\mathcal{D}] \ra [\widehat{\mathcal{C}},\mathcal{D}]$ and $G$ in the other direction given by $FT = \widehat{T}$, $F\alpha = \widehat{\alpha}$, $GX = XI$, and for $\widehat{\alpha}:X\ra Y$ let $G\widehat{\alpha} = \alpha$ be the natural transformation given by $\alpha_A = \widehat{\alpha}_{\underline{1_A}}$.
\end{document}
