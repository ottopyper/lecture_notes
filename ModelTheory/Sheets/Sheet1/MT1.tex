\documentclass[]{article}


\usepackage{amsmath}
\usepackage{amssymb}
\usepackage{amsthm}
\usepackage{graphicx}
\usepackage{parskip}
\usepackage{xcolor}
\usepackage{pagecolor}
\usepackage[margin=1.2in]{geometry}
\usepackage{enumerate}


\usepackage[utf8]{inputenc}
\usepackage[english]{babel}

\usepackage{mathtools}
\DeclarePairedDelimiter\bra{\langle}{\rvert}
\DeclarePairedDelimiter\ket{\lvert}{\rangle}
\DeclarePairedDelimiterX\braket[2]{\langle}{\rangle}{#1 \delimsize\vert #2}

\definecolor{thmcolour}{rgb}{0,0,0}
\definecolor{defcolour}{rgb}{0,0,0}
\definecolor{textcolour}{rgb}{0,0,0}
\definecolor{backgroundcolour}{rgb}{1,1,1}

\pagecolor{backgroundcolour}
\color{textcolour}

\newtheoremstyle{custhm}
{%space above
1em
}{%space below
1em
}{%body font
\color{thmcolour}
}{%indent amount
-0em
}{%head font
\bfseries\color{thmcolour}
}{%head punct
}{%after head space
1em
}{%head spec
\thmname{#1}
\if\relax\detokenize{#2}\relax:
\else\thmnumber{ #2}:\fi
\if\relax\detokenize{#3}\relax
\else\thmnote{ (#3)}\fi
}

\newtheoremstyle{remark}
{%space above
}{%space below
}{% body font
}{%indent amount
-0em
}{%head font
\bfseries
}{%head punct
}{%after head space
0em
}{%head spec
\if\relax\detokenize{#3}\relax \thmname{#1}:
\else \thmname{#3}:
\fi
}

\newtheoremstyle{cusdef}
{%space above
1em
}{%space below
1em
}{%body font
\color{defcolour}
}{%indent amount
-0em
}{%head font
\bfseries\color{defcolour}
}{%head punct
}{%after head space
1em
}{%head spec

%if numbered, include number
%if named, include name

\thmname{#1}
\if\relax\detokenize{#2}\relax:
\else\thmnumber{ #2}:\fi
\if\relax\detokenize{#3}\relax
\else\thmnote{ (#3)}\fi
}

\theoremstyle{custhm}
\newtheorem{theorem}{Theorem}[section]
\theoremstyle{cusdef}
\newtheorem{defin}[theorem]{Definition}
\theoremstyle{custhm}
\newtheorem{lemma}[theorem]{Lemma}
\theoremstyle{custhm}
\newtheorem{cor}[theorem]{Corollary}

\theoremstyle{custhm}
\newtheorem{prop}[theorem]{Proposition}

\theoremstyle{custhm}
\newtheorem*{theorem*}{Theorem}

\theoremstyle{cusdef}
\newtheorem*{defin*}{Definition}

\theoremstyle{remark}
\newtheorem*{remark*}{Remark}


%\marginpar{to describe which lecture it is}

\newcommand{\N}{\mathcal{N}}
\newcommand{\Z}{\mathbb{Z}}
\newcommand{\Q}{\mathbb{Q}}
\newcommand{\R}{\mathbb{R}}
\newcommand{\C}{\mathbb{C}}
\newcommand{\e}{\mathrm{e}}
\newcommand{\ra}{\rightarrow}
\newcommand{\lef}{\left(}
\newcommand{\res}{\right)}
\newcommand{\ie}{\textit{i.e.}}
\newcommand{\eps}{\varepsilon}
\newcommand{\E}{\mathbb{E}}
\newcommand{\suminf}{\sum_{n=0}^{\infty}}
\newcommand{\suminfa}[1]{\sum_{#1=0}^{\infty}}
\renewcommand{\P}{\mathbb{P}}
\newcommand{\undf}[1]{\textit{\textbf{#1}}}
\renewcommand{\L}{\mathcal{L}}
\renewcommand{\it}[1]{\textit{#1}}
\newcommand{\M}{\mathcal{M}}
\renewcommand{\phi}{\varphi}
\newcommand{\proves}{\vdash}
\newcommand{\lra}{\leftrightarrow}
\renewcommand{\value}{|\cdot|}
\newcommand{\val}[1]{\left|#1\right|}
\newcommand{\valk}{(K,|\cdot|)}
\renewcommand{\bar}{\overline}
\renewcommand{\O}{\mathcal{O}}
\newcommand{\A}{\mathcal{A}}

\renewcommand{\lnot}{\neg}
\newcommand{\false}{\bot}
\newcommand{\true}{\top}

\title{Model Theory: Sheet 1}
\author{Otto Pyper}
\date{}

\begin{document}

\maketitle
\clearpage

\textbf{1}. Let $\phi_n$ be the $\L$-sentence
\[
\exists x_1\dots\exists x_n \bigwedge_{i\ne j}\neg(x_i=x_j)
\]
Then let $T' = T\cup\{\phi_n:n\in\N\}$. $T$ has arbitrarily large finite models, so $T'$ is finitely satisfiable. Hence $T'$ has a model, which must both be infinite and a model of $T$.\qed

\textbf{2}. (ii) is equivalent to the claim that either $T\models \phi$ or $T\models\neg\phi$, which is the definition of completeness as in (i).

(i)$\implies$(iii): Let $\M,\mathcal{N}$ be models of $T$, and $\phi$ and $\L$-sentence. By definition, $T\models\phi$ implies $\M\models\phi$ and $\N\models \phi$. Otherwise, $T\models\neg\phi$ and so do $\M,\N$, so $\M\equiv\N$.

(iii)$\implies$(i): Suppose $T$ not complete. Then there exists a sentence $\phi$ such that both $T\cup\{\phi\}$ and $T\cup\{\neg\phi\}$ are consistent, and thus have models. But any two models are elementarily equivalent, and no model can satisfy both $\phi$ and $\neg\phi$.\qed

\textbf{3}. Let $\M,\N$ be two infinite sets of cardinality $\kappa$. Then there exist bijections $f:M\ra\kappa$, $g:N\ra\kappa$, so we define $h:M\ra N$ by $h = g^{-1}\circ f$. Owing to the lack of function symbols, relation symbols or constant symbols in $\L$, h is in fact a homomorphism $h:\M\ra\N$, and since it is bijective it is also an isomorphism. So $\M\equiv \N$, and the theory of infinite sets is $\kappa$-categorical for any infinite $\kappa$.

The theory of infinite sets has no finite models, so by Vaught's Test it is complete.\qed

\textbf{4}. (note: heavily AC dependent) We have $\L$ the language of groups: $\L = \{+,0\}$ (chosen thusly as our groups are going to be abelian).

Let $\phi_n$ denote the $\L$-sentence $\forall x (x=0\lor nx\neq 0)$, and $\psi_m$ the sentence $\forall x\exists y(x=my)$. Our theory of nontrivial torsion-free divisible abelian groups is then
\begin{align*}
T &= \{\forall x\exists y (x+y = y+x = 0)\}\\
&\cup\{\forall x(x+0=0+x=x)\}\\
&\cup\{\forall x\forall y\forall z ((x+y)+z = x + (y+z))\}\\
&\cup\{ \forall x\forall y (x+y = y+x) \}\\
&\cup\{\phi_n:n\in\mathbb{N}\}\\
&\cup\{\psi_n:n\in\mathbb{N}\}
\end{align*}

This theory is complete because it is $\kappa$-categorical for all $\kappa \ge \aleph_1$; we need only use $\kappa = \aleph_1$.

First, note that the divisibility property is unique, \it{i.e.} for $x\in G$ and $n > 0$, there is a unique $y\in G$ such that $x = ny$. Indeed, if $ny = nz$, then $n(y-z) = 0$ and since $G$ is torsion-free we must have $y-z = 0$, so $y = z$. In particular, given any $x\in G$ and $n > 0$ we can express the unique such element as $x/n$.

We can then add elements in the natural way: $x/n + y/n = (x+y)/n$, $m(x/n) = (mx)/n$, and so $x/n+y/m = (mx+ny)/(mn)$.

Denote by $\langle X\rangle$ the torsion-free divisible abelian group generated by $X$, \it{i.e.}
\begin{align*}
\langle X\rangle = \left\lbrace \frac{a_1x_1+\dots+a_kx_k}{n}: a_i\in\Z,k,n\in \mathbb{N},\ n>0,x_i\in X\right\rbrace
\end{align*}
Let $G$ be a model for $T$ as above, of cardinality $\kappa$ - this exists by Lowenheim-Skolem, noting that $T$ is satisfiable since $(\Q,+)$ is a model.

Let $(P,<)$ be the partial order defined on $\mathcal{P}G$ by $X \le Y$ iff $\langle X\rangle \le \langle Y \rangle$ as groups.

Let $\mathcal{C} = \{C_i:i\in I\}$ be a chain in $P$. Claim that $C = \bigcup_{i\in I}C_i$ is an upper bound for $\mathcal{C}$. Indeed, it is clear that $\langle C_i\rangle \le \langle C\rangle$ for each $i\in I$, so this result is immediate.

So by Zorn's Lemma, $P$ has a maximal element $S$. Then we must have $\langle S\rangle = G$, else we have some $x\in G\backslash \langle S\rangle$, so $\langle S\cup\{x\}\rangle$ is a strictly larger subgroup.

We then need to use AC again to obtain an independent $S'\subset S$ with $\langle S' \rangle = \langle S\rangle$, \it{i.e.} for all $x\in S'$, $x\not\in \langle S'\backslash\{x\}\rangle$. Informally, this amounts to going through $S$ and removing unnecessary elements until all remaining elements are necessary. We lose nothing each time we throw away an unnecessary element.

Finally, and this was the whole point, we must have $|S'| = \kappa$. If $S'$ were countable, then so would be $G$, as we would be able to identify $\langle S'\rangle$ with finite rational sequences, the set of which is countable. We of course cannot have $|S'| > \kappa$ either.

Thus for any $G$ with $|G| = \kappa$, we have $G = \langle S_G \rangle$ for some independent $S_G \subset G$ with $|S_G| = \kappa$. So for any two $G,H$, any bijection $f:S_G\ra S_H$ induces a group isomorphism $\bar{f}:G\ra H$ in the natural way (\it{e.g.} if $S_G = \{x_i: i<\kappa\}$ and $S_H = \{y_i: i<\kappa\}$, we have a group isomorphism $G\ra H$ by `change any $x$ to a $y$'). Thus $G\cong H$.

So $T$ is $\kappa$-categorical for $\kappa = \aleph_1$, and indeed for any $\kappa \ge \aleph_1$ by a very similar proof (a minor tweak about cardinality is needed at the end).

Hence $T$ is complete.\qed

\textbf{5}. The phrasing of the question is highly suggestive; we want to reframe subsets of $V$ as theories, and $k$-colourings as models.

Let $\L = \{e,c\}$ where $e,c$ are both binary relation symbols, and let $T$ assert that $e$ is an edge relation and $c$ an equivalence relation: $\forall x \neg e(x,x)$, $\forall x \forall y (e(x,y)\iff e(y,x))$, $c$ the usual equivalence relation axioms (symmetry, reflexivity, transitivity).

Also include axioms that assert $c$ is a colouring, \it{i.e.} $\forall x\forall y (e(x,y)\implies \neg c(x,y))$.

Moreover, we require that $c$ is a $k$-colouring: $\forall x_1\dots\forall x_{k+1}\bigvee_{i\ne j}c(x_i,x_j)$ - \it{i.e.} there are at most $k$ equivalence classes of $c$.

We denote this whole theory as $T$. Its models are precisely $k$-colourable graphs equipped with a $k$-colouring.

Now onto the specific case of a graph $(V,E)$. Let its vertex set be written $V = \{v_i: i\in I\}$. We augment our language to $\L^\ast = \L \cup \{v_i: i\in I\}$, and augment our theory to $T^\ast = T\cup \{ v_i\ne v_j :i\ne j \}\cup\{e(v_i,v_j):v_iv_j\in E\}$.

By assumption, every finite subset of $T^\ast$ is satisfied by some finite subgraph of $(V,E)$, so by the compactness theorem $T^\ast$ has a model $(V',E',C')$ with $V\subset V'$, $E\subset E'$ and $C'$ a $k$-colouring of $(V',E')$. Thus we may exhibit a $k$-colouring of $(V,E)$ by simply restricting $C'$. Hence $(V,E)$ is $k$-colourable.\qed

\textbf{6}. $\implies$ was done in lectures.

For $\impliedby$, we need to show that $h:\M\xhookrightarrow{}\N$ is an $\L$-embedding.

Firstly, let $\phi(x_1,\dots,x_n)$ be $R(x_1,\dots,x_n)$ for $R$ an $n$-ary relation symbol in $\L$. Then for any $a_i\in M$, $\M\models \phi(a_1,\dots,a_n)$ states $(a_1,\dots,a_n)\in R^\M$, so by assumption we have $(a_1,\dots,a_n)\in R^M\iff (a_1,\dots,a_n)\in R^\M$, so $h$ preserves relations.

Next, let $c$ be a constant symbol in $\L$, and let $\phi(x)$ be the $L$-formula $(x=c)$. We then have $\M\models \phi(c^\M)$, and so $\N\models\phi(c^\M)$, hence $c^\M = c^\N$.

For a general $n$-ary function symbol $f\in \L$, we may adopt a similar approach. Let $\phi(x_1,\dots,x_n,y)$ be the statement $f(x_1,\dots,x_n) = y$. Then we have, for $a_1,\dots,a_n,b\in M$:
\begin{align*}
f^\M(a_1,\dots,a_n) = b &\iff \M \models \phi(a_1,\dots,a_n,b)\\
&\iff \N \models \phi(a_1,\dots,a_n,b)\\
& \iff f^\N(a_1,\dots,a_n) = b
\end{align*}
From which we conclude $f^\M(a_1,\dots,a_n) = f^\N(a_1,\dots,a_n)$, hence $h$ is indeed an embedding.\qed

\textbf{7}. We probably haven't covered this exactly yet, but I'm assuming the substructure \it{generated by} $A$ is the smallest substructure $\M \subset \N$ such that $A\subset M$ - the notion of \it{smallest} is well defined since if $\M,\M'$ are substructures then so is $\M\cap\M'$.

In particular, we need only show that there exists some substructure of $\N$ satisfying the given property.

The outline of the argument is to find a set $A^\ast \subset N$ closed under all function symbols in $\L$, with relations induced by $\N$. Then this is a substructure containing $A$.

Define $A_0 = A\cup\{c^\N:c\in\L\textrm{ is constant}\}$. Then given $A_n$, define $A_{n+1} = A_n \cup \{ f^\N(a_1\dots,a_{\alpha(f)}): f\in \L \textrm{ a function symbol, }a_i\in A_n \}$.

We then define $A^\ast = \bigcup_{n=0}^{\infty}A_n$. Then $A^\ast$ is closed under all functions $f^\N$, since for $a_1,\dots,a_k\in A^\ast$ we must have $a_1,\dots,a_k\in A_m$ for some $m$, hence $f^\N(a_1,\dots,a_m)\in A_{m+1}\subset A^\ast$. Hence restricting $R^\N$ to $A^{\alpha(R)}$ gives an $\L$-structure $\mathcal{B}$ with $A\subset B$.

We now show that $\mathcal{B}$ satisfies the desired property; we do this by induction. For the base case, consider $b\in A_0$. Then either $b$ is a constant (so done), or $b = x^\N(a_i)=x^\mathcal{B}(a_i)$ for some $a_i\in A$, $x$ a variable.

Now suppose $b\in A_{n}$. Then $b = f^\N(x_1,\dots,x_k) = f^\mathcal{B}(x_1,\dots,x_k)$, with $x_i\in A_{n-1}$. Hence by induction, $b = f^\mathcal{B}(t_1^\mathcal{B}(a_1^1,a_2^1,\dots,a_{n_1}^1),\dots) = t^\mathcal{B}(a_1,\dots,a_m)$ for an $\L$-term $t$ and some $a_i\in A$.

Then since $A\subset B$, this result also holds for $\langle A\rangle \subset \mathcal{B}$ (these structures should be the same, but that isn't technically important).\qed

\textbf{8}. (a) The universe of $\N$ is given by $N = \bigcup_{i < \alpha} M_i$. For each function symbol $f\in \L$, we interpret $f^\N = \bigcup_{i < \alpha}f^{\M_i}$ (as set-theoretic functions), and similarly $R^{\N} = \bigcup_{i < \alpha}R^{\M_i}$. Equivalently, for $x_j \in N$ we can define $f^{\N}(x_1,\dots,x_n) = f^{\M_i}(x_1,\dots,x_n)$ for some sufficiently large $i$; this is well-defined since such an $i$ must exist, and all such $\M_i$ agree on the value $f^{\M_i}(x_1,\dots,x_n)$. Similarly for the relation symbols $R$.

So $\N$ is a well-defined $\L$-structure. From the above it is also clear that $\M_i\subset \N$, by definition of $\M_i$; $f^{\M_i}(x_1,\dots,x_n)\eqqcolon f^\N(x_1,\dots,x_n)$, and similarly for the relation symbols.

(b) Since $\M_i\subset \N$, we have that $\M_i$ and $\N$ agree on all quantifier-free formulae as in Q6. So we need only show this holds for quantified formulae.

Suppose that for all $i$ and $a_1,\dots,a_n\in \M_i$, $\M_i$ and $\N$ agree on $\phi(a_1,\dots,a_n)$. Consider the formula $\psi$ given by $\forall x_n \phi(x_1,\dots,x_n)$, and suppose that $\M_i \models \psi(a_1,\dots,a_{n-1})$. Then for every $b\in M_i$, we have $\M_i\models \phi(a_1,\dots,a_{n-1},b)$. Let $d\in \N$. Then $d\in M_j$ for some $j$, and since $\M_i\preceq \M_j$ we have $\M_j \models \phi(a_1,\dots,a_{n-1},d)$. By induction on $\phi$, $\M_j\models \phi(a_1,\dots,a_{n-1},d) \iff \N \models \phi(a_1,\dots,a_{n-1},d)$. Hence for all $d\in \N$, $\N\models \phi(a_1,\dots,a_{n-1},d)$. Therefore $\N\models \psi(a_1,\dots,a_{n-1})$.

Thus $\M_i \preceq \N$ for all $i < \alpha$.\qed

\textbf{9}. Here is the plan:
\begin{enumerate}
	\item define a theory $T$
	\item augment language to $\L^\ast$
	\item augment theory
	\item use DLS to get model
	\item find an elementary substructure
	\item show it's isomorphic to a substructure of $\M$ that contains $A$
\end{enumerate}

We take $T =$ Th$(\M)$, and let $\mathcal{B}$ be the substructure generated by $A$. We then augment our language to $\L^\ast = \L\cup B$, and augment our theory to $T^\ast$ by adding in:
\begin{itemize}
	\item $\{t(a_1,\dots,a_n)=t^\M(a_1,\dots,a_n): t \textrm{ an }\L\textrm{-term}, a_i\in B\}$
	\item $\{R(a_1,\dots,a_m):R^\M(a_1,\dots,a_m),a_i\in B\}$
	\item $\{b\ne c:b\ne c\in B\}$
\end{itemize}

So any model for $T^\ast$ is a model for Th$(\M)$, and contains a substructure isomorphic to $\mathcal{B}$. $T^\ast$ is clearly satisfied by $\M$, with interpretations $a^\M = a$ for $a\in B$.

Hence by DLS there exists a model $\N^\ast$ for $T^\ast$ with $|N| \le |\L^\ast| + \aleph_0$. We have $|\L^\ast| = |\L| + |B|$, and $|B| \le \aleph_0\cdot |A|$ (countable union of $A$-sized sets). Hence $|N^\ast| \le |\L| + \aleph_0 + |A| + \aleph_0 = |A| + |\L| + \aleph_0$.

We are nearly done, excluding the fact that the universe of $\N^\ast$ might look nothing like that of $\M$ in real terms; so we need to embed $\N^\ast$ into $\M$. We do this using a chain of embeddings $(f_i)_{i=0}^{\infty}$, defined as follows.

Define $A^\ast = \{ a^{\N^\ast} : a\in A \}$, and let $X_0$ be the substructure generated by $A^\ast$. $f_0:\chi_0\ra\M$ is then the embedding induced by the map $a^{\N^\ast}\mapsto a$. $f_0$ is in fact an isomorphism between $\chi_0$ and $\mathcal{B}$ as $\chi_0$ obeys the same rules as $\mathcal{B}$ since they were hard-coded into the theory.

Then suppose we have defined embeddings $f_0\subset f_1\subset\cdots\subset f_n$, each $f_i : \chi_i \xhookrightarrow{} \M$ with $A\subset X_i$.

\textbf{9}. Another go. We will start with the smallest substructure $\mathcal{B}$ with $A\subset B$, and extend it to an elementary substructure without adding in too much stuff.

First note that $|B| \le |A| + \aleph_0$ as seen above (choice). Moreover, $\mathcal{B}\subset \M$, so $\mathcal{B}$ and $\M$ agree on quantifier-free formulae.

Observe that if $\psi$ is the sentence $\forall x_1\dots\forall x_n \phi(x_1,\dots,x_n)$ with $\phi$ quantifier-free, then $\M\models \psi$ iff for all $a_1,\dots,a_n\in M$, $\M\models\phi(a_1,\dots,a_n)$, hence $\mathcal{B}\models \phi(b_1,\dots,b_n)$ for all $b_i\in B$, and so $\mathcal{B}\models \psi$ also. 

We will construct a sequence of structures $(\chi_n)_{n<\alpha}$ for $\alpha = |\L| + \aleph_0$ the cardinality of the set of $\L$-formulae (each formula is derived in finitely many steps, each set of formulae derivable in $n$ steps has cardinality $\le |\L| + \aleph_0$, hence so does their countable union).

$\chi_0 = \mathcal{B}$.

Partially order the $\L$-formulae by $\phi \le \psi$ iff $\phi$ is a subformula of $\psi$. This partial order is well-founded, since each formula has only finitely many subformulae. Hence we may extend this partial order to a well order, and write the set as $\mathcal{P} = \{\phi_i: i<\kappa\}$

The aim will be to have $\chi_\beta$ agree with $\M$ on all formulae $\chi_\delta: \delta < \beta$. Then $\chi_\kappa \equiv \M$.

Suppose that we have $\chi_\beta$ as above, and that $\beta = \delta^+$. We must have $\phi_\beta = \neg\phi_\alpha,\phi_\alpha\land\phi_\gamma$, or $\forall x\phi_\alpha$ for some $\alpha,\gamma < \beta$. In the first two cases it is clear that agreement on $\phi_\alpha$, $\phi_\gamma$ induces agreement on $\phi_\beta$, so if $\phi_\beta$ is of that form then we define $\chi_{\beta^+} = \chi_\beta$.

If instead $\phi_\beta$ is $\forall x \phi_\alpha$, then we observe that the only way in which $\M$ and $\chi_\beta$ can disagree is if there exist some $x_1,\dots,x_n\in M$, $\M\models \neg \phi_\beta(x_1,\dots,x_n)$ but for every $y_1,\dots,y_n\in  X_\beta$, $\chi_\beta \models \phi_\beta(y_1,\dots,y_n)$, else we are quickly done by the induction hypothesis.

Hence there exists $m\in M $ such that $\M\models\neg\phi_\alpha(x_1,\dots,x_n,m)$, but for every $y_1,\dots,y_n,z \in X_\beta$ we have $\chi_\beta\models \phi_\alpha(y_1,\dots,y_n,z)$. We then define $\chi_{\beta^+} $ to be the substructure generated by $\chi_\beta$ and $m$, and we hope that $\chi_{\beta+}\models \neg \phi_\beta(m)$.

\textbf{9}. Third time's the charm.

We'll ignore the subset $A$ for now, and will just aim to show that if we have an $\L$-structure $\M$, $\M$ infinite, then we can find a substructure $\N \preceq \M$ with $|N|\le |\L| +\aleph_0$.

The plan will be to find a chain of substructures $(\A_n)_{n=0}^{\infty}$ where each $\A_i$ satisfies more and more of the sentences of Th$(\M)$, such that their union satisfies them all.

We start with $\A_0$ generated by the constant terms, {\it i.e.} the interpretations of all the closed terms of $L_0 = \L$. $\A_0$ then satisfies any statement in $T$ of the form $\forall \overrightarrow{x} \phi(\overrightarrow{x})$, with $\phi$ quantifier-free - this is since it is a substructure, so it agrees with $\M$ on all quantifier free-formulae.

This is a decent start. The next problem is that formulae like $\exists x\phi$ in $T$ don't necessarily have witnesses in $\A_0$. So for every formula $\exists y \forall \overrightarrow{x} \phi(\overrightarrow{x},y)$ with $\phi$ quantifier-free, we find a witness $m$ and add it to $\A_0$ to obtain a new generated substructure $\A_1$.
\end{document}
