\documentclass[]{article}


\usepackage{amsmath}
\usepackage{amssymb}
\usepackage{amsthm}
\usepackage{graphicx}
\usepackage{parskip}
\usepackage{xcolor}
\usepackage{pagecolor}
\usepackage[margin=1.2in]{geometry}
\usepackage{enumerate}
\usepackage{enumitem}
\usepackage{tikz}
\newcommand*\circled[1]{%
   \tikz[baseline=(C.base)]\node[draw,circle,inner sep=1.2pt,line width=0.2mm,](C) {#1};}
\newcommand*\Myitem{%
   \stepcounter{enumi}\item[\circled{\theenumi}]}

\usepackage[utf8]{inputenc}
\usepackage[english]{babel}

\usepackage{mathtools}
\DeclarePairedDelimiter\bra{\langle}{\rvert}
\DeclarePairedDelimiter\ket{\lvert}{\rangle}
\DeclarePairedDelimiterX\braket[2]{\langle}{\rangle}{#1 \delimsize\vert #2}

\definecolor{thmcolour}{rgb}{0,0,0}
\definecolor{defcolour}{rgb}{0,0,0}
\definecolor{textcolour}{rgb}{0,0,0}
\definecolor{backgroundcolour}{rgb}{1,1,1}

\pagecolor{backgroundcolour}
\color{textcolour}

\newtheoremstyle{custhm}
{%space above
1em
}{%space below
1em
}{%body font
\color{thmcolour}\em
}{%indent amount
-0em
}{%head font
\bfseries\color{thmcolour}
}{%head punct
}{%after head space
1em
}{%head spec
\thmname{#1}\if\relax\detokenize{#2}\relax:\else\thmnumber{ #2}:\fi\if\relax\detokenize{#3}\relax\else\thmnote{ (#3)}\fi
}

\newtheoremstyle{ex}
{%space above
1em
}{%space below
1em
}{%body font
\color{thmcolour}
}{%indent amount
-0em
}{%head font
\bfseries\color{thmcolour}
}{%head punct
}{%after head space
1em
}{%head spec
\thmname{#1}\if\relax\detokenize{#2}\relax:\else\thmnumber{ #2}:\fi\if\relax\detokenize{#3}\relax\else\thmnote{(#3)}\fi
}

\newtheoremstyle{remark}
{%space above
}{%space below
}{% body font
}{%indent amount
-0em
}{%head font
\bfseries
}{%head punct
}{%after head space
1em
}{%head spec
\if\relax\detokenize{#3}\relax\thmname{#1}:\else\thmname{#3}:\fi
}

\newtheoremstyle{numremark}
{%space above
}{%space below
}{% body font
}{%indent amount
-0em
}{%head font
\bfseries
}{%head punct
}{%after head space
1em
}{%head spec
\thmname{#1}\thmnumber{ #2}:
}

\newtheoremstyle{cusdef}
{%space above
1em
}{%space below
1em
}{%body font
\color{defcolour}
}{%indent amount
-0em
}{%head font
\bfseries\color{defcolour}
}{%head punct
}{%after head space
1em
}{%head spec
%if numbered, include number
%if named, include name
\thmname{#1}\if\relax\detokenize{#2}\relax:\else\thmnumber{ #2}:\fi\if\relax\detokenize{#3}\relax\else\thmnote{ (#3)}\fi
}

\theoremstyle{custhm}
\newtheorem{theorem}{Theorem}[section]
\theoremstyle{cusdef}
\newtheorem{defin}[theorem]{Definition}
\theoremstyle{custhm}
\newtheorem{lemma}[theorem]{Lemma}
\theoremstyle{custhm}
\newtheorem{cor}[theorem]{Corollary}

\theoremstyle{custhm}
\newtheorem{prop}[theorem]{Proposition}

\theoremstyle{ex}
\newtheorem{ex}[theorem]{Example}

\theoremstyle{custhm}
\newtheorem*{theorem*}{Theorem}

\theoremstyle{cusdef}
\newtheorem*{defin*}{Definition}

\theoremstyle{remark}
\newtheorem*{remark*}{Remark}

\theoremstyle{remark}
\newtheorem{remark}[theorem]{Remark}

\theoremstyle{numremark}
\newtheorem{numremark}[theorem]{Remark}

\setcounter{section}{-1}

%\marginpar{to describe which lecture it is}

\newcommand{\Na}{\mathbb{N}}
\newcommand{\Z}{\mathbb{Z}}
\newcommand{\Q}{\mathbb{Q}}
\newcommand{\R}{\mathbb{R}}
\newcommand{\C}{\mathbb{C}}
\newcommand{\e}{\mathrm{e}}
\newcommand{\ra}{\rightarrow}
\newcommand{\lef}{\left(}
\newcommand{\res}{\right)}
\newcommand{\ie}{\textit{i.e.}}
\newcommand{\eps}{\varepsilon}
\newcommand{\E}{\mathbb{E}}
\newcommand{\suminf}{\sum_{n=0}^{\infty}}
\newcommand{\suminfa}[1]{\sum_{#1=0}^{\infty}}
\renewcommand{\P}{\mathbb{P}}
\newcommand{\undf}[1]{\textit{\textbf{#1}}}
\renewcommand{\L}{\mathcal{L}}
\renewcommand{\it}[1]{\textit{#1}}
\newcommand{\M}{\mathcal{M}}
\renewcommand{\phi}{\varphi}
\newcommand{\proves}{\vdash}
\newcommand{\lra}{\leftrightarrow}
\renewcommand{\value}{|\cdot|}
\newcommand{\val}[1]{\left|#1\right|}
\newcommand{\valk}{(K,|\cdot|)}
\renewcommand{\bar}{\overline}
\renewcommand{\O}{\mathcal{O}}
\newcommand{\Th}{\textrm{Th}}
\newcommand{\tp}{\textrm{tp}}

\renewcommand{\lnot}{\neg}
\newcommand{\false}{\bot}
\newcommand{\true}{\top}
\newcommand{\n}{\mathcal{N}}
\newcommand{\N}{\mathcal{N}}
\newcommand{\ac}[1]{\mathcal{#1}}
\newcommand{\acf}{\textrm{ACF}}
\newcommand{\F}{\mathbb{F}}
\newcommand{\A}{\mathcal{A}}
\newcommand{\rg}{\textrm{RG}}
\newcommand{\D}{\mathcal{D}}
\newcommand{\sman}{S_n^\M(A)}
\newcommand{\gp}{\textrm{gp}}

\renewcommand{\subset}{\subseteq}

\title{Model Theory}
\author{Lectures by Gabriel Conant}
\date{}

\begin{document}
	
	\maketitle
	\clearpage
	\tableofcontents
	\clearpage

\section{Review of First Order Logic}

A \undf{language} is a set $\L$ of function symbols, relation symbols, and constant symbols. Additionally, each function/relation symbol has an assigned \it{arity} $n\ge 1$.

By convention, we view constant symbols as `function symbols of arity 0'.

An \undf{$\L$-structure} $\mathcal{M}$ consists of:
\begin{itemize}
	\item a non-empty set $M$ (the \undf{universe} of $\mathcal{M}$)
	\item for every function symbol $f$ of arity $n$, a function $f^{\M}:M^n\ra M$
	\item for every relation symbol $R$ of arity $n$, a subset $R^\M\subseteq M^n$
	\item for every constant symbol $c$, an element $c^\M \in M$ (\ie\ identified with the unique element in its image)
\end{itemize}

\undf{Syntax}: we build formulas using symbols in $\L$ along with $$ \land\ \lnot\ \forall\ =\ (\ )\ ,$$and countably many variable symbols.

\undf{$\L$-term}: these are our way of creating new functions by composing the ones we already have.
\begin{itemize}
	\item constant symbols and variables are terms
	\item if $t_1,\dots,t_n$ are terms and $f$ is an $n$-ary function symbol, then $f(t_1,\dots,t_n)$ is a term
\end{itemize}

Given a structure $\M$ and a term $t$, we are going to interpret the term in the structure in exactly the way you might expect. Inductively, define (for appropriate $r$) $t^\M:M^r\ra M$ as:
\begin{itemize}
	\item constant symbol $c$: $c^\M$ (case $r = 0$)
	\item variable $x$: identify function ($r = 1$)
	\item general term $f(t_1,\dots,t_n):\ f^\M(t_1^\M,\dots,t_n^\M)$
\end{itemize}

\undf{$\L$-formulas}: new relations. We have the following \it{atomic} $L$-formulas:
\begin{itemize}
	\item If $t_1$ and $t_2$ are terms, then $(t_1=t_2)$ is a formula
	\item If $R$ is an $n$-ary relation symbol and $t_1,\dots,t_n$ are terms, then $R(t_1,\dots,t_n)$ is a formula
\end{itemize}
We can then create more complicated formulas. Given formulae $\phi$ and $\psi$:
\begin{itemize}
	\item $\lnot\phi$
	\item $(\phi \land \psi)$
	\item $\forall x\phi$ for any variable $x$
\end{itemize}

An occurrence of a variable $x$ is \undf{free} in $\phi$ if $x$ does not occur in the scope of $\forall x$. Otherwise, the occurrence is \undf{bound}.

For instance, if $\phi$ is the statement $\forall x \lnot(f(x)=y)$, $x$ is bound and $y$ is free. 

\undf{Notation}: Given a formula $\phi$, we write $\phi(x_1,\dots,x_n)$ to denote that $x_1,\dots,x_n$ are the free variables of $\phi$.

Given a formula $\phi(x_1,\dots,x_n)$, a structure $\M$, $a_1,\dots,a_n\in M$, we define ``$\bar{a}$ satisfies $\phi$ in $\M$'', written $\M\models\phi(a_1,\dots,a_n)$, as follows:
\begin{itemize}
	\item If $\phi$ is $(t_1 = t_2)$ then $\M\models \phi(\bar{a})$ iff $t_1^\M(\bar{a}) = t_2^\M(\bar{a})$
	\item If $\phi$ is $R(t_1,\dots,t_n)$ then $\M\models\phi(\bar{a})$ iff $(t_1^\M(\bar{a}),\dots,t_n^\M(\bar{a})\in R^\M$
	\item $\M\models (\phi\land\psi)(\bar{a})$ iff $\M \models \phi(\bar{a})$ and $\M\models \psi(\bar{a})$
	\item $\M \models \lnot \phi(\bar{a})$ iff $\M\not\models \phi(\bar{a})$
	\item Suppose $\phi$ is $\forall w \psi(x_1\dots,x_n,w)$. Then $M\models \phi(\bar{a})$ iff for all $b\in M$, $\M\models \psi(\bar{a},b)$
\end{itemize}

We emphasise that the focus of this course will not be on the precise definitions and semantics, so much as the meaning of what we are doing. All we seek is a first order logic that works for us, so that we can use it to do interesting things.

\undf{Abbreviations}: We have \it{global} abbreviations such as
\begin{itemize}
	\item $(\phi\lor\psi)$ is $\lnot(\lnot\phi\land\lnot\psi)$
	\item $(\phi\ra\psi)$ is $(\lnot\phi\lor\psi)$
	\item $(\phi\lra\psi)$ is $(\phi\ra\psi)\land(\psi\ra\phi)$
	\item $\exists x \phi$ is $\lnot\forall x\lnot \phi$
\end{itemize}

We note that the last equivalence in a semantic sense hinges on the assumption that universes are non-empty. Since we will be almost exclusively be studying infinite structures, we will not worry about this.

We also have \it{local} abbreviations, often specific to the language we are studying. For instance, in $\L = \{+,\cdot,<,0,1\}$ (the language of ordered rings):
\begin{itemize}
	\item $x+y$ is $+(x,y)$
	\item $x < y$ is $<(x,y)$
	\item $x \le y$ is $(x<y) \land (x=y)$
	\item $x < y < z$ is $(x < y)\land (y<z)$
	\item $x^2$ is $x\cdot x$
	\item $nx$ is $\underbrace{x+x+\cdots+x}_{n\textrm{ times}}$
\end{itemize}

An \undf{$\L$-sentence} is an $\L$-formula with no free variables. For instance, $\forall x (f(x)\ne y)$ is \it{not} a sentence, but $\exists y\forall x(f(x)\ne y)$ \it{is} a sentence. Sentences can be thought of as actually saying something meaningful.

If $\phi$ is a sentence and $\M$ is a structure, then we have the notion of $\M\models\phi$, ``$\M$ satisfies $\phi$'' or ``$\M$ models $\phi$''.

\begin{defin*}[$L$-theory]
	An \undf{$L$-theory} is a set of $L$-sentences.

Given a theory $T$, we write $\M\models T$ (``$\M$ is a \undf{model} of $T$) if $\M\models\phi$ for all $\phi \in T$.

T is \undf{satisfiable} if it has a model.
\end{defin*}

\undf{Example}: $T = \{\lnot \exists x(x=x)\}$ - this sentence claims there are no elements in the universe. In our setting, this is unsatisfiable (though it is technically a matter of opinion).

Similarly, $\exists x(x=x)$ (``The Axiom of Non-Triviality'') is always satisfied in any $\L$-structure.

\undf{Recall}: $T$ is \undf{consistent} if it does not prove a contradiction (\it{e.g.} $(\phi\land\lnot\phi))$

A consequence of \undf{G{\"o}del's Completeness Theorem} is that a theory is satisfiable iff it is consistent. This is a very important theorem, though we will mostly be focussing on the model theoretic aspect (satisfiability).

\section{Lecture 2}

We now consider a fixed language $\L$.

An $\L$-theory $T$ is \undf{finitely satisfiable} if every finite subset of $T$ is satisfiable. This leads us to one of the most important theorems for getting Model Theory off the ground:

\begin{theorem*}[Compactness Theorem]
An $\L$-theory $T$ satisfiable iff it is finitely satisfiable
\end{theorem*}

Another important theorem of Model Theory is the following.

\begin{theorem*}[Downward Lowenheim-Skolem Theorem]
	Any satisfiable $\L$-theory has a model of cardinality at most $|\L|+\aleph_0$
\end{theorem*}

The proofs of the above are non-examinable; see Part II notes for details.

\begin{theorem*}[(Upward) Lowenheim-Skolem  Theorem]
Suppose $T$ is an $\L$-theory with infinite models. Then $T$ has a model of cardinality $\kappa$ for any $\kappa \ge |\L|+\aleph_0$
\end{theorem*}

We note that by the `cardinality' of a structure we mean the cardinality of its universe.

\begin{proof}
What we need to do here is build a model of this theory, but do it such that it's not just a model of the theory but that it also has some extra properties of our choosing. This is a common technique in model theory.

We want more elements, so we add more symbols to our language and more sentences claiming various properties about these symbols.

Let $\L^\ast = \L \cup \{c_i : i < \kappa\}$ where each $c_i$ is a new constant symbol.

Then let $T^\ast = T\cup \{c_i\ne c_j: i\ne j\}$. Suppose $\Sigma \subset T^\ast$ is finite. Then $\Sigma \subset T\cup\{c_i\ne c_j:i,j\in I\}$ for some finite set $I$.

Let $\M\models T$ be an infinite $\L$-structure. Expand $\M$ to an $\L^{\ast}$ structure $\M^\ast$ by interpreting $c_i^{\M^\ast}$ as distinct elements for $i\in I$, and interpreting $c_i^{\M^\ast}$ for $i\not\in I$ arbitrarily. Note that this is `physically' the same structure, all we have changed is its interpretation.

Then $M^{\ast}\models\Sigma$, so $T$ is finitely satisfiable. Hence by the Compactness Theorem $T^\ast$ is satisfiable. Then by DLST, $T^\ast$ has a model $\mathcal{N}^\ast$ of cardinality at most $|\L^\ast|+\aleph_0 = \kappa$. Moreover, every model has cardinality \it{at least} $\kappa$, so $\mathcal{N}^\ast$ indeed has cardinality $\kappa$.

Then let $\mathcal{N}$ be the reduct of $\mathcal{N}^\ast$ to $\L$ (same universe, different interpretation). Then $\mathcal{N}\models T$ and $|N| = \kappa$.
\end{proof}

\section*{Complete Theories}

\begin{defin}[Semantic Entailment]
Let $T$ be an $\L$-theory and $\phi$ an $\L$-sentence. Then $T\models \phi$ (`$T$ \undf{models} $\phi$, `$T$ \undf{implies} $\phi$') if any model of $T$ is also a model of $\phi$.
\end{defin}

\begin{ex}\ 
\begin{enumerate}[label=\arabic*)]
\item $\{\phi,\psi\}\models \phi\land\psi$
\item If $T$ is consistent then $T\models \exists x(x=x)$ (also if it's not consistent). So $\emptyset \models \exists x(x=x)$ since we assume all models are non-empty.
\item Let $T$ be the theory of groups in the language of groups $\L = \{\ast,e\}$.

Then $T\models \forall x\forall y\forall z\left((x\ast y = e\land x\ast z = e)\ra y=z\right)$, since in any group inverses are unique.
\end{enumerate}
\end{ex}

\begin{defin}[Complete Theory]
An $\L$-theory $T$ si \undf{complete} if, for any $\L$-sentence $\phi$, we have $T\models\phi$ or $T\models \neg\phi$.
\end{defin}

\begin{ex}\ 
\begin{enumerate}[label=\arabic*)]
\item The theory of groups is not complete. Consider $\forall x\forall y (x\ast y = y \ast x)$ - this asserts that the group is abelian. Since there are some groups with this property and some without it, then neither $T\models \phi$ nor $T\models \neg\phi$.
\item ZFC is not complete (if it is consistent); consider the Continuum Hypothesis.
\end{enumerate}
\end{ex}

\begin{defin}[Theory of a structure]
	Let $\M$ be an $\L$-structure. The \undf{theory of $\M$} is
	\begin{align*}
	\textrm{Th}(\M) = \textrm{Th}_\L(\M) \coloneqq \{\phi : \phi \textrm{ is an $\L$-sentence and }\M \models \phi\}
	\end{align*}
\end{defin}
Note that $\Th(\M)$ is complete, since for every $\phi$ either $\phi \in \Th(\M)$ or $M\models \not \phi$. However, this makes $\Th(\M)$ complicated as a set; every sentence or its negation is in the set, including many that are pointless or redundant. We want to look for complete theories that have a much more efficient presentation.

\begin{defin}[Elementarily Equivalent]
Two $\L$-structures $\M$ and $\N$ are \undf{elementarily equivalent}, written $\M\equiv \mathcal{N}$ if $\Th(\M) = \Th(\mathcal(N))$.
\end{defin}
Note that $\equiv$ is an equivalence relation on $\L$-structures. To emphasise that this only a discussion of $\L$-structures for a specific language $\L$, we may sometimes write $\equiv_\L$.

\begin{remark*}[Exercise] (Sheet 1 Question 2) Let $T$ be an $\L$-theory. TFAE
\begin{enumerate}[label=\roman*)]
\item $T$ is complete
\item For an $\L$-sentence $\phi$, if $T\not\models \phi$ then $T\models \neg \phi$. We remark that for a model $\M$, $\M\not\models \phi\implies \M\models \neg \phi$, but this is \it{not} the case for \it{theories} in general.
\item Any two models of $T$ are elementarily equivalent.
\end{enumerate}
\end{remark*}
\begin{ex} Let $\L = \emptyset$ and $T = \{ \phi_n:n\ge 2 \}$ where $\phi_n$ is
	\[
	\exists x_1\dots\exists x_n \bigwedge_{i\ne j} x_i\ne x_j
	\]
	$T$ is then the \undf{theory of infinite sets}; its models are all of the infinite $\L$-structures. So, as $\L$-structures, $\N\equiv \Z\equiv \Q\equiv \R\equiv \C\equiv \mathcal{P}(\C) \equiv $ any infinite set.
\end{ex}

\begin{theorem}[Vaught's Test]
Let $T$ be an $\L$-theory such that
\begin{enumerate}[label=\alph*)]
\item $T$ has no finite models
\item $\exists \kappa \ge |\L| +\aleph_0$ such that any two models of $T$ of cardinality $\kappa$ are elementarily equivalent
\end{enumerate}
Then $T$ is complete.
\end{theorem}
\begin{proof}
Suppose $T$ is not complete. Then there is a sentence $\phi$ such that $T\cup\{\neg\phi\}$ is satisfiable, and $T\cup \{\phi\}$ is satisfiable.

By (a), these theories have infinite models. By Lowenheim-Skolem, these theories have models of size $\kappa$. But these are both models of $T$ and hence are elementarily equivalent $\bot$ by (b).
\end{proof}


Showing that two structures are elementarily equivalent is often difficult to do directly, so we need to find other ways around it.

\section{Homomorphisms}

Let $\L$ be a language.

\begin{defin}[$\L$-Homomorphism]
Let $\M$ and $\mathcal{N}$ be $\L$-structures. A function $h:M\ra N$ is an \undf{$\L$-homomorphism} if
\begin{enumerate}[label=\roman*)]
	\item for any $n$-ary function symbol $f$ and $a_1,\dots,a_n\in M$
	\[
	h(f^\M(a_1,\dots,a_N)) = f^\N(h(a_1),h(a_2),\dots,h(a_n))
	\]
	\item for any $n$-ary relation symbol $R$ and $a_1,\dots,a_n\in M$
	\[
	(a_1,\dots,a_n)\in R^\M \iff (h(a_1),\dots,h(a_n))\in R^\N
	\]
	\item for any constant symbol $c$, $h(c^\M) = c^\N$.
\end{enumerate}

We write $h:\M\ra\N$ for $\L$-homomorphisms $h$.

If $h$ is also injective, then $h$ is an \undf{$\L$-embedding}. If $h$ is also bijective, then $h$ is an \undf{$\L$-isomorphism}.
\end{defin}

\begin{theorem}[]
Suppose $h:\M\ra\n$ is an $\L$-isomorphism. Then for any $\L$-formula $\phi(x_1,\dots,x_n)$ and $a_1,\dots,a_n\in M$, we have
\[
\M\models\phi(a_1,\dots,a_n)\iff \n\models \phi(h(a_1),\dots,h(a_n))
\]
\end{theorem}
\begin{proof}
Often in situations like this, we will need to induct on the complexity of the formula, with the base case simply being the terms, and then atomic formulae, then all formulae.

\underline{Claim}: For any $\L$-term $t(x_1,\dots,x_n)$ and $a_1,\dots,a_n\in M$
\[
h(t^\M(a_1,\dots,a_n)) = t^\n(h(a_1),\dots,h(a_n))
\]
Proof of claim: induction on terms. If $t$ is a constant symbol $c$, then $h(t^\M) = h(c^\M) = h(c^\n) = t^\N$ since $h$ preserves functions (and thus constant symbols).

If $t$ is a variable $x_1$, then $h(t^\M(a_1)) = h(a_1) = t^\n(h(a_1))$ since variables are interpreted as the identity function.

Let $f$ be an $m$-ary function symbol. Assume the result for terms $t_1,\dots,t_m$ whose free variables are among $x_1,\dots,x_n$. Let $t$ be $f(t_1,\dots,t_m)$. Given $a_1,\dots,a_n\in M$:
\begin{align*}
h(t^\M(\overline{a}) &= h(f^\M(t_1^\M(\overline{a}),\dots,t_m^\M(\overline{a})))\\
&=f^\n(h(t_1^\M(\overline{a})),\dots,h(t_m^\M(\overline{a})))\\
&=f^\n(t_1^\n(h(\overline{a})),\dots,t_m^\n(h(\overline{a})))\\
&=t^\n(h(\overline{a}))
\end{align*}
So the claim is proven. Now we prove the theorem by induction on $\phi$.

\underline{Base case}: $\phi$ is atomic.
\begin{enumerate}[label=\arabic*)]
	\item $\phi$ is $t_1 = t_2$:
	\begin{align*} M\models \phi(\overline{a}) &\iff t_1^\M(\bar{a}) = t_2^\M(\bar{a})\\
		 &\iff h(t_1^\M(\bar{a})) = h(t_2^\M(\bar{a}))\textrm{ ($h$ injective)}\\
		 &\iff t_1^\n(h(\bar{a})) = t_2^\n(h(\bar{a}))\textrm{ (by claim)}\\
		 &\iff \N\models\phi(h(\bar{a}))
	\end{align*}
	\item $\phi$ is $R(t_1,\dots,t_n)$ (Exercise).
\end{enumerate}
\underline{Induction Step}: Assume the result for $\phi$ and $\psi$.

Exercise: check $\phi\land\psi$ and $\neg\phi$.

We will do $\forall x_n \phi(x_1,\dots,x_n)$, with free variables $x_1,\dots,x_{n-1}$. Fix $a_1,\dots,a_{n-1}\in M$.
\begin{align*}
M\models \forall x_n \phi(a_1,\dots,a_{n-1},x_n) &\iff \textrm{ for all }b\in M,\ \M\models \phi(a_1,\dots,a_{n-1},b)\\
&\iff \textrm{ for all }b\in M,\ \n\models \phi(h(a_1),\dots,h(a_{n-1}),h(b))\textrm{ (induction)}\\
&\iff \textrm{ for all }c\in N,\ \n\models \phi(h(a_ 1),\dots,h(a_{n-1}),c)\textrm{ ($h$ surjective)}\\
&\iff \N\models\forall x_n \phi(h(a_1),\dots,h(a_{n-1}),x_n)
\end{align*}
And so we are done. In particular, $\L$-isomorphisms preserve all formulae.
\end{proof}

\begin{remark*}[Notation]
We write $\M \cong \n$ if there is an $\L$-isomorphism $h:\M\ra\n$.
\end{remark*}

\begin{cor}
If $\M\cong \N$ then $\M \equiv \N$.
\end{cor}

Note that, as we can see, $\cong$ is stronger than $\equiv$; $\cong$ says that two structures are more or less the same, whereas $\equiv$ only makes an assertion about first order statements satisfied by the models.

\begin{cor}
$h:\M\ra\N$ is an $\L$-embedding iff for any quantifier-free the conclusion of Theorem 2.2 holds for all quantifier-free formulas $\phi(x_1,\dots,x_n)$. That is to say, $\L$-embeddings preserve all quantifier-free formulas.
\end{cor}
\begin{proof}
$(\implies)$ is done by the proof of 2.2; we only used the surjectivity of $h$ for the quantifier step.
For $(\impliedby)$, see Sheet 1, Question 6.
\end{proof}

An embedding is precisely characterised by preserving quantifier-free formulae. This motivates the question, what about maps that preserve all formulas? We know that isomorphisms will do, but is that all of them? The answer is in fact no, in general.

\begin{defin}[Elementary $L$-Embedding]
$h:\M\ra\N$ is an \undf{elementary $\L$-embedding} if for any $L$-formula $\phi(\bar{x})$ and $\bar{a}$ from $M$, $\M\models \phi(\bar{a})$ iff $\N\models \phi(h(\bar{a}))$.
\end{defin}

Note that isomorphisms are elementary embeddings, but elementary embeddings need not be isomorphisms.

\begin{defin}[Elementary Substructure]
Let $\M$ and $\N$ be $\L$-structures with $M\subset N$. Let $h:M \xhookrightarrow{} N$ be the inclusion map. Then $\M$ is a \undf{substructure} of $\N$ (respectively, \undf{elementary substructure}), written $\M\subset \N$ (respectively $\M\preceq\N$) if $h$ is an $\L$-embedding (respectively, elementary embedding).

Similarly, $\N$ is an \undf{extension} of $\M$ (respectively, \undf{elementary extension}).
\end{defin}	

\begin{remark*}[Note]
If $\M\preceq\N$ then $M\subset N$ and $\M\equiv \N$.
\end{remark*}
\begin{ex}
Let $\M = (2\Z, <)$ and $\N = (\Z, <)$.

Then $\M \subset \N$ and $\M \equiv \N$, but $\M\not\preceq\N$, for instance $\M\models \neg\exists x(0 < x < 2)$, but this is of course untrue for $\N$.

So the inclusion map might be an embedding, but it is not necessarily elementary.
\end{ex}


\section{Categoricity}

\underline{Q}: Suppose $\M\equiv\N$. Then is it true that $\M \cong \N$?

\underline{A}: No - \it{e.g.} theory of infinite sets, any two infinite sets are elementarily equivalent but many are obviously not isomorphic. More generally, if $\M$ is infinite then Th$(\M)$ has models of arbitrarily large size.

So a theory with infinite models \it{never} has a unique model up to isomorphism, as models of different cardinalities cannot be isomorphic (since an isomorphism contains a bijection).

\begin{defin}[$\kappa$-categorical]
An $\L$-theory $T$ is \undf{$\kappa$-categorical} if it has a unique model of size $\kappa$ up to isomorphism.
\end{defin}
Our main focus for theories here will be those $T$ that have infinite models and $\kappa \ge |\L| + \aleph_0$.

\begin{ex}\ 
\begin{enumerate}[label=\arabic*)]
	\item Th$(\mathbb{N})$ in $\L = \emptyset$ is $\kappa$-categorical for all $\kappa \ge \aleph_0$ (Sheet 1 \#3)
	\item Th$(\Q,+)$ is $\kappa$-categorical iff $\kappa > \aleph_0$ (related to Sheet 1 \#4)
	\item Th$(\Q,<)$ is $\kappa$-categorical iff $\kappa = \aleph_0$
	\item Th$(\Z,+)$ is $\kappa$-categorical for no $\kappa$
\end{enumerate}
\end{ex}
***Non-Examinable***
\begin{theorem*}[Morley's Theorem (1965)]
Let $T$ be a complete theory in a countable language. If $T$ is $\kappa$-categorical for some $\kappa > \aleph_0$, then it is $\kappa$-categorical for all $\kappa > \aleph_0$.
\end{theorem*}
***End of non-examinable section***

\begin{defin}[Theory of Dense Linear Orders]
Let $\L = \{<\}$ (binary relation) be the language of partial orders. Define \undf{DLO (dense linear orders)} to be the following theory
\begin{itemize}
	\item $\forall x\neg (x < x)$
	\item $\forall x \forall y \forall z ((x < y \land y < z) \ra x < z)$ (partial order)
	\item $\forall x\forall y ((x\ne y) \ra (x < y \lor y < x))$ (linear order)
	\item $\forall x \forall y(x < y \ra \exists z (x < z < y))$ (dense)
	\item $\forall x \exists y \exists z (y < x < z)$ (no endpoints)
\end{itemize}
Note that $(\Q,<)\models$ DLO.
\end{defin}

\begin{theorem}[Cantor, 1895]
DLO is $\aleph_0$-categorical.
\end{theorem}
\begin{proof}
``Back and Forth Construction''.

Fix countable models $\M,\N\models$ DLO. Let $M = \{a_n:n\ge 0\}$ and $N = \{b_n:n\ge 0\}$. With these enumerations we will construct an isomorphism between $\M$ and $\N$.

We will inductively construct a sequence $(h_n)_{n=0}^{\infty}$ of functions such that:
\begin{enumerate}[label=\arabic*)]
	\item $h_n:X_n\ra Y_n$ is an order-preserving bijection, where $X_n\subset M$ and $Y_n\subset N$ are finite
	\item $X_n\subset X_{n+1}$, $Y_n\subset Y_{n+1}$ and $h_n\subset h_{n+1}$
	\item $a_n \in X_n$ and $b_n \in Y_n$
\end{enumerate}
Once we have done this, we will have a sequence of increasing functions with domains and ranges getting bigger and bigger. We can then let $h = \bigcup_{n=0}^{\infty}h_n$. Then $h$ is an order-preserving bijectionfrom $M$ to $N$, which in this language is precisely an $\L$-isomorphism.

\underline{Base case}: Let $X_0 = \{a_0\}$, $Y_0 = \{b_0\}$, and $h_0 = \{(a_0,b_0)\}$; this trivially satisfies all the desired properties.

Now assume we have $h_n:X_n\ra Y_n$ as above.

\underline{Forth}: Construct an order-preserving bijection $h_\ast:X_\ast\ra Y_\ast$ extending $h_n$ with $a_{n+1}\in X_\ast$. Enumerate $X_n = \{x_1,\dots,x_k\}$ such that $x_1 <^\M \dots <^\M x_k$. Let $y_i = h_n(x_i)$. Then $y_1 <^\N \dots <^\N y_k$ since $h_n$ is order-preserving.

Define $h_\ast = h_n \cup \{(a_{n+1},b)\}$ where $b\in N$ is chosen as follows.

\underline{Case 1}: $a_{n+1} = x_i$ for some $i\le k$. Let $b = y_i$.

\underline{Case 2}: $x_k <^\M a_{n+1}$. Choose $b\in N$ such that $y_k <^\N b$.

\underline{Case 3}: $a_{n+1} <^\M x_1$. Choose $b\in N$ such that $b <^\N y_1$.

\underline{Case 4}: $x_i <^\M a_{n+1} <^\M x_{i+1}$ for some $ i < k$. Choose $b \in N$ such that $y_i < ^\N b < ^\N y_{i+1}$.

\underline{Back}: Construct order-preserving $h_{n+1}:X_{n+1}\ra Y_{n+1}$ extending $h_\ast$ such that $b_{n+1}\in Y_{n+1}$; details are an exercise (though it is basically the same as the above).
\end{proof}
\begin{cor}
DLO is a complete theory.
\end{cor}
\begin{proof}
Apply Vaught's Test. Note that DLO clearly has no finite models.

If $\M,\N\models$ DLO are countable then $\M \cong \N$, so $\M \equiv \N$.
\end{proof}

So $(\Q,<) \equiv (\R,<)\equiv $ any dense linear order without endpoints. In particular, any two such orders cannot be distinguished by a first order statement in the language of partial orders.

\begin{remark*}[More Notions]
Let $\L$ be a language. Suppose $\M$ is an $\L$-structure. Fix a collection $(\M_i)_{i\in I}$ of substructures of $\M$. Let $\N = \bigcap_{i\in I}M_i$; assume $N \ne \emptyset$ (this will always happen as along as there are some constant symbols, say). Then we have a canonical $L$-structure $\N$ with universe $N$, by interpreting the language in the only way that makes sense. That is, $f^\N = f^\M |_\N = f^{\M_i} |_\N$, $R^\N = R^\M \cap N^{\alpha(R)} = R^{\M_i}\cap N^{\alpha(R)}$, $c^\N = c^\M = c^{\M_i}$.

Note $\N \subset \M_i$ for all $i\in I$.
\end{remark*}
\begin{defin}[Generated Substructure]
Given a structure $\M$ and a non-empty set $A\subset M$, the \undf{substructure of $\M$ generated by $A$} is the intersection of all substructures of $\M$ containing $A$.
\end{defin}

\begin{defin}[Chain of $\L$-structures]
Let $\alpha$ be a limit ordinal. A collection $(\M_i)_{i<\alpha}$ of $\L$-structures is a \undf{chain} if $\M_i \subset \M_{j}$ for all $i < j$.

If in fact the condition above is strengthened to $\preceq$, then we say it is an \undf{elementary chain}.

If $(M_i)_{i<\alpha}$ is a chain then we have a well-defined structure $\bigcup_{i < \alpha}\M_i$.
\end{defin}


\section{Algebraically Closed Fields}

\begin{remark*}[Recall]
$(K,+,\cdot,0,1)$ is a \undf{field} if $(K,+,0)$ and $(K\backslash\{0\},\cdot,1)$ are abelian groups and $\forall x\forall y\forall z((x\cdot(y+z) = x\cdot y + x\cdot z))$.

$K$ is \undf{algebraically closed} if every non-constant polynomial over $K$ has a root in $K$.
\end{remark*}
Let $\L = \{+,\cdot,0,1\}$, the language of fields.

\begin{defin}[ACF]
The first order $\L$-theory axiomatising algebraically closed fields is known as \undf{ACF} - all the above statements can be given as first order $\L$-sentences.

In particular, this contains the field axioms and for every $d\ge 1$ the claim that every degree $d$ polynomial has a root:
\begin{align*}
\forall v_0\forall v_1\dots\forall v_{d-1}\exists x(x^d + v_{d-1}x^{d-1}+\dots+v_1x+v_0 = 0)
\end{align*}
We take this statement for every $d$, {\it i.e.} we have infinitely many.
\end{defin}
\begin{remark*}
ACF is not complete, since it does not specify characteristic - hence different models are distinguishable by a first order property.
\end{remark*}
\begin{defin}[$\textrm{ACF}_0,\ \textrm{ACF}_p$]
For $n\ge 1$, let $\chi_n$ be the $\L$-sentence
\begin{align*}
	\underbrace{1+1+\dots+1}_{n} = 0
\end{align*}
We then have the theory of algebraically closed fields of characteristic zero, \undf{ACF$_0$}:
\begin{align*}
\acf_0 = \acf\cup\{\neg\chi_n:n\ge1\}
\end{align*}
For a prime $p$, we have $\acf_p = \acf\cup\{\chi_p\}$
\end{defin}

\begin{theorem}
$\acf_0$, $\acf_p$ are $\kappa$-categorical for all $\kappa > \aleph_0$.
\end{theorem}
\begin{proof}
The \undf{transcendence degree} of $K\models\acf$ is the cardinality of the largest algebraically independent susbet of $K$.

For example, trdeg$(\bar{Q}) = 0$, trdeg$(\overline{\Q(\pi)}) = 1$, trdeg$(\C) = 2^{\aleph_0}$, trdeg$(\overline{\Q(x_i)}_{i<\kappa})=\kappa$

\begin{remark*}[Facts]\ 
\begin{enumerate}[label=(\arabic*)]
	\item Suppose $K,L\models\acf$. Then $K\cong L$ iff trdeg$(K)$ = trdeg$(L)$, char$(K)$ = char$(L)$, and $|K| = |L|$
	\item If $K\models \acf$ and $\kappa = $trdeg$(K)$, then $|K| = \aleph_0 + \kappa$
\end{enumerate}
\end{remark*}
\underline{Conclusion}: If $K,L\models \acf_0$ (or $\acf_p$) are uncountable and $|K| = |L|$, then $K\cong L$.
\end{proof}
\begin{cor}
$\acf_0$ and $\acf_p$ are complete.
\end{cor}
\begin{proof}
Vaught's Test.
\end{proof}
\begin{remark*}
$\acf_0$, $\acf_p$ are not $\aleph_0$-categorical.

The countable models are precisely the countable $\acf_p$s of trdeg $n$ for $n\in \N\cup\{\aleph_0\}$.
\end{remark*}

\begin{defin}[Polynomial Map]
Let $K$ be a field. A function $\Phi: K^m\ra K^n$ is a \undf{polynomial map} if
\begin{align*}
\Phi = \big(p_1(x_1,\dots,x_m),p_2(x_1,\dots,x_m),\dots,p_n(x_1,\dots,x_m)\big)
\end{align*}
where $p_i \in K[\bar{x}]$ for each $i$.
\end{defin}
\begin{theorem}[Ax-Grothendieck]
Let $K\models \acf$ and suppose $\Phi:K^n\ra K^n$ is an injective polynomial map. Then $\Phi$ is surjective.
\end{theorem}
\begin{proof}
First, suppose that $K = \bar{\mathbb{F}}_p$ for some prime $p$. Recall that $\bar{\mathbb{F}}_p = \bigcup_k \mathbb{F}_{p^k}$. Fix $m$ such that all coefficients in $\Phi$ come from $\mathbb{F}_{p^m}$. Note that $\bar{\mathbb{F}}_p = \bigcup_k \mathbb{F}_{p^{km}}$.

Then for any $k\ge 1$, $\Phi$ induces an injective polynomial map from $\mathbb{F}_{p^{km}}^n \ra \mathbb{F}_{p^{km}}^n$m , which therefore is surjective since the sets we are dealing with are finite.
\begin{align*}
\Phi\left(\bar{\mathbb{F}}_p^n\right) &= \Phi\left(\bigcup_k\mathbb{F}_{p^{km}}^n\right)\\
&= \bigcup_k\Phi\left(\mathbb{F}_{p^{km}}^n\right) = \bigcup_k\mathbb{F}_{p^{km}}^n\\
&=\bar{\mathbb{F}}_p^n
\end{align*}
Now, given $n,d\ge 1$, let $\psi_{n,d}$ be the $\L$-sentence which says:

``Every injective polynomial map with $n$ coordinates, each of which is a polynomial in $n$ variables and degree $\le d$, is surjective.''

Exercise: show that this is first order.
\end{proof}

We've shown $\bar{\F}_p \models \psi_{n,d}$ for all primes $p$ and $n,d$.

So for any prime $p$, $\acf_p\models\psi_{n,d}$ for all $n,d$ since $\acf_p$ is complete.

Now consider $\acf_0$. For contradiction, suppose that there exists soem $n,d$ such that $\acf_0\not\models\psi_{n,d}$. Then $\acf_0\models\neg\psi_{n,d}$ since $\acf_0$ is complete. By Compactness, there is a finite set $\Sigma\subset \acf_0$ such that $\sigma \models \neg \psi_{n,d}$. So $\Sigma\subset \acf\cup\{\neg\chi_1,\dots,\neg\chi_m\}$ for some $m$. Choose a prime $p > m$. Then $\acf_p\models \Sigma$.

So $\acf_p\models \neg\psi_{n,d}$, which is a contradiction.
\begin{theorem}[Lefschetz Principle]
Let $\phi$ be an $\L$-sentence. TFAE
\begin{enumerate}[label=(\arabic*)]
	\item $\acf_0\models \phi$ {\it i.e.} $\phi$ is true in every $K\models \acf$
	\item $\acf_0\cup\{\phi\}$ is consistent, {\it i.e.} $\phi$ is true in some $K\models \acf_0$
	\item There is some $n > 0$ such that $\acf_p\models\phi$ for all $p > n$ {\it i.e.} $\phi$ is true in every $K\models \acf$ of sufficiently large characteristic
	\item For all $n > 0$ there exists $p > n$ such that $\acf_p\cup\{\phi\}$ is consistent, {\it i.e.} $\phi$ is true in some $K\models \acf$ of arbitrarily large characteristic.
\end{enumerate}
\end{theorem}


\section{Diagrams \& Extensions}

Let $\M$ be an $\L$-structure.

\begin{theorem}[Remark!]
%CHANGE THIS TO A NUMBERED REMARK!
If $h:\M\ra\N$ is an $\L$-embedding then after identifying $a\in M$ with $h(a) \in N$, we can view $\M$ as a substructure of $\N$.

Similarly, if $h$ is an elementary embedding then $\M$ can be viewed as an elementary substructure of $\N$.
\end{theorem}

Given $A\subset M$, let $\L_A = \L\cup\{\underline{a}:a\in A\}$, where $\underline{a}$ is a new constant symbol. We underline it to differentiate it from the element in $A$.

Then $\M$ is canonically an $\L_A$-structure, with $\underline{a}^\M = a$.

\begin{defin}[Diagram]
The \undf{diagram of $\M$}, written $\\D(\M)$, is the $\L_M$-theory consisting of all quantifier-free $\L_M$-sentences $\phi$ such that $\M\models \phi$.

Similarly, the \undf{elementary diagram of $\M$}, written $\Th_M(\M)\coloneqq \Th_{\L_M}(M)$.
\end{defin}

\begin{prop}
Suppose $\M$ is an $\L$-structure and $\N^\ast$ is an $\L_M$-structure such that $\N^\ast\models \\D(\M)$. Let $\N$ be the reduct of $\N^\ast$ to $\L$. Define $h:\M\ra\N$ such that $h(a) = \underline{a}^{\N^\ast}$. Then $h$ is an $\L$-embedding.

Moreover, if $\N^\ast\models \Th_M(\M)$, then $h$ is an elementary embedding.
\end{prop}
\begin{proof}
Use Corollary 3.4. Let $\phi(x_1,\dots,x_n)$ be a quantifier-free $\L$-formula, and fix $a_1,\dots,a_n\in M$. Then $\M\models \phi(a_1,\dots,a_n)$ iff $\M\models \phi(\underline{a_1},\dots,\underline{a_n})$ iff $\phi(\underline{a_1},\dots,\underline{a_n})\in \\D(\M)$ iff $\N^\ast\models\phi(\underline{a_1},\dots,\underline{a_n})$ iff $\N\models \phi(h(a_1),\dots,h(a_n))$.

The ``moreover'' statement is similar (just drop the quantifier-free claim).
\end{proof}

\subsection*{Application to Groups}

Recall that an abelian group $G$ is \undf{orderable} if there is a linear order $<$ on $G$ such that for all $x,y,z\in G$ , if $x<y$ then $x+z < y+z$.

Note that any orderable abelian group is torsion-free, since $x > 0\implies nx > 0$ for every $n$. Similarly for $x < 0$. We now prove the converse:

\begin{theorem}[Levi 1942]
Any torsion-free abelian group is orderable.
\end{theorem}
\begin{proof}
Let $\L^0 = \{+,0\}$ be the language of (abelian) groups. Set $\L = \L^0 \cup\{<\}$, where $<$ is a binary relation symbol. Let $\sigma$ be the $\L$-sentence
\begin{align*}
\forall x \forall y\forall z ( x< y \ra x + z < y + z)
\end{align*}
Now let $G$ be a torsion-free abelian group, viewed as an $\L^0$ structure.

Define the $\L_G$-theory
\begin{align*}
T = \underbrace{\D(G)}_{\L_G^0\textrm{-theory}}\cup\{\textrm{axioms for linear order \& abelian groups}\}\cup\{\sigma\}
\end{align*}

Suppose $T$ has a model $\M$. Then $(M,+^\M,0^\M,<^\M)$ is an ordered abelian group, and $G\subset (M,+^\M,0^\M)$ by Prop 6.3. So $G$ is a subgroup of an ordered abelian group, which is thus orderable.

So all that remains is to show that $T$ has a model.

Fix $\Sigma \subset T$ finite. Let $A = \{a\in G: \underline{a}\textrm{ appears in some }\L_G^0\textrm{-sentence in }\Sigma\}$, and let $H = \langle A\rangle \le G$. Then $H\cong \Z^n$ for some $n\ge 0$ by the structure theorem for (torsion-free) finitely generated abelian groups. View $H$ as an $\L_A$-structure such that $\underline{a}^H = a$ and $<^H$ is the lexicographic ordering. Then $H\subset_{\L_A^0} G$, and so $H\models \phi$ for any $\phi \in \D(G)$, using only extra constants from $A$ by Corollary 3.4.

So $H\models \Sigma$. So done by Compactness.
\end{proof}


\section*{Quantifier Elimination}

\underline{Idea}: Let $T$ be an $\L$-theory and let $\M\models T$. Then $X\subset M^n$ is \undf{definable} if there is an $\L$-formula $\phi(x_1,\dots,x_n)$ such that $X = \{\bar{a}\in M^n : \M\models \phi(\bar{a})\}$.

\underline{Goal}: Study definable subsets of models of $T$.

Unfortunately, quantifiers make this difficult. $X$ itself might be nice, by the projection $Y = \{(a_1,\dots,a_{n-1})\in M^{n-1}:(\bar{a},\bar{b})\in X\textrm{ for some }b\in M\}$ (defined by $\exists x_n \phi(\bar{x})$) might be complicated.

\begin{defin}[Quantifier Elimination]
An $\L$-theory $T$ has \undf{quantifier elimination} if for any $\L$-formula $\phi(x_1,\dots,x_n)$ there is a quantifier-free $\L$-formula $\psi(x_1,\dots,x_n)$ such that
\begin{align*}
T\models \forall \bar{x}\big((\phi(\bar{x})\leftrightarrow \psi(\bar{x}))
\end{align*}
That is to say, $\phi$ and $\psi$ define the same set in any $\M\models T$.
\end{defin}

\begin{ex}\ 
\begin{enumerate}[label=(\arabic*)]
\item $T = \Th(F)$, where $F$ is a field. Let $\phi(w,x,y,z)$ be the statement ``$\left(\begin{array}{cc} w & x\\ y & z\end{array}\right)\textrm{ has an inverse}$'', {\it i.e.} there exist $s,t,u,v$ forming a matrix that inverts it.

Then $T\models \forall w\forall x\forall y\forall z\big(\phi(w,x,y,z)\leftrightarrow wz-xy\ne 0\big)$.

\section{Lecture 7}

\item $T = \Th(\R,+,\cdot,0,1)$. $\phi(x)$ is $\exists y(x = y^2)$. Note that $\phi$ defines $\R^{\ge 0}$. We can in fact not write this quantifier-free:

Suppose $\psi(x)$ is quantifier-free. In this case the terms are just polynomials, so $\psi(x)$ is a Boolean combination of polynomial equations. So $\psi$ defines a finite or cofinite subset of $\R$, and in particular cannot define the positive reals. Hence $T$ does not have QE.

We will later see that $\Th(\R,+,\cdot,<,0,1)$ {\it does} have QE. Note that $x < y \iff \exists z: (z\ne 0 \land y-x = z^2)$. In particular, adding the ordering relation doesn't really add anything new to what we can define in this language.

So $(\R,+,\cdot,0,1)$ and $(\R,+,\cdot,<,0,1)$ have the same definable sets, which means they are very similar structures. So the important thing to note is that QE is very language-dependent, and while we might not immediately have QE we might be able to just go out and look for it.
\end{enumerate}
\end{ex}

We will now discuss some quantifier elimination tests.

\begin{lemma}
Suppose $T$ is an $\L$-theory such that for any q.f. formula $\phi(x_1,\dots,x_m,y)$ there is a q.f. $\psi(x_1,\dots,x_n)$ such that $T\models \forall \bar{x}\big(\exists y \phi(\bar{x},y)\leftrightarrow \psi(\bar{x})\big)$. Then $T$ has QE.
\end{lemma}
\begin{proof}
Induction on formulas (exercise).
\end{proof}

So we only need to eliminate one quantifier at a time.

\begin{theorem}
Let $T$ be an $\L$-theory. TFAE:
\begin{enumerate}[label=\roman*)]
	\item $T$ has QE
	\item Suppose $\M,\N\models T$ and $\A\subset\M,\A\subset\N$. Then for any q.f. formula $\phi(\bar{x},y)$ and any tuple $\bar{a}$ of parameters from $A$, if $\M\models \exists y\phi(\bar{a},y)$ then $\N\models \exists y \phi(\bar{a},y)$.
	\item For any $\L$-structure $\A$, $T\cup \mathcal{D}(\A)$ is a complete $\L_A$-theory.
\end{enumerate}
\end{theorem}
\begin{proof}
(i)$\implies$(iii). Assume $T$ has QE. Let $\A$ be an $\L$-structure and suppose $\M,\N\models T\cup \D(\A)$. We want to show that $\M \equiv_{\L_A} \N$.

Let $\sigma$ be an $\L_A$-sentence such that $\M\models \sigma$. WTS $\N\models \sigma$. Write $\sigma$ as $\phi(\underline{a}_1,\dots,\underline{a}_n)$ for some $\L$-formula $\phi(x_1,\dots,x_n)$ and $a_1,\dots,a_n\in A$. By QE, there is a q.f. $\psi(x_1,\dots,x_n)$ such that $T\models \forall \bar{x}\big(\phi(\bar{x})\leftrightarrow \psi(\bar{x})\big)$.

Since $\M\models T$ and $\M\models \phi(\bar{a})$, we have $\M\models \psi(\bar{a})$. Since $\M\models\D(\A)$, we have $\psi(\underline{a}_1,\dots,\underline{a}_n)\in\D(\A)$. But $\N$ models the diagram, so $\N\models\psi(\underline{a}_1^\N,\dots,\underline{a}_n^\N)$. Since $\N\models T$, $\N\models \phi(\underline{a}_1,\dots,\underline{a}_n)$, {\it i.e.} $\N\models \sigma$.

(iii)$\implies$(ii) Let $\M,\N,\A,\phi(\bar{x},y),\bar{a}$ be as in the hypothesis of (ii). Since $\A\subset \M$ and $\A\subset \N$, we have that $\M,\N\models T\cup \D(\A)$ by Cor 2.4. By (iii), this is a complete theory and so $\M\equiv_{\L_A}\N$. So $\M\models \exists y\phi(\bar{a},y)$, which is an $\L_A$-sentence, implies $\N\models \exists y \phi(\bar{a},y)$.

(ii)$\implies$(i) By Lemma 6.1, it suffices to fix q.f. $\phi(\bar{x},y)$ and find q.f. $\psi(\bar{x})$ such that $T\models \forall \bar{x}\big(\exists y \phi(\bar{x},y)\leftrightarrow\psi(\bar{x})\big)$.

Let $\L^\ast = \L \cup \{c_1,\dots,c_n\}$ where $c_i$ is a new constant symbol. Let $\Gamma = \{\psi(\bar{c}): \psi(\bar{x})$ is a q.f. $\L$-formula and $T\models \forall\bar{x} \big(\exists y\phi(\bar{x},y)\rightarrow \psi(\bar{x})\big)\}$.

\underline{Claim}: $T\cup \Gamma \models \exists y\phi(\bar{c},y)$.

First, assume the claim holds. By Compactness, there exists a q.f. $\psi_1(\bar{x}),\dots,\psi_m(\bar{x})$ such that $T\cup\{\psi_1(\bar{c}),\dots,\psi_m(\bar{c})\}\models \exists y\phi(\bar{c},y)$ and $T\models \forall \bar{x}\big(\exists y\phi(\bar{x},y)\rightarrow \bigwedge_{i=1}^{m}\psi_i(\bar{x})\big)$. Let $\psi(\bar{x})$ be $\bigwedge_{i=1}^{m}\psi_i(\bar{x})$. Then $T\models \big(\psi(\bar{c}) \rightarrow \exists y\phi(\bar{c},y)\big)$.

So $T\models \forall \bar{x}\big(\psi(\bar{x}) \rightarrow \exists y \phi(\bar{x},y)\big)$ (exercise ``generalisation''). So $T\models \forall \bar{x}\big(\psi(\bar{x})\leftrightarrow \exists y\phi(\bar{x},y)\big)$.

\underline{Proof of Claim}: Suppose not. There is $\N\models T\cup\Gamma \cup \{\neg \exists y\phi(\bar{c},y)\}$. Let $a_i = c_i^\N$ and let $\A\subset\N$ be the substructure generated by $a_1,\dots,a_n$. Then $\N\models T$, $\A\subset\N$, and $\N\models \neg \exists y\phi(\bar{a},y)$.

By ES1 \#7, any $b \in A$ is of the form $t^\N(\bar{a})$ for some $\L$-term $b$. So we can view $\D(\A)$ as an $\L^\ast$-theory by replacing $\underline{b}$ with $t(c_1,\dots,c_n)$. Let $\Sigma\models T\cup\D(\A)\cup\{\exists y\phi(\bar{c},y)\}$. If we build $\M\models \Sigma$ then $\M\models T$, $\A\subset \M$ and $\M\models \exists y \phi(\bar{a},y)$, contradicting (ii).

So it suffices to show $\Sigma$ has a model, which we will do by compactness. Suppose this fails. Then by compactness there are q.f. $\psi_1(\bar{x}),\dots,\psi_m(\bar{x})$ such that $\psi_1(\bar{c}),\dots,\psi_m(\bar{c})\in\D(\A)$ and $$ T\cup \left\lbrace \bigwedge_{i=1}^{m}\psi_i(\bar{c})\right\rbrace \cup \left\lbrace \exists y \phi(\bar{c},y)\right\rbrace$$ is unsatisfiable. Let $\psi(\bar{x})$ be $\neg \bigwedge_{i=1}^{m}\psi_i(\bar{x})$. Then $T\models \big(\exists y\phi(\bar{c},y)\rightarrow \psi(\bar{c})\big)$. So $T\models \forall \bar{x} \big(\exists y \phi(\bar{x},y)\rightarrow\psi(\bar{x})\big)$. So $\psi(\bar{c})\in \Gamma$. So $\N\models \psi(\bar{c})$. since $\N\models \D(\A)$, we have $\N\models \neg \psi(\bar{c})$. Contradiction.
\end{proof}


\section{Lecture 8}

\begin{remark*}
Recall Theorem 6.2 (QE test)
\begin{enumerate}[label=\arabic*)]
	\item In condition (iii), we may assume that $\A\subset\M$ for some model $\M\models T$. Otherwise, $T\cup \D(\A)$ is inconsistent and thus complete
	\item In both conditions (ii) and (iii), we may assume that $\A$ is finitely generated
\end{enumerate}
\end{remark*}

\begin{theorem}
$\acf$ has quantifier elimination.
\end{theorem}
\begin{proof}
We apply Theorem 6.2(iii). Fix a finitely-generated $\L$-structure $\A$ (in the language of fields). We want to show $\acf\cup \D(\A)$ is complete. We use Vaught's Test.

Fix $K_1,K_2\models\acf\cup \D(\A)$ uncountable with $|K_1| = |K_2|$. Then $\A$ is a finitely generated integral domain contained in $K_1$ and $K_2$.

So char$(K_1)$ = char\((K_2)\). Let $F_i$ be the field of fractions of $\A$ in $K_i$. There is a field isomorphism $\tau : F_1\ra F_2$ fixing $\A$ pointwise. Since $\A$ is finitely generated, trdeg$(F_i)$ is finite. So trdeg$(K_1/F_1)$ = trdeg$(K_2/F_2)$.

So $\tau$ extends to an isomorphism $\tau^\ast: K_1\ra K_2$ fixing $\A$.
\end{proof}
We now see a very common application of quantifier elimination of $\acf$.

\begin{defin}[Constructible Set]
Let $F$ be a field. Then $X\subset F^n$ is \undf{constructible} if it is a Boolean combination of subsets of $F^n$ defined by $p(x_1,\dots,x_n) = 0$, where $p \in F[x_1,\dots,x_n]$.
\end{defin}
\begin{cor}[Chevalley]
If $K\models \acf$ and $X\subset K^n$ is constructible, then the projection
\begin{align*}
Y = \{(a_1,\dots,a_{n-1})\in K^{n-1}:(\overline{a},b)\in X\textrm{ for some }b\in K\}
\end{align*}
is constructible.
\end{cor}

\begin{remark*}[Compare]
Consider $X = \{(x,y)\in \R^2:x = y^2\}$. Then $Y = \R^{\ge 0}$.

Exercise: think about more examples in the rationals.
\end{remark*}
\begin{proof}
Note that $X\subset K^n$ is constructible iff there is a quantifier-free formula $\phi(x_1,\dots,x_n,y_1,\dots,y_m)$ ($y_i$ parameters) and parameters $b_1,\dots,b_m\in K$ such that $X$ is defined by $\phi(\bar{x},\bar{b})$.

Fix quantifier-free formula $\phi(\bar{x},\bar{y})$ and $\bar{b}$ such that $\phi(\bar{x},\bar{b})$ defines $X$. Let $\psi(x_1,\dots,x_{n-1},\bar{y})$ be $\exists x_n\phi(\bar{x},\bar{y})$. Then $\psi(\bar{x},\bar{b})$ defines $Y$. Then by $QE$ $\psi(\bar{x},\bar{y})$ is equivalent to some quantifier-free formula. So $Y$ is constructible.
\end{proof}

\subsection*{Rado Graphs}

We work with the language of graphs $\L = \{E\}$, $E$ a binary relation.

\begin{defin}[Rado Graph]
A \undf{Rado Graph} is a graph $(V,E)$ such that $V\ne \emptyset$ and for any finite disjoint $X,Y\subset V$ there is some $v\in V$ such that $E(v,x)$ for all $x\in X$ and $\neg E(v,y)$ for all $y\in Y$.
\end{defin}
\begin{defin}[RG]
We let $\rg$ be the theory of Rado graphs in the language of graphs. In particular:
\begin{align*}
	\rg &= \{ \forall x \neg E(x,x),\forall x\forall y(E(x,y)\ra E(y,x)) \}\\
		&\cup \left\lbrace \forall x_1,\dots,x_k\forall y_1,\dots,y_k\left(\bigwedge_{i,j}x_i\ne x_j \ra \exists v\left(\bigwedge_{i=1}^{k}E(x_i,v) \land \bigwedge_{i=1}^{k}\neg E(y_i,v)\right)\right):k\ge 1\right\rbrace
\end{align*}
\end{defin}

\begin{theorem}
$\rg$ is $\aleph_0$-categorical.
\end{theorem}
\begin{proof}\ 
\begin{enumerate}[label=\arabic*)]
	\item $\rg$ has a countable model.
	
	Let $A = (V,E)$ be any finite graph. Set $A_0 = A$. Given $A_n$, define $V(A_{n+1}) = V(A_n)\cup\{v_{X,Y}:X,Y\subset V(A_n)\textrm{ disjoint}\}$, with new edges $E(v_{X,Y},x)$ for all $x\in X$ (and no others). So $A_0\subset A_1\subset A_2\subset\dots$ is a chain of substructures.
	
	Let $M = \bigcup_{n=0}^{\infty}A_n$. Then $M\models\rg$. Moreover, $M$ is countable since each $A_n$ is finite. 
	
	\item Any two countable models are elementarily equivalent.
	
	Let $\M,\N\models \rg$ countable. We show $\M\cong \N$ via a back and forth argument. Enumerate $M = \{a_n : n\ge 0\}$ and $N = \{b_n:n\ge 0\}$. Let $h_0: a_0\mapsto b_0$. Given $h_n:X_n\ra Y_n$, extend to include $a_{n+1}$ and $b_{n+1}$.
	
	Partition $X_n$ into the neighbourhood of $a_{n+1}$ and its complement. We then partition $Y_n$ by its image under $h_n$, and use the Rado axioms to find a vertex $b$ connected to $h_n(\Gamma(a_{n+1}))$ and none of its complement. Similarly find appropriate $a$ for $b_{n+1}$, and extend $h_n$ to include these pairs, giving us $h_{n+1}$.
\end{enumerate}
Hence $\rg$ is $\aleph_0$-categorical.
\end{proof}
\begin{cor}
$\rg$ is complete
\end{cor}
\begin{proof}
Use Vaught's Test. Note that $\rg$ has no finite models.
\end{proof}
\begin{remark*}[Claim]
If $\M\models \rg$ then every finite graph is an induced subgraph of $\M$.
\end{remark*}
\begin{proof}
The proof of Theorem 7.6 shows this when $\M$ is countable. Then for any $\M\models \rg$ there exists a countable $\M_0$ such that $\M_0\le \M$ by DLST ({\it c.f.} Sheet 1 Question 9).
\end{proof}
\begin{remark*}[Exercise]
	Suppose $\M,\N\models\rg$ countable and $f: X\ra Y$ is a graph isomorphism for some finite $X\subset \M$ and $Y\subset \N$. Then $f$ extends to an isomorphism from $\M$ to $\N$.
\end{remark*}
\section{Lecture 9}

\begin{theorem}
RG has QE
\end{theorem}
\begin{proof}
\underline{Option 1}: Theorem 6.2(iii). Consider $RG \cup \D(\A)$, where $\A$ is a finite graph (see last exercise).

\underline{Option 2}: Theorem 6.2(ii). Fix $\M,\N\models \textrm{RG}$ and $\A\subset \M\cap \N$. Fix a q.f. formula $\phi(x_1,\dots,x_n,y)$ and $a_1,\dots,a_n\in A$. Assume $\M\models \phi(\bar{a},b)$ for some $b\in M$. Want to show that $\N\models \exists \phi(\bar{a},y)$.

Write $\phi(\bar{x},y)$ as $$\bigvee_{s=1}^{k}\bigwedge_{t=1}^{\ell_s}\theta_{s,t}(\bar{x},y)$$ where each $\theta_{s,t}$ is atomic or negated atomic (this is known as \textit{disjunctive normal form}). $\exists s\le k$ such that $$\M\models\bigwedge_{t=1}^{\ell_s}\theta_{s,t}(\bar{a},b)$$
Each $\theta_{s,t}$ is one of: $x_i = x_j, x_i = y,E(x_i,x_j),E(x_i,y)$, or the negation of one of the above. If we have $x_i = y$ appearing then $b = a_i\in A\subset N$. So $\N \models \phi(\bar{a},b)$ since $\phi$ is q.f.. We can assume that no $x_i = y$ appears. Let $X = \{a_i:\M\models E(a_i,b)\}$ and $Y = \{a_i : \M\models \neg E(a_i,b)\}$. $X$ and $Y$ are finite disjoint subsets of $A\subset N$. Choose $c\in N$ such that $\N\models E(a_i,c)$ iff $a_i \in X$, and $c\not\in\{a_1,\dots,a_n\}$. We do this by finding a new element connected to everything in $X$ and nothing in $Y$, and ensure that $c$ is not connected to this new vertex either.

Then $\N\models \bigwedge_{t}\theta_{s,t}(\bar{a},c)$ (check). So $\N\models \phi(\bar{a},c)$.
\end{proof}

\section*{Types}

\underline{Motivation}: Given $\M$, we want to understand ``potential behaviour'' of elements in elementary extensions.

\underline{Terminology}: Given an $\L$-structure $\M$ and $A\subset M$, we call an $\L_A$-formula an \undf{$\L$-formula with parameters from $A$}. We write these as $\phi(\bar{x},\bar{a})$ where $\phi(\bar{x},\bar{y})$ is an $\L$-formula and $\bar{a}$ is from $A$. (Identifiy $a$ with $\underline{a}^\M$ - this should not cause problems in most cases).

Now suppose $\M$ is an $\L$-structure and $\N\succeq\M$. If $a\in N\backslash M$ then the $\L_N$-formula $x = a$ doescribes the new behvaiour in a trivial way.

OTOH: If $\phi(x)$ is an $\L$-formula with parameters from $\M$ and $\N\models \phi(a)$ for some $a\in N$, then $\N\models \exists x\phi(x)$ so $\M\models \exists \phi(x)$.

\underline{Idea}: New behaviour cannot be controlled with one formula at a time.

\underline{Notation}: Let $p$ be a set of formulas in free variables $x_1,\dots,x_n$. We also write $p(x_1,\dots,x_n)$. Given $\M$ and $a_1,\dots,a_n\in M$, we write $\M\models p(a_1,\dots,a_n)$ if $\M\models \phi(\bar{a})$ for all $\phi\in p$. We say ``$\bar{a}$ realises $p$ (in $\M$)''. Also write $\bar{a}\models p$. We call $p$ \undf{consistent} if it is realised in some structure.

\begin{remark*}[Exercise]
$p$ is consistent iff every finite subset of $p$ is consistent.
\end{remark*}

\begin{defin}[$n$-type]
Let $\M$ be an $\L$-structure and fix $A\subset M$. An \undf{$n$-type over $A$ w.r.t. $\M$} is a set $p$ of $\L$-formulae with parameters from $A$ in free variables $x_1,\dots,x_n$ such that $p\cup \Th_A(\M)$ is consistent.

We say $p$ is \undf{complete} if, for every $\L_A$-formula $\phi(x_1,\dots,x_n)$, either $\phi \in p$ or $\neg \phi \in p$.

Let $S_n^\M(A)$ denote the set of all complete $n$-types over $A$ w.r.t. $\M$.
\end{defin}

\begin{ex}
Given $a_1,\dots,a_n\in M$, let $\textrm{tp}^\M(a_1,\dots,a_n/A)$ be the set of all $\L_A$-formulae $\phi(x_1,\dots,x_n)$ such that $\M\models \phi(\bar{a})$. Then $\tp^\M(\bar{a}/A)\in S_n^\M(A)$, and $\bar{a}\models \tp^\M(\bar{a}/A)$.
\end{ex}
\begin{prop}
If $p \in S_n^\M(A)$, then there is $\N\succeq\M$ with $|N| \le |M| + |\L|$, and $\bar{a}\in N^n$ such that $p = \tp^\N(\bar{a}/A)$.
\end{prop}
\begin{proof}
By assumption, $p\cup\Th_A(\M)$ is consistent. We want to show that $p \cup \Th_M(\M)$ is consistent, which is not \it{quite} what we have.

Fix $\Sigma \subset p \cup \Th_M(\M)$ finite. So $\Sigma \subset p \cup \{\phi_1,\dots,\phi_t\}$ where $\phi_i$ is an $\L_M$-sentence, and $\M\models \phi_i$. Let $\phi^\ast$ be $\bigwedge_{i=1}^{t}\phi_i$. We can write $\phi^\ast$ as $\phi(\underline{b}_1,\dots,\underline{b}_m)$ where $b_1,\dots,b_m\in \M\backslash A$ and $\phi(x_1,\dots,x_n)$ is an $\L_A$-formula. Since $\M\models \phi(b_1,\dots,b_m)$, we have that $\M\models \exists \bar{v}\phi(v_1,\dots,v_m)$. So $\exists \bar{v}\phi(\bar{v})\in \Th_A(\M)$. Since $p\cup \Th_A(\M)$ is consistent, there is $\N\models \Th_A(\M)$ and $\bar{a} \in N^n$ such that $\N\models p(\bar{a})$.

Since $\N\models \exists \bar{v}\phi(\bar{v})$, there is $\bar{c} \in N^m$ such that $\N\models \phi(\bar{c})$. Expand $\N$ to an $\L_M$-structure such that $\underline{b}_i^\N = c_i$ and $\underline{b}^\N$ is arbitrary for $b\in M\backslash (A\cup \{b_1,\dots,b_m\})$. Then $\N\models \phi(\underline{b}_1,\dots,\underline{b}_m)$, \it{i.e.} $\N\models \phi^\ast$. So $\N\models \Sigma$.
\end{proof}

\section{Lecture 10}

\begin{remark}
If $\M\le \N$ and $A \subset M$, then $S_n^\M(A) = S_n^\N(A)$.
\end{remark}
\begin{proof}
It is enough to show that $\Th_A(\M) = \Th_A(\N)$. If $\phi(x_1,\dots,x_m)$ is an $\L$-formula $a_1,\dots,a_m\in A$, then $\M\models \phi(\bar{a})$ iff $\N\models \phi(\bar{a})$ since $\M\preceq \N$.
\end{proof}
\begin{remark}
	$p$ is an $n$-type over $A$ wrt $\M$ iff for any finite $q \subset p$, $\exists \bar{a}\in \M^n$ such that $\bar{a}\models q$.
\end{remark}
\begin{proof}
($\implies$): Choose $\N\succeq \M$ realising $p$. Fix finite $a \subset p$. Let $\phi(\bar{x})$ be the conjunction of all $\L_A$-formulae in $q$. $\N\models \exists \bar{x}\phi(\bar{x})$, the $\L_A$-sentence. So $\M\models \exists \bar{x}\phi(\bar{x})$ since $\N\succeq \M$.
\end{proof}

\begin{ex}
Suppose $K\models \acf$, and $A\subset K$. We aim to describe $S_n^K(A)$.

Fix $p \in S_n^K(A)$. By QE, we only need to consider q.f. formulae in $p$.

\underline{Note}: $\phi\land \psi \in p$ iff $\phi,\psi \in p$, and $\neg \phi \in p$ iff $\phi \not\in p$.

So it suffices to focus on atomic formulae, in variables $x_1,\dots,x_n$ with parameters from $A$, \ie polynomial equations in $F[\bar{x}]$, where $F$ is the subfield generated by $A$. Le t $I_p = \{f(\bar{x})\in F[\bar{x}]:f(\bar{x}) = 0\textrm{ is in }p\}$. Then $I_p$ is a prime ideal. In fact, this map $p\mapsto I_p$ is a bijection between $S_n^K(A)$ and the set of prime ideals in $F[\bar{x}]$ (\ie\ Spec$(F[\bar{x}])$).

For example, the set of $1$-types with all parameters $S_1^K(K) = \{p_a:a\in K\}\cup\{q\}$ where $p_a$ contains $x = a$, and $q$ contains $x\ne a$ for all $a \in K$. In particular, $|S_1^K(K)| = |K|$.
\end{ex}
\begin{ex}
Let $\M\models \rg$. We will describe $S_1^\M(M)$.

For $a\in M$, let $p_a\in S_1^\M(M)$ be the unique type containing $x = a$.

Why is this unique? Suppose $x = a$ is in $p,q$ distinct. Choose $\phi(x)$ such that $\phi(x) \in p$ and $\neg\phi(x)\in q$. Then $x =a \land \phi(x)$, $x = a \land \neg\phi(x)$ both consistent. $\false$

For $V\subset M$, set $p_V$ as follows:
\begin{align*}
p_V =& \{x\ne a: a\in M\}\\
\cup&\{E(x,a):a\in V\}\\
\cup&\{\neg E(x,a):a\in M\backslash V\}
\end{align*}

Then $p_V$ is a $1$-type wrt $\M$, and by QE this determines a unique, complete $1$-type. So we have $S_1^\M(M) = \{p_a:a\in M\}\cup\{p_V:V\subset M\}$, and $|S_1^\M(M)| = 2^{|M|}$ \ie there is a type for every subset of the model.
\end{ex}
\underline{Note}: In general, $|S_n^\M(A)| \le 2^{|A| + |\L| + \aleph_0}$

\subsection{Type Spaces}

Let $\M$ be an $\L$-structure, $A\subset M$. Given $\L_A$-formula $\phi(x_1,\dots,d_n)$, define $[\phi(\bar{x})] = \{p \in S_n^\M(A):\phi(\bar{x}) \in p\}$.

\underline{Basic Properties}:
\begin{enumerate}[label = \arabic*.]
	\item $\sman = \left[\bigwedge_{i=1}^{n}x_i = x_i\right]$
	\item $\left[\phi(\bar{x})\land\psi(\bar{x})\right] = [\phi(\bar{x})]\cap [\psi(\bar{x})]$
	\item $\left[\neg \phi(\bar{x})\right] = \sman\backslash \left[\phi(\bar{x})\right]$
\end{enumerate}
We then define a topology on $\sman$ by using $[\phi(\bar{x})]$ for all $\L_A$-formulae $\phi(\bar{x})$ as a basis of open sets. Here, $S$ is for ``Stone''; see: Stone space.

\begin{theorem}
$\sman$ is a totally disconnected compact Hausdorff space.
\end{theorem}
\begin{proof}
Showing that the topology is well-defined is an exercise (Sheet 2 \#7).

\underline{Hausdorff}: Fix distinct $p,q \in \sman$. Find $\phi(\bar{x})$ such that $\phi(\bar{x}) \in p$ and $\neg\phi(\bar{x}) \in q$. Then $p \in [\phi(\bar{x})]$ and $q \in [\neg \phi(\bar{x})]$.

\underline{Compactness}: It suffices to consider open covers consisting of basic open sets. Fix a collection of $\L_A$-formulae $(\phi_i(\bar{x}))_{i\in I}$ such that $\sman = \bigcup_{i=I}\left[\phi_i(\bar{x})\right]$.

Let $\Sigma = \{\neg \phi_i(\bar{x}):i\in I\}$. Then $\Sigma \cup \Th_A(\M)$ is inconsistent. Otherwise, $\N\models \Th_A(\M)$ and $\bar{a}\in N^n$ such that $\bar{a} \models \Sigma$. Let $p = \tp^\N(\bar{a}/A)$. Then $p \in \sman$ but $p \in [\phi_i(\bar{x})]$ for all $i \in I\ \false$.

Then by the Compactness Theorem, there is some finite $I_0 \subset I$ such that $\{\neg \phi_i(\bar{x}):i\in I_0\}\cup\Th_A(\M)$ is inconsistent. $(\ast)$

We show $\sman = \bigcup_{i\in I_0}[\phi_i(\bar{x})]$. Fix $p \in \sman$. Choose $\N\models \Th_A(\M)$ and $\bar{a} \in \N^n$ such that $\bar{a}\models p$. By $(\ast)$, there exists $i \in I_0$ such that $\N\models \phi_i(\bar{a})$. So $\phi_i(\bar{x})\in p$ (since $p$ is complete). So $p \in [\phi_i(\bar{x})]$.

\underline{Totally Disconnected}: A compact Hausdorff space is totally disconnected iff any two distinct  points can be separated by clopen sets (not just open sets). Note that in this case the basic open sets are clopen (they are closed because their compliment is open).
\end{proof}

We now have a long-term goal: to analyse countable models of complete theories.

For example, DLO and RG are $\aleph_0$-categorical. For ACF$_p$, the countable models are $K_\alpha$ for $\alpha \in \N\cup\{\aleph_0\}$ where $K_\alpha$ has transcendence degree $\alpha$.

\subsection*{Saturated Models}

\begin{defin}
Let $\M$ be an infinite $\L$-structure, and let $\kappa \ge |\L| + \aleph_0$. Then $\M$ is \undf{$\kappa$-saturated} if for any $A\subset M$, with $|A| < \kappa$, every type in $\sman$ is realised in $\M$ for all $n\ge 1$.
\end{defin}

\begin{remark}\ 
\begin{enumerate}[label = \alph*)]
	\item Restricting to complete types is not important since since any $n$-type over $A$ wrt $\M$ can be extended to some $p \in S_n^\M(A)$ (Sheet 2 \#6).
	\item (Sheet 2 \#8) It suffices to assume $n = 1$ to prove $\kappa$-saturation.
	\item If $\M$ is $\kappa$-saturated then $|M| \ge \kappa$.
	\begin{proof}
		$\{x \ne a:a\in \M\}$ is a $1$-type over $M$ wrt $\M$, and is not realised in $\M$.
	\end{proof}
\end{enumerate}
\end{remark}
\begin{defin}[Partial elementary map, $\kappa$-homogeneous]
Let $\M,\N$ be $\L$-structures, and suppose $A\subset M,B\subset N$. Then a function $f:A\ra B$ is \undf{partial elementary} if for any $\L$-formula $\phi(x_1,\dots,x_n)$ and $a_1,\dots,a_n\in A$, $\M\models \phi(\bar{a})$ iff $\N\models \phi(f(\bar{a}))$.

Given $\kappa\ge |\L| + \aleph_0$, $\M$ is \undf{$\kappa$-homogeneous} if, for any $A\subset M$ with $|A| < \kappa$, any partial elementary $f:A\ra M$, and any $c\in M$, there exists $d \in M$ such that $f\cup \{(c,d)\}$ is partial elementary. That is to say, any partial elementary map can be extended.

Let $T$ be a complete $\L$-theory. Fix $\M,\N\models T$. Then $S_n^\M(\emptyset) = S_n^\N(\emptyset)$ since $\Th(\M) = \Th(\N) = T$.
\end{defin}
\begin{defin}
$S_n(T) \coloneqq S_n^\M(\emptyset)$ for some (equivalently, any) $\M\models T$.
\end{defin}
\begin{prop}
$\M\models T$ is $\aleph_0$-saturated iff $\M$ is $\aleph_0$-homogeneous and $\M$ realises all types in $S_n(T)$ for all $n\ge 1$.
\end{prop}
\begin{proof}
($\implies$) Assume $\M\models T$ is $\aleph_0$-saturated. Then $\M$ realises all types in $S_n(T)$ sicne $\emptyset$ is finite. Fix finite $A\subset M$, partial elementary $f:A\ra M$, and $c\in M$. Define $p\in S_1(f(A))$ such that $\phi(x,f(\bar{a}))\in p$ iff $\M\models \phi(c,\bar{a})$.

\underline{Notation}: $f(\tp^\M(c/A)) = p$. $p \in S_1(f(A))$, \it{e.g.} $p$ is finitely satisfiable in $\M$: if $\phi(x,f(\bar{a}))\in p$ then $\M\models \exists x \phi(x,\bar{a})$, so $\M\models \exists x \phi(x,f(\bar{a}))$.

Let $d \in M$ realise $p$. Then $f\cup\{(c,d)\}$ is partial elementary.

($\impliedby$) Fix $a_1,\dots,a_n\in M$ and $p\in S_1^\M(\{a_1,\dots,a_n\})$. Want to show that $\M$ realises $p$.

Set $q = \{\phi(x,y_1,\dots,y_n):\phi(x,\bar{a})\in p\}$. Then $q \in S_{n+1}(T)$. Let $d,b_1,\dots,b_n \in M$ such that $(d,\bar{b})\models q$. Then $\tp^\M(\bar{b}) = \tp^\M(\bar{a})$. So $f:b_i\ra a_i$ for all $i$ is partial elementary.

Let $c\in M$ such that $f\cup \{(d,c)\}$ is partial elementary. Then $\tp^\M((c,\bar{a})) = \tp^\M((d,\bar{b})) = q$. So $(c,\bar{a})\models q$, \ie\ $c\models p$.
\end{proof}

This tells us that if we want to build a saturated model, we at least need to be able to build homogeneous models.

\underline{Notation}: Given $\M$, $\bar{a},\bar{b}\in M^n$, write $\bar{a}\equiv^\M\bar{b}$ if $\tp^\M(\bar{a}) = \tp^\M(\bar{b})$. So $\M$ is $\aleph_0$-homogeneous iff whenever $\bar{a}\equiv^\M\bar{b}$ and $c\in M$, there exists $d\in M$ such that $(\bar{a},c) \equiv^\M (\bar{b},d)$.

\begin{lemma}
For any $\M\models T$, there is $\N\succeq\M$ such that $|N|\le |M| + |\L|$ and $\N$ is $\aleph_0$-homogeneous.
\end{lemma}
\begin{proof}
\underline{Claim}: For any $\M\models T$, there is $\N\succeq \M$ such that $|N|\le |M|+|\L|$ and $\forall \bar{a},\bar{b},c$ from $M$, such that $\bar{a}\equiv^\M\bar{b}$, there exists $d\in N$ such that $(\bar{a},c) \equiv^\N(\bar{b},d)$.

\underline{Proof of Claim}: Enumerate all $(\bar{a},\bar{b},c)$ as $(\bar{a}_\alpha,\bar{b}_\alpha,c_\alpha)_{\alpha < |M|}$. We build an elementary chain $(M_\alpha)_{\alpha < |M|}$ such that $\M_0 = \M$ and $|\M_\alpha| \le |M| + |\L|$ for all $\alpha$.

For $\alpha$ a limit, let $\M_\alpha = \bigcup_{i<\alpha}M_i$. Then $|M_\alpha| \le |\alpha|(|M|+|\L|) = |M| + |\L|$.

Given $M_\alpha$, look at $(\bar{a}_\alpha,\bar{b}_\alpha,c_\alpha)$. We have $\bar{a}_\alpha \equiv^\M\bar{b}_\alpha$. Let $f_\alpha : \bar{a}_\alpha \ra \bar{b}_\alpha$ be partial elementary. Apply Prop to find $\M_{\alpha+1} \ge \M_\alpha$ such that $|M_{\alpha+1}|\le |M_\alpha| + |\L| \le |M| + |\L|$, and there exists $d \in M_{\alpha+1}$ realising $f_\alpha(\tp(c_\alpha/\bar{a}_\alpha))$. Then $(\bar{a}_\alpha,c_\alpha) \equiv^\M(\bar{b}_\alpha,d)$. Let $\N = \bigcup_{\alpha < |M|}M_\alpha$. Then $|N| \le |M|(|M|+|\L|) = |M| + |\L|$.\qed

We now build $\M = \N_0 \le \N_1\le \N_2\le \dots$ such that $|N_i| \le |M|+|\L|$, and $\forall \bar{a},\bar{b},c$ from $\N_i$ if $\bar{a}\equiv \bar{b}$ then there exists $d\in N_{i+1}$ such that $(\bar{a},c) \equiv (\bar{b},d)$. We do this by iterating the claim. Then let $\N = \bigcup_{i<\aleph_0}\N_i$. Then $|N|\le |M| + |\L|$.

$\N$ is $\aleph_0$-homogeneous: any $\bar{a},\bar{b},c$ from $\N$ all lie in $N_i$ for some $i$, so we find a solution in $\N_{i+1}$.
\end{proof}

\section{Lecture 12}
%fine, I'll start using his system...
\underline{Recall}: $\M$ is $\kappa$-saturated $\implies |\M| \ge \kappa$.

\begin{defin}[Saturated]
$\M$ is \undf{saturated} if it is $|M|$-saturated.
\end{defin}

Let $T$ be a complete consistent theory with infinite models and $\L$ is countable.

\begin{theorem}
$T$ has a countable, saturated model iff $S_n(T)$ is countable for all $n\ge1$.
\end{theorem}
\begin{proof}
($\implies$) If $\M\models T$ is countable and saturated, then $S_n(T)$ is countable since $M^n$ is countable and $p\mapsto \bar{a}\models p$ is injective.

($\implies$) Enumerate $\bigcup_{n\ge 1}S_n(T) = \{p_1,p_2,p_3,\dots\}$. Fix $\M_0\models T$ countable. Build a chain $\M_0\preceq\M_1\preceq\M_2\preceq\dots$ such that $\M_i$ realises $p_i$ and is countable (by Prop 8.4).

Let $\N = \bigcup_{n\ge 1}\M_n$. $\N\models T$ is countable. Apply Lemma 10.6 to obtain $\M\succeq \N$ countable and $\aleph_0$-homogeneous. So $\M$ is saturated by Prop 10.5.
\end{proof}

\begin{ex}\ 
\begin{enumerate}[label=(\arabic*)]
	\item $\acf_p$. Let $F = \Q$ if $p = 0$, and $\F_p$ otherwise. Then $S_n(T)\leftrightarrow \textrm{Spec}(F[x_1,\dots,x_n])$. So $S_n(T)$ is countable since every ideal in $F[\bar{x}]$ is finitely generated. So $\acf_p$ has a countable saturated model, which is the model of countably infinite transcendence degree $\aleph_0$, $\overline{F[x_1,x_2,\dots]}$. Note that if $K\models \acf_p$ and trdeg$(K) = n < \aleph_0$, then the $(n+1)$-type saying ``$x_1,\dots,x_{n+1}$ algebraically independent'' is not realised in $K$.
	
	\item TFDAG (torsion-free divisible abelian groups) has a countable saturated model, which is the $\Q$-vector space of dimension $\aleph_0$.
	
	\item Let $T = \Th(\Z,+,0)$. Given $n\ge 1$, let $\delta_n(x)$ be the $\L$-formula $\exists y(x = ny)$. Let $\mathbb{P}$ be the set of primes. Given $X\subset \mathbb{P}$, let $q_x = \{\delta_n(x):n\in X\}\cup\{\neg\delta_n(x):n\in \mathbb{P}\backslash X\}$. Note $q_X$ is finitely satisfiable in $\Z$. So we can extend it to a complete type; there exists $p_X \in S_1(T)$ such that $q_X \subset p_X$ (Sheet 2).
	
	If $X \ne Y$ then $p_X \ne p_Y$, so $|S_1(T)| = 2^{\aleph_0}$. So $T$ does not have a countable saturated model.
\end{enumerate}
\end{ex}

\begin{prop}
If $\M,\N\models T$ are countable and saturated, then $\M\cong\N$.
\end{prop}
\begin{proof}
(Sketch). Enumerate $\M = \{a_n:n\ge 1\},\N = \{b_n:n\ge 1\}$. Build partial elemenatary maps $f_0\subset f_1\subset \dots$ such that $a_n \in \textrm{dom}(f_n)$, $b_n\in \textrm{Im}(f_n)$, dom$(f_n)$ is finite.

Let $f_0 = \emptyset$. Note that this is partial elementary since $\M\equiv \N$.

Given $f_n$, let $d\in N$ realise $f_n\left(\tp\left(a_{n+1}/\textrm{dom}(f_n)\right)\right)$. Now let $c\in M$ realise $f_\ast^{-1}(\tp(b_{n+1}/\textrm{Im}(f_n)\cup\{d\}))$, where $f_\ast = f_n\cup\{(a_{n+1},d)\}$. Let $f_{n+1} = f_\ast \cup \{(c,b_{n+1})\}$.

Then let $f = \bigcup f_n$. Then by construction $f$ is an $\L$-isomorphism from $\M$ to $\N$.
\end{proof}

\subsection*{Omitting Types}

Let $\M$ be an $\L$-structure.

\begin{defin}[Isolated type]
$p\in \sman$ is \undf{isolated} if it is an isolated point wrt the Stone space topology, \ie\ $\{p\}$ is open.
\end{defin}
\begin{remark*}[Example]
If $a\in A\subset M$, then $\tp^\M(a/A)$ is isolated since $\{\tp^\M(a/A)\} = [x = a]$.
\end{remark*}
\begin{prop}
|Given $p\in \sman$, TFAE:
\begin{enumerate}[label=\roman*)]
	\item $p$ is isolated
	\item $\{p\} = [\phi(\bar{x})]$ for some $\L_A$-formula $\phi(\bar{x})$ (we say $\phi(\bar{x})$ \undf{isolates} $p$)
	\item There is an $\L_A$-formula $\phi(\bar{x})\in p$ such that for any $\L_A$-formula $\psi(\bar{x})$, if $\psi(\bar{x})\in p$ then $\Th_A(\M)\models \forall \bar{x}(\phi(\bar{x})\ra\psi(\bar{x}))$
\end{enumerate}
\end{prop}
\begin{proof}
(i)$\iff$(ii) follows by definition of the basis for the topology.

(ii)$\implies$(iii). Assume $\phi(\bar{x})$ isolates $p$. Fix an $\L_A$-formula $\psi(\bar{x})\in p$. WTS $\M\models \forall \bar{x}(\phi(\bar{x})\ra\psi(\bar{x}))$. Suppose $\bar{a}\in M^n$ such that $\M\models \phi(\bar{a})$. Then $\tp^\M(\bar{a}/A)\in [\phi(\bar{x})]$, so $p = \tp^\M(\bar{a}/A)$. So $\M\models \psi(\bar{a})$.

(iii)$\implies$(ii). Assume (iii). Then for all $\L_A$-formulae $\psi(\bar{x})\in p$, we have $\phi(\bar{x})\subset[\psi(\bar{x})]$ since any $q \in [\phi(\bar{x})]$ is realised by $\bar{a} \in N^n$ in some $\N\models \Th_A(\M)$. So $\N\models \psi(\bar{a})$. So $q \in [\psi(\bar{x})]$. If $q \in [\phi(\bar{x})]$ then $p\subset q$. So $[\phi(\bar{x})] = \{p\}$.
\end{proof} 
\section{dummy}
\section{Lecture}

Let $T$ be a complete, consistent theory.

\begin{prop}
If $p \in S_n(T)$ is isolated then $p$ is realised in any $\M\models T$.
\end{prop}
\begin{proof}
Fix $p \in S_n(T)$ isolated by $\phi(\bar{x})\in p$. Fix $\M\models T$. By Prop 8.4, there is $\N\succeq \M$ realising $p$. So $\N\models \exists \bar{x}\phi(\bar{x})$. So $\M\models \exists \bar{x}\phi(\bar{x})$. Fix $\bar{a}\in M^n$ such that $\M\models \phi(\bar{a})$. We then show $\bar{a}\models p$.

Fix $\psi(\bar{x})\in p$. Then $T\models \forall \bar{x}(\phi(\bar{x})\ra\psi(\bar{x}))$. So $\M\models \psi(\bar{a})$.
\end{proof}

\begin{theorem*}[Omitting Types Theorem]
Assume $\L$ is countable, and $p \in S_n(T)$ is non-isolated. Then there is countable $\M\models T$ such that $p$ is not realised in $\M$ (\it{i.e.} $\M$ \undf{omits} $p$)
\end{theorem*}

This is a relatively complicated argument.

\begin{proof} (Henkin construction; non-examinable)
Let $\L^\ast = \L\cup C$, where $C$ is a countably infinite set of new constant symbols.

An $\L^\ast$-theory $T^\ast$ has the \undf{witness property} if for any $\L^\ast$-formula $\phi(x)$ there is a constant symbol $c\in C$ such that $T^\ast \models(\exists x\phi(x)\ra \phi(c))$.

\underline{Fact}: (Part II) Suppose $T^\ast$ is a complete, satisfiable $\L^\ast$-theory with the witness property.

Define $\sim$ on $C$ such that $c\sim d$ iff $T^\ast \models c = d$. Let $M = C/\sim$ and define an $\L^\ast$-structure $\M$ on $M$ such that
\begin{align*}
	\left\lbrace \begin{array}{ll} c^\M = [c] \textrm{ ($\sim$-equivalence class)} \\ f^\M([c_1],\dots,[c_n]) = [d] \textrm{ iff } T^\ast \models f(c_1,\dots,c_n) = d\\
	R^\M = \{([c_1],\dots,[c_n])\in M^n:T^\ast\models R(c_1,\dots,c_n)\} \end{array}\right.
\end{align*}
Then $\M$ is a well-defined $\L^\ast$ structure and $\M\models T^\ast$ - this requires checking. In particular, for any $\L$-formula $\phi(x_1,\dots,x_n)$ and $c_1,\dots,c_n\in C$, $\M\models \phi([c_1],\dots,[c_n])$ iff $T^\ast\models \phi(c_1,\dots,c_n)$, the $\L^\ast$-sentence.

We call $\M$ the \undf{Henkin model} of $T^\ast$.

Fix $p \in S_N(T)$ non-isolated.

\underline{Goal}: Build a complete, satisfiable $L^\ast$-theory $T^\ast\supseteq T$ with the witness property such that $\forall c_1,\dots,c_n\in C$ there is $\psi(\bar{x})\in p$ such that $T^\ast\models \neg\psi(c_1,\dots,c_n)$.

Then given such a $T^\ast$, the Henkin model omits $p$ since it denies some formula from $p$ on every tuple of (equivalence classes of) constants.

Enumerate all $\L^\ast$-sentences $\phi_0,\phi_1,\dots$ and also enumerate $C^n = \{\bar{c}_0,\bar{c}_1,\dots\}$. We build a satisfiable $\L^\ast$-theory $T^\ast = T\cup\{\theta_0,\theta_1,\dots\}$ such that
\begin{enumerate}[label=\arabic*)]
	\setcounter{enumi}{-1}
\item $\models \theta_i\ra\theta_j$ for all $i > j$ (this is for convenience)
\item Either $\models \theta_{3i+1}\ra\phi_i$ or $\models \theta_{3i+1}\ra\neg\phi_i$ (completeness)
\item If $\phi_i$ is $\exists v\psi(v)$ for some $\psi$ and $\models \theta_{3i+1}\ra\phi_i$, then $\models \theta_{3i+2}\ra\psi(c)$ for some $c\in C$

(witness property: $T^\ast \models (\exists v\psi(v)\ra\psi(c))$ as $\N\models T^\ast$ and $\N\models \exists v\psi(v)$, and $\N\models \phi_i$ so $\N\models \psi(c)$)

\item $\models \theta_{3i+3}\ra\neg\psi(\bar{c}_i)$ for some $\psi(\bar{x})\in p$ (omit $p$)
\end{enumerate}

We construct this model inductively. Let $\theta_0$ be $\forall v(v=v)$ ($\theta_0$ does nothing). Now suppose we have $\theta_0,\dots,\theta_m$ as above.

\underline{Case 1}: $m+1 = 3i+1$. If $T\cup\{\theta_m,\phi_i\}$ is satisfiable then $\theta_{m+1}$ is $\theta_m\land\phi_i$. Otherwise, let $\theta_{m+1}$ be $\theta_m\land \neg\phi_i$. Then $T\cup\{\theta_{m+1}\}$ is satisfiable, since in either case we're adding the conjunction of two axioms. This relies on the inductive hypothesis that $T\cup\{\theta_m\}$ is satisfiable.

\underline{Case 2}: $m+1 = 3i+2$. Suppose $\phi_i$ is $\exists v\psi(v)$ for some $\psi$, and $\models \theta_m\ra \phi_i$ (if this fails we simply let $\theta_{m+1}$ be $\theta_m$). Choose $c\in C$ not used in $\theta_m$. Let $\theta_{m+1}$ be $\theta_m\land\psi(c)$. This satisfies the above hypotheses. Moreover, $T\cup\{\theta_{n+1}\}$ is satisfiable: Let $\N\models T\cup\{\theta_m\}$. Then $\N\models \phi_i$. Choose $a\in N$ such that $\N\models \psi(a)$. Re-interpret $c^\N = a$. Then $\N\models T\cup\{\theta_{m+1}\}$.

\underline{Case 3}: $m+1 = 3i+3$. Let $\bar{c}_i = (c_1,\dots,c_n)$. WLOG assume $x_1,\dots,x_n$ are not used in $\theta_m$. We build an $\L$-formula $\phi(x_1,\dots,x_n)$ from $\theta_m$ as follows:
\begin{itemize}
	\item replace $c_t$ by $x_t$ for all $t \le n$.
	\item then replace any $c\in C\backslash\{c_1,\dots,c_n\}$ by a new variable $v_c$ and add $\exists v_c$ to the front.
\end{itemize}
Then $\phi(\bar{x})$ does not isoalte $p$. By Prop 11.6, $\exists \psi(\bar{x})\in p$ such that $\not\models \forall\bar{x}(\phi(\bar{x})\ra\psi(\bar{x}))$.

Let $\theta_{m+1}$ be $\theta_m\land \neg\psi(c_1,\dots,c_n)$. $T\cup\{\theta_{m+1}\}$ is satisfiable: Choose $\N\models T$ such that $\N\not\models \forall \bar{x}(\phi(\bar{x})\ra\psi(\bar{x}))$. Pick $\bar{a}\in N^n$ such that $\N\models \phi(\bar{a})\land \neg\psi(\bar{a})$. Make $\N$ an $\L^\ast$-structure:

Interpret $c_t^\N$ as $a_t$. If $c\in C\backslash\{c_1,\dots,c_n\}$, then $c^\N$ is a witness to $\exists v_c$ in $\N\models \phi(\bar{a})$. Then $\N\models \theta_m$ and $\N\models \neg\psi(c_1,\dots,c_t)$. So $\N\models \theta_{m+1}$.
\end{proof}

\section{Prime \& Atomic Models}

$T$ is a complete, consistent $\L$-theory with infinite models.

\begin{defin}\ 
\begin{enumerate}[label=\arabic*)]
	\item $\M$ is \undf{atomic} if every $n$-type over $\emptyset$ realized in $\M$ is isolated
	\item $\M$ is \undf{prime} if for any $\N\models T$ there is an elementary embedding $\M\xhookrightarrow{}\N$
\end{enumerate}
\end{defin}

\begin{remark*}[Example]
$K\models \acf_0$. Then $\bar{\Q}\subseteq K$. So $\bar{\Q}\preceq K$ by QE.
\end{remark*}
\begin{theorem}
Assume $\L$ is countable. Then $\M\models T$ is prime iff it is countable and atomic.
\end{theorem}
So up to issues of cardinality, we can think of prime and atomic as the same thing.
\begin{proof}
$\implies$: Assume $\M\models T$ is prime. Then $\M$ is countable since $T$ has a countable model (by DLST), into which $\M$ embeds. Suppose $p \in S_n(T)$ is non-isolated. By OTT there is some $\N\models T$ omitting $p$. Since $\M\preceq\N$, $\M$ omits $p$. So $\M$ is atomic.

$\impliedby$: Assume $\M\models T$ is countable and atomic. Fix $\N\models T$. WTS $\M\preceq\N$. Enumerate $M = \{a_n:n\ge 1\}$. We build partial elementary $f_0\subset f_1\subset f_2\dots$ from $M$ to $N$ such that $a_n\in \textrm{dom}(f_n)$ and dom$(f_n)$ is finite. Then $f = \bigcup f_n$ is an elementary embedding from $\M$ to $\N$.

We start as before with $f_0 = \emptyset$, which is partial elementary since $\M\equiv \N$. Now suppose we have $f_n$. Let $\phi(x_1,\dots,x_{n+1})$ be a n $\L$-formula isolating $\tp^\M(a_1,\dots,a_{n+1})$, which exists since $\M$ is atomic. $\M\models \exists x_{n+1}\phi(a_1,\dots,a_n,x_{n+1})$, so $\N\models \exists x_{n+1}\phi(f_n(a_1),\dots,f_n(a_n),x_{n+1})$. Pick $b\in N$ such that $\N\models \phi(f(a_1),\dots,f_n(a_n),b)$. Fix an $\L$-formula $\psi(x_1,\dots,x_{n+1})$ such that $\M\models \psi(a_1,\dots,a_{n+1})$. WTS $\N\models \psi(f_n(a_1),\dots,f_n(a_n),b)$.

By Prop 11.6, $T\models \forall x_1\dots\forall x_{n+1}(\phi(\bar{x})\ra\psi(\bar{x}))$. So $\N\models \psi(f_n(a_1),\dots,f_n(a_n),b)$. So $f_{n+1} = f_n\cup\{(a_{n+1},b)\}$ is partial elementary.
\end{proof}

\begin{theorem}
Assume $\L$ is countable. TFAE:
\begin{enumerate}[label=\roman*)]
	\item $T$ has a prime model
	\item $T$ has an atomic model
	\item For all $n\ge 1$, the isolated types in $S_n(T)$ are dense.
\end{enumerate}
\end{theorem}
\begin{proof}
We have (i)$\iff$(ii) by Theorem 13.2 (and Sheet 1 \#9).

(ii)$\implies$(iii): Let $\M\models T$ be atomic. Fix $n\ge 1$, and an $\L$-formula $\phi(\bar{x})$ such that $[\phi(\bar{x})]\ne \emptyset$. WTS $[\phi(\bar{x})]$ contains an isolated type. Note $\M\models \exists \bar{x}\phi(\bar{x})$. Choose $\bar{a}\in M^n$ such that $\M\models \phi(\bar{a})$. Then $\tp^\M(\bar{a})$ is isolated (since $\M$ is atomic) and it is in $[\phi(\bar{x})]$.

(iii)$\implies$(ii): [Henkin construction, non-examinable] Let $\L^\ast = \L\cup\{c_1,c_2,\dots\}$. Let $\phi_0,\phi_1,\dots$ enumerate all $\L^\ast$-sentences. We build $T^\ast = T \cup \{\theta_0,\theta_1,\dots\}$ such that $T^\ast$ is complete, satisfiable, has the witness property, and such that the Henkin model of $T^\ast$ is atomic (as an $\L$-structure). This is similar to the OTT.

Let $\theta_0$ be $\forall x(x=x)$. Suppose we have $\theta_0,\theta_1,\dots,\theta_m$.

The cases $m+1\in \{3i+1,3i+2\}$ are identical to the proof of OTT.

\underline{Case $m+1 = 3i+3$}: Choose $n\ge i$ such that all new constants used in $\theta_n$ are in $\{c_1,\dots,c_n\}$. Let $\psi(x_1,\dots,x_n)$ be an $\L$-formula such that $\theta_m$ is $\psi(c_1,\dots,c_n)$. By induction, $T\cup\{\theta_m\}$ is consistent. So $T\cup\{\psi(x_1,\dots,x_n)\}$ is consistent, and so $[\psi(\bar{x})]\ne \emptyset$. Then by (iii), there is some isolated type $p \in [\psi(\bar{x})]$. Let $\phi(\bar{x})$ isolate $p$. Let $\theta_{m+1}$ be $\theta_m\land \phi(c_1,\dots,c_n)$.

$T\cup\{\theta_{m+1}\}$ is consistent: choose $\N\models T$ with $\bar{a}\in N^n$ realising $p$. Expand $\N$ to an $\L^\ast$-structure such that $c_i^\N = a_i$ (for $i\le n$). Then $\N\models T\cup\{\theta_{m+1}\}$.

Now let $\M\models T^\ast$ be the Henkin model. WTS $\M$ is atomic (as an $\L$-structure). For arbitrarily large $n$, we have $\phi(x_1,\dots,x_n)$ isolating $p \in S_n(T)$ such that $T^\ast \models \phi(c_1,\dots,c_n)$. So $\tp^\M(c_1^\M,\dots,c_n^\M)$ is isolated for all $n\ge1$.

For any tuple $\bar{a}$ from $\M$, WTS $\tp^\M(\bar{a})$ is isolated. WLOG the coordinates of the tuple are distinct, \it{i.e.} $(a,b,c)$ isolated by $\psi(x_1,x_2,x_3)$, $(a,a,b,c)$ isolated by $\psi(x_1,x_3,x_4)\land x_2 = x_3$.

So $\bar{a}$ is a sub-tuple of $([c_1],\dots,[c_n])$ for some $n$.

\underline{General fact}: Given any $\M$ and $a_1,\dots,a_n\in M$, if $\tp^\M(a_1,\dots,a_n)$ is isolated by $\phi(x_1,\dots,x_n)$, then for all $\emptyset\ne I\subset \{1,\dots,n\}$, $\tp^\M((a_i)_{i\in I})$ is isolated by $(\exists x_i)_{i\not\in I}\phi(x_1,\dots,x_n)$.
\end{proof}

\section{Lecture}
$T$ is a complete theory in a countable language with infinite models.

\begin{remark*}[Recall]
For any $n\ge1$, $|S_n(T)|\le 2^{\aleph_0}$.
\end{remark*}
\begin{lemma}
	For any $n\ge 1$, if $|S_n(T)| < 2^{\aleph_0}$ then $S_n(T)$ is countable and the isolated types are dense.
\end{lemma}
Note that this is a purely topological result, as is seen in the proof.
\begin{proof}
	$S_n(T)$ is a second countable, totally disconnected, compact, Hausdorff space. Let $X$ be any such space. We show that if $X$ is uncountable \it{or} the isolated points are not dense, then $|X| \ge 2^{\aleph_0}$.

	Let $\mathcal{B}$ be a countable basis for $X$ consisting of clopen sets, and assume $\mathcal{B}$ is closed under intersections and complements (is a \it{Boolean algebra})

	\underline{Case 1}: $X$ is uncountable.

	\underline{Claim}: If $U\in \mathcal{B}$ and $|U| > \aleph_0$ then $\exists V\in \mathcal{B}$ such that $|U\cap V|,|U\backslash V| > \aleph_0$.
	
	\underline{Proof of claim}: Suppose not. Let $\mathcal{C} = \{V\in \mathcal{B}:|U\cap V| > \aleph_0\}$. Fix $V_1,V_2\in \mathcal{C}$. Set $W = V_1\cap V_2$. If $W\not\in \mathcal{C}$ then $|U\backslash W| > \aleph_0$. Note $U\backslash W = (U\backslash V_1)\cup(U\backslash V_2)$, so WLOG $|U\backslash V_1| > \aleph_0$, which is a contradiction since $V_1\in \mathcal{C}$. So $\mathcal{C}$ is a collection of non-empty closed sets, and $\mathcal{C}$ is closed under intersections. Since $X$ is compact, there is some $p \in X$ such that $p \in V$ for all $V \in \mathcal{C}$. We then show that $$U = \{p\}\cup\bigcup_{V\in \mathcal{B}\backslash\mathcal{C}}U\cap V$$
	Then $U$ is a countable union of countable sets, and hence countable.

	To do this, we fix $q \in U$ such that $q \ne p$. There is $V\in \mathcal{B}$ such that $q \in V$ and $p \not \in V$. So $V \in \mathcal{B}\backslash \mathcal{C}$. So $q \in U\cap V$.$\ \  \qedsymbol$

	\underline{Notation}: $2^\omega$ is the set of seqeuences of $0,1$ indexed by $\mathbb{N}$. $2^{<\omega}$ is the set of finite sequences of $0,1$. We have a partial order on $2^\omega \cup 2^{<\omega}$ given by proper initial segment.

	We build a collection $\{U_\sigma\}_{\sigma\in 2^{<\omega}}$ such that $\forall \sigma \in 2^{<\omega}$, $U_\sigma \in \mathcal{B}$, $|U_{\sigma}| > \aleph_0$, $U_{\sigma} = U_{\sigma 0}\cup U_{\sigma 1}$, and $U_{\sigma0}\cap U_{\sigma1} = \emptyset$. Let $U_{\emptyset} = X$. Given $U_\sigma$, let $V \in \mathcal{B}$ be as in the Claim. Let $U_{\sigma 0} =  U_\sigma \cap V$ and $U_{\sigma 1} = U_\sigma\backslash V$.

	Now, for any $\alpha \in 2^\omega$, there is $p_\alpha \in \bigcap_{i\ge 0}U_{\alpha\upharpoonright i}$, where $\alpha \upharpoonright i$ is the infinite sequence $\alpha$ cut off after the first $i$ entries. By construction, $\alpha \ne \beta\implies p_\alpha \ne p_\beta$. So $|X|\ge 2^{\aleph_0}$.

	\underline{Case 2}: The isolated points in $X$ are not dense.

	We will build $\{U_\sigma\}_{\sigma \in 2^{<\omega}}$ as above, but just iwth $U_\sigma \ne \emptyset$.

	Let $U_\emptyset$ be a non-empty clopen set with no isolated points. Suppose we have $U_\sigma$. $U_\sigma$ has no isolated points, so there exist distinct $p,q \in U_\sigma$. Partition $U_\sigma$ into $U_{\sigma 0}$ and $U_{\sigma 1}$ with $p \in U_{\sigma 0}$ and $q \in U_{\sigma 1}$ by Hausdorffness. As before, $|X|\ge 2^{\aleph_0}$.
\end{proof}

\begin{theorem}\ 
	\begin{enumerate}[label=\alph*)]
		\item Suppose $|S_n(T)|<2^{\aleph_0}$ for all $n$. Then $T$ has a prime model and a countable saturated model.
		\item If $T$ has a countable model, then $T$ has a prime model.
	\end{enumerate}
\end{theorem}
\begin{proof} (a): Apply Lemma 14.1, Theorem 13.3 and Theorem 11.2.

	(b): Apply Theorem 11.2, Lemma 14.1, Theorem 13.3.
\end{proof}

\begin{remark*}[Fact]
$\Th(\Z,+,0)$ has no countable saturated model (Ex 11.3(c)), and no prime model (Baldwin, Blass, Glass, Kuecker 1972). This is (essentially) because the type of $1$ is not isolated; there is no way to pin down what $1$ really is.

On the other hand, $\Th(\Z,+,0,1)$ then there is a prime model and no countable saturated model.
\end{remark*}

\begin{defin}
	For $\kappa \ge \aleph_0$, let $I(T,\kappa)$ be the number of models of $T$ of size $\kappa$ up to isomorphism.
\end{defin}
\begin{remark*}
	$1 \le I(T,\kappa) \le 2^\kappa$, where the upper bound is given by the number of $\L$-strucutres of size $\kappa$, which are determined (essentially) by picking subsets of a set of size $\kappa$ (relations, graphs of functions).

	[Recall Morley's Theorem: If $I(T,\kappa) =1 $ for some $\kappa > \aleph_0$, then $I(T,\kappa) = 1$ for all $\kappa > \aleph_0$.]
\end{remark*}
\begin{prop}
	If $I(T,\aleph_0) < 2^{\aleph_0}$, then $S_n(T)$ is countable for all $n\ge 1$ (and so $T$ has a prime model and a countable, saturated model).
\end{prop}
\begin{proof}
	Assume $I(T,\kappa) = \kappa < 2^{\aleph_0}$. Let $(\M_i)_{i<\kappa}$ be all countable models of $T$. Fix $n$. Let $X_i$ be the set of $p \in S_n(T)$ realised in $\M_i$. Each $X_i$ is countable and $S_n(T) = \bigcup_{i<\kappa}X_i$. So $|S_n(T)|\le \kappa < 2^{\aleph_0}$. So $S_n(T)$ is countable by Lemma 14.1.
\end{proof}

\begin{remark*}[Example]
	$T = \acf_p$. $I(T,\aleph_0) = \aleph_0$. Also $T = $ TFDAG.
\end{remark*}

\begin{remark*}[Vaught's Conjecture (1961)]
	If $I(T,\aleph_0) < 2^{\aleph_0}$, then $I(T,\aleph_0) \le \aleph_0$.
\end{remark*}
This is, of course, trivial if one assumes CH - but it is an open problem of ZFC, and one of the oldest in model theory.
\begin{remark*}[Morley (1970)]
	If $I(T,\aleph_0) < 2^{\aleph_0}$ then $I(T,\aleph_0)\le \aleph_1$.
\end{remark*}

\section{Lecture}

\begin{remark*}[Examples of $I(T,\aleph_0)$]\ \\
	\underline{$I(T,\aleph_0) = 2^{\aleph_0}$}:\ 
	\begin{enumerate}
		\item $T = \Th(\Z,+,0)$ [$|S_1(T)| = 2^{\aleph_0}\implies I(T,\aleph_0) = 2^{\aleph_0}$]
		\item $T = \Th(\Z,<)$. In this case, $S_n(T)$ is countable for all $n$ (via Sheet 2 \#5)
		
		Given a linear order $\A$, let $\M_\A = \Z\cdot \A$ (\it{i.e.} replace each point in $\A$ with a copy of $\Z$). Then $\M_\A\models T$. Now $A\not\cong \mathcal{B} \implies \M_\A \not\cong \M_{\mathcal{B}}$. By Cantor, \# of countable linear orders is $2^{\aleph_0}$, and so $I(T,\aleph_0) = 2^{\aleph_0}$.
	\end{enumerate}

	\underline{$I(T,\aleph_0) = \aleph_0$}: $\acf_p$, TFDAG.

	\underline{$I(T,\aleph_0) = 1$}: (\it{i.e.} $T$ is $\aleph_0$-categorical) DLO, RG, InfSets.
\end{remark*}

\begin{remark}
	If $T$ is $\aleph_0$-categorical then its unique countable model is saturated and prime by Prop 14.4.
\end{remark}

\begin{theorem}[Ryll-Narzewski/Enegler/Svenonius 1959]
	Let $T$ be a complete theory in a countable language with infinite models. TFAE:
	\begin{enumerate}[label=\emph{\roman*)}]
		\item $T$ is $\aleph_0$-categorical
		\item $\forall n\ge 1$, every type in $S_n(T)$ is isolated.
		\item $\forall n\ge 1$, $S_n(T)$ is finite.
		\item $\forall n\ge 1$, the number of $\L$-formulae in $x_1,\dots,x_n$ is finite, up to equivalence in $T$.
	\end{enumerate}
\end{theorem}
\begin{proof}
	(i)$\implies$(ii): Every type over $\emptyset$ is realised in the unique countable model, which is an atomic model (Remark 15.1).

	(ii)$\implies$(iii): Suppose $X$ is a compact space, and every point is isolated. Then $(\{p\})_{p\in X}$ is an open cover, which has a finite subcover; hence $X$ itself is finite.

	(iii)$\implies$(ii): If $X$ is Hausdorff and finite then all points are isolated.

	(ii)/(iii)$\implies$(iv): Fix $n\ge 1$. Let $S_n(T) = \{[p_1,\dots,p_k\}$ and let $\phi_i(\bar{x})$ isolate $p_i$. Then for any $\L$-formula $\psi(\bar{x})$, we have that
	\begin{align*}
		T\models\forall\bar{x}\left(\psi(\bar{x})\leftrightarrow\bigvee_{\psi\in p_i}\phi_i(\bar{x})\right)
	\end{align*}
	by Prop 11.6.

	(iv)$\implies$(ii): Fix $n\ge 1$. Let $\phi_1(\bar{x}),\dots,\phi_k(\bar{x})$ represent all $\L$-formulae in $x_1,\dots,x_n$. Then $p \in S_n(T)$ is isolated by
	\begin{align*}
		\bigwedge_{\phi_i\in p}\phi_i(\bar{x}) \land \bigwedge_{\phi_i\not\in p}\neg\phi_i(\bar{x})
	\end{align*}

	(ii)$\implies$(i): If (ii) holds, then every model of $T$ is atomic. So every model of $T$ is $\aleph_0$-homogeneous (Sheet 3 \#1a). Moreover, every model of $T$ realises all types in $S_n(T)$ by Prop 12.1. So every countable model of $T$ is saturated by Prop 10.5. So $T$ is $\aleph_0$-categorical by Prop 11.4.
\end{proof}
\begin{cor}
	Let $G$ be an infinite group, and $T = \Th(G)$ (in the language of groups) is $\aleph_0$-categorical. Then $G$ has finite exponent.
\end{cor}
\begin{proof}
	Want to show there is some $n\in \mathbb{N}$ such that $g^n = 1$ for all $g\in G$. Suppose not.

	\underline{Case 1}: $G$ is torsion-free. WLOG $G$ is countable. Then $T\cup \{x^n\ne 1_G : n\ge 1\}$ is finitely satisfiable, so by DLST it has a countable model $H\not\cong G\ \false$.

	\underline{Case 2}: There is some $g \in G$ of infinite order. For $k \ge 1$, let $p_k = \tp(g,g^k)\in S_2(T)$. If $k < \ell$ then $p_k$ contains $x_2 = x_1^k$, but $p_\ell$ does not. So $S_2(T)$ is infinite. Contradiction.
\end{proof}

\underline{Fact}: Any abelian group of finite exponent has an $\aleph_0$-categorical compelte theory.

\begin{cor}
	Suppose $T$ is a complete $\aleph_0$-categorical $\L$-theory, with $\L$ countable. Then, for any $\L_0 \subset \L$, $T\upharpoonright \L_0 = \{\phi\in T:\phi \textrm{ is an }\L\textrm{-sentence}\}$ is still $\aleph_0$-categorical.
\end{cor}
\begin{proof}
	Apply Theorem 15.2(iv). If there are only finitely many $\L$-formulae modulo $T$, then there's only finitely many $\L_0$-formulae modulo the restriction of $T$ to $\L_0$ - and these characterise $\aleph_0$-categoricity.
\end{proof}
\begin{ex}\ [$I(T,\aleph_0) = 3$] Let $\L = \{<,c_0,c_1,c_2,\dots\}$.\ \\
	Let $T = $ DLO $\cup\ \{c_n<c_{n+1}:n\ge0\}$.

	\underline{Claim}: $T$ is complete.
	\begin{proof}(Vaught's Test)
		Fix countable $\M,\N\models T$. Want to show that $\M\equiv \N$. It suffices to show that the reducts to any finite sublanguage are isomorphic.
		
		\underline{Note}: DLO $\cup\ \{c_0 < c_1<\dots < c_n\}$ is $\aleph_0$-categorical (\it{e.g.} as in proof of Sheet 2 \#4).

		\underline{Claim}: $I(T,\aleph_0) = 3$.
		\underline{Proof of Claim}: $\M_1$ is $(\Q,<)$ with $c_n^{\M_1} = n$ (no upper bound for the constant $c_n$s). $\M_2$ is $(\Q,<)$ with $\sqrt{2} - 1/n < c_n^{\M_2} < \sqrt{2}$ (upper bound exists, but no supremum).
		$\M_3$ is $(\Q,<)$ with $c_n^{\M_3} = 1 - 1/n$ (supremum exists).

		These are three countable models of the theory, and no two of them are isomorphic since the upper bound properties must be preserved by isomorphism.

		If $\M\models T$, then $\M\cong \M_i$ for some $i$, depending on which sup properties it has.

		This can be modified to obtain $I(T,\aleph_0) = k$ for all $k\ge 3$ (Sheet 3 \#2).
	\end{proof}
\end{ex}

\section{Lecture}

\begin{theorem}[Vaught 1959]
	Suppose $T$ is a complete $\L$-theory, with $\L$-countable. Then $I(T,\aleph_0) \ne 2$.
\end{theorem}
\begin{proof}
	Assume for contradiction that $I(T,\aleph_0) = 2$. By Prop 14.4, $T$ has a prime model $\M$ and a countable, saturated model $\N$. By Theorem 15.2, there exists some non-isolated $p\in S_n(T)$ for some $n\ge 1$. So $\M$ omits $p$, and there exists $\bar{a}\in N^n$ realising $p$. Let $T^\ast = \Th_{\bar{a}}(\N)$. Then $\N$ is still saturated as an $\L_{\bar{a}}$-structure (Sheet 3 \#3). So $T^\ast$ has a prime model $\mathcal{B}$ by Theorem 14.2(b).

	Let $\A = \mathcal{B}\upharpoonright\L\models T$. So $\A$ realises $p$. So $\A\not\cong\M$, and hence $\A\cong \N$. So then $\mathcal{B} = \N$ (Sheet 3 \#3). So the prime model of $T^\ast$ is saturated. So $T^\ast$ is $\aleph_0$-categorical by Theorem 15.2, so $T$ is $\aleph_0$-categorical by Corollary 15.4. $\false$
\end{proof}

\section*{Uncountably Saturated Models}

\begin{theorem}
	For any infinite $\M$ and any $\kappa \ge |\L| + \aleph_0$, there is an $\N\succeq\M$ such that $\N$ is $\kappa^+$-saturated and $|N| \le |M|^\kappa$.
\end{theorem}
\begin{proof}
	Fix $\kappa \ge |\L| + \aleph_0$. Notation: $X\subset_\kappa Y$ means $X\subset Y$ and $|X| \le \kappa$.

	\underline{Claim}: For any $\M$, there is $\N\succeq\M$ such that $|N|\le |M|^\kappa$ and $\N$ realises all types in $S_1^\M(A)$ for all $A\subset_\kappa M$.

	\underline{Proof of Claim}: \# subsets of $\M$ of size $\le \kappa$ is $|M|^\kappa$ and if $|A|\le \kappa$, then $\S_1^\M(A)| \le 2^\kappa\le |M|^\kappa$. Enumerate all such types as $(p_\alpha)_{\alpha < |M|^\kappa}$ ($\alpha$ ordinal). Build an elementary chain $(\M_\alpha)_{\alpha < |M|^\kappa}$ such that $\M_0 = \M$, for limit $\alpha$ $\M_\alpha = \bigcup_{i<\alpha}\M_i$, and $\M_{\alpha+1} \succeq \M_\alpha$ realises $p_\alpha$ such that $\M_{\alpha+1} \le |M_\alpha| + |\L|$ (by Prop 8.4). Let $\N = \bigcup_\alpha \M_\alpha$. Then $|N| \le |M|^\kappa$ and $\N$ realises all $p_\alpha$. $\qedsymbol$

	Fix $\M$. Now build elementary chain $(\N_\alpha)_{\alpha < \kappa^+}$ such that $|N_\alpha| \le |M|^\kappa$, and:
	\begin{enumerate}
		\item $\N_0 = \M$
		\item $\alpha$ limit, $\N_\alpha = \bigcup_{i<\alpha}\N_i$
		\item Given $\alpha < \kappa^+$, let $\N_{\alpha+1}\succeq \N_\alpha$ such that $|N_{\alpha+1}| \le |N_\alpha|^\kappa$ and $\N_{\alpha+1}$ realises all types over all sets $A\subset_\kappa N_{\alpha}$ (by the Claim).
	\end{enumerate}
	Then let $\N = \bigcup_{\alpha < \kappa^+}\N_\alpha$. By induction on $\alpha$, $|N|\le |M|^\kappa$. $\N$ is $\kappa^+$-saturated since $A\subset_\kappa N\implies A\subset_\kappa \N_\alpha$ for some $\alpha < \kappa^+$.
\end{proof}

Let $T$ be an $\L$-theory with infinite models.

\begin{cor}
	If $\kappa \ge |\L|+\aleph_0$ and $2^\kappa = \kappa^+$, then $T$ has a saturated model of size $\kappa$.
\end{cor}
\begin{proof}
	By 16.2, $T$ has a $\kappa^+$-saturated model of size $(|\L|+\aleph_0)^\kappa = 2^\kappa = \kappa^+$.
\end{proof}

\underline{Fact}: If $\kappa > |\L| + \aleph_0$ is regular and $2^\lambda \le \kappa$ for all $\lambda < \kappa$, then $T$ has a saturated model of size $\kappa$.

\begin{remark*}[Basic Facts]\ 
	\begin{enumerate}[label = \arabic*)]
		\item If $\M\equiv \N$, $|M| = |N|$ and $\M,\N$ are saturated, then $\M\cong \N$ (Sheet 3 \#4).
		\item Suppose $\kappa > |\L| + \aleph_0$. Then $\M$ is $\kappa$-saturated iff $\M$ is $\kappa$-homogeneous and for all $\N\equiv \M$, if $|N| < \kappa$ then $\N$ elementarily embeds into $\M$ (we say $\M$ is \undf{$\kappa$-universal}).
	\end{enumerate}
\end{remark*}

\section*{Stability}

Let $T$ be a complete theory with infinite models.

\begin{defin}
	Given $\kappa \ge |\L| + \aleph_0$, we say $T$ is \undf{$\kappa$-stable} if for any $\M\models T$, $|M| = \kappa$, we have $|S_1(M)| = \kappa$. $T$ is \undf{stable} if it is $\kappa$-stable for some $\kappa \ge |\L| + \aleph_0$.
\end{defin}

\begin{ex}\ 
	\begin{enumerate}
		\Myitem $\acf_p$, TFADG are $\kappa$-stable for all $\kappa \ge \aleph_0$ (see Example 9.3).
		\Myitem $T = \Th(\Z,+,0,1, (\equiv_n )_{n\ge 2})$, where $x\equiv_n y\iff \exists z(x-y = nz)$. $T$ has QE.

		Fix $\M\models T$. Given $f:\{\textrm{primes}\}\ra \mathbb{N}$ such that $0\le f(n) < n$, we have a complete $1$-type $p_f = \{x\ne a:a\in \M\}\cup \{x\equiv_n f(n):n\textrm{ is prime}\}\in S_1(M)$.

		By QE, $S_1(M) = \{\tp(a/M):a\in M\}\cup\{p_f\}_f$. So $|S_1(M)| = |M| + 2^{\aleph_0}$. Thus $T$ is $\kappa$-stable iff $\kappa \ge 2^{\aleph_0}$.
		\Myitem If $\M\models\rg$ then $|S_1(M)| = 2^{|M|}$ (Example 9.4). So RG is not $\kappa$-stable for any $\kappa$.
	\end{enumerate}
\end{ex}

\section{Lecture}

Again $T$ is complete with infinite models.

\begin{remark*}[Recall]
	A cardinal $\kappa$ is \undf{regular} if every unbounded subset of $\kappa$ has size $\kappa$ (\it{i.e.} cofinality $\kappa$). For instance, $\aleph_0$, $\aleph_1$, $2^{\aleph_0}$ are regular. $\aleph_\omega$ is not regular, as $\{\aleph_n:n\ge 1\}$ is unbounded.
\end{remark*}

\begin{theorem}
	Suppose $T$ is $\kappa$-stable and $\kappa$ is regular. Then $T$ has a saturated model of size $\kappa$.
\end{theorem}
\begin{proof}
	\underline{Step 1}: If $\M\models T$, $|M| = \kappa$, then $\exists \N \succeq \M$ such that $|N| = \kappa$ and $\N$ realises all $1$-types over $\M$.

	\underline{Proof of Step 1}: Enumerate $S_1(M) = \{p_\alpha:\alpha < \kappa\}$ by $\kappa$-stability. Build chain.

	\underline{Step 2}: Build $(\M_\alpha)_{\alpha < \kappa}$ such that $|M_0| = \kappa$. $\M_{\alpha+1}\ge \M_\alpha$ realises all $1$-types over $\M_\alpha$ and $|\M_{\alpha+1}| = \kappa$. Let $\M = \bigcup_{\alpha<\kappa} \M_\alpha$. Then $|M| = \kappa$. If $A\subset M$, $|A| <\kappa$, then $A\subset\M_\alpha$ for some $\alpha < \kappa$ since $\kappa$ is regular. So $\M$ realises all $1$-types over $A$.
\end{proof}

\begin{theorem}
	Suppose $\L$ is countable and $T$ is $\aleph_0$-stable. Then $T$ is $\kappa$-stable for all $\kappa\ge \aleph_0$.
\end{theorem}
\begin{proof}
	Fix $\kappa \ge \aleph_0$. Assume $T$ is not $\kappa$-stable, so we have $\M\models T$, $|M| = \kappa$ such that $|S_1(\M)| > \kappa$. Then there exists an $\L_M$-formula $\phi(x)$ such that $|[\phi(x)]|>\kappa$ (\it{e.g.} $[x=x]=S_1(M)$).

	\underline{Claim}: For any $\L_M$-formula $\phi(x)$ if $|[\phi(x)]| > \kappa$ then there exists an $\L_M$-formula $\psi(x)$ such that $|[\phi \land \psi]| > \kappa$, $|[\phi \land \neg \psi]| > \kappa$.

	\underline{Proof of Claim}: Adapt the claim from the proof of Lemma 14.1.

	Now build $\{\phi_\sigma\}_{\sigma\in 2^{<\omega}}$ such that $[\phi_\sigma]$ is partitioned $[\phi_{\sigma 0}]\sqcup [\phi_{\sigma 1}] $ and $[\phi_\sigma]\ne \emptyset$. Let $\N\preceq \M$ be such that $|N| = \aleph_0$ (by DLST) and $\N$ contains all parameters used in all $\phi_\sigma$s (possible since everything is countable).

	For $\alpha \in 2^\omega$, find $p_\alpha\in S_1(N)$ such that $\phi_{\alpha \upharpoonright i}\in p_\alpha$ for all $i > 0$. This gives us $|S_1(N)| = 2^{\aleph_0}$, so $T$ is not $\aleph_0$-stable.
\end{proof}

\begin{cor}
	If $T$ is $\aleph_0$-stable then $T$ has a saturated model of size $\kappa$ for all regular $\kappa \ge \aleph_0$.
\end{cor}

\begin{remark*}[Fact]
	$\aleph_0$-stable theories have saturated models of all infinite cardinalities - but this is harder to prove.
\end{remark*}

\begin{defin}
	Fix $\M\models T$, and an $\L$-formula $\phi(x_1,\dots,x_m,y_1,\dots,y_n)$. Then $p \in S_m(M)$ is \undf{definable wrt $\phi(\bar{x},\bar{y})$} if there exists an $\L_M$-formula $\psi(y_1,\dots,y_n)$ such that for all $\bar{b}\in M^n$: $$\phi(\bar{x},\bar{b})\in p \iff \M\models \psi$$
	We say $p \in S_m(M)$ is \undf{definable} if it is definable wrt any $\L$-formula $\phi(\bar{x},\bar{y})$ (any $\bar{y})$.
\end{defin}

\begin{ex}\ \\
	\begin{enumerate}[label = \arabic*)]
		\item $p = \tp(a/M)$ where $a \in M$. Given $\phi(x,\bar{y})$, let $\psi(\bar{y})$ be $\phi(a,\bar{y})$ (this is an $\L_M$-formula). So realised types are fairly trivially definable.
		\item $T$ is DLO. $\M$ is $\Q$. Choose $p \in S_1(\Q)$ such that $x < b$ is in $p$ iff $\sqrt{2}<b$. Let $\phi(x,y)$ be $x < y$. Then $\{b \in \Q: \phi(x,b)\in p\} = (-\infty,\sqrt{2})\cap\Q$. By QE, any definable subset of $\Q$ is a finite Boolean combination of intervals with endpoints in $\Q$.
	\end{enumerate}
\end{ex}

\begin{remark*}[Notation]
	Let $x$ be a tuple of variables. $M^x$ denotes $M^{|x|}$.
\end{remark*}

\begin{defin}
	Let $\phi(x,y)$ be an $\L$-formula ($x,y$ tuples). Then $\phi(x,y)$ has the \undf{order property wrt $T$} if there exists $\M\models T$ and $(a_i)_{i\ge 0}$, $(b_i)_{i\ge0}$ such that for all $i\ge 0$ we have $a_i \in M^x,b_i\in M^y$, and, for all $i,j$, $\M\models \phi(a_i,b_j)$ iff $i\le j$. 

	\it{E.g.} In $\Q$, $x\le y$ has the order property (wrt DLO). Let $a_i = b_i = i$.
\end{defin}

\begin{remark*}[Fundamental Theorem of Stability (Shelah 1976)]
	\emph{
	TFAE:
	\begin{enumerate}[label=\arabic*)]
		\item $T$ is stable.
		\item For any $\M\models T$, any $p \in S_n(M)$ is definable.
		\item No $\L$-formula has the order property wrt $T$.
	\end{enumerate}
	}
\end{remark*}

For now we only show 2$\implies$1.

\begin{proof}
	\underline{FTS2$\implies$FTS1}: Assume $2$. Fix $\kappa \ge |\L| + \aleph_0$ such that $\kappa^{|\L|+\aleph_0} = \kappa$ (\it{e.g.} $\kappa = 2^{|\L|+\aleph_0}$). We show $T$ is $\kappa$-stable.

	Fix $\M\models T$, $|M| = \kappa$. Let $X = \{\L\textrm{-formulae }\phi(x,\bar{y})\}$ and $Y = \{\textrm{all }\L_M\textrm{-formulae }\psi(\bar{y})\}$ (any $\bar{y}$). Given $p \in S_1(M)$, define $F_p : X\ra Y$ such that $F_p(\phi(x,\bar{y}))$ witnesses that $p$ is definable wrt $\phi(x,\bar{y})$. Then $p\mapsto F_p$ is an injective function (exercise) from $S_1(M)$ to $Y^X$. So $|S_1(M)| \le |Y^X| = \kappa^{|\L|+\aleph_0} = \kappa$.
\end{proof}

\section{Lecture}
$T$ complete, infinite models.
\begin{proof}\underline{FTS1$\implies$FTS3}:
	Suppose $\phi(x,y)$ has the order proprety with respect to $T$ ($x,y$ tuples). Fix $\kappa \ge |\L| + \aleph_0$. WTS $T$ is not $\kappa$-stable.

	\underline{Claim}: There is a  linear order $(I,<)$ such that $|I| > \kappa$ and there is $J \subseteq I$ such that $|J| = \kappa$ and $J$ is dense.

	\underline{Proof of Claim}: Fix minimal $\lambda \le \kappa$ such that $2^\lambda < \kappa$. Let $I = \Q^\lambda$, and the order given lexicographically: given distinct $f,g\in I$, set $f<g$ iff $f(\alpha) < g(\alpha)$ where $\alpha < \lambda$ is least such that $f(\alpha) \ne g(\alpha)$.

	Let $J = \{f\in I:\exists \alpha < \lambda \textrm{ such that }f(x) = 0\textrm{ for all }x\ge \alpha\}$. Then $|J| \le \max_{\alpha < \lambda}2^{\alpha}\le \kappa$. \qed

	Let $I$ be as in the claim. Consider
	\begin{align*}
		T \cup\{\phi(x_i,y_j):i,j\in I,i\le j\}\cup\{\neg\phi(x_i,y_j):i,j\in I,i>j\}
	\end{align*}
	This is finitely satisfiable since $\phi(x,y)$ has the order property wrt $T$. So $\exists \N\models T$ and $(a_i)_{i\in I}$ and $(b_i)_{i\in I}, a_i\ni N^d,b_i\in N^y$ and $\N\models \phi(a_i,b_j)$ iff $i\le j$. By DLST $\exists \M\preceq\N$ such that $|M| = \kappa$ and $b_i \in M^y$ for all $i\in J$. We show $|S_x(M)| \ge |I| > \kappa$, and thus $T$ is not $\kappa$-stable (ES3 \#5).
	
	For $i\in I$, let $p_i = \tp(a_i/M)$. Fix $i,j\in I$ such that $i<j$. $\exists k\in J$ such that $i < k < j$. So $\phi(x,b_k)\in p_i$ and $\phi(x,b_k)\not\in p_j$. So $p_i\ne p_j$. \qedsymbol

	Lastly, we have \underline{FTS3$\implies$FTS2}: Fix an $\L$-formula $\phi(x,y)$ such that $\phi$ does not have the order property wrt $T$. Now fix $\M\models T$. We show that any $p \in S_x(M)$ is definable wrt $\phi(x,y)$.

	This proof is long and topological.

	Let $X = S_x(M)$ and $Y= S_y(M)$. Let $A = \{\tp(a/M):a\in M^x\}\subseteq X$. $A$ is dense in $X$: Given an $\L_M$-formula $\psi(x)$, if $[\psi(x)]\ne\emptyset$ then $\M\models \exists x\psi(x)$, so $\exists a \in M^x$ such that $\M\models \psi(a)$, so $\tp(a/M)\in [\psi(x)]\cap A$. Also let $B = \{\tp(b/M):b\in M^y\}$. Then $B$ is dense in $Y$ by the same argument.

	Identify $A$ with $M^x$ and $B$ with $M^y$. Let $2 = \{0,1\}$ (as a discrete space). Define $f:A\times B\ra 2$ such that $f(a,b) = 1$ iff $\M\models \phi(a,b)$.

	\underline{Notation}: Given $a\in A$, $b\in B$, let $f_b:A\ra 2$ and $f^a:B\ra 2$ be the corresponding fiber functions (\it{e.g.} $f_b(a) = f(a,b)$).

	Given $b\in B$, let $\hat{f}_b:X\ra 2$ such that for $p \in X=S_x(M)$, $\hat{f}_b(p) = 1$ iff $\phi(x,b)\in p$. Now claim that $\hat{f}_b$ extends $f_b$: given $a\in A$, $\hat{f}_b(a) = 1$ iff $\phi(x,b)\in \tp(a/M)$ iff $\M\models \phi(a,b)$ iff $f_b(a) = 1$ by definition. Moreover, $\hat{f}_b$ is continuous: $\hat{f}_b^{-1}(\{1\}) = [\phi(x,b)]$, and $\hat{f}_b^{-1}(\{0\}) = [\neg\phi(x,b)]$ and these are both clopen sets.

	Similarly, given $a \in A$, $\hat{f}^a:Y\ra 2$ such that $\hat{f}^a(q) = 1$ iff $\phi(a,y)\in q$.

	This brings us to...

	\underline{Main Claim}: There is a separately continuous function $F:X\times Y\ra 2$ extending $f$. By separately continuous, we mean that if one coordinate is fixed the resulting function is continuous.

	For now, we suppose that this is true. Fix $ p \in S_x(M) = X$. WTS $p$ is definable wrt $\phi(x,y)$. $F^p:Y\ra 2$ is continuous. Set $D = (F^p)^{-1}(\{1\})$. $D$ is clopen in $Y = S_y(M)$. So $D = [\psi(y)]$ for some $\L_M$-formula $\psi(y)$ (ES2\#7). Fix $b\in M^y = B$.
	\begin{align*}
		\phi(x,b)\in p &\iff \hat{f}_b(p)=1\\
		&\iff F_b(p) = 1\quad \textrm{ $F_b,\hat{f}_b$ are continuous extensions of $f_b$ and $A$ is dense}\\
		&\iff F^p(b)=1\\
		&\iff \tp(b/M)\in D = [\psi(y)]\\
		&\iff M\models \psi(b)
	\end{align*}

	\underline{Proof of Main Claim}

	\undf{Goal}: show that $\forall p \in X,q\in Y$ there exist open neighbourhood $U_{pq}$ of $p$ and $V_{pq}$ of $q$ such that for all $a\in A\cap U_{pq}$ and $b\in B\cap V_{pq}$, $\hat{f}_b(p) = \hat{f}^a(q)=t_{pq}$ (definition of $t_{pq}$). ($\ast$)

	Suppose this fails. Fix $p\in X, q\in Y$ such that for all open neighbourhoods $U$ containing $p$, $V$ containing $q$, there exists $a\in A\cap U$ and $b\in B\cap V$ such that $\hat{f}_b(p)\ne \hat{f}^a(q)$. We build $(a_n)_{n\ge 0}$ from $A = M^x$ and $(b_n)_{n\ge 0}$ from $B = M^y$ such that for all $n$:

	\begin{enumerate}
		\item $\hat{f}_{b_n}(p) \ne \hat{f}^{a_n}(q)$
		\item $\forall i<n$, $f(a_n,b_i) = \hat{f}_{b_i}(p)$
		\item $\forall i < n$, $f(a_i,b_n) = \hat{f}^{a_i}(q)$
	\end{enumerate}

	Suppose we have this. Pass to subsequences so that $(\hat{f}_{b_n}(p))_n$ and $(\hat{f}^{a_n}(q))_n$ are constant.

	\underline{Case 1}: For all $n$, $\hat{f}_{b_n}(p) = 0$ and $\hat{f}^{a_n}(q) = 1$


	Given this, define $F:X\times Y\ra \{0,1\}$ such that $F(p,q) = t_{pq}$.
	
	$F$ extends $f$: for $a\in A,b\in B$, $a\in A\cap U_{ab}\implies t_{ab} = \hat{f}^a(b) = f(a,b)$.
	
	$F$ is separately continuous: \it{e.g.} $p \in X, t\in 2$:
	\begin{align*}
		(F^p)^{-1}(\{t\}) = \bigcup\left\lbrace V_{pq}:q\in Y,t_{pq} = t\right\rbrace
	\end{align*}

	So $\M\models \phi(a_i,b_j)$ for all $i < j$ and $\M\models \neg\phi(a_i,b_j)$ for all $i > j$.

	WLOG $(f(a_n,b_n))_n$ is constant (by passing to a further subsequence). If it is $1$, then $\M\models \phi(a_i,b_j)\iff i\le j$. If it is $0$, let $b_i' = b_{i+1}$. Then $\M\models \phi(a_i,b_j')\iff i\le j$.

	\underline{Case 2}: For all $n$, $\hat{f}_{b_n}(p) = 1$ and $\hat{f}^{a_n} = 0$.

	By a similar argument to above (passing to subsequences when necessary), we get WLOG that $\M\models \phi(a_i,b_j)\iff i \ge j$. But it turns out this is enough - fix $k\ge 1$. For $i,j\le k$, let $a_i'' = a_{k-i}$ and $b_i'' = b_{k-i}$. Then $\M\models \phi(a_i'',b_j'')\iff i\le j$. This does not give us an infinite instance of the order property, but rather arbitrarily long finite sequences; it is a consequence of compactness that there then exists an infinite sequence (\it{c.f.} ES3 \#6).

	So all we have left is to build the sequence with the above properties, which we do inductively:

	Pick $a_0,b_0$ such that $\hat{f}_{b_0}(p)\ne\hat{f}^{a_0}(q)$ (choose $U = X,V = Y$ in condition $\neg$($\ast$)).

	Suppose we have $(a_i)_{i < n}$ and $(b_i)_{i<n}$. Pick $a_n,b_n$ as follows. Let $U = \bigcap_{i<n}\{u\in X:\hat{f}_{b_i}(u) = \hat{f}_{b_i}(p)\}$. This is an open neighbourhood of $p$. Similarly, let $V = \bigcap_{i<n}\{v\in Y:\hat{f}^{a_i}(v) = \hat{f}^{a_i}(q)\}$.

	Now by $\neg(\ast)$ there exists $a_n\in A\cap U$ and $b_n\in B\cap V$ such that $\hat{f}_{b_n}(p) \ne \hat{f}^{a_n}(q)$.

	So for all $i < n$, $f(a_n,b_i) = \hat{f}_{b_i}(a_n) = \hat{f}_{b_i}(p)$, and $f(a_i,b_n) = \hat{f}^{a_i}(b_n) = \hat{f}^{a_i}(q)$.

	This concludes the proof of the main claim, and hence the proof of the entire theorem.
\end{proof}

\section{Lecture}

This was not Shelah's original proof; that uses more combinatorics and involves colouring binary tress; this is a more recent proof.

\begin{numremark}
	The previous proof did not use Hausdorffness of $X$ or $Y$. We could instead repeate the previous proof with a coarser topology on $Y$ generated by $[\phi(a,y)]$ for all $a \in M^x$. Then given $p \in S_x(M)$, we get a $\phi$-definition $\psi(y)$ to be a Boolean combination of $\phi(a,y)$ for $a\in M^x$.
\end{numremark}

\begin{numremark}
	Assume that $T$ is complete with infinite models, and $\L$ is countable.
	\begin{enumerate}[label=\arabic*)]
		\item $\aleph_0$-stable $\implies$ $\kappa$-stable for all $\kappa \ge \aleph_0$
		\item stable $\implies$ $\kappa$-stable for all $\kappa^{\aleph_0} = \kappa$.
	\end{enumerate}
\end{numremark}

\begin{remark*}[Shelah 1976]
	Let $\Sigma(T) = \{\kappa: T \textrm{ is }\kappa\textrm{-stable}\}$. Then $\Sigma(T)$ is one of :
	\begin{itemize}
		\item $\{\kappa \ge \aleph_0\}$ (sometimes known as $\omega$-stable). This includes $\acf_p$, TFDAG, InfSets.
		\item $\{\kappa \ge 2^{\aleph_0}\}$. Examples include $\Th(\Z,+,0)$.
		\item $\{\kappa^{\aleph_0} = \kappa\}$. For instance, $\Th(\Z^\omega,+,0)$, or any separably closed field that is not $\acf$.
		\item $\emptyset$. These theories include RG, DLO.
	\end{itemize}

	The first two cases together are known as \it{superstable}, the third is \it{strictly stable}, the first three are \it{stable} and the fourth is \it{unstable}.


\end{remark*}

\begin{remark*}[Motto]
	Stable algebraic structures are ``nice''.
\end{remark*}
\begin{remark*}[Example]
	(fields) $T = \Th(F)$ where $F$ is a field.

	[Macintyre 1971/Cherlin-Shelah 1980] If $T$ is superstable then $F\models \acf$.

	[Macintyre-Shelah-Wood 1975] If $F$ is separably closed then $T$ is stable.
\end{remark*}
\begin{remark*}[Stable Fields Conjecture]
	If $T$ is stable then $F$ is separably closed.	
\end{remark*}

\underline{Open problem}: Is $\Th(\C(t))$ stable? This is often a test question for the above.

\section{Lecture}

\subsection*{Stable Groups}

\begin{defin}[Expansion of a Group]
	Let $\L$ be a language. An $\L$-structure $G$ is an \undf{expansion of a group} if $\L$ contains the language of groups and the reduct of $G$ to the group language is a group.

	We conflate the structure $G$ with its universe for notational convenience.
\end{defin}

\begin{remark*}[Examples]\ 
	\begin{enumerate}
		\item If $G$ is an abelian group then $\Th(G,+,0)$ is stable (folklore - no attributed author).
		\item If $G$ is a free group then $\Th(G,\cdot,1)$ is stable (Sela 2006)
		\item If $P = \{2^n:n\ge 1\}$ then $\Th(\Z,+,0,P)$, where $P$ is a unary relation that determines whether or not something is a power of two, is stable (Mousa-Scanlon 2007).
		\item Let $G$ be an algebraic group over some $K\models \acf$. Consider the expansion of $G$ by relation symbols for all subsets of $G^n$ ($n\ge 1$) definable in the field language. Then $\Th(G)$ is $\omega$-stable.
	\end{enumerate}
\end{remark*}

Let $T = \Th(G)$ where $G$ is an expansion of a group. Given an $\L_G$-formula $\phi(x)$ (note we could have multiple variables instead) let $\phi(G) = \{a\in G:G\models \phi(a)\}$. Recall that $X\subseteq G$ is \undf{definable} if $X = \phi(G)$ for some $\L_G$-formula $\phi(x)$.

\begin{defin}
	Let $\phi(x,y_1,\dots,y_n)$ be an $\L$-formula. Define $H_\phi$ to be the collection of all finite-index subgroups of $G$ of the form $\phi(G,\bar{b})$ for some $\bar{b}\in G^n$. Let $G^{0}(\phi) = \bigcap_{H\in H_{\phi}}H$, which we write as $\bigcap H_\phi$. If $H_\phi = \emptyset$ then $G^{0}(\phi) = G$.
\end{defin}

\begin{ex}
	$G = (\Z,+,\cdot,0,1)$. $\phi(x,y)$ is $\exists z(x=y\cdot z)$ (colloquially, $y$ divides $x$). Then $\phi(\Z,m) = m\Z$. So $G^{0}(\phi) = \{0\}$.
\end{ex}

\begin{theorem}[Baldwin-Saxl 1976]
	Assume $T = \Th(G)$ is stable. Then for any $\L$-formula $\phi(x,y_1,\dots,y_n)$, there is a finite $\mathcal{F}\subseteq H_\phi$ such that $G^{0}(\phi) = \bigcap \mathcal{F}$. Moreover, $G^0(\phi)$ is definable by an $\L$-formula.
\end{theorem}
\begin{proof}
	Fix $\phi(x,\bar{y})$. WLOG $H_\phi \ne \emptyset$.

	\underline{Claim 1}: There exists $m\ge 1$ such that for all finite $H\subseteq H_\phi$ then there exists $\mathcal{F}\subseteq H$, $|\mathcal{F}| = m$, such that $\bigcap H = \bigcap \mathcal{F}$.

	\underline{Proof of Claim 1}: Suppose not. Fix $m\ge 1$. Then there exists finite $H\subseteq H_\phi$ such that for all $\mathcal{F}\subseteq H$, $|\mathcal{F}| = m$, we have $\bigcap H\subsetneq \bigcap \mathcal{F}$. After thinning $H$, we may assume $|H| > m$ and if $\mathcal{F}\subsetneq H$ then $\bigcap H\subsetneq \bigcap \mathcal{F}$. Let $H = \{H_1,\dots,H_k\}$ ($k>m$).

	For $1\le i \le k$, choose $g_i \in \big(\bigcap_{j\ne i}H_j\big)\backslash H_i$. Given $I \subseteq \{1,\dots,k\}$, let $g_I = \prod_{i\in I}g_i$. Then $g_I \in H_j$ if and only if $j\not\in I$:
	\begin{align*}
		j\not\in I &\implies g_i \in H_j\ \forall i\ni I \implies g_I\in H_j\\
		j\in I &\implies g_{I\backslash\{j\}}\in H_j\textrm{ and }g_j\not\in H_j\implies g_I\not\in H_j
	\end{align*}

	Choose $\bar{b}_j\in G^n$ such that $H_j = \phi(G,\bar{b}_j)$. Let $a_i = g_{<i}\coloneqq \prod_{j<i}g_j$. Then $G\models \phi(a_i,\bar{b}_j)$ iff $g_{<i}\in H_j$ iff $i\le j$. Since $k$ can be chosen arbitarily large, we get the order property for $\phi(x,\bar{y})$ by ES3 \#6. \qed

	Fix $m\ge 1$ as in Claim 1. Let $\psi(x,\bar{y}_1,\dots,\bar{y}_n)$ be $\bigwedge_{i=1}^{m}\phi(x,\bar{y}_i)$.

	\underline{Claim 2}: $H_\psi$ contains a minimal element.

	\underline{Proof of Claim 2}: Suppose not. There is $H_0 > H_1 > H_2 >\dots$ with $H_i \in H_\psi$. Choose $g_i \in H_i\backslash H_{i+1}$. Then $g_i \in H_j$ iff $j \le i$. So $\psi$ has the order property wrt $T$. \qed

	Let $H$ be a minimal element of $H_\psi$. Note that $H = \bigcap \mathcal{F}$ where $\mathcal{F}\subseteq H_\phi$, $|\mathcal{F}| = m$.

	\underline{Claim 3}: $H = G^0(\phi)$.
	
	\underline{Proof of Claim 3}: $G^0(\phi) \le H$ be definition of $G^0(\phi)$. Fix $K\in H_\phi$. WTS $H\le K$.

	By Claim 1, $H\cap K\in H_\psi$. By minimality of $H$, $H= H\cap K$, \it{i.e.} $H\le K$. \qed

	\underline{Claim 4}: $G^0(\phi)$ is definable by an $\L$-formula.

	\underline{Proof of Claim 4}: Let $k = [G:G^0(\phi)]$. Then $a\in G^0(\phi)$ iff for all $\bar{b}\in g^n$, if $\phi(G,\bar{b})$ is a subgroup of $G$ of index at most $k$, then $\phi(a,\bar{b})$.

	This is expressible by an $\L$-formula (all the above statements are first-order, exercise; the key is that the index is now uniformly bounded.\qed

	This concludes the proof of the whole theorem.
\end{proof}

\section{Lecture}

\underline{Setting}: $G$ is an expansion of a group. $T = \Th(G)$. Assume $T$ is stable.

\begin{defin}
	Let $G^0$ be the intersection of all definable finite index subgroups of $G$.
\end{defin}
\underline{Note}: $G^0 = \bigcap_{\phi}G^0(\phi)$.

\begin{ex}\ 
	\begin{enumerate}[label = \arabic*)]
		\item $\Th(\Z,+,0) = T$. If $G\models T$ then $G^0 = \bigcap_{n\ge 1}nG$. So if $G = \Z$, $G^0 = \{0\}$. If $G$ is $\aleph_0$-saturated, then $G^0$ is nontrivial.
		\item $G$ is an algebraic group over some $\acf$. $G^0$ is the connected component of $G$ wrt the Zariski topology.
	\end{enumerate}
\end{ex}

\begin{numremark}
	$G^0$ is a normal subgroup.
\end{numremark}
\underline{Goal}: Another description of $G^0$.

To work towards this goal, we will look at a more general notion of being a coset of a finite index subgroup.

\begin{defin}[bi-generic]
	$X\subseteq G$ is \undf{bi-generic} if $\exists a_1,\dots,a_n,b_1,\dots,b_n\in G$ such that $G = \bigcup_{i=1}^{n}a_iXb_i$.

	So cosets are special types of bi-generic subsets.
\end{defin}

\begin{lemma}
	If $X\subseteq G$ is definable, then either $X$ of $G\backslash X$ is bi-generic.
\end{lemma}
\begin{proof}
	Suppose not. We build $(a_i)_{i\ge 1}$ and $(b_i)_{i\ge 1}$ such that $a_ib_j \in X$ iff $i \le j$. [So if $\phi(x)$ defines $X$ then $\phi(x\cdot y)$ has the order property. Note that this glosses over the fact that $\phi$ might have parameters, but this is easy to deal with. Let $a_0 \in X$ and $b_0 = 1$.
	
	Choose $a_n \not \in \bigcup_{i<n} Xb_i^{-1}$, which exists since $X$ is not bi-generic. In particular, this says that $a_nb_i\not\in X\ \forall i<n$, and choose $b_n\not\in \bigcup_{i\le n}a_i^{-1}(G\backslash X)$ - this says that $a_ib_n\in X\ \forall i \le n$. These two statements gives us the order property up to $n$.
\end{proof}

\begin{ex}
	$\Th(\Z,+,<,0)$ is unstable. $\mathbb{N}$ and $\Z\backslash \mathbb{N}$ are definable and not bi-generic.
\end{ex}

\begin{lemma}
	If $X,Y\subseteq G$ are definable and $X\cup Y$ is bi-generic, then $X$ or $Y$ is bi-generic.
\end{lemma}
\begin{proof}
	Assume $G = \bigcup_{i = 1}^{n}a_i(X\cup Y)b_i = \underbrace{\bigcup_{i=1}^{n}a_iXb_i}_{A}\cup\underbrace{\bigcup_{i=1}^{n}a_iYb_i}_B$. Note that $A$ and $B$ are both definable also.

	If $A$ is bi-generic, then so is $X$.

	Suppose $A$ is not bi-generic. Then $G\backslash A$ is bi-generic by 21.5. So $B$ is bi-generic since $G\backslash A \subseteq B$). So $Y$ is bi-generic.
\end{proof}

\begin{defin}
	$p \in S_1(G)$ is bi-generic if every $X \in p$ is bi-generic.

	\underline{Convention}: Identify $\L_G$-formula $\phi(x)$ with $\phi(G)$.
\end{defin}

\begin{prop}
	There is a bi-generic type $p \in S_1(G)$.
\end{prop}
\begin{proof}
	Let $q = \{\neg X:X\subseteq G\textrm{ is definable and not bi-generic}\}$. Then $q$ is finitely satisfiable in $G$: Fix $X_1,\dots,X_n\subseteq G$ definable, not bi-generic. Then $X_1\cup \dots\cup X_n$ is not bi-generic by 21.7. So $\neg X_1\cap\dots\cap \neg X_n = \neg(X_1\cup\dots\cup X_n)\ne \emptyset$.

	Now extend $q$ to some $p\in S_1(G)$. If $X\in p$ then $\neg X\not\in q$, so $X$ is bi-generic.
\end{proof}

\begin{defin}
	Given $p \in S_1(G)$ and $g \in G$, define $\gp = \{gX:X \in p\}$.

	\underline{ES4}: $\gp \in S_1(G)$.
\end{defin}

\begin{defin}
	Given $p \in S_1(G)$ and $X\subseteq G$ definable, let
	\begin{align*}
		H^p_X = \{g \in G :\ \forall a \in G, aX\in p \iff aX \in gp\}
	\end{align*}
\end{defin}
\begin{theorem}
	If $p \in S_1(G)$ and $X\subseteq G$ is definable, then $H^p_X$ is a definable subgroup of $G$. Moreover, if $p$ is bi-generic then $H^p_X$ has finite index.
\end{theorem}
\begin{proof}
	Fix $p \in S_1(G)$ and definable $X \in G$. \underline{Exercise}: $H^p_X$ is a subgroup. Since $p$ is definable, there exists $\psi(y)$ such that for all $a \in G$, $aX \in p\iff G\models \psi(a)$. [$\psi(y)$ is the definition for $p$ wrt the formula $\phi(x,y)$ given by ``$y^{-1}\cdot x \in X$'']

	So $g \in H^p_X\iff G \models \forall a(\psi(a)\leftrightarrow \psi(g^{-1}a))$, which is a formula in one free variable $g$ and hence the subgroup $H^p_X$ is definable.

	Given $q,r\in S_1(G)$, write $q\sim r$ iff: $\forall a,b\in G, aXb \in q \iff aXb \in r$; this is easily shown to be an equivalence relation.

	\underline{Main Claim}: There are only finitely many bi-generic types in $S_1(G)$ mod $\sim$.
	
	For now, we assume this claim. Now assume $p$ is bi-generic. Suppose $\exists g_1,g_2,\dots$ in $G$ such that $\forall i\ne j,g_i^{-1}g_j\not\in H^p_X$. So for all $i\ne j$, there exists $a \in G$ such that $aX \in p\iff aX\not\in g_i^{-1}g_jp$, \it{i.e.} $g_iaX\in g_ip \iff g_iaX\not\in g_jp$. In particular then, for all $i\ne j$ we have $g_ip \not\sim g_jp$. But each $g_ip$ is bi-generic [ES4], and by the claim we have only finitely many such types; contradicton.
\end{proof}

\begin{cor}
	If $p \in S_1(G)$ is bi-generic, then
	\begin{align*}
		G^0 = \bigcap_{X\textrm{ def. }\subseteq G}H^p_X
	\end{align*}
	So $G^0 = \{g\in G: gp = p\}=:$ \emph{Stab}$(p)$.
\end{cor}
\begin{proof}
	$G^0 \le H^p_X$ for all definable $X\subseteq G$, so it must also lie within their intersection. For the other direction, it suffices to fix definable finite index normal subgroup $K\triangleleft G$, and show $\bigcap_{X\subseteq G}H^p_X \le K$.

	Check $H^p_K = K$.
\end{proof}

\end{document}  

