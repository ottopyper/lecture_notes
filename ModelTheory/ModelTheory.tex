\documentclass[]{article}


\usepackage{amsmath}
\usepackage{amssymb}
\usepackage{amsthm}
\usepackage{graphicx}
\usepackage{parskip}
\usepackage{xcolor}
\usepackage{pagecolor}
\usepackage[margin=1.2in]{geometry}



\usepackage[utf8]{inputenc}
\usepackage[english]{babel}

\usepackage{mathtools}
\DeclarePairedDelimiter\bra{\langle}{\rvert}
\DeclarePairedDelimiter\ket{\lvert}{\rangle}
\DeclarePairedDelimiterX\braket[2]{\langle}{\rangle}{#1 \delimsize\vert #2}

\definecolor{thmcolour}{rgb}{0,1,1}
\definecolor{defcolour}{rgb}{1,0,1}
\definecolor{textcolour}{rgb}{1,1,1}
\definecolor{backgroundcolour}{rgb}{0,0,0}

\pagecolor{backgroundcolour}
\color{textcolour}

\newtheoremstyle{custhm}
{
	%space above
	1em
}{
	%space below
	1em
}{
	%body font
	\color{thmcolour}
}{
	%indent amount
	-0.5em
}{
	%head font
	\bfseries\large\color{thmcolour}
}{
	%head punct
}{
	%after head space
	0em
}{
	%head spec
	
	\if\relax\detokenize{#3}\relax \thmname{#1}\thmnumber{ #2}:
	%stuff
	\else \thmname{#3}\thmnumber{ #2}:
	%stuff
	
	\fi
}

\newtheoremstyle{remark}
{
%space above
}{
%space below
}{
% body font
}{
%indent amount
-0.6em
}{
%head font
\bfseries
}{
%head punct
}{
%after head space
0em
}{
%head spec
\if\relax\detokenize{#3}\relax \thmname{#1}:
\else \thmname{#3}:
\fi
}

\newtheoremstyle{cusdef}
{%space above
1em
}{%space below
1em
}{%body font
\color{defcolour}
}{%indent amount
-0em
}{%head font
\bfseries\large\color{defcolour}
}{%head punct
}{%after head space
0em
}{%head spec
\if\relax\detokenize{#2}\relax\underline{\thmname{#3}}:
\else\underline{\thmname{#3}\thmnumber{ #2}}:
\fi
}

\theoremstyle{custhm}
\newtheorem{theorem}{Theorem}[section]
\theoremstyle{cusdef}
\newtheorem{defin}[theorem]{Definition}
\theoremstyle{custhm}
\newtheorem{lemma}[theorem]{Lemma}

\theoremstyle{cusdef}
\newtheorem*{defin*}{Definition}

\theoremstyle{remark}
\newtheorem*{remark*}{Remark}


%\marginpar{to describe which lecture it is}

\newcommand{\N}{\mathbb{N}}
\newcommand{\Z}{\mathbb{Z}}
\newcommand{\Q}{\mathbb{Q}}
\newcommand{\R}{\mathbb{R}}
\newcommand{\e}{\mathrm{e}}
\newcommand{\ra}{\rightarrow}
\newcommand{\lef}{\left(}
\newcommand{\res}{\right)}
\newcommand{\ie}{\textit{i.e.}}
\newcommand{\eps}{\varepsilon}
\newcommand{\E}{\mathbb{E}}
\newcommand{\suminf}{\sum_{n=0}^{\infty}}
\newcommand{\suminfa}[1]{\sum_{#1=0}^{\infty}}
\renewcommand{\P}{\mathbb{P}}
\newcommand{\undf}[1]{\underline{\textbf{#1}}}
\renewcommand{\L}{\mathcal{L}}
\renewcommand{\it}[1]{\textit{#1}}
\newcommand{\M}{\mathcal{M}}
\renewcommand{\phi}{\varphi}
\newcommand{\proves}{\vdash}
\newcommand{\lra}{\leftrightarrow}

\renewcommand{\lnot}{\neg}
\newcommand{\false}{\bot}
\newcommand{\true}{\top}

\title{Model Theory}
\author{Lectures by Gabriel Conant}
\date{}

\begin{document}

\maketitle
\clearpage
\tableofcontents
\clearpage

\section{Review of First Order Logic}

A \undf{language} is a set $\L$ of function symbols, relation symbols, and constant symbols. Additionally, each function/relation symbol has an assigned \it{arity} $n\ge 1$.

By convention, we view constant symbols as `function symbols of arity 0'.

An \undf{$\L$-structure} $\mathcal{M}$ consists of:
\begin{itemize}
	\item a non-empty set $M$ (the \undf{universe} of $\mathcal{M}$)
	\item for every function symbol $f$ of arity $n$, a function $f^{\M}:M^n\ra M$
	\item for every relation symbol $R$ of arity $n$, a subset $R^\M\subseteq M^n$
	\item for every constant symbol $c$, an element $c^\M \in M$ (\ie\ identified with the unique element in its image)
\end{itemize}

\undf{Syntax}: we build formulas using symbols in $\L$ along with $$ \land\ \lnot\ \forall\ =\ (\ )\ ,$$and countably many variable symbols.

\undf{$\L$-term}: these are our way of creating new functions by composing the ones we already have.
\begin{itemize}
	\item constant symbols and variables are terms
	\item if $t_1,\dots,t_n$ are terms and $f$ is an $n$-ary function symbol, then $f(t_1,\dots,t_n)$ is a term
\end{itemize}

Given a structure $\M$ and a term $t$, we are going to interpret the term in the structure in exactly the way you might expect. Inductively, define (for appropriate $r$) $t^\M:M^r\ra M$ as:
\begin{itemize}
	\item constant symbol $c$: $c^\M$ (case $r = 0$)
	\item variable $x$: identify function ($r = 1$)
	\item general term $f(t_1,\dots,t_n):\ f^\M(t_1^\M,\dots,t_n^\M)$
\end{itemize}

\undf{$\L$-formulas}: new relations. We have the following \it{atomic} $L$-formulas:
\begin{itemize}
	\item If $t_1$ and $t_2$ are terms, then $(t_1=t_2)$ is a formula
	\item If $R$ is an $n$-ary relation symbol and $t_1,\dots,t_n$ are terms, then $R(t_1,\dots,t_n)$ is a formula
\end{itemize}
We can then create more complicated formulas. Given formulae $\phi$ and $\psi$:
\begin{itemize}
	\item $\lnot\phi$
	\item $(\phi \land \psi)$
	\item $\forall x\phi$ for any variable $x$
\end{itemize}

An occurrence of a variable $x$ is \undf{free} in $\phi$ if $x$ does not occur in the scope of $\forall x$. Otherwise, the occurrence is \undf{bound}.

For instance, if $\phi$ is the statement $\forall x \lnot(f(x)=y)$, $x$ is bound and $y$ is free. 

\undf{Notation}: Given a formula $\phi$, we write $\phi(x_1,\dots,x_n)$ to denote that $x_1,\dots,x_n$ are the free variables of $\phi$.

Given a formula $\phi(x_1,\dots,x_n)$, a structure $\M$, $a_1,\dots,a_n\in M$, we define ``$\bar{a}$ satisfies $\phi$ in $\M$'', written $\M\models\phi(a_1,\dots,a_n)$, as follows:
\begin{itemize}
	\item If $\phi$ is $(t_1 = t_2)$ then $\M\models \phi(\bar{a})$ iff $t_1^\M(\bar{a}) = t_2^\M(\bar{a})$
	\item If $\phi$ is $R(t_1,\dots,t_n)$ then $\M\models\phi(\bar{a})$ iff $(t_1^\M(\bar{a}),\dots,t_n^\M(\bar{a})\in R^\M$
	\item $\M\models (\phi\land\psi)(\bar{a})$ iff $\M \models \phi(\bar{a})$ and $\M\models \psi(\bar{a})$
	\item $\M \models \lnot \phi(\bar{a})$ iff $\M\not\models \phi(\bar{a})$
	\item Suppose $\phi$ is $\forall w \psi(x_1\dots,x_n,w)$. Then $M\models \phi(\bar{a})$ iff for all $b\in M$, $\M\models \psi(\bar{a},b)$
\end{itemize}

We emphasise that the focus of this course will not be on the precise definitions and semantics, so much as the meaning of what we are doing. All we seek is a first order logic that works for us, so that we can use it to do interesting things.

\undf{Abbreviations}: We have \it{global} abbreviations such as
\begin{itemize}
	\item $(\phi\lor\psi)$ is $\lnot(\lnot\phi\land\lnot\psi)$
	\item $(\phi\ra\psi)$ is $(\lnot\phi\lor\psi)$
	\item $(\phi\lra\psi)$ is $(\phi\ra\psi)\land(\psi\ra\phi)$
	\item $\exists x \phi$ is $\lnot\forall x\lnot \phi$
\end{itemize}

We note that the last equivalence in a semantic sense hinges on the assumption that universes are non-empty. Since we will be almost exclusively be studying infinite structures, we will not worry about this.

We also have \it{local} abbreviations, often specific to the language we are studying. For instance, in $\L = \{+,\cdot,<,0,1\}$ (the language of ordered rings):
\begin{itemize}
	\item $x+y$ is $+(x,y)$
	\item $x < y$ is $<(x,y)$
	\item $x \le y$ is $(x<y) \land (x=y)$
	\item $x < y < z$ is $(x < y)\land (y<z)$
	\item $x^2$ is $x\cdot x$
	\item $nx$ is $\underbrace{x+x+\cdots+x}_{n\textrm{ times}}$
\end{itemize}

An \undf{$\L$-sentence} is an $\L$-formula with no free variables. For instance, $\forall x (f(x)\ne y)$ is \it{not} a sentence, but $\exists y\forall x(f(x)\ne y)$ \it{is} a sentence. Sentences can be thought of as actually saying something meaningful.

If $\phi$ is a sentence and $\M$ is a structure, then we have the notion of $\M\models\phi$, ``$\M$ satisfies $\phi$'' or ``$\M$ models $\phi$''.

\begin{defin*}[$L$-theory]
	An \undf{$L$-theory} is a set of $L$-sentences.

Given a theory $T$, we write $\M\models T$ (``$\M$ is a \undf{model} of $T$) if $\M\models\phi$ for all $\phi \in T$.

T is \undf{satisfiable} if it has a model.
\end{defin*}

\undf{Example}: $T = \{\lnot \exists x(x=x)\}$ - this sentence claims there are no elements in the universe. In our setting, this is unsatisfiable (though it is technically a matter of opinion).

Similarly, $\exists x(x=x)$ (``The Axiom of Non-Triviality'') is always satisfied in any $\L$-structure.

\undf{Recall}: $T$ is \undf{consistent} if it does not prove a contradiction (\it{e.g.} $(\phi\land\lnot\phi))$

A consequence of \undf{G{\"o}del's Completeness Theorem} is that a theory is satisfiable iff it is consistent. This is a very important theorem, though we will mostly be focussing on the model theoretic aspect (satisfiability).

\end{document}
